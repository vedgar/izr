\chapter{Turing-izračunljivost}\label{ch:Turing}

\section{Zašto nam treba još jedan model?}

Uveli smo dva modela izračunljivosti brojevnih funkcija --- RAM-izračunljivost i parcijalna rekurzivnost --- i dokazali da su ekvivalentni, usprkos bitno različitim pristupima (imperativnom odnosno funkcijskom) definiciji algoritma. To je dobar argument za Church--\!Turingovu tezu, da su izračunljive funkcije iste u svim modelima --- odnosno u modernijem obliku, da su sve programske paradigme jednako snažne. Ipak, pritom su zanemarena dva aspekta izračunljivosti.

% \subsection{Elementarne operacije}

Prvi je izračunljivost funkcija koje nisu brojevne. Rekli smo u uvodnom poglavlju, i opravdali to mnogo puta kasnije --- skup $\N$ je idealan za matematički tretman izračunljivosti, ali nije baš vjeran onom što se događa u praksi. Glavna razlika leži u konačnosti odnosno beskonačnosti skupova relevantnih za izračunavanje.

Recimo, u RAM-modelu, skup mogućih stanja pojedinog registra je beskonačan. Često je problematično shvaćanje beskonačnosti kao "jako mnogo, pa još malo više". Zapravo, bliže definiciji beskonačnosti bilo bi "jako mnogo, pa onda još beskonačno više" --- ma koliko velik konačni skup uzeli od beskonačnog skupa, ostatak je jednako velik kao da nismo uzeli ništa.

Konkretno, lako se zavarati primjerima u kojima u RAM-registrima stoje brojevi poput $0$, $150$ ili $2^{257}$, i misliti da su RAM-registri nešto kao obični procesorski registri, samo "malo veći". Kodiranje pokazuje koliko je to daleko od istine: u RAM-registar možemo staviti i brojeve poput broja $e_0$ iz primjera~\ref{pr:Qflatkod}, koji ima $579\,690\,725\,279$ \emph{znamenaka}! Štoviše, takva procedura nije ništa neuobičajeno; ona odgovara prirodnom postupku računanja $\f{univ}(\kr{2,4},e_0)$, odnosno konstrukciji rezultata $Q^\flat$-izračunavanja s $(2,4)$, na univerzalnom RAM-stroju (dobivenom kompiliranjem simboličke definicije funkcije $\f{univ}$).

Stvarna računala, naravno, takve složene strukture (zamislimo $e_0$ kao \emph{bytecode}, neku vrstu strojnog koda) ne kodiraju pomoću prirodnih brojeva, već preko nizova bitova, bajtova ili većih procesorskih \emph{riječi}. Ključno je da je skup $\Gamma$ svih stanja pravog procesorskog registra \emph{konačan}. Način na koji se onda reprezentira potencijalna beskonačnost ulaznih podataka (što moramo, jer jedino beskonačni skupovi imaju netrivijalnu teoriju izračunljivosti), kao i međurezultata u izračunavanju, je kroz neograničenost same memorije, odnosno broja memorijskih ćelija koje sadrže po jedan element iz $\Gamma$.

I ovdje vrijedi univerzalni princip da na konačnim skupovima možemo dopustiti bilo kakve transformacije kao elementarne korake, reprezentirajući ih tablicama --- dok s beskonačnim skupovima moramo biti oprezni, ograničavajući transformacije samo na one izračunljive. Kako su ti beskonačni skupovi u pravilu trivijalno izomorfni s $\N$, kao osnovne korake dopuštamo samo prelazak na sljedeći odnosno prethodni (ako već nije~$0$) prirodni broj.

Zato smo na konfiguracijama RAM-stroja dozvoljavali samo one prijelaze kod kojih se stanje pojedinog registra mijenja za najviše $1$ u svakom koraku; dok smo za stanje programskog brojača dozvoljavali skokove na proizvoljnu legalnu vrijednost --- upravo jer legalnih vrijednosti programskog brojača za fiksni RAM-stroj ima konačno mnogo. Također, osnovna ideja od koje dolazi i ime RAM-stroja, \emph{random access}, znači da njegova "memorijska sabirnica" može adresirati proizvoljni registar u jednom koraku --- što možemo upravo jer adresa relevantnih registara za fiksni RAM-algoritam ima konačno mnogo.

Ako želimo beskonačnost prikazati ne kroz veličinu sadržaja pojedinog registra nego kroz broj ćelija potrebnih da se zapiše podatak, zapravo imamo "transponirani" model: na pojedinoj ćeliji (jednom kad dođemo do nje) ćemo moći napraviti proizvoljnu transformaciju u jednom koraku, jer je skup $\Gamma$ mogućih stanja pojedine ćelije konačan --- ali pristup do pojedine ćelije više neće moći biti \emph{random access}, već ćemo u jednom koraku samo moći adresu trenutno adresirane ćelije povećati ili smanjiti (osim ako je već jednaka $0$) za jedan. To je \emph{jezični model} izračunavanja, tako nazvan jer odgovara onom kako (barem zapadni) jezici funkcioniraju: od konačnog broja slova u abecedi nizanjem možemo dobiti proizvoljno komplicirane riječi, rečenice i tekstove. Da bismo povećali izražajnost, ne uvodimo nova slova, već pišemo dulje rečenice.

Možda ovdje treba objasniti kako moderna računala postižu \emph{random access} i na potencijalno neograničenoj memoriji. Objašnjenje je jednostavno: varaju. Njihova memorija \emph{nije} potencijalno neograničena, jer ovisi o veličini adresnog prostora. Jedno $64$-bitno računalo, koliko god mu virtualne memorije dali, ne može adresirati više od $2^{64}$ bajtova. To varanje u stvarnom svijetu prolazi jer s trenutnom tehnologijom ne možemo uopće sastaviti funkcionalnu memoriju od $2^{64}$ bajtova (što je više od osamnaest milijuna terabajta), ali za $32$-bitna računala to ograničenje (na $4\,GiB$) je bilo vrlo stvarno i nezgodno, i uostalom jedan od glavnih razloga za prijelaz na $64$-bitnu arhitekturu.

Ista vrsta varanja koja je nedavno došla do granice svojih mogućnosti je internetski protokol IPv4, koji je dobro spomenuti jer daje uvid u to kako se takvi problemi mogu riješiti iteriranjem adresiranja: NAT (\emph{Network Address Translation}) pokazuje da ako se u jednom koraku (DNS) može adresirati $t=2^{32}$ računala, u dva koraka (DNS + \emph{router}) se može u idealnom slučaju adresirati $t^2=2^{64}$ računala. Više nam vjerojatno neće nikada trebati jer ćemo u međuvremenu prijeći na IPv6, ali u $n$ koraka mogli bismo adresirati $t^n$ računala, zamišljajući generaliziranu IP-adresu kao $n$-znamenkasti broj u bazi $t$ --- ili $32n$-znamenkasti broj u bazi $2$. Jedina razlika kod jezičnog modela je što umjesto binarnog zapisa adrese korisimo unarni, u kojem su elementarne operacije samo inkrementi dekrement (koji ostavlja nulu fiksnom).

Iako je unarni zapis eksponencijalno lošiji od binarnog (i svih ostalih pozicijskih) zapisa --- broj koji u bazama $2$, $3$, $4$,~\ldots\ ima nekoliko desetaka znamenaka, zapisan unarno može imati milijarde milijardi milijardi~\ldots\ "znamenaka" --- opet, to je samo razlika u složenosti, odnosno u performansama algoritma, ne u samom postojanju algoritma, i kao takva neće nam biti važna. Ono što nam jest važno je da u jezičnom modelu adresiranje pojedine ćelije zahtijeva netrivijalne algoritme, a bilo kakva promjena sadržaja ćelije je elementarna operacija --- upravo suprotno od RAM-modela.

\subsection{Računanje bez računala}

Drugi aspekt koji smo u potpunosti zanemarili u brojevnom modelu je povijesni. Danas, kad smo na svakom koraku okruženi računalima, a većina nas jedno nosi u džepu, lako je zaboraviti da računala ne postoje oduvijek. Zapravo, u upotrebljivom obliku postoje tek nekoliko desetaka godina. S druge strane, algoritmi postoje već milenijima: Euklidov algoritam nastao je prije modernog brojenja godina, a neki babilonski algoritmi potječu sa samih početaka pisane povijesti. Što znači algoritam ako nema računala na kojem se može izvršavati?

Odgovor je jednostavan: algoritme su izvršavali ljudi. Iako ljudski mozak po svojoj prirodi nije savršen supstrat za doslovno slijeđenje instrukcija kroz višeznamenkaste brojeve koraka, moderno obrazovanje svjedoči da se može tome na\-u\-či\-ti. Doista, većina zadataka iz školske matematike može se svesti na provođenje nekog algoritma. Mnoge takve zadatke računala rješavaju brže i točnije od ljudi, što se vidi kroz uspjeh aplikacija kao što je \texttt{photomath}. Iako su ljudi bolji u pronalaženju (logičkih, analogijskih ili asocijativnih) veza među pojmovima, potreba za rješavanjem problema iz stvarnog života koji se mogu precizno klasificirati uvjetovala je pronalazak mnogih algoritama daleko prije izuma računala. Motivacija je uvijek bila ista: optimizacija i specijalizacija ljudskog rada. Jedan čovjek može osmisliti algoritam, koji poslije milijuni ljudi mogu provoditi i tako rješavati stvarne probleme, ne razumijevajući nužno zašto algoritam radi. No da bi to uspjelo, koraci algoritma moraju biti takvi da njegovo provođenje ne zahtijeva nikakav angažman pored onog koji je specificiran algoritmom --- niti ikakvo vanjsko znanje osim poznavanja ulaznih podataka, i nekoliko elementarnih vještina za koje smatramo da su svojstvene svim radno sposobnim ljudima, poput čitanja i pisanja fiksnog skupa simbola te odlučivanja na osnovi pročitanog.

Britanski matematičar Alan Mathison Turing prvi je uspješno formalizirao taj koncept. U svom članku~\cite{turing}, prije više od 80 godina i svakako prije nastanka digitalnih elektroničkih računala, Turing govori o ljudima koji računaju decimale nekih konkretno zadanih realnih brojeva --- za što danas znamo da je vrlo slično računanju vrijednosti nekih konkretno zadanih brojevnih funkcija. Iako je i prije tog članka bilo pokušaja formalizacije algoritma, Turingov se ističe po tome što detaljno motivira svoje definicije, koristeći tada poznate činjenice vezane uz ljudsku percepciju i kogniciju. Čovjekov fizički rad za vrijeme provođenja algoritma, i njegova misaona stanja kroz koja prolazi, nisu samo incidentni dio opisa algoritma --- oni su u tom članku suštinski ugrađeni u definiciju. Tako možemo biti sigurni da doista modeliramo ono što se događa u stvarnom svijetu kad čovjek provodi algoritam, a ne matematičku apstrakciju kao što je recimo $\lambda$-račun.

%\subsection{Turingov stroj}

Odakle onda u priči Turingovi \emph{strojevi}? Turing je bio svjestan fenomena da je lako pomisliti kako opisujemo postupak koji ne zahtijeva nikakvo eksterno znanje, a da to zapravo nije istina. Razumijevanje napisanog ili izgovorenog jezika, prepoznavanje objekata na slikama, pa čak i osnove socijalnog ponašanja, vještine su koje smo toliko duboko internalizirali da nam se čine elementarnima --- a zapravo pretpostavljaju ogromne količine znanja o svijetu koji nas okružuje. Jednostavni primjer: rečenice hrvatskog jezika "Ana i Marija su sestre."\ i "Ana i Marija su majke."\ imaju potpuno istu sintaksnu strukturu, ali fundamentalno različitu semantiku, za čiju je konstrukciju potrebno netrivijalno znanje o ljudskoj biologiji (mogu biti sestre jedna drugoj, ali ne mogu biti majke jedna drugoj) --- a ipak nam se svaka od te dvije semantike nametne sasvim prirodno čitajući odgovarajuću rečenicu, i uopće ne razmišljamo kako bi moglo biti drugačije, sve dok ih ne vidimo jednu pored druge.

Da bi svoje čitatelje uvjerio kako njegove elementarne operacije doista ne zahtijevaju nikakvo implicitno pretpostavljeno znanje, Turing je paralelno opisao i zamišljeni, idealizirani, \emph{stroj} koji može provoditi te operacije. Taj stroj, odnosno njegovu matematičku formalizaciju (neki detalji su kasnije promijenjeni radi lakšeg razumijevanja) danas nazivamo Turingovim strojem. Ipak, treba razumjeti da "ove operacije su toliko elementarne da bi ih mogao provoditi i mehanički stroj" nije poziv na konstrukciju stvarnog stroja, već apel na intuiciju da smo osnovni opis algoritma lišili svega suvišnog.

U literaturi postoje brojne varijante Turingovog stroja --- mi slijedimo~\cite{sipser}, uz neke male modifikacije kako bismo lakše dokazivali teoreme. Za početak ponovimo definicije.

Ulazna abeceda, ili samo \emph{abeceda}, je konačan neprazan skup. Obično je označavamo sa $\Sigma$ i smatramo fiksnom. Njene elemente zovemo \emph{znakovima} --- ne\-od\-re\-đe\-ne znakove (znakovne varijable) pišemo malim grčkim slovima s početka alfabeta ($\alpha$, $\beta$, $\gamma$), dok konkretne znakove pišemo u fontu fiksne širine: $\t a$, $\t b$, $\t 0$, $\t 1$. \emph{Riječ} (nad $\Sigma$) je bilo koji konačan niz znakova, najčešće označen slovom $w$. Riječi pišemo konkatenacijom znakova: recimo, riječ $(\t0,\t1,\t1)$ pišemo kao $\t{011}$. Dakle, skup svih riječi je skup $\Sigma^*$ svih konačnih nizova znakova. Duljinu riječi označavamo s $\dulj{w}$. Prazni niz (duljine $0$) zovemo \emph{praznom rječju} i označavamo s $\varepsilon$. \emph{Jezik} (nad $\Sigma$) je bilo koji podskup od $\Sigma^*$. \emph{Jezična funkcija} (nad $\Sigma$) je bilo koja parcijalna funkcija $\varphi:\Sigma^*\rightharpoonup\Sigma^*$.

\begin{napomena}[{name=[abeceda je dio identiteta jezične funkcije]}]
Kao i mjesnost kod brojevnih funkcija i relacija, tako i abecedu kod jezika i jezičnih funkcija smatramo dijelom njihovog identiteta; preslikavanje nad $\{\t a,\t b\}$ koje riječi pridružuje njen reverz (obrnuto čitanu riječ), različito je od preslikavanja nad $\{\t a,\t c\}$ zadanog istim pravilom. Ili, jezik svih riječi koje se sastoje samo od znakova $\t a$ i $\t b$ je različit kao jezik nad $\{\t a,\t b,\t c\}$ i kao jezik nad $\{\t a,\t b,\t d\}$ --- iako se u ovom slučaju radi o skupu s istim elementima. Motivacija je slična kao u slučaju praznih relacija: komplementi su različiti, a i karakteristične funkcije jezika su različite (jer imaju različite domene).
\end{napomena}

\begin{definicija}[{name=[Turingov stroj]}]
Neka je $\Sigma$ abeceda. \emph{Turingov stroj} (nad $\Sigma$) je matematički (idealizirani) stroj, obično zapisan kao uređena sedmorka $\mathcal T=(Q,\Sigma,\Gamma,\bl,\delta,q_0,q_z)$, koji sadrži:
\begin{itemize}
    \item konačni skup \emph{stanja} $Q$, s istaknutim elementima\newline $q_0\in Q$ (\emph{početno} stanje) i $q_z\in Q$ (\emph{završno} stanje);
    \item konačnu \emph{radnu abecedu} $\Gamma\supset\Sigma$, s istaknutim elementom $\bl\in\Gamma\setminus\Sigma$ (\emph{praznina});
    \item konačnu \emph{funkciju prijelaza} $\delta:(Q\setminus\{q_z\})\times\Gamma\to Q\times\Gamma\times\{-1,1\}$.\qedhere
\end{itemize}
\end{definicija}

Umjesto registara, Turingov stroj ima \emph{ćelije} (također adresirane prirodnim brojevima), svaka od kojih u svakom trenutku izračunavanja sadrži proizvoljni element od $\Gamma$. Kao što su kod RAM-stroja na početku izračunavanja svi registri osim ulaznih bili inicijalizirani na $0$, tako će kod Turingovog stroja sve ćelije osim ulaznih biti inicijalizirane na $\bl$. Po analogiji s $\N_+=\N\setminus\{0\}$ označimo $\Gamma_+:=\Gamma\setminus\{\bl\}$.

\begin{definicija}[{name=[Turing-konfiguracije i prijelazi među njima]}]
Neka je $\mathcal T=(Q,\Sigma,\Gamma,\bl,\delta,q_0,q_z)$ Turingov stroj. \emph{Konfiguracija} od $\mathcal T$ je bilo koja uređena trojka $(q,n,t)\in Q\times\N\times\Gamma^\N$, takva da je niz $t$ skoro svuda $\bl$ (odnosno, $t^{-1}[\Gamma_+]$ je konačan skup). Komponente konfiguracije zovu se redom \emph{stanje}, \emph{pozicija} i \emph{traka}. Konfiguracija je \emph{završna} ako joj je stanje završno ($q_z$). \emph{Početna konfiguracija} s ulazom $w=\alpha_0\alpha_1\dotsm\alpha_{\dulj{w}-1}\in\Sigma^*$ je trojka $(q_0,0,w\bl\ldots)$, gdje je traka definirana s $(w\bl\ldots)_i:=\begin{cases}
\alpha_i,&i<\dulj{w}\\
\,\bl,&\text{inače}
\end{cases}$.

Za konfiguracije $c=(q,n,t)$ i $d=(q',n',t')$ istog Turingovog stroja $\mathcal T$ kažemo da $c$ \emph{prelazi} u $d$, pišući $c\leadsto d$, ako je $c$ završna i $c=d$, ili uz oznake $\delta(q,t_n)=:(p,\beta,d)$ vrijedi $q'=p$, $n'=\max\,\{n+d,0\}$, $t'_n=\beta$ te $t'_i=t_i$ za sve $i\in\N\setminus\{n\}$.
\end{definicija}

Traku možemo zamisliti kao s jedne strane ograničen, a s druge strane neograničen, niz ćelija, u kojem su od nekog mjesta nadalje samo prazne ćelije (one u kojima piše praznina). Ulaz za Turingov stroj je riječ nad $\Sigma$, koja se na početku izračunavanja zapiše na lijevi kraj trake redom (ostatak trake je prazan). U svakom koraku, funkcija prijelaza trenutno stanje i sadržaj trenutne ćelije preslikava u novo stanje, novi znak trenutne ćelije te pomak ulijevo ili udesno na susjednu ćeliju, koja time postaje trenutna u idućem koraku (pomak ulijevo od početne ćelije rezultira ostajanjem na mjestu). To se događa dok konfiguracija ne postane završna, i tada, ako je traka oblika $v\bl\ldots$ za neku riječ $v\in\Sigma^*$, kažemo da je $v$ izlaz Turingovog stroja s ulazom $w$.

\begin{lema}[{name=[determinističnost Turingovih strojeva]}]\label{lm:Turingdet}
Svaka konfiguracija Turingovog stroja prelazi u jedinstvenu konfiguraciju.% tog stroja.
\end{lema}
\begin{proof}
Neka je $\mathcal T$ Turingov stroj te $c=(q,n,t)$ proizvoljna njegova konfiguracija. Ako je $q=q_z$, tada $c\leadsto c$, i ni u koju drugu konfiguraciju jer $\delta$ nije definirana u $(q_z,t_n)$. Ako pak $c$ nije završna, postoje jedinstveni $p$, $\beta$ i $d$ takvi da je $\delta(q,t_n)=(p,\beta,d)$, koji jednoznačno (zajedno s $q$, $n$ i $t$) određuju $q'$, $n'$ i $t'$ takve da $c\leadsto(q',n',t')$.
\end{proof}

\begin{definicija}[{name=[Turing-izračunljiva jezična funkcija]}]\label{def:Tcomputefi}
Neka je $\Sigma$ abeceda, neka je $w\in\Sigma^*$ riječ te neka je $\mathcal T$ Turingov stroj nad~$\Sigma$. \emph{$\mathcal T$\!-izračunavanje s $w$} je niz $(c_n)_{n\in\N}$ konfiguracija od $\mathcal T$, takav da je $c_0$ početna konfiguracija s ulazom $w$, a za svaki $i\in\N$, $c_i\leadsto c_{i+1}$. Kažemo da to izračunavanje \emph{stane} ako postoji $n_0\in\N$ takav da je $c_{n_0}$ završna konfiguracija.

Neka je $\varphi$ jezična funkcija nad $\Sigma$. Kažemo da $\mathcal T$ \emph{računa} $\varphi$ ako za sve $w\in\Sigma^*$ vrijedi:
\begin{itemize}
    \item Ako je $w\in\dom\varphi$, tada $\mathcal T$-izračunavanje s $w$ stane i završna konfiguracija mu je oblika $\bigl(q_z,n,\varphi(w)\bl\ldots\bigr)$ za neki $n\in\N$ (pozicija nije bitna).
    \item Ako $w\notin\dom\varphi$, tada $\mathcal T$-izračunavanje s $w$ ne stane.
\end{itemize}
Za jezičnu funkciju $\varphi$ kažemo da je \emph{Turing-izračunljiva} ako postoji Turingov stroj koji je računa.
\end{definicija}

Kao i za RAM-model, mogli bismo dokazati da za svaki Turingov stroj $\mathcal T$ i njegov ulaz $w$ postoji jedinstveno $\mathcal T$-izračunavanje s $w$, ali više ne vrijedi da svaki Turingov stroj računa neku jezičnu funkciju. Traka u završnoj konfiguraciji ne mora biti oblika $v\bl\ldots$ za $v\in\Sigma^*$: može sadržavati znakove iz $\Gamma_+\!\setminus\Sigma$, ili sadržavati neki znak iz $\Sigma$ nakon prve praznine. Za takve Turingove strojeve nećemo reći da računaju ikakvu funkciju. Zapravo ih nećemo uopće promatrati, ali dobro ih je imati na umu ako iskazujemo teoreme univerzalno po svim Turingovim strojevima.

\subsection{Primjer Turing-izračunavanja}

\begin{primjer}[{name=[funkcija koja riječi parne duljine preslikava u prvu polovicu]}]\label{pr:pola}
Nad $\Sigma:=\{\t a,\t b\}$ promotrimo funkciju $\varphi_h:\Sigma^*\rightharpoonup\Sigma^*$, čija je domena skup svih riječi parne duljine, a $\varphi_h(\alpha_1\alpha_2\dotsm\alpha_{2k}):=\alpha_1\alpha_2\dotsm\alpha_k$ (prva polovica riječi). Konkretno, za $w_0:=\t{aababa}$ vrijedi $\dulj{w_0}=6$, pa je $w_0\in\dom{\varphi_h}$ te je $\varphi_h(w_0)=\t{aab}$.

Funkcija $\varphi_h$ je Turing-izračunljiva: računa je Turingov stroj
\begin{equation}
    \mathcal T_h:=(\{\textsc a,\textsc b,\textsc c,\textsc d,\textsc e,\textsc f\},\Sigma,\{\bl,\t a,\t b,\t c,\t d\},\bl,\delta_h,\textsc a,\textsc e)\text,
\end{equation}
čija je funkcija prijelaza (kao konačna funkcija) zadana tablicom
\begin{equation}\label{eq:deltapola}
\begin{array}{c|ccccc}
    \delta_h &\bl               &\t a               &\t b               &\t c               &\t d               \\\hline
    \textsc a&(\textsc e,\bl,+1)&(\textsc b,\t c,+1)&(\textsc b,\t d,+1)&(\textsc f,\t c,-1)&(\textsc f,\t d,-1)\\
    \textsc b&(\textsc c,\bl,-1)&(\textsc b,\t a,+1)&(\textsc b,\t b,+1)&(\textsc f,\t c,-1)&(\textsc f,\t d,-1)\\
    \textsc c&(\textsc f,\bl,-1)&(\textsc d,\bl ,-1)&(\textsc d,\bl ,-1)&(\textsc f,\t c,-1)&(\textsc f,\t d,-1)\\
    \textsc d&(\textsc f,\bl,-1)&(\textsc d,\t a,-1)&(\textsc d,\t b,-1)&(\textsc a,\t a,+1)&(\textsc a,\t b,+1)\\
    \textsc f&(\textsc f,\bl,-1)&(\textsc f,\t a,-1)&(\textsc f,\t b,-1)&(\textsc f,\t c,-1)&(\textsc f,\t d,-1)
\end{array}\text.
\end{equation}
Kao što vidimo, takav način zadavanja Turingovog stroja nije naročito čitljiv.  Zato se obično crtaju dijagrami. Recimo, $\mathcal T_h$ bismo mogli prikazati dijagramom
\begin{equation}
\begin{tikzpicture}[baseline=(D)]\label{dia:Th}
\node[state,initial] (A) {$\textsc a$};
\node[state,right of=A] (B) {$\textsc b$};
\node[state,below of=B] (C) {$\textsc c$};
\node[state,left of=C] (D) {$\textsc d$};
\node[state,accepting,above left of=D] (E) {$\textsc e$};
\draw
(A) edge[below] node{\bl} (E)
(A) edge node[above]{\t a:\t c} node[below]{\t b:\t d} (B)
(B) edge[loop right] node{$\Sigma$} (B)
(B) edge[dashed,right] node{\bl} (C)
(C) edge[dashed] node[above]{\t a:\bl} node[below]{\t b:\bl} (D)
(D) edge[right] node[align=center]{\t c:\t a\\\t d:\t b} (A)
(D) edge[dashed,loop left] node{$\Sigma$} (D)
;
\end{tikzpicture}\text.
\end{equation}
Recimo ponešto o konvencijama pri crtanju takvih dijagrama. Crtamo konačni usmjereni graf, čiji su vrhovi stanja, a bridovi su prijelazi. Opće pravilo je da se prijelaz $\delta(p,\alpha)=(q,\beta,d)$ prikazuje kao strelica od vrha $p$ prema vrhu $q$, na kojoj piše $\alpha\mathord:\beta$. Strelicu crtamo kao punu ako je $d=1$ (pomak udesno), a kao iscrtkanu ako je $d=-1$ (pomak ulijevo).

Više prijelaza s istim $p$, $q$ i $d$ prikazujemo strelicom s više oznaka $\alpha_1\mathord:\beta_1,\dotsc,\alpha_k\mathord:\beta_k$. Ako se znak ne mijenja ($\alpha=\beta$), umjesto $\alpha\!:\!\alpha$ pišemo samo $\alpha$. Ako već imamo oznaku za skup $S\subseteq\Gamma$, možemo je napisati na brid umjesto pojedinih elemenata (kao što smo na dijagramu učinili sa $\Sigma$).

    Početno stanje označavamo "strelicom niotkud" --- kako je na dijagramu označeno stanje $\textsc a$. Završno stanje označavamo dvostrukim krugom --- kako je na dijagramu označeno stanje $\textsc e$. I još jedno pravilo koje bitno povećava preglednost dijagrama: ne pišemo stanja ni prijelaze koji više ne mogu voditi do završnog stanja. Običaj je prilikom konstrukcije Turingovog stroja imati jedno stanje $q_x$ (u slučaju $\mathcal T_h$ to je stanje $\textsc f$), takvo da $\delta$ svaku "nemoguću situaciju" $(q,\alpha)$ preslika u $(q_x,\alpha,-1)$. Specijalno to vrijedi i za $q=q_x$ (za sve $\alpha\in\Gamma$) pa Turingov stroj, ako se ikad nađe u nekoj od tih nedozvoljenih konfiguracija, više nikada neće stati.

    Efektivno, to je "stanje greške" i ne moramo ga (kao ni prijelaze prema njemu) crtati na dijagramu: podrazumijevamo da riječi za koje se dogodi neka od tih "nemogućih situacija" nisu u domeni funkcije koju Turingov stroj računa. Važno je da ne mora vrijediti obrat: Turingov stroj ne mora nikada ući u stanje $q_x$, a da ipak nikada ne stane. Možemo to usporediti s instrukcijom RAM-stroja $i.\;\goto\;i$ --- ako $\textsc{pc}$ ikad postane $i$, RAM-stroj sigurno ne stane, ali može ne stati i na druge načine.

    Konfiguracije Turingovog stroja u konkretnom izračunavanju obično se označavaju skraćeno: ispod trenutno čitanog znaka napišemo trenutno stanje, odnosno umjesto $\bigl(q,n,(t_m)_{m\in\N}\bigr)$ pišemo $t_0t_1\ldots\smash{\underset{q}{t_n}}t_{n+1}\ldots$ --- u takvom zapisu ne pišemo (ali podrazumijevamo) $\bl\ldots$ na kraju.

	Koristeći tu notaciju, početna konfiguracija $\mathcal T_h$ s ulazom $\t{abaa}$ je $\underset{\textsc a}{\t a}\t{baa}$. Tada je $\mathcal T_h$-izračunavanje s $\t{abba}$
\begin{multline}
\underset{\textsc a}{\t a}\t{baa}\leadsto
\t c\underset{\textsc b}{\t a}\t{aa}\leadsto
\t{cb}\underset{\textsc b}{\t a}\t a\leadsto
\t{cba}\underset{\textsc b}{\t a}\leadsto
\t{cbaa}\underset{\textsc b }{\bl}\leadsto
\t{cba}\underset{\textsc c}{\bl}\leadsto
\t{cb}\underset{\textsc d}{\t a}\leadsto
\t{c}\underset{\textsc d }{\t b}\t{a}\leadsto
\underset{\textsc d}{\t c }\t{ba}\leadsto\\
\leadsto\t{a}\underset{\textsc a}{\t b}\t{a}\leadsto
\t{ad}\underset{\textsc b}{\t a}\leadsto
\t{ada}\underset{\textsc b}{\bl}\leadsto
\t{ad}\underset{\textsc c}{\t a}\leadsto
\t{a}\underset{\textsc d}{\t d}\leadsto
\t{ab}\underset{\textsc a}{\bl}\leadsto
\t{ab}\bl\underset{\textsc e}{\bl}\lcirclearrowleft\!\text,
\end{multline}
iz čega se vidi izlazni podatak $\t{ab}=\varphi_h(\t{abba})$. S druge strane, $\mathcal T_h$-izračunavanje s $\t{aba}$
\begin{multline}
\underset{\textsc a}{\t a}\t{ba}\leadsto
\t{c}\underset{\textsc b}{\t b}\t{a}\leadsto
\t{cb}\underset{\textsc b }{\t a}\leadsto
\t{cba}\underset{\textsc b}{\bl}\leadsto
\t{cb}\underset{\textsc c}{\t a}\leadsto
\t{c}\underset{\textsc d}{\t b}\leadsto
\underset{\textsc d}{\t c}\t{b}\leadsto
\t{a}\underset{\textsc a}{\t b}\leadsto
\t{ad}\underset{\textsc b}{\bl}\leadsto
\t{a}\underset{\textsc c}{\t d}\leadsto
\t{}\underset{\textsc f}{\t a}\t d\leadsto
\t{}\underset{\textsc f}{\t a}\t d\leadsto\dotsb%\;\text,
\end{multline}
očito ne stane, dakle $\t{aba}\notin\dom{\varphi_h}$.
\end{primjer}

\section{Prateće funkcije jezičnih funkcija}\label{sec:tikp}

Kao što je već najavljeno, cilj je pokazati ekvivalentnost Turing-modela s RAM-modelom i funkcijskim modelom izračunljivosti. Za početak ćemo koristeći pristup sličan onome iz poglavlja~\ref{ch:univ} dokazati da su Turing-izračunljive jezične funkcije "parcijalno rekurzivne".

Prvo moramo precizirati što to uopće znači. Neka je $\Sigma$ abeceda i $\varphi$ jezična funkcija nad njom. Očito $\varphi$ ne možemo dobiti kompozicijom, primitivnom rekurzijom i minimizacijom iz inicijalnih (brojevnih) funkcija. I da smislimo neke "inicijalne jezične funkcije", kompozicija je jasna (čak jednostavnija nego u brojevnom slučaju, jer su sve funkcije jednomjesne), ali kako definirati primitivnu rekurziju kad riječ nema jedinstvenog sljedbenika? Kako definirati minimizaciju kad kanonski leksikografski uređaj na $\Sigma^*$ nije dobar uređaj? Kako uopće definirati izračunljive jezike ("relacije") kad karakteristična funkcija jezika nije ni brojevna ni jezična funkcija?

Sve su to problemi o kojima smo već govorili, ponajviše u uvodu. No sada znamo dovoljno o kodiranju da nas to ne treba previše obeshrabriti. Kodirat ćemo $\Sigma^*$ nekom funkcijom $\sigma:=\N\Sigma^*$ (po uzoru na kodiranje $\N^*$ označavamo $\kr w:=\sigma(w)$) te ćemo pomoću tog kodiranja definirati prateću funkciju $\N\varphi=\sigma\circ\varphi\circ\sigma^{\,-1}$, za koju je dobro definirano što znači da je parcijalno rekurzivna.
Štoviše, pazit ćemo da $\sigma$ bude \emph{bijekcija} između $\Sigma^*$ i $\N$, tako da će prateće funkcije totalnih funkcija biti opet totalne funkcije. Onda će i \emph{rekurzivne} jezične funkcije biti dobro definirane, kao totalne izračunljive jezične funkcije, odnosno one čije su prateće funkcije rekurzivne.

\subsection{Kodiranje znakova i riječi}\label{sec:kSigma}

Prvo kodirajmo abecedu $\Sigma$. Kako je to konačan skup iskoristit ćemo \texttt{enum}-tehniku, kao što smo napravili kod kodiranja tipova instrukcija RAM-stroja~\eqref{eq:enumIns}. Ovdje počinjemo brojiti od $1$ (želimo $0\nin\im{\N\Sigma}$) iz tehničkog razloga koji će uskoro biti jasan.

Mali filozofski problem je u tome što su elementi od $\Sigma$ "apstraktni znakovi" bez ikakve imanentne semantike i međusobnog odnosa, tako da ne možemo fiksirati jedan poredak kao što smo to učinili za tipove RAM-instrukcija. Ipak, za svaku konkretnu abecedu moći ćemo fiksirati kodiranje (\emph{encoding}), a za apstraktne abecede moći ćemo univerzalno kvantificirati tvrdnje u obliku "Za svako kodiranje od $\Sigma$~\ldots", podrazumijevajući ("Zermelov teorem za konačne skupove") da kodiranje \emph{postoji}. U praksi, ti znakovi će obično biti dio standarda Unikod, pa ćemo ih moći poredati, ako baš nemamo pametniji kriterij, po rednim brojevima u Unikodu.

\begin{definicija}[{name=[kodiranje abecede]}]
Neka je $\Sigma$ abeceda. Označimo $b':=\card\Sigma$. %(taj broj zovemo \emph{baza} abecede $\Sigma$, iz razloga koji će uskoro postati jasan).
\emph{Kodiranje} od $\Sigma$ je bilo koja bijekcija $\N\Sigma:\Sigma\leftrightarrow[1\dd b']$.
\end{definicija}

% \subsection{Pomaknute baze i kodiranje riječi}

Sada bismo mogli kao u definiciji~\ref{def:kodProg} kodirati riječi kao kodove konačnih nizova kodova njihovih znakova --- ali to ne bi bilo bijektivno. Osim toga, kodovi RAM-instrukcija mogli su biti proizvoljno veliki, dok za fiksnu abecedu kodovi znakova mogu biti najviše $b'$ --- što sugerira da nam je dovoljno jednostavnije kodiranje.

Najprirodnije kodiranje ograničenih nizova, toliko prirodno da možda nismo ni svjesni kako ga koristimo (za $b'=10$) svaki put kad čitamo i pišemo višeznamenkaste brojeve, je \textbf{zapis u bazi} $b'$. Problem su početne nule ($(001121)_3=(1121)_3$), što smo riješili početkom kodiranja znakova od $1$, ali smo zato dobili uključenu gornju granicu. Možemo li imati znamenku $b'$ u bazi $b'$? Drugi problem su jednočlane abecede: obični pozicijski brojevni sustavi definiraju se samo za baze od $2$ nadalje. Postoji i unarni zapis, ali on nije pozicijski --- ili jest?

Zapravo, unarni zapis sasvim odgovara uobičajenom brojevnom zapisu u bazi $b'$ za $b'=1$: recimo, $(1111)_1=1\cdot1^3+1\cdot1^2+1\cdot1^1+1\cdot1^0=4$, i jedini problem je uključena gornja granica --- u bazi $1$ imamo znamenku $1$. Ali nemamo znamenku $0$, i ne smijemo je imati ako želimo jedinstven zapis --- no to upravo rješava prvi problem.

\begin{definicija}[{name=[zapis broja u pomaknutoj bazi]}]
Neka je $b\in\N_+$, $n\in\N$ te $z_0,z_1,\dotsc,z_{n-1}\in[1\dd b]$. Zapis u \emph{pomaknutoj bazi} $b$ je zapis
\begin{equation}\label{eq:pomakbaza}
    x=\sum_{i<n}z_i\cdot b^i=:(z_{n-1}\dotsm z_0)_b\text,
\end{equation}
dakle isti kao obični zapis u bazi $b$, samo sa znamenkama iz $[1\dd b]$ umjesto iz $[0\dd b\rangle$.
\end{definicija}

\begin{lema}[{name=[egzistencija i jedinstvenost zapisa u pomaknutoj bazi]}]\label{lm:pomakbaza}
Za svaki $b\in\N_+$, svaki $x\in\N$ ima jedinstven zapis u pomaknutoj bazi $b$.
\end{lema}
\begin{proof}
Dokaz egzistencije je isti kao i dokaz da svaki broj ima zapis u običnoj bazi $b$: znamenke konstruiramo tako da uzastopno dijelimo $x$ s $b$ dok ne dobijemo nulu pa ostatke tih dijeljenja zapišemo obrnutim redom. Jedina razlika je što ovdje ostatci moraju biti između $1$ i $b$: ako $x$ nije djeljiv s $b$ ništa se ne mijenja, a ako jest smanjimo količnik za $1$. Recimo, $18$ podijeljeno s $3$ je $5$ i pomaknuti ostatak $3$. Pomaknuti ostatak je primitivno rekurzivna operacija, po primitivno rekurzivnoj verziji teorema o grananju.
\begin{equation}
    \f{mod}'(x,b):=\begin{cases}
        b,&b\mid x\\
        x\bmod b,&\text{inače}
    \end{cases}
\end{equation}

Za dokaz jedinstvenosti treba vidjeti da brojevi različitih duljina zapisa u istoj bazi, kao i brojevi iste duljine zapisa koji se u nekoj znamenci razlikuju, moraju biti različiti. Obje te nejednakosti posljedica su od
\begin{equation}
\label{eq:injNSz}
    (tz_{m-1}\dotsm z_0)_b\le
    (tb\dotsm b)_b<
    \bigl((t+1)1\dotsm1\bigr)_b\le
    \bigl((t+1)z_{m-1}'\dotsm z_0'\bigr)_b\text,
\end{equation}
--- za prvu je $t=0$, a za drugu je $t$ prva znamenka slijeva u kojoj se zapisi razlikuju. Sama stroga nejednakost u~\eqref{eq:injNSz} dobije se iz
\begin{multline}
    t\cdot b^m+b\cdot b^{m-1}+b\cdot b^{m-2}+\dotsb+b=
    t\cdot b^m+b^m+b^{m-1}+\dotsb+b<{}\\
    {}<(t+1)b^m+1\cdot b^{m-1}+\dotsb+1\cdot b+1\text.
\end{multline}
To po tranzitivnosti zapravo znači da smo zapise poredali po duljini, a zapise iste duljine leksikografski (tzv.\ \emph{shortlex ordering}).
\end{proof}

\begin{primjer}[{name=[\emph{shortlex ordering} svih riječi nad dvočlanom abecedom]}]
\emph{Shortlex ordering} $\{\t a,\t b\}^*$ je: $\underset{0}{\varepsilon}$,
$\underset{1}{\t a}$,
$\underset{2}{\t b}$,
$\underset{3}{\t{aa}}$,
$\underset{4}{\t{ab}}$,
$\underset{5}{\t{ba}}$,
$\underset{6}{\t{bb}}$,
$\underset{7}{\t{aaa}}$,
$\underset{8}{\t{aab}}$,
$\underset{9}{\t{aba}}$,~\ldots.
\end{primjer}

\begin{definicija}[{name=[kodiranje riječi]}]
Neka je $\Sigma$ abeceda te $\N\Sigma$ njeno kodiranje.

	Definiramo kodiranje skupa $\Sigma^*$ svih riječi nad $\Sigma$, s
\begin{equation}\label{eq:kodNSz}
	\N\Sigma^*(w):=\kr{\alpha_{n-1}\dotsm\alpha_0}:=\bigl(\N\Sigma(\alpha_{n-1}\mspace{-1mu})\dotsm\mspace{1mu}\N\Sigma(\alpha_0)\bigr)_{b'}=\sum_{i<n}\N\Sigma(\alpha_i)\cdot(b')^i\text,
\end{equation}
za svaku riječ $w=\alpha_{n-1}\dotsm\alpha_0\in\Sigma^*$ (uz oznake $n:=\dulj{w}$ i $b':=\card\Sigma$).
\end{definicija}

\begin{primjer}[{name=[prateća funkcija jezične funkcije]}]
U primjeru~\ref{pr:pola} uveli smo abecedu $\Sigma=\{\t a,\t b\}$ i riječ $w_0=\t{aababa}$ nad njom. Za tu abecedu je $b'=2$; fiksirajmo kodiranje $\N\Sigma(\t a):=1$, $\N\Sigma(\t b):=2$. Tada je $\kr{w_0}=(112121)_2=73$.
Također je $w_0\in\dom{\varphi_h}$, pa je $73\in\dom{\N\varphi_h}$; a iz $\varphi_h(w_0)=\t{aab}$ vidimo $\N\varphi_h(73)=\kr{\t{aab}}=(112)_2=8$.

Za broj $153$ pretvaranje u pomaknutu bazu $2$ daje
$\begin{array}{rrrrrrr}
153 & 76 & 37 & 18 & 8 & 3 & 1\\\hline
1 & 2 & 1 & 2 & 2 & 1 & 1
\end{array}$, dakle $153=(1122121)_2=\kr{\t{aabbaba}}$. Kako je $\dulj{\t{aabbaba}}=7$ neparan broj, zaključujemo $\t{aabbaba}\notin\dom{\varphi_h}$, pa $153\notin\dom{\N\varphi_h}$.
\end{primjer}

Za dekodiranje trebamo, analogno propoziciji~\ref{prop:lhpartprn}, duljinu zapisa i ekstrakciju pojedine znamenke u zadanoj pomaknutoj bazi. %Ipak, prvo to napravimo za obične zapise u bazi, jer je jednostavnije a trebat će nam. Nakon toga ćemo vidjeti što treba promijeniti za zapis u pomaknutoj bazi.

\begin{lema}[{name=[rad sa zapisima u pomaknutoj bazi]}]\label{lm:ldcpprn}
% Možda: umjesto specifikacije napraviti definiciju s točkicom (b=0->b=1)
Postoje primitivno rekurzivne funkcije $\f{slh}^2$\!, $\f{sdigit}^3$\!, $\f{sconcat}^3$ i $\f{sprefix}^3$,\newline takve da za sve $b\in\N_+$, za sve $i,n,m\in\N$ vrijedi:
\begin{labeling}{$\f{sconcat}(m,n,b)$}
    \item[$\f{slh}(n,b)$] je duljina zapisa broja $n$ u pomaknutoj bazi $b$. \item[$\f{sdigit}(n,i,b)$] za $i<\f{slh}(n,b)$, je vrijednost $i$-te znamenke zapisa broja $n$ u pomaknutoj bazi $b$, gdje brojimo od nule slijeva.
    \item[$\f{sconcat}(m,n,b)$] je broj čiji zapis u pomaknutoj bazi $b$ se dobije konkatenacijom istih takvih zapisa za $m$ i $n$ redom. Pišemo ga i kao $m\conc bn$.
    \item[$\f{sprefix}(n,i,b)$] za $i\le\f{slh}(n,b)$, je broj čiji je zapis u pomaknutoj bazi $b$ prefiks (početak) duljine $i$ istog takvog zapisa broja $n$.
\end{labeling}
\end{lema}
%Za fiksni $b$, $\f{sconcat}(m,n,b)$ još pišemo $m\conc b n$, a $\f{Prefix}(m,n,b)$ pišemo $m\preceq_b n$.
\noindent\begin{align}
\label{eq:deflh}
    \f{slh}(n,b)&:=(\mu t\le n)\big(\!
    \textstyle\sum_{i\le t}\!b^i>n
    \big)\\
\label{eq:defconcat}
    \f{sconcat}(m,n,b)&:=m\conc bn:=m\cdot b^{\,\f{slh}(n,b)}+n\\
\label{eq:defprefix}
    \f{sprefix}(n,i,b)&:=(\mu m\le n)(\exists t\le n)\bigl(n=m\conc bt\land\f{slh}(m,b)=i\bigr)\\
\label{eq:defdigit}
    \f{sdigit}(n,i,b)&:=\f{mod}'\bigl(\f{sprefix}\bigl(n,\f{Sc}(i),b\bigr),b\bigr)
\end{align}
\begin{proof}
Za $\f{slh}$, treba uočiti da zapisa duljine $i$ ima točno $b^i$. Dakle, samo trebamo zbrajati takve brojeve dok ne prijeđemo $n$. Očito je $\f{slh}(n,b)\le n$.~\eqref{eq:deflh}

Sada jednostavno možemo dobiti konkatenaciju: $m$ pomaknemo udesno toliko da $n$ taman stane u dobiveni prostor, nakon čega ga pribrojimo.~\eqref{eq:defconcat}

Pomoću konkatenacije možemo dobiti $\f{sprefix}$: prefiks je ono što se može konkatenirati s nečim tako da se dobije početni zapis $n$. Očito su dijelovi manji ili jednaki $n$.~\eqref{eq:defprefix} %I ovdje: podriječ ima manji kod.

Sada možemo napisati i $\f{sdigit}$: to je zadnja znamenka prefiksa čija je duljina za jedan veća od pozicije znamenke.~\eqref{eq:defdigit}
%Sve te funkcije su primitivno rekurzivne, koristeći dosadašnje rezultate.
\end{proof}

\begin{propozicija}[{name=[bijektivnost kodiranja riječi]}]\label{pp:bijkr}
Neka je $\Sigma$ abeceda i $\N\Sigma$ njeno kodiranje.

    Tada je $\N\Sigma^*$ bijekcija između $\Sigma^*$ i $\N$.
\end{propozicija}
\begin{proof}
Označimo $b':=\card\Sigma$. Ako je $w\ne w'$ za $w,w'\in\Sigma^*$, tada su $w$ i $w'$ ili različitih duljina (pa je $\f{slh}(\kr w,b')\ne\f{slh}(\kr{w'},b')$), ili su jednake duljine ali se na nekom mjestu razlikuju (pa je za neki $i$, $\f{sdigit}(\kr w,i,b')\ne\f{sdigit}(\kr{w'},i,b')$).

U svakom slučaju mora biti $\kr w\ne\kr{w'}$, dakle $\N\Sigma^*$ je injekcija.

Za surjektivnost, neka je $x\in\N$ proizvoljan. Prema lemi~\ref{lm:pomakbaza}, postoje znamenke $\vec z\in[1\dd b']^*$ takve da je $x=(\vec z)_{b'}$. Ako sad za svaku $z_i$ označimo $\alpha_i:=\N\Sigma^{-1}(z_i)\in\Sigma$, riječ sastavljena od tih znakova ima upravo kod $x$.
\end{proof}

%Sada možemo formalno dokazati da Turing-izračunljivost jezične funkcije povlači parcijalnu rekurzivnost njene prateće funkcije.

\begin{definicija}[{name=[prateća funkcija jezične funkcije]}]\label{def:kodfi}
Neka je $\Sigma$ abeceda, $\N\Sigma$ njeno kodiranje te $\varphi:\Sigma^*\rightharpoonup\Sigma^*$ jezična funkcija nad njom. \emph{Prateća funkcija} od $\varphi$ je jednomjesna funkcija $\N\varphi$ s domenom $\N\Sigma^*[\dom\varphi]$, zadana s
\begin{equation}\label{eq:kodfidef}
    \N\varphi(\kr{w}):\simeq\kr{\varphi(w)}\text.
\end{equation}
Kako je svaki prirodni broj kod jedinstvene riječi iz $\Sigma^*$, definicija je dobra.
\end{definicija}

\subsection{Kodiranje stanja, trake i funkcije prijelaza}

Kroz čitavu ovu točku imat ćemo fiksiranu abecedu $\Sigma_0$, njeno kodiranje $\N\Sigma_0$ uz oznaku $b':=\card\Sigma_0$, Turing-izračunljivu jezičnu funkciju $\varphi_0$ nad $\Sigma_0$ te fiksni Turingov stroj $\mathcal T_0=(Q_0,\Sigma_0,\Gamma_0,\bl,\delta_0,q_0,q_z)$ koji je računa. Cilj će nam biti, kodirajući komponente i izračunavanje stroja $\mathcal T_0$, dokazati da je $\N\varphi_0$ parcijalno rekurzivna funkcija. To je slično pristupu u poglavlju~\ref{ch:univ}, samo je jednostavnije jer ne moramo pisati interpreter za proizvoljni RAM-program zadan kodom, već samo "ručno" prevesti jedan konkretni Turingov stroj $\mathcal T_0$ u funkcijski jezik.

%\subsection{Kodiranje stanja, radne abecede i trake}

Prvo kodirajmo skup $Q_0$. Kako se radi o konačnom skupu, možemo jednostavno staviti $a:=\card Q_0$ i fiksirati bijekciju $\N Q_0:Q_0\leftrightarrow[0\dd a\rangle$. Za konstruktor trebamo samo fiksirati kod početnog stanja --- prirodnim se čini definirati $\N Q_0(q_0):=0$ --- a što se komponenata tiče, sve što trebamo je usporedba s $q_z$, što ćemo dobiti ako i njegov kod bude fiksan broj --- recimo, $\N Q_0(q_z):=1$.

Hm, hoće li onda $\N Q_0$ biti dobro definirana --- što ako je $q_0=q_z$? U definiciji Turingovog stroja ništa ne sprečava da se to dogodi. Ipak, $q_0\ne q_z$ smijemo pretpostaviti bez smanjenja općenitosti.

\begin{lema}[{name=[možemo pretpostaviti $q_0\ne q_z$]}]\label{lm:bsomp-q0neqz}
	Za svaki Turingov stroj $\mathcal T$ koji računa neku jezičnu funkciju $\varphi$, postoji \emph{ekvivalentni} (računa istu funkciju $\varphi$) Turingov stroj $\mathcal T'$, kojem se početno i završno stanje razlikuju.
\end{lema}
\begin{proof}
Ako već u $\mathcal T$ vrijedi $q_0\ne q_z$, stavimo $\mathcal T':=\mathcal T$. Inače vrijedi $q_0=q_z$ i tvrdimo da je tada $\varphi$ identiteta na $\Sigma^*$. Doista, za svaku riječ $w\in\Sigma^*$, početna konfiguracija stroja $\mathcal T$ s ulazom $w$, $(q_0,0,w\bl\ldots)=(q_z,0,w\bl\ldots)$, ujedno je i završna konfiguracija, pa izračunavanje uvijek stane ($\varphi$ je totalna) i iz završnog oblika trake čitamo $\varphi(w)=w$.

Dakle, sad samo trebamo konstruirati Turingov stroj s različitim početnim i za\-vrš\-nim stanjem, koji računa identitetu. Jedan takav je
\begin{align}
    \mathcal T'&:=(\{0,1\},\Sigma,\Sigma\cup\{\bl\},\bl,\delta',0,1)\text,\\
%\shortintertext{gdje je}
    \delta'(0,\alpha)&:=(1,\alpha,1)\text{ za sve $\alpha\in\Sigma\cup\{\bl\}$.}
\end{align}
Tada za svaku $w\in\Sigma^*$, $\mathcal T'$-izračunavanje s $w$ je $(0,0,w\bl\ldots)\leadsto(1,1,w\bl\ldots)\lcirclearrowleft$, odakle je izlazni podatak opet $w$, odnosno $\mathcal T'$ računa identitetu.
\end{proof}

Slično možemo kodirati i $\Gamma_0$ --- ali kako već imamo kodiranje skupa $\Sigma_0\subset\Gamma_0$, želimo da znakovi ulazne abecede imaju iste kodove. Stoga označimo $b:=\card\Gamma_0$ i proširimo bijekciju $\N\Sigma_0:\Sigma_0\leftrightarrow[1\dd b']$ na bijekciju $\N\Gamma_0:\Gamma_0\leftrightarrow[0\dd b\rangle$, tako da $\bl$ preslikamo u $0$, a ostale elemente iz $\Gamma_0\!\setminus\Sigma_0$ bijektivno u skup $\langle b'\dd b\rangle$.

Ovaj put smo upotrijebili nulu, kao $\N\Gamma_0(\bl)$. To opravdava frazu "traka je s konačnim nosačem" (praznine su kodirane nulama, pa ne-nula ima konačno mnogo), a bit će esencijalno i za kodiranje trake. Naime, sada imamo sličan problem kao sa stanjem registara RAM-stroja: da bismo kodirali proizvoljnu traku, moramo nekako skupiti kodove beskonačno mnogo ćelija, ali takve da su svi osim konačno mnogo njih jednaki $0$. Alternativno, želimo "kodiranje" konačnih nizova, ali tako da dodavanje nule na kraj ne promijeni kod.

U slučaju RAM-registara to smo riješili rastavom na prim-faktore, odnosno malom modifikacijom kodiranja $\N^*$ --- umjesto od $1$, eksponenti su išli od $0$. Možemo li ovdje naći neku malu modifikaciju kodiranja $\Sigma^*$ (zapis u bazi) da dobijemo analogni rezultat?

Možemo, i slična ideja funkcionira: kontraprimjer za injektivnost preslikavanja koje broji znakove od $0$ bit će upravo putokaz kako treba napisati kodiranje trake. Umjesto od $1$, znamenke će ići od $0$. Tamo smo rekli da je $(001121)_3=(1121)_3$, no to upravo znači da je taj broj prirodno gledati kao kod trake $\t{abaa\bl\bl\bl\ldots}=\t{abaa\bl\ldots}$. Drugim riječima, ovdje imamo \textbf{obični zapis u (ne pomaknutoj$\mspace{1mu}$) bazi $b$}.

\begin{definicija}[{name=[kodiranje trake Turingovog stroja]}]
Za proizvoljnu traku $t:\N\to\Gamma_0$ (koja je skoro svuda $\bl$), definiramo \emph{kod trake} \begin{equation}
	\knk{t}=\knk{t_0t_1t_2\ldots\bl\bl\ldots}:=\sum_{i\in\N}\N\Gamma_0(t_i)\cdot b^i\text{, gdje je $b:=\card\Gamma_0$.}
\end{equation}
%(Usporedite s~\eqref{eq:defkreg}.)
	Zbog konačnog nosača i činjenice da je $\N\Gamma_0(\mspace{0.5mu}\bl)=0$, samo konačno mnogo članova tog "reda potencija" bit će pozitivno, pa je suma dobro definirana.
\end{definicija}

S ovom idejom postoje dvije smetnje pri konstrukciji početne konfiguracije. Prvo, \textbf{moramo promijeniti bazu} iz $\card\Sigma_0=b'$ u $\card\Gamma_0=b$, jer na traci se mogu naći i znakovi s većim kodovima, kao i praznina (\textbf{uvijek je $b>b'$}, zbog $\Gamma\supset\Sigma$). Drugo, u zapadnom svijetu pišemo riječi slijeva nadesno (i tako doživljavamo traku), ali smo zapis brojeva preuzeli iz arapskog jezika koji se piše u suprotnom smjeru. Zato dodavanje nule \emph{slijeva} u zapis broja u bazi $b$ odgovara dodavanju prazne ćelije \emph{zdesna} pri pomaku udesno od zadnje posjećene ćelije.

Računarci s iskustvom mrežnog programiranja znaju za taj problem. Radi se o \emph{byteorder}-dilemi, sasvim analognoj upravo opisanom problemu, iz sličnih razloga. Većina modernih procesora pamti višebajtne podatke tako da na početnoj adresi podatka stoji najmanje značajni bajt, jer se jedino tako postiže neovisnost o veličini ćelije %se na taj način semantika \emph{casta} pokazivača podudara sa semantikom \emph{casta} pokazanih vrijednosti. Nakon \texttt{int x=0x61626364;}, izraz \texttt{(char)x} ima vrijednost \texttt{0x64}, pa bi se moglo očekivati da i \texttt{*(char*)\&x} ima istu vrijednost. (Gledano s druge strane, 
(nakon \texttt{y=25;} na adresi \texttt{\&y} je bajt $25$, neovisno o tome je li varijabla \texttt y tipa \texttt{char}, \texttt{int} ili \texttt{long~long}), što pojednostavljuje računanje. %kojeg je \texttt y tipa.) %No to je jedino moguće ako se \texttt x stavlja na adrese $[p\dd p+\texttt{sizeof x}\rangle$ tako da se na adresu $p$ stavi bajt \texttt{0x64}, na adresu $p+1$ bajt \texttt{0x63} (za \emph{cast} u \texttt{short}),~\ldots, odnosno "obrnutim" redom (\emph{little-endian}).

Naravno, taj redoslijed je obrnut u odnosu na to kako smo navikli pisati brojeve --- i kako smo uostalom napisali taj broj u naredbi koja inicijalizira \texttt y. To posebno dolazi do izražaja u mrežnom programiranju, jer mrežni standardi propisuju da se višebajtni brojevi preko mreže prenose u redoslijedu \emph{big-endian}, kako bi bilo lakše pratiti što se događa u slučaju greške. Zato mnoge standardne biblioteke imaju funkcije za pretvaranje (\texttt{ntoh}/\texttt{hton}) redoslijeda bajtova iz mrežnog (koji je lakše čitati) u procesorski (s kojim je lakše računati) i obrnuto. Mi ćemo napraviti nešto slično i usput prenijeti zapis iz jedne baze u drugu.

\begin{lema}[{name=[primitivna rekurzivnost \emph{input\slash output} sustava trake]}]\label{lm:recodeprn}
Postoji primitivno rekurzivna funkcija $\f{Recode}^3$ takva da za svaku riječ $w\in\Sigma_0^*$ vrijede jednakosti
\begin{align}
\SwapAboveDisplaySkip
    \label{eq:recodein}\f{Recode}\bigl(\kr w,b',b\bigr)&=\knk{w\bl\ldots}\text,\\
    \label{eq:recodeout}\f{Recode}\bigl(\knk{w\bl\ldots},b,b'\bigr)&=\kr w\text.
\end{align}
\end{lema}
\begin{proof}
Kao što smo već rekli, $\f{Recode}(n,b_1,b_2)$ samo treba ekstrahirati znamenke od $n$ u bazi $b_1$ te ih slagati obrnutim redom u bazi $b_2$. Kako je $w\in\Sigma_0^*$, u zapisu nema znamenke $0$, pa
možemo koristiti funkcije $\f{slh}$ i $\f{sdigit}$ bez obzira na to je li baza $b_1$ pomaknuta.
\begin{equation}
	\f{Recode}(n,b_1,b_2):=\quad\sum_{\mathclap{i\,<\,\f{slh}(n,b_1\!)}}\f{sdigit}(n,i,b_1\mspace{-1mu})\cdot{b_2}^i
\end{equation}
Neka je $w=\alpha_0\alpha_1\dotsm\alpha_{l-1}\in\Sigma^*$ proizvoljna ($l:=\dulj{w}$). Tada, ako označimo $n:=\kr w$, vrijedi $\f{slh}(n,b')=l$ i $d_i:=\f{sdigit}(n,i,b')=\N\Sigma_0(\alpha_i)=\N\Gamma_0(\alpha_i)$ za sve $i<l$. Iz toga
\begin{equation}
    \f{Recode}(n,b',b)=\sum_{i<l}d_i\cdot b^i=\sum_{i=0}^{l-1}\N\Gamma_0(\alpha_i)\cdot b^i=\knk{w\bl\ldots}\text,
\end{equation}
i analogno~\eqref{eq:recodeout}. Ključno je bilo da se $\N\Sigma_0$ i $\N\Gamma_0$ podudaraju na svim $\alpha\in\Sigma_0$.

    Primitivna rekurzivnost slijedi iz lema~\ref{lm:sumprodrek} i~\ref{lm:ldcpprn} te primjera~\ref{pr:addmulpow}.
\end{proof}

\begin{primjer}[{name=[korištenje \emph{input\slash output} sustava trake]}]
Neka je $\Sigma:=\{\t a,\t b,\t c\}\subset\Gamma:=\{\bl,\t a,\t b,\t c,\t A,\t B,\t C\}$, neka je kodiranje zadano redom kojim su znakovi napisani, a $w:=\t{bbc}$. Tada je $\kr w=(223)_3=27$, a $\knk{w\bl\ldots}=(322)_7=163$, dakle $\f{Recode}(27,3,7)=163$ i $\f{Recode}(163,7,3)=27$.
\end{primjer}

%\subsection{Kodiranje funkcije prijelaza}

Na redu je kodiranje funkcije prijelaza $\delta_0$. Mogli bismo se uplašiti, znajući koliko kodiranje funkcija može biti komplicirano (sjetite se indeksa). Ali $\delta_0$ je \emph{konačna} funkcija (\emph{lookup table}), pa zapravo neće biti problema.

Prvo, $\delta_0$ je funkcija s dva ulaza i tri izlaza, tako da je zapravo kodiramo kroz tri dvomjesne koordinatne funkcije (napomena~\ref{nap:brip}). Drugo, već imamo kodiranja za ulaze i prva dva izlaza, $\N Q_0:Q_0\leftrightarrow[0\dd a\rangle$ i $\N\Gamma_0:\Gamma_0\leftrightarrow[0\dd b\rangle$, samo još trebamo kodirati pomake. Kako oni već jesu "numerički" u $\mathbb Z$, najlakše ih je samo povećati za $1$ da upadnu u $\N$, tako da "pomak ulijevo" $-1$ kodiramo brojem $0$, a "pomak udesno" $1$ kodiramo brojem $2$.

\begin{lema}[{name=[primitivna rekurzivnost funkcije prijelaza]}]\label{lm:newssdprn}
Postoje primitivno rekurzivne dvomjesne funkcije $\f{newstate}$, $\f{newsymbol}$ i $\f{direction}$ koje preslikavaju $\bigl(\N Q_0(q),\N\Gamma_0(\gamma)\bigr)$ redom u $\N Q_0(q')$, $\N\Gamma_0(\gamma')$ i $1+d$, gdje je
$(q',\gamma',d):=\begin{cases}
\delta_0(q,\gamma),& q\ne q_z\\
(q,\gamma,0),&\text{inače}\end{cases}$.
\end{lema}
\begin{proof}
Za proizvoljni par $(k,g)\in[0\dd a\rangle\times[0\dd b\rangle$, ako je $k\ne 1$ označimo $(q',\gamma',d):=\delta_0\bigl(\N Q_0^{-1}(k),\N\Gamma_0^{-1}(g)\bigr)$ te definirajmo $\f{newstate}(k,g):=\N Q_0(q')$, $\f{newsymbol}(k,g):=\N\Gamma_0(\gamma')$ i $\f{direction}(k,g):=1+d$. Također, za $k=1$, za svaki $g\in[0\dd b\rangle$, dodefinirajmo $\f{newstate}(1,g):=\f{direction}(1,g):=1$ i $\f{newsymbol}(1,g):=g$.

Koliko god to komplicirana pravila bila, njima definirane funkcije su konačne (sve tri imaju domenu $[0\dd a\rangle\times[0\dd b\rangle$ s $ab$ elemenata), pa prema korolaru~\ref{kor:kon0} za svaku od njih postoji primitivno rekurzivna funkcija koja se s njom podudara na $[0\dd a\rangle\times[0\dd b\rangle$ (proširenje nulom). Sada je iz definicije tih triju funkcija jasno da vrijedi tvrdnja leme (sjetimo se, $1=\N Q_0(q_z)$, a $[0\dd a\rangle$ i $[0\dd b\rangle$ su slike od $\N Q_0$ i $\N\Gamma_0$ redom).
\end{proof}

\begin{primjer}[{name=[kodirana tablica prijelaza]}]\label{pr:polatable}
U primjeru~\ref{pr:pola} smo vidjeli funkciju prijelaza $\delta_h$, definiranu s~\eqref{eq:deltapola}. Ako stanja i znakove kodiramo kao
%\begin{equation}
$
    \begin{array}{r|cccccc}
         Q_h& \textsc a& \textsc b& \textsc c& \textsc d& \textsc e& \textsc f \\\hline
         \N Q_h& 0 & 3 & 4 & 5 & 1 & 2
    \end{array}\text{ i }
    \begin{array}{r|cc|ccc}
        \Gamma_h& \t a & \t b & \t c & \t d & \bl \\\hline
        \N\Gamma_h & 1 & 2 & 3 & 4 & 0
    \end{array}\text,
    $
%\end{equation}
tada su funkcije $\f{newstate}$, $\f{newsymbol}$ i $\f{direction}$ redom zadane tablicama:
\begin{equation*}
\begin{array}{r|ccccccc}
%\f{newstate}
    &0&1&2&3&4&5&\cdots\\\hline
0& 1 & 3 & 3 & 2 & 2 & \multicolumn{1}{|c}{0} & \cdots\\
1& 1 & 1 & 1 & 1 & 1 & \multicolumn{1}{|c}{0} & \cdots\\
2& 2 & 2 & 2 & 2 & 2 & \multicolumn{1}{|c}{0} & \cdots\\
3& 4 & 3 & 3 & 2 & 2 & \multicolumn{1}{|c}{0} & \cdots\\
4& 2 & 5 & 5 & 2 & 2 & \multicolumn{1}{|c}{0} & \cdots\\
5& 2 & 5 & 5 & 0 & 0 & \multicolumn{1}{|c}{0} & \cdots\\ \cline{2-6}
6& 0 & 0 & 0 & 0 & 0 & 0 & \cdots\\
\vdots&\vdots&\vdots&\vdots&\vdots&\vdots&\vdots&\ddots
\end{array}\text,\;\;
\begin{array}{r|ccccccc}
%\f{newsymbol}
    &0&1&2&3&4&5&\cdots\\\hline
0& 0 & 3 & 4 & 3 & 4 & \multicolumn{1}{|c}{0} & \cdots\\
1& 0 & 1 & 2 & 3 & 4 & \multicolumn{1}{|c}{0} & \cdots\\
2& 0 & 1 & 2 & 3 & 4 & \multicolumn{1}{|c}{0} & \cdots\\
3& 0 & 1 & 2 & 3 & 4 & \multicolumn{1}{|c}{0} & \cdots\\
4& 0 & 0 & 0 & 3 & 4 & \multicolumn{1}{|c}{0} & \cdots\\
5& 0 & 1 & 2 & 1 & 2 & \multicolumn{1}{|c}{0} & \cdots\\ \cline{2-6}
6& 0 & 0 & 0 & 0 & 0 & 0 & \cdots\\
\vdots&\vdots&\vdots&\vdots&\vdots&\vdots&\vdots&\ddots
\end{array}\text,\;\;
%\end{equation}
%\begin{equation}
%\label{eq:directiontable}
\begin{array}{r|ccccccc}
%\f{direction}
    &0&1&2&3&4&5&\cdots\\\hline
0& 2 & 2 & 2 & 0 & 0 & \multicolumn{1}{|c}{0} & \cdots\\
1& 1 & 1 & 1 & 1 & 1 & \multicolumn{1}{|c}{0} & \cdots\\
2& 0 & 0 & 0 & 0 & 0 & \multicolumn{1}{|c}{0} & \cdots\\
3& 0 & 2 & 2 & 0 & 0 & \multicolumn{1}{|c}{0} & \cdots\\
4& 0 & 0 & 0 & 0 & 0 & \multicolumn{1}{|c}{0} & \cdots\\
5& 0 & 0 & 0 & 2 & 2 & \multicolumn{1}{|c}{0} & \cdots\\ \cline{2-6}
6& 0 & 0 & 0 & 0 & 0 & 0 & \cdots\\
\vdots&\vdots&\vdots&\vdots&\vdots&\vdots&\vdots&\ddots
\end{array}\text.\!\!\!\qedhere
\end{equation*}
\end{primjer}

\subsection{Kodiranje Turing-izračunavanja}

Upravo dokazana lema donekle je analogna lemi~\ref{lm:NextRegCountprn}. Vrijeme je za analogon leme~\ref{lm:RegCountprn}. Za razliku od tih lema, koje su definirale \emph{uniformne} funkcije koje su primale RAM-program kao argument, ove će se odnositi na fiksni Turingov stroj $\mathcal T_0$ --- jer je tako lakše, a ne treba nam puna općenitost; sve što će nam kasnije trebati su \emph{neki} indeksi izračunljivih funkcija, a njih smo dobili s RAM-strojevima.

\begin{lema}[{name=[primitivna rekurzivnost $\mathcal T_0$-izračunavanja]}]\label{lm:StateHeadTapeprn}
%Postoje primitivno rekurzivne dvomjesne funkcije $\f{State}$, $\f{Position}$ i $\f{Tape}$, takve da za svaku riječ $w\in\Sigma_0^*$, za svaki $n\in\N$, njihove vrijednosti u $(\kr w,n)$ kodiraju stanje, poziciju i traku konfiguracije Turingovog stroja $\mathcal T_0$ nakon $n$ koraka izračunavanja s $w$.
Funkcije definirane sa
\begin{equation}
    \f{State}(\kr w,n):=\N Q_0(q_n)\text,\quad
    \f{Position}(\kr w,n):=p_n\text,\quad
    \f{Tape}(\kr w,n):=\knk{t_n}\text,
\end{equation}
gdje je $\bigl((q_n,p_n,t_n)\bigr)_{n\in\N}$
$\mathcal T_0$-izračunavanje s $w$,
primitivno su rekurzivne.
\end{lema}
\begin{proof}
Kao i kod RAM-izračunavanja, upotrijebit ćemo simultanu primitivnu rekurziju. Za početak, trebaju nam funkcije $\f G_1$, $\f G_2$ i $\f G_3$, koje daju njihove vrijednosti za $n=0$. Stanje počinje od $q_0$ koje smo kodirali nulom, pozicija počinje od nule (lijevi rub), a traka počinje od $w\bl\ldots$, čiji kod se iz koda $x=\kr w$ može dobiti primjenom funkcije $\f{Recode}$.
\begin{equation}
    \f G_1(x):=\f G_2(x):=0\text,\qquad
    \f G_3(x):=\f{Recode}(x,b',b)\text.
\end{equation}
Dakle, $\f G_1=\f G_2=\f Z$ (inicijalna funkcija), a $\f G_3$ je primitivno rekurzivna po lemi~\ref{lm:recodeprn} i propoziciji~\ref{prop:konst}.

	Za korak, trebamo funkcije $\f H_1$, $\f H_2$ i $\f H_3$, od kojih svaka prima po pet argumenata: k\^od riječi s kojom se računa $x=\kr w$, broj dosad napravljenih koraka izračunavanja $n$, k\^od prethodnog stanja $q=\N Q_0(q_n)$, prethodnu poziciju $p=p_n$ i k\^od prethodne trake $t=\knk{t_n}$. Redom trebaju vratiti $\N Q_0(q_{n+1})$, $p_{n+1}$ i $\knk{t_{n+1}\mspace{-2mu}}$. U tome će nam pomoći funkcije (redom) $\f{newstate}$, $\f{direction}$ i $\f{newsymbol}$, ali na čemu ih pozvati? Prvi argument je kod trenutnog stanja $q$, a drugi je kod trenutno čitanog znaka $g$, koji možemo dobiti kao vrijednost znamenke od $t$ mjesne vrijednosti $b^p$. Ovdje ne možemo koristiti $\f{sdigit}$ jer se na traci mogu nalaziti i praznine, a i jer se mjesne vrijednosti broje zdesna u zapisu --- ali standardni postupak za ekstrakciju znamenke~\eqref{eq:exznamb} funkcionira.

Sada $\f H_1$ jednostavno treba vratiti $\f{newstate}(q,g)$. $\f H_2$ treba vratiti $p$ pomaknut za $\f{direction}(q,g)$, ali za jedan manje jer smo pomake za $d$ kodirali s $d+1$. $\f H_3$ je najkompliciranija: treba $p$-tu znamenku u zapisu $t$ u bazi $b$ zamijeniti znamenkom vrijednosti $\f{newsymbol}(q,g)$. To radimo tako da od $t$ oduzmemo trenutnu mjesnu vrijednost te znamenke $g\cdot b^p$ pa dodamo mjesnu vrijednost nove. Dakle, funkcije
\begin{align}
    \f H_1(x,n,q,p,t)&:=\f{newstate}(q,g)\text,\\
    \f H_2(x,n,q,p,t)&:=\f{pd}\bigl(p+\f{direction}(q,g)\bigr)\text,\\
    \f H_3(x,n,q,p,t)&:=t\dotminus g\cdot b^p+\f{newsymbol}(q,g)\cdot b^p\text,\\
    \label{eq:exznamb}\text{uz pokratu }g&:=t\div b^p\bmod b\text,
\end{align}
zadovoljavaju parcijalnu specifikaciju koju smo napisali. Sada su $\f{State}$, $\f{Position}$ i $\f{Tape}$ definirane simultanom primitivnom rekurzijom iz $\f G_1$, $\f G_2$, $\f G_3$, $\f H_1$, $\f H_2$ i $\f H_3$, pa su primitivno rekurzivne po propoziciji~\ref{prop:simultrek}. Tvrdnju leme sada dokazujemo matematičkom indukcijom po $n$, sasvim analogno kao u RAM-modelu.

Za $n=0$, početno stanje je $q_0$, pa je $\f{State}(x,0)=\f G_1(x)=\f Z(x)=0=\N Q_0(q_0)$ doista njegov kod. Početna pozicija je $0=\f Z(x)=\f G_2(x)=\f{Position}(x,0)$. Početna traka je $w\bl\ldots$, čiji kod $\knk{w\bl\ldots}$ je prema~\eqref{eq:recodein} jednak $\f{Recode}(\kr w,b',b)=G_3(\kr w)=\f{Tape}(\kr w,0)$.

Sada pretpostavimo da je nakon $k$ koraka, $q:=\f{State}(\kr w,k)$ kod stanja, $p:=\f{Position}(\kr w,k)$ pozicija, a $t:=\f{Tape}(\kr w,k)$ kod trake stroja. Ako je ta konfiguracija završna, tada je $q=\N Q_0(q_z)=1$, pa prema definicijama iz leme~\ref{lm:newssdprn} vrijedi $\f{newstate}(q,g)=\f{direction}(q,g)=1$ i $\f{newsymbol}(q,g)=g$. Tada $\f H_1$ daje $q$, $\f H_2$ daje $\f{pd}(p+1)=\f{pd}\bigl(\f{Sc}(p)\bigr)=p$, a $\f H_3$ daje $t\dotminus g\cdot b^p+g\cdot b^p=t$ (jer je $g=t\div b^p\bmod b\le t\div b^p\le\frac{t}{b^p}$, pa je $g\cdot b^p\le t$, odnosno $\dotminus$ je zapravo $-$). Dakle $\f{State}(\kr w,k+1)=\f{State}(\kr w,k)$, i slično za $\f{Position}$ i $\f{Tape}$, odnosno završna konfiguracija ostaje ista, kao što i treba.

Ako ta konfiguracija nije završna, tad je $q\in[0\dd a\rangle\setminus\{1\}$, pa su vrijednosti $\f{newstate}$, $\f{newsymbol}$ i $\f{direction}$ zadane preko funkcije $\delta_0$ te predstavljaju upravo novo stanje, novi znak na trenutnoj poziciji i pomak na novu poziciju. Recimo, ako s $g$ označimo kod trenutno čitanog znaka, a s $d$ treću komponentu odgovarajuće vrijednosti $\delta_0(q,g)$, tada je $\f{direction}(q,g)=d+1$, pa je
\begin{multline}
    \f{Position}(\kr w,k+1)=\f H_2\bigl(\kr w,k,\f{State}(\kr w,k),\f{Position}(\kr w, k),\f{Tape}(\kr w,k)\bigr)=\\
    =\f H_2(\kr w,k,q,p,t)=
    \f{pd}\bigl(p+\f{direction}(q,g)\bigr)=\f{pd}(p+d+1)=\\
    =\max\,\{p+d+1-1,0\}=\max\,\{\f{Position}(\kr w,k)+d,0\}\text.
\end{multline}
Slično se dobiju i odgovarajuće vrijednosti za $\f{State}(\kr w,k+1)$ i $\f{Tape}(\kr w,k+1)$.
\end{proof}

%\subsection{Prepoznavanje završne konfiguracije i čitanje rezultata}

Pomoću funkcije $\f{State}^2$ možemo prepoznati završnu konfiguraciju --- tada će vrijednost te funkcije biti jednaka $\N Q_0(q_z)=1$. Štoviše, ako $\mathcal T_0$-izračunavanje s $w$ stane, tada je broj koraka nakon kojeg se to dogodi upravo najmanji broj za koji vrijedi ta jednakost. (Nemamo probleme kao u primjeru~\ref{pr:Final'}, jer nam je Turingov stroj fiksan i ne prenosimo ga kao kod, a svaki prirodni broj je kod neke riječi iz $\Sigma_0^*$.)

\begin{lema}[{name=[parcijalna rekurzivnost brojenja koraka do zaustavljanja]}]\label{lm:stopprek}
Funkcija $\f{stop}^1$\!, definirana sa
\begin{equation}
    \f{stop}(\kr w):\simeq\text{"broj koraka nakon kojeg $\mathcal T_0$-izračunavanje s $w$ stane",}
\end{equation}
    parcijalno je rekurzivna ($\dom{\f{stop}}=\dom{\N\varphi_0}=\{\kr w\mid w\in\dom{\varphi_0}\}=:\kr{\dom{\varphi_0}}$).
\end{lema}
\begin{proof}
Tvrdimo da je
\begin{equation}\label{eq:stopdef}
    \f{stop}(x)\simeq\mu n\bigl(\f{State}(x,n)=1\bigr)
\end{equation}
dobivena minimizacijom primitivno rekurzivne relacije.

Neka je $x\in\N$ proizvoljan i označimo $w:=(\N\Sigma_0^*)^{-1}(x)$. Ako je $w\in\dom{\varphi_0}$, tada po definiciji~\ref{def:Tcomputefi} $\mathcal T_0$-izračunavanje s $w$ stane --- označimo s $n_0:=\f{stop}(x)$ broj koraka nakon kojeg se to dogodi. Prema lemi~\ref{lm:StateHeadTapeprn} vrijedi $\f{State}(x,n_0)=\N Q_0(q_z)=1$, ali isto tako za sve $n<n_0$ vrijedi $\f{State}(x,n)\ne 1$ --- jer stanje još nije završno, a $\N Q_0$ je injekcija. Dakle $n_0$ je jednak $\mu n\bigl(\f{State}(x,n)=1\bigr)$.

Ako pak $w\notin\dom{\varphi_0}$, tada $\mathcal T_0$-izračunavanje s $w$ ne stane, pa ne postoji $n\in\N$ takav da vrijedi $\f{State}(x,n)=1$ (opet, jer je $\N Q_0$ injekcija) te izraz $\mu n\bigl(\f{State}(x,n)=1\bigr)$ nema smisla, baš kao ni $\f{stop}(x)$.
\end{proof}

Sada napokon možemo dokazati teorem zbog kojeg smo sve ovo radili.

\begin{teorem}[{name=[parcijalna rekurzivnost pratećih Turing-izračunljivih funkcija]}]\label{tm:tikp}
Neka je $\Sigma_0$ abeceda, $\N\Sigma_0$ njeno kodiranje te $\varphi_0$ Turing-izračunljiva funkcija nad njom. Tada je prateća funkcija $\N\varphi_0$ parcijalno rekurzivna.
\end{teorem}
\begin{proof}
Po pretpostavci, postoji Turingov stroj $\mathcal T_0=(Q_0,\Sigma_0,\Gamma_0,\bl,\delta_0,q_0,q_z)$ koji računa $\varphi_0$. Označimo $b':=\card\Sigma_0$ i $b:=\card\Gamma_0$. Primjenjujući na taj $\mathcal T_0$ sve što smo napravili u ovoj točki (kodiramo mu stanja, radnu abecedu u skladu s $\N\Sigma_0$, traku, funkciju prijelaza te izračunavanje s proizvoljnom rječju), dobijemo (među ostalim) funkcije $\f{Recode}$, $\f{Tape}$ i $\f{stop}$, sa svojstvima iz odgovarajućih lema. Tvrdimo da je
\begin{equation}\label{eq:kodfiprek}
    \N\varphi_0(x)\simeq
    \f{Recode}\bigl(\f{Tape}\bigl(x,\f{stop}(x)\bigr),b,b'\bigr)\text,
\end{equation}
pa je $\N\varphi_0$ parcijalno rekurzivna kao kompozicija takvih. Sama parcijalna jednakost~\eqref{eq:kodfiprek} je zapravo~\eqref{eq:kodfidef} iz definicije~\ref{def:kodfi}, samo izrečena u "izračunljivom obliku". Za svaki $x\in\N$ imamo dva slučaja ovisno o tome je li $w:=(\N\Sigma_0^*)^{-1}(x)\in\dom{\varphi_0}$.

Ako nije, tada $\N\varphi_0(x)$ nema smisla, a niti desna strana od~\eqref{eq:kodfiprek} nema smisla --- $\mathcal T_0$ računa $\varphi_0$, pa $w\notin\dom{\varphi_0}$ znači da $\mathcal T_0$-izračunavanje s $w$ ne stane. Tada $x\notin\dom{\f{stop}}$ prema lemi~\ref{lm:stopprek}, a onda po korolaru~\ref{kor:comptot} također $x$ nije ni u domeni desne strane.

Ako je pak $w\in\dom{\varphi_0}$, tada $\mathcal T_0$-izračunavanje s $w$ stane, a prema lemi~\ref{lm:stopprek} to se dogodi nakon $s:=\f{stop}(\kr w)=\f{stop}(x)$ koraka. Prema lemi~\ref{lm:StateHeadTapeprn}, kod trake završne konfiguracije je onda $t:=\f{Tape}(x,s)$, što je $\knk{\varphi_0(w)\bl\ldots}$ jer $\mathcal T_0$ računa $\varphi_0$. Sada je prema~\eqref{eq:recodeout}, $\f{Recode}(t,b,b')=\kr{\varphi_0(w)}$, kao što i treba biti.
\end{proof}

\section{Pretvorba RAM-stroja u Turingov stroj}\label{sec:RAM>Turing}

Pokušajmo sada dokazati obrat teorema~\ref{tm:tikp}. Kako za parcijalno rekurzivne funkcije imamo kompajler u RAM-strojeve, možemo ga iskoristiti i za kompiliranje u Turingove strojeve. To je također poznata pojava u računarstvu: dolaskom nove arhitekture, često ne moramo iznova kompilirati izvorni kod. Ako već imamo kod na sličnoj razini (kao što je recimo već kompilirani RAM-program), možemo ga direktno prevesti ("transpilirati") u kompilirani kod za drugu arhitekturu. To ćemo učiniti ovdje.

Kroz čitavu ovu točku imat ćemo fiksiranu abecedu $\Sigma_0$, njeno kodiranje $\N\Sigma_0$, kodiranje $\kr{\cdots}=\N\Sigma_0^*$ zapisom u pomaknutoj bazi $b'=\card\Sigma_0$, jezičnu funkciju $\varphi_0$ nad $\Sigma_0$ takvu da je $\N\varphi_0\in\mathscr Comp_1$ te fiksni RAM-algoritam $P_0^1$ koji računa $\N\varphi_0$.

Cilj će nam biti konstruirati Turingov stroj $\mathcal T_0$ koji računa $\varphi_0$, tako da primi ulaz $w$ na traci, kodira ga u $x:=\kr w$, od toga konstruira početnu konfiguraciju RAM-stroja $\mathcal S_0$ s programom $P_0$ i ulazom $x$ ("stavi $x$ u $\reg1$\!") te simulira redom korake kroz koje $P_0$-izračunavanje s $x$ prolazi. Ako\slash kad $\mathcal S_0$ uđe u završnu konfiguraciju, $\mathcal T_0$ će sadržaj $\reg0$ "dekodirati" u riječ $v:=\varphi_0(w)$, premjestiti $v$ na početak trake i obrisati sve ostalo s nje. Ako se to nikad ne dogodi, $\mathcal T_0$ će vječno simulirati rad $\mathcal S_0$, odnosno nikad neće otići u završno stanje --- što i treba da bi računao $\varphi_0$, jer činjenica da $P_0$-izračunavanje s $\kr w$ ne stane znači da $w\notin\dom{\varphi_0}$. Efektivno, definiciju prateće funkcije smo "izvrnuli" iznutra van: $\N\varphi_0=\N\Sigma_0^*\circ\varphi_0\circ(\N\Sigma_0^*)^{-1}$ znači $\varphi_0=(\N\Sigma_0^*)^{-1}\circ\N\varphi_0\circ\N\Sigma_0^*$.

Stroj $\mathcal T_0$ ćemo konstruirati u pet dijelova (\emph{fragmenata}), koji provode razne faze opisanog postupka simulacije stroja $\mathcal S_0$. Da bismo ga opisali, potrebno je precizirati stanja, radnu abecedu $\Gamma$ te prijelaze.

Stanja ćemo uvoditi postepeno, kako budemo specificirali funkciju $\delta$. Dodat ćemo još jedno stanje, stanje greške $q_x$, uz standardnu konvenciju da za bilo koji par $(q,\alpha)$ na kojem nismo eksplicitno definirali $\delta$, vrijedi $\delta(q,\alpha):=(q_x,\alpha,1)$. To je samo birokratski detalj da bi $\delta$ bila totalna, jer $\mathcal T_0$ nikada neće otići u stanje $q_x$: jedini način da računa beskonačno dugo bit će da odgovarajuće $P_0$-izračunavanje ne stane.

S druge strane, radnu abecedu već možemo preciznije specificirati. Prvo, u njoj se moraju nalaziti svi znakovi iz $\Sigma_0$, kodirani po $\N\Sigma_0$ brojevima iz $[1\dd b']$. Znak ${\N\Sigma_0}^{-1}(t)$ označit ćemo s $\alpha_t$. Oni predstavljaju "znamenke" od $1$ do $b'$ u pomaknutoj bazi $b'$.

Za označavanje početka i kraja (upotrijebljenog dijela) trake koristimo graničnik, koji se obično piše $\t\$$. Također moramo nekako odvojiti dio za ulaz i dio za računanje (simulaciju): za to ćemo koristiti separator koji se obično piše $\t\#$. Traka će u početku izgledati otprilike kao $\t\$\bl^lv\t\#u\bl\ldots$, a kasnije kao $\t\$\bl^k\t\#uv\t\$\bl\ldots$, gdje je $v\in\Sigma_0^*$ dio za ulaz odnosno izlaz, a $u\in(B^m)^*$ dio za reprezentaciju konfiguracije od $\mathcal S_0$ --- o kojem ćemo govoriti nešto kasnije. Sve u svemu, zasad znamo
%\begin{equation}
    $\Gamma:=\Sigma_0\dcup\{\t\$,\t\#\}\dcup B^m$,
%\end{equation}
gdje je $B^m$ skup koji ćemo kasnije definirati --- i sadržavat će prazninu $\bl$.

%\subsection{Inicijalizacija RAM-stroja}

Cilj prve faze je dovesti traku u oblik $\t\$w\t\#\bl\ldots$, tako da možemo početi kodirati. Kako je traka na početku $w\bl\ldots$, moramo pomaknuti riječ za jednu ćeliju udesno.

\begin{lema}[{name=[prvi fragment transpiliranog stroja]}]\label{lm:faza1}
    Postoji fragment Turingovog stroja koji prevodi početnu konfiguraciju $\underset{n_{\t\$}}{w}=(n_{\t\$},0,w\bl\ldots)$ u konfiguraciju $\t\$\underset{q_0}{w}\t\#=(q_0,\dulj{w},\t\$w\t\#\bl\ldots)$.
\vspace{-6pt}
\end{lema}

\begin{proof}
    Standardni način da se to napravi je otići na kraj riječi, pa pomicati slova s kraja redom za po jedno mjesto udesno. To možemo i na Turingovom stroju (pokušajte!), ali možemo i jednostavnije: obrisat ćemo prvi znak (zamijeniti ga s $\t\$$) i \emph{zapamtiti ga} u stanju stroja --- imat ćemo po jedno stanje $n_\alpha$ za prijenos svakog znaka $\alpha\in\Sigma_0$ te još dva posebna stanja $n_{\t\$}$ i $n_{\bl}$.

    Tada, ako u stanju $n_\alpha$ čitamo znak $\beta$, zamjenjujemo ga s $\alpha$ i pamtimo u stanju (idemo u stanje $n_\beta$)~\eqref{d:11}. To objašnjava i zašto je početno stanje $n_{\t\$}$. Nakon što pročitamo i pomaknemo čitavu riječ, nađemo se u stanju $n_{\bl}$ na prvoj $\bl$ nakon riječi. Zamijenimo je s \t\# i pomaknemo se ulijevo~\eqref{d:12}.
\end{proof}
\noindent\begin{gather*}
    \SwapAboveDisplaySkip
\label{d:11}\tag{$\delta$1a}
    \delta(n_\alpha,\beta):=(n_{\beta},\alpha,1)\text{, za sve }\alpha\in\{\t\$\}\dcup\Sigma_0\text{, }\beta\in\Sigma_0\dcup\{\bl\}\\
\label{d:12}\tag{$\delta$1b}
    \delta(n_{\bl},\bl):=(q_0,\t\#,-1)
\end{gather*}

\begin{primjer}[{name=[prvi fragment transpiliranog stroja]}]
Za $\Sigma_0:=\{\t a,\t b\}$, prvi fragment je prikazan dijagramom
\begin{equation}
\begin{tikzpicture}[baseline=(q0)]\label{dia:T1}
\clip (-1,-2.53) rectangle (8.5,2.16);
    \node[state,initial] (q0) {$n_{\t\$}$};
\node[state] (na) [above right=0.25 and 1.5 of q0] (na) {$n_{\t a}$};
\node[state,below right=0.25 and 1.5 of q0] (nb) {$n_{\t b}$};
    \node[state,below right=0.25 and 1.5 of na] (q1) {$n_{\bl}$};
\node[state,right of=q1] (q2) {$q_0$};
\draw
(q0) edge node[above=0.1] {\t a:\t\$} (na)
(q0) edge node[below=0.1] {\t b:\t\$} (nb)
(na) edge[loop above] node[below left=0.2]{\t a} (na)
(nb) edge[loop below] node[above right=0.2]{\t b} (nb)
(na) edge node[above=0.1]{\bl:\t a} (q1)
(nb) edge[below] node{\bl:\t b} (q1)
(nb) edge[bend left=15] node[left]{\t a:\t b} (na)
(na) edge[bend left=15] node[right]{\t b:\t a} (nb)
(q0) edge[bend right=90,looseness=1.5] node[above left=0.4 and 1.8]{\bl:\t\$} (q1)
(q1) edge[dashed] node[above]{\bl:\t\#} (q2)
;
\end{tikzpicture}\text.
\end{equation}
Za ulaz \t{aab} izračunavanje počinje s
   $
    \underset{\mathclap{n_{\t\$}}}{\t a}\t{ab}\leadsto\t\$\underset{\mathclap{n_{\t a}}}{\t a}\t b\leadsto\t{\$a}\underset{\mathclap{n_{\t a}}}{\t b}\leadsto\t{\$aa}\underset{\mathclap{n_{\t b}}}{\bl}\leadsto\t{\$aab}\underset{\mathclap{n_{\bl}}}{\bl}\leadsto\t{\$aa}\underset{\mathclap{q_0}}{\t b}\t\#
    $.
\end{primjer}

Sljedeći korak je konstruirati početnu konfiguraciju od $\mathcal S_0$, desno od separatora \t\# na poziciji $\dulj{w}$. Znamo da to mora biti element od $(B^m)^*$, pa pokušajmo s unarnim kodiranjem: cilj nam je desno od \t\# napisati $r_1^{\kr w}$ za jedan konkretni element $r_1\in B^m$. %Što je točno $r_1$ (i čitav $B^m$), reći ćemo u idućoj točki.%, ali zasad ga samo tretiramo kao zasebni znak (baš kao \bl).

\begin{lema}[{name=[drugi fragment transpiliranog stroja]}]\label{lm:faza2}
Postoji fragment Turingovog stroja koji prevodi konfiguraciju\newline $\T{\$}{q_0}{\t{$w$}}{\#}=(q_0,\dulj w,\t\$w\t\#\bl\ldots)$ u konfiguraciju $\t\$\bl^{\dulj w}\t\#\underset{\mathclap{p_0}}{r_1}^{\!\kr w}=(p_0,\dulj w+2,\t\$\bl^{\dulj w}\t\#r_1{}^{\!\kr w}\bl\ldots)$.
\end{lema}
\vspace{-1em}
\noindent\begin{gather*}
    \label{d:21}\tag{$\delta$2a}\delta(q_0,\alpha_t):=(q_1,\alpha_{t-1},1)\text{, za sve }t\in[2\dd b']\\
\label{d:22}\tag{$\delta$2b}\delta(q_1,\alpha):=(q_1,\alpha,1)\text{, za sve }\alpha\in\Sigma_0\mathbin{\dcup}\{\t\#,r_1\}\\
\label{d:23}\tag{$\delta$2c}\delta(q_1,\bl):=(q_0,r_1,-1)\\
\label{d:24}\tag{$\delta$2d}\delta(q_0,\gamma):=(q_0,\gamma,-1)\text{, za }\gamma\in\{\t\#,r_1\}\\
    \label{d:25}\tag{$\delta$2e}\delta(q_0,\alpha_1):=(q_0,\alpha_{b'},-1)\\
\label{d:26}\tag{$\delta$2f}\delta(q_0,\gamma):=(q_2,\gamma,1)\text{, za }\gamma\in\{\t\$,\bl\}\\
\label{d:27}\tag{$\delta$2g}\delta(q_2,\alpha_{b'}):=(q_1,\bl,1)\\
\label{d:28}\tag{$\delta$2h}\delta(q_2,\t\#):=(p_0,\t\#,1)
\end{gather*}
\begin{proof}
To je pretvorba zapisa između različitih pomaknutih baza: broj $\kr w$ čitamo u pomaknutoj bazi~$b'$ pomoću znamenaka $\alpha_t,t\in[1\dd b']$ te pišemo u pomaknutoj bazi~$1$ pomoću znamenke $r_1$. To se može riješiti tako da "dekrementiramo" $w$, za svaki uspješni dekrement dodajući po jedan $r_1$ na kraj.

Zapis u pomaknutoj bazi dekrementiramo slično kao i zapis u običnoj bazi. Ako zadnji znak nije $\alpha_1$, smanjimo njegov kod za $1$~\eqref{d:21} i odemo desno~\eqref{d:22} do prvog razmaka, koji zamijenimo s $r_1$~\eqref{d:23} i vraćamo se ponovo lijevo~\eqref{d:24}. Ako naiđemo na $\alpha_1$, zamijenimo ga "najvećom znamenkom" $\alpha_{b'}$ i nastavljamo smanjivati dalje lijevo~\eqref{d:25}. Ako smanjujući dođemo do lijevog ruba broja~\eqref{d:26}, prvu znamenku mu obrišemo~\eqref{d:27}. Ako smo ostali bez znamenaka, gotovi smo: prijeđemo desno iza separatora, u stanje $p_0$~\eqref{d:28}.

Za dokaz da to stvarno dovede do željene konfiguracije dovoljno je primijetiti da se nakon svakog prolaza lijevo-desno  nađemo ponovo u stanju $q_0$, s pozicijom pri desnom kraju trake oblika $\t\$\bl^{\dulj w-\dulj{w'}}w'\t\#r_1^{\kr w-\kr{w'}}$. Na početku je to istina jer je $w=w'$ i još nemamo nijedan $r_1$, u svakom koraku ostaje istina jer se dekrementom $\kr{w'}$ smanjuje za $1$, dok se broj znakova $r_1$ povećava za $1$, a na kraju imamo $\t\$\bl^{\dulj w}\t\#r_1^{\kr w}$, jer je tada $w'=\varepsilon$ s kodom $\kr{\varepsilon}=0$.
\end{proof}

\begin{primjer}[{name=[drugi fragment transpiliranog stroja]}]
Za $\Sigma_0:=\{\t a,\t b,\t c,\t d\}$, drugi fragment je prikazan dijagramom
\begin{equation}
\begin{tikzpicture}[baseline=(q4)]\label{dia:T2}
%\clip (-1,-2.53) rectangle (8.5,2.16);
\node[state] (q2) {$q_0$};
\node[state,below of=q2] (q3) {$q_1$};
\node[state,below right=0.75 and 2 of q2] (q4) {$q_2$};
\node[state,right of=q4] (p0) {$p_0$};
\draw
(q2) edge node[left,align=left] {\t d:\t c\\\t c:\t b\\\t b:\t a} (q3)
(q2) edge[dashed, loop left] node[left]{$\t\#,r_1,\t a$:\t d} (q2)
(q2) edge node[above,xshift=3]{$\t\$,\bl$} (q4)
(q3) edge[loop left] node[left]{$\t\#,r_1,\Sigma_0$} (q3)
(q3) edge[dashed,bend left=60] node[left]{\bl:$r_1$} (q2)
(q4) edge node[below,xshift=3]{\t d:\bl} (q3)
(q4) edge node[above]{\t\#} (p0)
;
\end{tikzpicture}\;\text.
\end{equation}
Za riječ $\t{aab}$ imamo šetnju kroz konfiguracije (pišemo samo neke od njih):
\begin{multline}
\T{\$aa}{q_0}{b}{\#}\leadsto
\T{\$aaa}{q_1}{\#}{}\leadsto
\T{\$aaa\#}{q_1}{\bl}{}\leadsto
\T{\$aaa}{q_0}{\#}{\t{$r_1$}}\leadsto^4
%\T{\$aa}{q_0}{a}{\#\t{$r_1$}}\leadsto
%\T{\$a}{q_0}{a}{d\#\t{$r_1$}}\leadsto
%\T{\$}{q_0}{a}{dd\#\t{$r_1$}}\leadsto{}\\
%{}\leadsto
\T{}{q_0}{\$}{ddd\#\t{$r_1$}}\leadsto
\T{\$}{q_2}{d}{dd\#\t{$r_1$}}\leadsto
    \T{\$\bl}{q_1}{d}{d\#\t{$r_1$}}\leadsto^4{}\\
    {}\leadsto^4
\T{\$\bl dd\#\t{$r_1$}}{q_1}{\bl}{}\leadsto
\T{\$\bl dd\#}{q_0}{\t{$r_1$}}{\t{$r_1$}}\leadsto
\T{\$\bl dd}{q_0}{\#}{\t{$r_1r_1$}}\leadsto
%{}\leadsto
\T{\$\bl d}{q_0}{d}{\#\t{$r_1^2$}}\leadsto
\T{\$\bl dc}{q_1}{\#}{\t{$r_1^2$}}\leadsto^3
    \T{\$\bl dc\#\t{$r_1^2$}}{q_1}{\bl}{}\leadsto^4{}\\
    {}\leadsto^4\T{\$\bl d}{q_0}{c}{\#\t{$r_1^3$}}\leadsto^*
\T{\$\bl d}{q_0}{b}{\#\t{$r_1^4$}}\leadsto^*
\T{\$\bl d}{q_0}{a}{\#\t{$r_1^5$}}\leadsto^*%{}\\
%{}\leadsto^*
\T{\$\bl c}{q_0}{d}{\#\t{$r_1^6$}}\leadsto^*
\T{\$\bl c}{q_0}{a}{\#\t{$r_1^9$}}\leadsto^*
    \T{\$\bl b}{q_0}{a}{\#\t{$r_1^{13}$}}\leadsto^*{}\\
    {}\leadsto^*
\T{\$\bl a}{q_0}{a}{\#\t{$r_1^{17}$}}\leadsto^*
\T{\$\bl\bl}{q_0}{d}{\#\t{$r_1^{18}$}}\leadsto^*
\T{\$\bl\bl}{q_0}{a}{\#\t{$r_1^{21}$}}\leadsto%{}\\
%{}\leadsto
\T{\$\bl}{q_0}{\bl}{d\#\t{$r_1^{21}$}}\leadsto
\T{\$\bl\bl}{q_2}{d}{\#\t{$r_1^{21}$}}\leadsto
    \T{\$\bl\bl\bl}{q_1}{\#}{\t{$r_1^{21}$}}\leadsto^*{}\\
    {}\leadsto^*
\T{\$\bl\bl\bl\#\t{$r_1^{21}$}}{q_1}{\bl}{}\leadsto^*
\T{\$\bl\bl}{q_0}{\bl}{\#\t{$r_1^{22}$}}\leadsto
\T{\$\bl\bl\bl}{q_2}{\#}{\t{$r_1^{22}$}}\leadsto%{}\\
%{}\leadsto
\T{\$\bl\bl\bl\#}{p_0}{\t{$r_1^{22}$}}{}\text,
\end{multline}
što je točno kako treba biti jer je $b'=4$, pa je $\kr{\t{aab}}=(112)_4=22$. "Invarijanta petlje" zadovoljena je u svim istaknutim konfiguracijama gdje se $q_0$ nalazi ispod zadnjeg znaka prije separatora --- npr.\ za $\t{\$\bl cd\#$r_1^6$}$ je $w'=\t{cd}$, čiji je kod $\kr{\mspace{-1mu}\t{cd}}=(34)_4=16$, a $\kr w-\kr{w'}=22-16=6$.
\end{primjer}

%I to je to. Spremni smo za simulaciju $P_0$-izračunavanja s $\kr w$.

\subsection{Registri kao tragovi trake}

Vrijeme je da opišemo kako ćemo točno reprezentirati konfiguraciju RAM-stroja $\mathcal S_0$ na traci.  Zapravo, vrijednost programskog brojača je jedna od konačno mnogo njih i na njoj su dopuštene proizvoljne transformacije, pa ju je bolje prikazati kroz stanje stroja. Na traci će stajati samo stanje registara --- i to samo onih relevantnih. U tu svrhu, označimo s $m:=m_{P_0^1}=\max\,\{m_{P_0},2\}$ širinu algoritma $P_0^1$; tada znamo da je dovoljno pratiti prvih $m$ registara.

Kako to najbolje učiniti? Kao što smo rekli na početku ovog poglavlja, jezični model je "transponirani" brojevni model: u jezičnom modelu broj ćelija nije ograničen, ali broj mogućih sadržaja svake ćelije jest, dok je u brojevnom modelu obrnuto. To znači da možemo stanje relevantnih (prvih $m$) registara prikazati kao riječ nad "abecedom $m$-bitnih procesorskih riječi"
\begin{equation}
\renewcommand{\arraystretch}{0.1}
    B^m:=\prod_{i<m}\{\t\textopenbullet,\t\textbullet\}=\bigl\{\left[\begin{array}{@{}l@{}}\beta_0\\[0mm]
    %\MTFlushSpaceAbove
        \vdotswithin{\beta}\\[1mm]
    \beta_{m-1}\end{array}\right]\bigm|(\forall i<m)(\beta_i\in\{\t\textopenbullet,\t\textbullet\})\bigr\}\text,
\end{equation}
tako da se dio trake desno od separatora sastoji od $m$ "redaka" (\emph{tragova}), u svakom od kojih piše $\t\textbullet^t\t\textopenbullet\dotsm$, gdje je $t$ sadržaj odgovarajućeg registra. U tom kontekstu, $B^m$ je "prostor stupaca" sa svim ($2^m$ njih) stupcima znakova $\t\textbullet$ i $\t\textopenbullet$ visine $m$. Recimo,
\begin{equation}\label{eq:B4}
\renewcommand{\arraystretch}{0.1}
    B^4:=\biggl\{
    \begin{array}{c}
    \t\textopenbullet\\
    \t\textopenbullet\\
    \t\textopenbullet\\
    \t\textopenbullet
    \end{array},
    \begin{array}{c}
    \t\textbullet\\
    \t\textopenbullet\\
    \t\textopenbullet\\
    \t\textopenbullet
    \end{array},
    \begin{array}{c}
    \t\textopenbullet\\
    \t\textbullet\\
    \t\textopenbullet\\
    \t\textopenbullet
    \end{array},
    \begin{array}{c}
    \t\textbullet\\
    \t\textbullet\\
    \t\textopenbullet\\
    \t\textopenbullet
    \end{array},
    \begin{array}{c}
    \t\textopenbullet\\
    \t\textopenbullet\\
    \t\textbullet\\
    \t\textopenbullet
    \end{array},
    \begin{array}{c}
    \t\textbullet\\
    \t\textopenbullet\\
    \t\textbullet\\
    \t\textopenbullet
    \end{array},
    \begin{array}{c}
    \t\textopenbullet\\
    \t\textbullet\\
    \t\textbullet\\
    \t\textopenbullet
    \end{array},
    \begin{array}{c}
    \t\textbullet\\
    \t\textbullet\\
    \t\textbullet\\
    \t\textopenbullet
    \end{array},
    \begin{array}{c}
    \t\textopenbullet\\
    \t\textopenbullet\\
    \t\textopenbullet\\
    \t\textbullet
    \end{array},
    \begin{array}{c}
    \t\textbullet\\
    \t\textopenbullet\\
    \t\textopenbullet\\
    \t\textbullet
    \end{array},
    \begin{array}{c}
    \t\textopenbullet\\
    \t\textbullet\\
    \t\textopenbullet\\
    \t\textbullet
    \end{array},
    \begin{array}{c}
    \t\textbullet\\
    \t\textbullet\\
    \t\textopenbullet\\
    \t\textbullet
    \end{array},
    \begin{array}{c}
    \t\textopenbullet\\
    \t\textopenbullet\\
    \t\textbullet\\
    \t\textbullet
    \end{array},
    \begin{array}{c}
    \t\textbullet\\
    \t\textopenbullet\\
    \t\textbullet\\
    \t\textbullet
    \end{array},
    \begin{array}{c}
    \t\textopenbullet\\
    \t\textbullet\\
    \t\textbullet\\
    \t\textbullet
    \end{array},
    \begin{array}{c}
    \t\textbullet\\
    \t\textbullet\\
    \t\textbullet\\
    \t\textbullet
    \end{array}
    \biggr\}\text.
\end{equation}
Konkretno, ako (za $m=4$) u nekom trenutku imamo konfiguraciju $(1,4,0,5,0,0,\dotsc\mspace{-1mu})$, uz $\dulj w=7$, traka će biti
\begin{equation}\label{eq:bitmap}
\renewcommand{\arraystretch}{0.1}
    \t{\$\bl\bl\bl\bl\bl\bl\bl\#}\begin{array}{@{}c@{}}
    \t\textbullet\\
    \t\textbullet\\
    \t\textopenbullet\\
    \t\textbullet
    \end{array}
    \begin{array}{@{}c@{}}
    \t\textopenbullet\\
    \t\textbullet\\
    \t\textopenbullet\\
    \t\textbullet
    \end{array}
    \begin{array}{@{}c@{}}
    \t\textopenbullet\\
    \t\textbullet\\
    \t\textopenbullet\\
    \t\textbullet
    \end{array}
    \begin{array}{@{}c@{}}
    \t\textopenbullet\\
    \t\textbullet\\
    \t\textopenbullet\\
    \t\textbullet
    \end{array}
    \begin{array}{@{}c@{}}
    \t\textopenbullet\\
    \t\textopenbullet\\
    \t\textopenbullet\\
    \t\textbullet
    \end{array}
    \begin{array}{@{}c@{}}
    \t\textopenbullet\\
    \t\textopenbullet\\
    \t\textopenbullet\\
    \t\textopenbullet
    \end{array}
    \begin{array}{@{}c@{}}
    \t\textopenbullet\\
    \t\textopenbullet\\
    \t\textopenbullet\\
    \t\textopenbullet
    \end{array}
    \begin{array}{@{}c@{}}
    \t\textopenbullet\\
    \t\textopenbullet\\
    \t\textopenbullet\\
    \t\textopenbullet
    \end{array}
    \begin{array}{@{}c@{}}
    \t\textopenbullet\\
    \t\textopenbullet\\
    \t\textopenbullet\\
    \t\textopenbullet
    \end{array}
    \begin{array}{@{}c@{}}
    \t\textopenbullet\\
    \t\textopenbullet\\
    \t\textopenbullet\\
    \t\textopenbullet
    \end{array}\dotsm\text,
\end{equation}
iz čega vidimo dvije stvari. Prvo, $\bl$ mora biti ovaj prvi element u $B^m$, koji ima sve kružiće prazne. Tako postižemo da traka bude prazna od nekog mjesta nadalje (konkretno, maksimum svih $m$ vrijednosti u registrima), odnosno da bude s konačnim nosačem. Drugo, naš znak $r_1$, koji smo upotrijebili da postavimo $\reg1$ na ulaznu vrijednost $\kr w$, je treći element u $B^m$, napisan transponirano kao $(\t{\textopenbullet\textbullet\textopenbullet}^{m-2})^\intercal$. Stupčano,
\begin{equation}
\renewcommand{\arraystretch}{0.1}
    \bl\,:=\begin{array}{c}
    \t\textopenbullet\\
    \t\textopenbullet\\
        \t\textopenbullet\\[0mm]
    %\MTFlushSpaceAbove
        \vdotswithin{\t\textopenbullet}\\[1mm]
    \t\textopenbullet
    \end{array}\text{,\qquad}
    r_1:=\begin{array}{c}
    \t\textopenbullet\\
    \t\textbullet\\
        \t\textopenbullet\\[0mm]
    %\MTFlushSpaceAbove
        \vdotswithin{\t\textopenbullet}\\[1mm]
    \t\textopenbullet
    \end{array}\text.
\end{equation}

\begin{primjer}[{name=[dodavanje znaka na kraj riječi]}]\label{pr:+a}
	Nad abecedom $\Sigma:=\{\t a,\t b\}$ (kodiranom s $\t a\mapsto1$, $\t b\mapsto2$) pogledajmo jezičnu funkciju "dopisivanje \t a zdesna", $\varphi_{\t a}(w):=w\t a$. Baza je $b'=2$, pa je prateća funkcija $\N\varphi_{\t a}(\kr w)=\kr{w\t a}=\kr w\conc2\kr{\t a}=\kr w\cdot2^{\,\f{slh}(\kr{\t a},2)}+\kr{\t a}=2^{\dulj{\t a}}\cdot\kr w+1$, odnosno $\N\varphi_{\t a}(x)=2x+1$. Očito je $\N\varphi_{\t a}\in\mathscr Comp_1$: računa je jednostavni RAM-program
\begin{equation}\label{eq:+aprog}
    P_{\t a}:=\begin{prog}
    0.&\incr0\\
    1.&\decr14\\
    2.&\incr0\\
    3.&\goto\;0
    \end{prog}
\end{equation}
	čija je širina $m=2$. Za ulaznu riječ $\t{ab}$ je $\kr{\t{ab}}=(\mspace{-0.5mu}12)_2=4$, pa je konfiguracija Turingovog stroja kroz faze (početna, nakon prve faze i nakon druge faze)
\begin{equation}\label{eq:+afaze}
\renewcommand{\arraystretch}{0.1}
    \T{}{n_{\t\$}}{a}{b}\leadsto^*
\T{\$a}{q_0}{b}{\#}\leadsto^*
    \T{\$\boo\boo\#}{\mund{p_0}}{\boi}{\boi\boi\boi}\,\text{.\qedhere}
\end{equation}
%te smo spremni za početak simulacije.
\end{primjer}

Rekli smo da ćemo vrijednost programskog brojača držati u stanju stroja --- i to je već učinjeno kroz supskript stanja $p_0$: ta nula znači da je \textsc{pc} upravo postao $0$. Općenito, ako je $P_0$ duljine $n:=n_{P_0}$ imat ćemo stanja $p_i,i\in[0\dd n]$, a ulazak u stanje $p_n$ značit će završetak $P_0$-izračunavanja.

\begin{definicija}[{name=[$m$-reprezentacija RAM-konfiguracije]}]
	Neka je $P^1$ RAM-algoritam širine $m$, te $c=(r_0,r_1,\dots,r_{m-1},0,0,\dots,pc)$ konfiguracija RAM-stroja s programom $P$. \emph{$m$-reprezentacija} od $c$ je konfiguracija sa stanjem $p_{pc}$, trakom oblika $\t\$\bl^k\t\#v\bl\dots$ i pozicijom većom od $k$ --- pri čemu je $v$ riječ nad $B^m$ u čijem $i$-tom tragu piše $\t\textbullet^{r_i}\t\textopenbullet^{\dulj v-r_i}$, za sve $i\in[0\dd m\rangle$.
%Za $k,m,n\in\N$ takve da je $m\ge 2$, za proizvoljnu RAM-konfiguraciju $c$ sa svojstvima $c(\reg j)=0$ za sve $j\ge m$ i $c(\textsc{pc})\le n$, \emph{reprezentacija} je Turing-konfiguracija
%\begin{equation}
%Turing_{kmn}(c):=\biggl(p_{c(\textsc{pc})},k+2,\t\$\bl^k\t\#\!\left[\begin{array}{@{}l@{}}
%\renewcommand{\arraystretch}{0.1}
%\t\textbullet^{c(\reg0)}\t\textopenbullet\dotsm\\
%\t\textbullet^{c(\reg1)}\t\textopenbullet\dotsm
%\MTFlushSpaceBelow
%\vdotswithin{\text\textbullet}
%\MTFlushSpaceBelow
%\t\textbullet^{c(\reg{m-1})}\t\textopenbullet\dotsm
%\end{array}\right.\biggr)\text{.\;\qedhere}
%\end{equation}
\end{definicija}

Recimo,~\eqref{eq:bitmap} je primjer trake $4$-reprezentacije konfiguracije $(1,4,0,5,0,0,\dots\mspace{-1mu})$. Također, zadnja konfiguracija u~\eqref{eq:+afaze} je $2$-reprezentacija početne konfiguracije RAM-stroja s programom $P_{\t a}$ iz~\eqref{eq:+aprog}, s ulazom \t{ab}.

Ipak, stanja $p_i$ nisu dovoljna za izvršavanje RAM-instrukcija: za svaki $i<n$ imamo još jedno stanje $s_i$, koje tome služi. Stanje $p_i$ je "pripremno stanje" za $s_i$, s jedinim zadatkom fiksiranja pozicije odmah nakon znaka $\t\#$.

%\begin{lema}[{name=[{treći fragment transpiliranog stroja, priprema za instrukciju}]}]\label{lm:pi>si}
%Za svaki $i\in[0\dd n\rangle$, postoji fragment Turingovog stroja koji prevodi bilo koju konfiguraciju oblika $(p_i,t,\t\$\bl^k\,\t\#v\bl\ldots)$, gdje su $v\in(B^m)^*$ i $t>k\in\N$, u konfiguraciju $(s_i,k+2,\t\$\bl^k\,\t\#v\bl\ldots)$.
%\end{lema}
%\begin{proof}
%Dijagramatski,
%\begin{tikzpicture}[baseline=0]
%\node[state] (pi) {$p_i$};
%\node[state, right of=pi] (si) {$s_i$};
%\draw
%(pi) edge[loop left,dashed] node[left]{$B^m$} (pi)
%(pi) edge[below] node{$\t\#$} (si)
%;
%\end{tikzpicture}\;.
%\end{proof}

%\subsection{Izvršavanje pojedinih RAM-instrukcija}

Da bismo implementirali prijelaze iz stanja $s_i$, moramo naučiti "adresirati bitove" na traci $m$-reprezentacije. Svaki element od $B^m$ vidimo kao $m$-bitnu riječ, koristeći za promjenu pojedinog bita istu tehniku kao "pravi" procesor: čitavu procesorsku riječ zamijenimo drugom, koja se podudara s njom u svim bitovima osim onog koji želimo promijeniti. Primjerice, ako na traci~\eqref{eq:bitmap} želimo "dekrementirati $\reg1$", zamijenit ćemo njen trinaesti znak $(\t{\textopenbullet\textbullet\textopenbullet\textbullet})^\intercal$ znakom $(\t{\textopenbullet\textopenbullet\textopenbullet\textbullet})^\intercal$, devetim na popisu~\eqref{eq:B4}.

\begin{definicija}[{name=[funkcija prijelaza u određenom tragu]}]\label{def:trag}
Neka je $j\in[0\dd m\rangle$, neka su $q,q'\in Q$, neka su $\beta,\beta'\in B:=\{\t\textopenbullet,\t\textbullet\}$ te neka je $d\in\{-1,1\}$. Pišemo $\delta_{(j)}(q,\beta):=(q',\beta',d)$ kao pokratu za:
$\delta(q,\gamma):=(q',\gamma',d)$\text,
za sve $\gamma,\gamma'\in B^m$ takve da je $\gamma_j=\beta$, $\gamma'_j=\beta'$ te $\gamma_i=\gamma'_i$ za sve $i\in[0\dd m\rangle\setminus\{j\}$.

U dijagramima, na strelici od $q$ prema $q'$ pišemo $\beta\t:\beta'\texttt@j$, ili $\beta\texttt@j$ ako je $\beta'=\beta$.
\end{definicija}


\subsection{Simulacija RAM-izračunavanja}

\begin{propozicija}[{name=[{treći fragment transpiliranog stroja, izvršavanje instrukcija}]}]\label{prop:gadgets}
    Neka je $P=\begin{prog}t.&I_t\end{prog}_{t<n}$ RAM-program. Označimo $m:=m_{P^1}$.

        Tada za svaki $i<n$ postoji fragment Turingovog stroja, koji prevodi svaku $m$-reprezentaciju RAM-konfiguracije $c$ kojoj je $c(\textsc{pc})=i$, u neku $m$-reprezentaciju od $d$, gdje $c\leadsto d$ po programu $P$.
\end{propozicija}
\begin{proof}
Traženi fragment počinje s radom u stanju $p_i$ na poziciji većoj ili jednakoj poziciji separatora $k$, a treba završiti u stanju $p_{i'}$ za $i':=d(\textsc{pc})$ na poziciji s istim svojstvom. Prvi dio je uvijek isti: dok god čitamo znak iz $B^m$, pomičemo se lijevo~\eqref{d:31}. Kad pročitamo \t\#, pomaknemo se desno~\eqref{d:32}. 
\begin{gather*}
\label{d:31}\tag{$\delta$3a}
    \delta(p_i,\gamma):=(p_i,\gamma,-1)\text{, za sve }\gamma\in B^m\\
\label{d:32}\tag{$\delta$3b}
    \delta(p_i,\t\#):=(s_i,\t\#,1)
\end{gather*}

Tako dođemo u stanje $s_i$, na poziciju neposredno desno od separatora. Zbog $c(\textsc{pc})=i<n=n_P$ konfiguracija $c$ sigurno nije završna, pa postoji instrukcija $I_i=I_{c(\textsc{pc})}$. Dalji postupak ovisi o tipu te instrukcije.

Ako je $I_i$ tipa $\inc$, recimo  $i.\;\incr j$, trebamo otići do kraja znakova \t\textbullet\ u tragu $j$~\eqref{d:41} i dopisati još jedan \t\textbullet\ tamo, a zatim se pripremiti za sljedeću instrukciju, onu rednog broja $i+1$~\eqref{d:42}.
\begin{gather*}
\SwapAboveDisplaySkip
\label{d:41}\tag{$\delta$3c}
    \delta_{(j)}(s_i,\t\textbullet):=(s_i,\t\textbullet,1)\\
\label{d:42}\tag{$\delta$3d}
    \delta_{(j)}(s_i,\t\textopenbullet):=(p_{i+1},\t\textbullet,-1)
\end{gather*}
Ako je $I_i$ tipa $\dec$, recimo $i.\;\decr jl$, trebamo kao i za $\inc$ doći do kraja znakova~\t\textbullet~\eqref{d:51}, a kada pročitamo \t\textopenbullet, pomaknuti se lijevo u novo "stanje odluke" $t_i$~\eqref{d:52}. Ako u tom stanju čitamo $\t\#$, znači da je $c(\reg j)=0$, pa prelazimo na instrukciju rednog broja $l$~\eqref{d:53}. Ako ne, tada u tragu $j$ zamijenimo \t\textbullet\ s \t\textopenbullet\ i prelazimo na sljedeću instrukciju~\eqref{d:54}.
\begin{gather*}
\SwapAboveDisplaySkip
\label{d:51}\tag{$\delta$3e}
    \delta_{(j)}(s_i,\t\textbullet):=(s_i,\t\textbullet,1)\\
\label{d:52}\tag{$\delta$3f}
    \delta_{(j)}(s_i,\t\textopenbullet):=(t_i,\t\textopenbullet,-1)\\
\label{d:53}\tag{$\delta$3g}
    \delta(t_i,\t\#):=(p_l,\t\#,1)\\
\label{d:54}\tag{$\delta$3h}
    \delta_{(j)}(t_i,\t\textbullet):=(p_{i+1},\t\textopenbullet,-1)
\end{gather*}
Ako je $I_i$ tipa $\goto$, recimo $i.\;\goto\;l$, samo odemo na pripremu za instrukciju $I_l$.
\begin{equation*}
\label{d:61}\tag{$\delta$3i}
    \delta(s_i,\gamma):=(p_l,\gamma,-1)\text{, za sve $\gamma\in B^m$}
\end{equation*}
	Lako je vidjeti da je pozicija ulaskom u $p_{i'}$ doista veća ili jednaka $k$ --- recimo, primjenom~\eqref{d:54} pozicija će biti $k+c(\reg j)-1\ge k$, jer je u tom slučaju $c(\reg j)>0$ --- dakle, postigli smo $m$-reprezentaciju od $d$.
%Sada samo još treba na svako stanje oblika $p_{i'}$ "našarafiti" konstrukciju iz leme~\ref{lm:pi>si}. Pozicija u svakom od tih stanja je strogo veća od $k$, ako je na početku u stanju $s_i$ bila $k+2$; recimo primjenom~\eqref{d:53} će pozicija biti upravo $k+2>k$, a primjenom~\eqref{d:54} će biti $k+c(\reg j)$, što je veće od $k$ jer je u tom slučaju $c(\reg j)>0$.
\end{proof}

%Radi jednostavnosti, pretpostavit ćemo da za svaku instrukciju (i još za $i=n$) imamo sva tri stanja $s_i$, $p_i$ i $t_i$ --- iako će stanja $t_i$ biti povezana s ostalim stanjima samo ako je instrukcija s rednim brojem $i$ tipa $\dec$; inače će biti nedostupna i ni na koji način neće utjecati na $\mathcal T_0$-izračunavanje. Nećemo ih ni crtati na dijagramima (baš kao ni stanje greške $q_x$), ali tako će formalna definicija biti jednostavnija.

Tipove fragmenata iz dokaza možemo prikazati dijagramima:
\begin{itemize}
    \item[$\rhd$] $i.\;\incr j$\;:
\begin{tikzpicture}[baseline=(si)]
\node[state] (pi) {$p_i$};
\node[state, right of=pi] (si) {$s_i$};
\node[state, right of=si] (pi1) {$p_{\!i\!+\!1}$};
\draw
(pi) edge[loop left,dashed] node[left]{$B^m$} (pi)
(pi) edge[above] node{$\t\#$} (si)
(si) edge[loop below] node[above right,xshift=4]{\t\textbullet\,\texttt@$j$} (si)
(si) edge[dashed] node[above]{\t\textopenbullet:\t\textbullet\,\texttt@$j$} (pi1)
;
\end{tikzpicture}
\vspace{-1.5cm}
    \item[$\rhd$] $i.\;\decr jl$\;:
\begin{tikzpicture}[baseline=(si)]
\node[state] (pi) {$p_i$};
\node[state, right of=pi] (si) {$s_i$};
\node[state, right of=si] (ti) {$t_i$};
\node[state, above right=0.5 and 2 of ti] (pl) {$p_l$};
\node[state, below right=0.5 and 2 of ti] (pi1) {$p_{\!i\!+\!1}$};
\draw
(pi) edge[loop left,dashed] node[left]{$B^m$} (pi)
(pi) edge[above] node{$\t\#$} (si)
(si) edge[loop below] node[above right,xshift=4]{\t\textbullet\,\texttt@$j$} (si)
(si) edge[dashed] node[above]{\t\textopenbullet\,\texttt@$j$} (ti)
(ti) edge node[above]{\t\#} (pl)
(ti) edge[dashed] node[above,xshift=4]{\t\textbullet:\t\textopenbullet\,\texttt@$j$} (pi1)
;
\end{tikzpicture}
\vspace{-0.8cm}
    \item[$\rhd$] $i.\;\goto\;l$\;:\qquad
\begin{tikzpicture}[baseline=(si)]
\node[state] (pi) {$p_i$};
\node[state, right of=pi] (si) {$s_i$};
\node[state, right of=si] (pl) {$p_l$};
\draw
(pi) edge[loop left,dashed] node[left]{$B^m$} (pi)
(pi) edge[above] node{$\t\#$} (si)
(si) edge[dashed] node[above]{$B^m$} (pl)
;
\end{tikzpicture}
\end{itemize}

%\subsection{Prepoznavanje završne konfiguracije}
\begin{propozicija}[{name=[treći fragment transpiliranog stroja]}]\label{pp:faza3}
    Neka RAM-algoritam $P^1$ širine $m$ računa funkciju $\f f^1$.
	Za svaki $x\in\N$ označimo s $t_x$ bilo koju $m$-reprezentaciju početne RAM-konfiguracije s ulazom $x$.

    Postoji fragment Turingovog stroja koji za sve $x\in\N$:
\begin{itemize}
    \item ako je $x\in\dom{\f f}$, prevodi svaku $t_x$ u neku $m$-reprezentaciju završne konfiguracije;
    \item inače, počevši od bilo koje $t_x$, nikada ne stigne u stanje $p_n$.
\end{itemize}
\end{propozicija}
\begin{proof}
	Za svaki $x\in\N$, označimo sa $(c_l)_l$ $P$-izračunavanje s $x$ i indukcijom po $l$ dokažimo da dosad sastavljeni fragment Turingovog stroja dostigne $m$-reprezentaciju od $c_l$. Baza je primjena lema~\ref{lm:faza1} i~\ref{lm:faza2}, kako je ilustrirano u~\eqref{eq:+afaze}. Korak (ako $c_l$ nije završna) je primjena propozicije~\ref{prop:gadgets}. Dakle, ako $P$-izračunavanje stane u $l_0$ koraka, postići će $m$-reprezentaciju završne konfiguracije $c_{l_0}$, došavši u stanje $p_n$.

No vrijedi i obrat: pri simulaciji jednog RAM-prijelaza $c\leadsto d$, naš fragment posjećuje samo stanja $p_i$, $s_i$ i $t_i$ za $i=c(\textsc{pc})$ ili za $i=d(\textsc{pc})$. Dakle, jedini način da dođe u stanje $p_n$ je da doista neka RAM-konfiguracija u izračunavanju preslika \textsc{pc} u $n$, što znači da $P$-izračunavanje s $x$ stane.
\end{proof}

%Općenito, ovaj dio je na neki način najkompliciraniji (zbog broja stanja --- za tipični RAM-program ovaj fragment će biti višestruko veći od čitavog ostatka stroja $\mathcal T_0$; a i zbog činjenice da ne mora stati, odnosno doći u stanje $p_n$). S druge strane, možemo reći i da je najjednostavniji, jer ima vrlo uniformnu strukturu: fragmentići (u simulacijama obično se zovu \emph{gadgets}) za pojedine instrukcije su jednostavni i lijepo se slažu. Pogledajmo jednostavni primjer.

\begin{korolar}[{name=[prve tri faze transpiliranog stroja]}]\label{kor:faza3}
	Dosad izgrađeni Turingov stroj (sastavljen od prva tri fragmenta) dostigne stanje $p_n$ ako i samo ako je pokrenut s ulazom $w\in\dom{\varphi_0}$, i u tom trenutku na traci nakon $\t\#$ u tragu $0$ piše $\raisebox{-1pt}{\t\textbullet}^y\raisebox{-1pt}{\t\textopenbullet}\dotsm$, gdje je $y=\N\varphi_0(\kr w)=\kr{\varphi_0(w)}$.
\end{korolar}
\begin{proof}
    Primijenimo propoziciju~\ref{pp:faza3} na RAM-program $P_0$ ($f=\N\varphi_0$) i $x:=\kr w$. % i $k:=\dulj{w}$. %Štoviše, jer se $f$ računa na $x$ koji jest kod, dovoljno je zahtijevati $f\approx\N\varphi_0$.
\end{proof}

\begin{primjer}[{name=[treći fragment transpiliranog stroja]}]\label{pr:+a2}
U primjeru~\ref{pr:+a} smo vidjeli RAM-program $P_{\t a}$~\eqref{eq:+aprog} duljine $4$ i širine $2$, koji računa prateću funkciju dopisivanja \t a na kraj riječi, $\N\varphi_{\t a}(x):=2x+1$.

Treći fragment Turingovog stroja koji odgovara $P_{\t a}$ prikazan je dijagramom
\begin{equation}
\begin{tikzpicture}[baseline=(s3)]
\node[state]                 (p0) {$p_0$};
\node[state,right of=p0]     (s0) {$s_0$};
%\node[state,right of=s0]     (t0) {$t_0$};
\node[state,below=1.5 of p0] (p1) {$p_1$};
\node[state,below=1.5 of s0] (s1) {$s_1$};
\node[state,right of=s1] (t1) {$t_1$};
\node[state,below=1.5 of p1] (p2) {$p_2$};
\node[state,below=1.5 of s1] (s2) {$s_2$};
\node[state,below=1.5 of t1] (p4) {$p_4$};
\node[state,below=1.5 of p2] (p3) {$p_3$};
\node[state,below=1.5 of s2] (s3) {$s_3$};
%\node[state,below=1.5 of p4] (s4) {$s_4$};
%\node[state,below=1.5 of t3] (t4) {$t_4$};
%\node[state,below=1.5 of p4] (p5) {$p_5$};
%\node[state,below=1.5 of s4] (s5) {$s_5$};
%\node[state,below=1.5 of t4] (t5) {$t_5$};
\node[left=4 of p0,align=left] {$0.\;\incr0\;:$};
\node[left=4 of p1,align=left] {$1.\;\decr14\;:$};
\node[left=4 of p2,align=left] {$2.\;\incr0\;:$};
\node[left=4 of p3,align=left] {$3.\;\goto\;0\;:$};
%\node[left=4 of p4,align=left] {$4.\;\incr0\;:$};
\draw
(p0) edge[loop left,dashed] node[left] {$B^2$} (p0)
(p1) edge[loop left,dashed] node[left] {$B^2$} (p1)
(p2) edge[loop left,dashed] node[left] {$B^2$} (p2)
(p3) edge[loop left,dashed] node[left] {$B^2$} (p3)
%(p4) edge[loop right,dashed] node[right] {$B^2$} (p4)
(p0) edge node[above] {$\t\#$} (s0)
(p1) edge node[above] {$\t\#$} (s1)
(p2) edge node[above] {$\t\#$} (s2)
(p3) edge node[above] {$\t\#$} (s3)
%(p4) edge node[right] {$\t\#$} (s4)
(s0) edge[loop right] node[right] {\t\textbullet\,\texttt@$0$} (s0)
(s1) edge[loop above] node[above] {\t\textbullet\,\texttt@$1$} (s1)
(s2) edge[loop right] node[right] {\t\textbullet\,\texttt@$0$} (s2)
(s1) edge[dashed] node[above] {\t\textopenbullet\,\texttt@$1$} (t1)
(t1) edge[dashed] node[above left,yshift=-4] {\t\textbullet:\t\textopenbullet\,\texttt@$1$} (p2)
(s0) edge[dashed] node[above left,yshift=-3] {\t\textopenbullet:\t\textbullet\,\texttt@$0$} (p1)
(s2) edge[dashed] node[above left,yshift=-3] {\t\textopenbullet:\t\textbullet\,\texttt@$0$} (p3)
(s3) edge[dashed,out=160,in=220,looseness=2] node[right,pos=0.6] {$B^2$} (p0)
%(t1) edge[out=267,in=20,looseness=1.5] node[right] {\t\#} (p4)
(t1) edge node[right] {\t\#} (p4)
;
\end{tikzpicture}\text{\hspace{-2.7cm}%
%.\,\qedhere%
}
\end{equation}
$P_{\t a}$-izračunavanje s $\kr{\t{ab}}=4$ je ($c$ skraćeno zapisujemo kao $c(\reg0)c(\reg1)c(\textsc{pc})$)
\begin{multline}
        040\leadsto
	141\leadsto
	132\leadsto
	233\leadsto
	230\leadsto
	331\leadsto
        322\leadsto
	423\leadsto
	420\leadsto
	521\leadsto
	512\leadsto{}\\{}\leadsto
	613\leadsto
        610\leadsto
	711\leadsto
	702\leadsto
	803\leadsto
	800\leadsto
	901\leadsto
        904\lcirclearrowleft
%\begin{array}{r|cccccccccccccccccccc}
%i&0&1&2&3&4&5&6&7&8&9&10&11&12&13&14&15&16&17&18&\dotsm\\\hline
%c_i(\reg0)&
%0&0&1&2&2&2&3&4&4&4&5&6&6&6&7&8&8&8&9&9\\
%c_i(\reg1)&
%4&3&3&3&3&2&2&2&2&1&1&1&1&0&0&0&0&0&0&0\\
%c_i(\textsc{pc})&
%0&1&2&3&0&1&2&3&0&1&2&3&0&1&2&3&0&4&5&5
%\end{array}
\end{multline}
	s izlaznim podatkom $9=2\cdot\mspace{1mu}4+1$, i odgovara mu šetnja kroz $2$-reprezentacije %Turing-konfiguracije %(ne pišemo, svima zajednički, početak \t{\renewcommand{\arraystretch}{0.1}\$\boo\boo\#})
    %$P_{\t a}$-izračunavanju s $\kr{\t{ab}}=4$, s izlaznim podatkom $9=2\cdot4+1$, po preslikavanju $Turing_{224}$ odgovara šetnja kroz Turing-konfiguracije
\begin{multline}\label{eq:+a2}
\renewcommand{\arraystretch}{0.1}
\T{\$\boo\boo\#}{\mund{p_{\!0}}}{\boi}{\boi\boi\boi}\leadsto^*
\T{\$\boo\boo}{\mund{p_{\!1}}}{\vphantom\boo\#}{\bii\boi\boi\boi}\leadsto^*
\T{\$\boo\boo\#\bii\boi}{\mund{p_{\!2}}}{\boi}{}\leadsto^*
\T{\$\boo\boo\#}{\mund{p_{\!3}}}{\bii}{\bii\boi}\leadsto^*
\T{\$\boo\boo}{\mund{p_{\!0}}}{\vphantom\boo\#}{\bii\bii\boi}\leadsto^*
\T{\$\boo\boo\#\bii}{\mund{p_{\!1}}}{\bii}{\bii}\leadsto^*
{}\\{}\leadsto^*
\renewcommand{\arraystretch}{0.1}
\T{\$\boo\boo\#\bii}{\mund{p_{\!2}}}{\bii}{\bio}\leadsto^*
\T{\$\boo\boo\#\bii\bii}{\mund{p_{\!3}}}{\bio}{\bio}\leadsto^*
\T{\$\boo\boo}{\mund{p_{\!0}}}{\vphantom\boo\#}{\bii\bii\bio\bio}\leadsto^*
\T{\$\boo\boo\#\bii\bii\bio}{\mund{p_{\!1}}}{\bio}{\bio}\leadsto^*
\T{\$\boo\boo\#}{\mund{p_{\!2}}}{\bii}{\bio\bio\bio\bio}\leadsto^*
{}\\{}\leadsto^*
\renewcommand{\arraystretch}{0.1}
\T{\$\boo\boo\#\bii\bio\bio\bio}{\mund{p_{\!3}}}{\bio}{\bio}\leadsto^*
\T{\$\boo\boo}{\mund{p_{\!0}}}{\vphantom\boo\#}{\bii\bio\bio\bio\bio\bio}\leadsto^*
\T{\$\boo\boo\#\bii\bio\bio\bio\bio}{\mund{p_{\!1}}}{\bio}{\bio}\leadsto^*
\T{\$\boo\boo}{\mund{p_{\!2}}}{\vphantom\boo\#}{\bio\bio\bio\bio\bio\bio\bio}\leadsto^*
{}\\{}\leadsto^*
\renewcommand{\arraystretch}{0.1}
\T{\$\boo\boo\#\bio\bio\bio\bio\bio\bio}{\mund{p_{\!3}}}{\bio}{\bio}\leadsto^*
\T{\$\boo\boo}{\mund{p_{\!0}}}{\vphantom\boo\#}{\bio\bio\bio\bio\bio\bio\bio\bio}\leadsto^*
\T{\$\boo\boo\#\bio\bio\bio\bio\bio\bio\bio}{\mund{p_{\!1}}}{\bio}{\bio}\leadsto^*
\T{\$\boo\boo\#}{\mund{p_{\!4}}}{\bio}{\bio\bio\bio\bio\bio\bio\bio\bio}
\,\text,
\end{multline}
koja predstavlja prirodni nastavak one u~\eqref{eq:+afaze}.
\end{primjer}

\begin{napomena}[{name=[zaustavljanje transpiliranog stroja ovisi samo o trećoj fazi]}]\label{nap:snstane}
Ostatak Turingovog stroja koji konstruiramo vezat će se s dosad izgrađenim dijelom isključivo kroz stanje $p_n$, a njegov rad će uvijek stati, pa ćemo na kraju moći zaključiti da čitav Turingov stroj stane ako i samo ako dođe u stanje $p_n$, odnosno ako i samo ako je ulazna riječ u domeni od $\varphi_0$.
\end{napomena}

%Time smo riješili pitanje domene, i potrebno je još samo natrag dekodirati $y$ (ako postoji) u riječ $v:=\varphi_0(w)$ i postaviti je na početak trake.

\subsection{Dekodiranje izlazne vrijednosti}

Ovaj će postupak biti sličan onome u drugoj fazi, samo u obrnutom smjeru. Imamo broj $y$ zapisan unarno (pomaknuta baza $1$) u tragu $0$, a trebamo ga zapisati u pomaknutoj bazi $b'$ --- skidajući po jedan \t\textbullet\ iz traga $0$ i inkrementirajući riječ iz $\Sigma_0^*$ svaki put. Ipak, nekoliko tehničkih detalja čine ovaj postupak kompliciranijim.

Prvo, umjesto pisanja konstantnog znaka $r_1\in B^m$, morat ćemo čitati bilo koji znak koji u tragu $0$ ima \t\textbullet. Srećom, tehniku za to smo napravili (definicija~\ref{def:trag}).

Drugo i važnije, kod dekrementiranja smo znali da riječ neće postati dulja, pa smo prazninama skraćivali broj koliko je već potrebno, dok ga nismo dekrementirali sasvim do nule. Kod inkrementiranja nemamo tu garanciju: može se dogoditi da mnogo puta moramo dodati po jednu znamenku slijeva (recimo, za $b'=5$, kad inkrementiramo $555$ u $1111$).

Zato je dobro riječ graditi unatrag, samo moramo biti sigurni da smo zauzeli dovoljno trake za to (da ne udarimo u lijevi rub, odnosno graničnik \t\$, prilikom dodavanja nove znamenke slijeva). Posljedica toga je da ne možemo učiniti intuitivno očitu stvar i upotrijebiti samo dio trake lijevo od separatora \t\#\ ---  jer je taj dio ograničen duljinom ulazne riječi, a izlazna riječ može biti bitno dulja od ulazne.

Spasit ćemo se tako što ćemo za rekonstrukciju izlazne riječi upotrijebiti dio \emph{desno} od separatora, unutar kojeg smo simulirali $P_0$-izračunavanje s $w$. Taj dio će sigurno biti dovoljno dugačak, jer sadrži barem $y=\kr{\varphi_0(w)}$ znakova \t\textbullet\ u tragu $0$.

\begin{lema}[{name=[duljina riječi nije veća od koda riječi]}]\label{lm:dulj<=kr}
Za svaku riječ $u$ vrijedi $\dulj u\le\kr u$.
\end{lema}
\begin{proof}
Označimo s $b'$ broj znakova abecede $\Sigma$ nad kojom je riječ $w$.
Svaki pribrojnik u sumi~\eqref{eq:kodNSz} je pozitivan, jer je svaka znamenka $\N\Sigma(\alpha_i)\in[1\dd b']$ te je $b'>0$ zbog $\Sigma\ne\emptyset$. Dakle, svaki pribrojnik je barem $1$, pa je suma veća ili jednaka broju pribrojnika. No suma je po definiciji $\kr w$, a broj pribrojnika je upravo $n=\dulj w$.

Za $b'=1$ vrijedi jednakost (jer je jedina moguća znamenka $1$).
\end{proof}

Štoviše, lema~\ref{lm:dulj<=kr} pokazuje da riječ možemo dekodirati korak po korak: u svakom trenutku će nam u tragu $0$ ostati $y'$ znakova \t\textbullet, a od desnog kraja će biti napisana riječ $u'$ čiji je kod $\kr{u'}=y-y'$, pa će sigurno stati u prostor od $y-y'$ ćelija.

\begin{lema}[{name=[četvrti fragment transpiliranog stroja]}]\label{lm:faza4}
Postoji fragment Turingovog stroja koji prevodi reprezentaciju završne konfiguracije $(p_n,\dulj w+2,\t\$\bl^{\dulj w}\t\#t\bl\ldots)$, gdje $t$ u tragu $0$ ima oblik $\t\textbullet^y\t\textopenbullet^{\dulj t-y}$, u konfiguraciju $(q_3,\dulj w+1,\t\$\bl^{\dulj w}\t\#uv\t\$\bl\ldots)$, gdje je $\kr v=y$ te $u\in(B^m)^{\dulj t-\dulj v}$.
\end{lema}

\begin{proof}
Kao što rekosmo, dokaz je sličan dokazu leme~\ref{lm:faza2}, samo u suprotnom smjeru. Prvo odemo do kraja upotrijebljenog (nepraznog) dijela trake~\eqref{d:71} i tamo zapišemo krajnji graničnik~\eqref{d:72}. Zatim se krećemo ulijevo kroz znakove \t\textopenbullet\ u tragu $0$~\eqref{d:73} dok ne dođemo do \t\textbullet, koji maknemo~\eqref{d:74} i prenosimo ga desno kroz "znamenke" i "stupce bitova"~\eqref{d:75}, dok ne dođemo do krajnjeg graničnika~\eqref{d:76}. Tada se opet krećemo ulijevo inkrementirajući riječ: zamjenjujući "devetke jedinicama"~\eqref{d:77} te inkrementirajući prvu "ne-devetku"~\eqref{d:78}, ili ako takve nema, dodajući "jedinicu" slijeva~\eqref{d:79}. Zatim prođemo kroz ostale znamenke do "praznog prostora"~\eqref{d:70} i opet izvršavamo cijeli postupak dok ne dođemo do separatora \t\#.
Argument zašto to funkcionira je praktički isti kao u dokazu leme~\ref{lm:faza2}. %Jedina bitna razlika je što je tamo~\eqref{dia:T2} "radni ciklus" imao dva stanja ($q_0$ i $q_1$), a ovdje ih ima tri ($q_3$, $q_4$ i $q_5$). To je posljedica toga da inkrementiramo "s krive strane", odnosno prenosimo kružiće slijeva nadesno, a kod kretanja zdesna nalijevo moramo pamtiti tražimo li tek znamenku koju ćemo inkrementirati (da bismo zamijenili "devetke jedinicama"), ili smo to već učinili (pa ostavljamo "devetke" na miru).
\end{proof}
\noindent\begin{gather*}
\SwapAboveDisplaySkip
	\label{d:71}\tag{$\delta$4a}\delta(p_n,\gamma):=(p_n,\gamma,1)\text{, za sve } \gamma\in \{\t\#\}\cup B^m\setminus\{\bl\}\\
\label{d:72}\tag{$\delta$4b}\delta(p_n,\bl):=(q_3,\t\$,-1)\\
\label{d:73}\tag{$\delta$4c}\delta_{(0)}(q_3,\t\textopenbullet):=(q_3,\t\textopenbullet,-1)\\
\label{d:74}\tag{$\delta$4d}\delta_{(0)}(q_3,\t\textbullet):=(q_4,\t\textopenbullet,1)\\
\label{d:75}\tag{$\delta$4e}\delta(q_4,\alpha):=(q_4,\alpha,1)\text{, za sve }\alpha\in\Sigma_0\dcup B^m\\
\label{d:76}\tag{$\delta$4f}\delta(q_4,\t\$):=(q_5,\t\$,-1)\\
\label{d:77}\tag{$\delta$4g}\delta(q_5,\alpha_{b'}):=(q_5,\alpha_1,-1)\\
\label{d:78}\tag{$\delta$4h}\delta(q_5,\alpha_t):=(q_3,\alpha_{t+1},-1)\text{, za sve }t\in[1\dd b'\rangle\\
\label{d:79}\tag{$\delta$4i}\delta(q_5,\gamma):=(q_3,\alpha_1,-1)\text{, za sve }\gamma\in B^m\\
\label{d:70}\tag{$\delta$4j}\delta(q_3,\alpha):=(q_3,\alpha,-1)\text{, za sve }\alpha\in\Sigma_0
\end{gather*}

\begin{primjer}[{name=[četvrti fragment transpiliranog stroja]}]
Za $\Sigma_0:=\{\t a,\t b,\t c,\t d\}$, četvrti fragment možemo prikazati dijagramom
\begin{equation}
\begin{tikzpicture}[baseline=(sn)]\label{dia:T4}
%\clip (-1,-2.53) rectangle (8.5,2.16);
\node[state] (sn) {$p_n$};
\node[state,right of=sn] (q5) {$q_3$};
\node[state,right of=q5] (q7) {$q_5$};
\node[state,above=1 of q7] (q6) {$q_4$};
\draw
(sn) edge[loop above] node[above,align=center] {\t\#
\\[-1mm]
$B^m_+$} (sn)
(sn) edge[dashed] node[above] {\bl:\t\$} (q5)
(q5) edge[loop above,dashed] node[above,xshift=-9,align=right] {\t\textopenbullet\,\texttt@$0$
\\[-1mm]
$\Sigma_0$} (q5)
(q5) edge node[above,xshift=-9,yshift=1] {\t\textbullet:\t\textopenbullet\,\texttt@$0$} (q6)
(q6) edge[loop right] node[right,align=left] {$B^m$
\\[-1mm]
$\Sigma_0$} (q6)
(q6) edge[dashed] node[right] {\t\$} (q7)
(q7) edge[dashed,loop right] node[right] {\t d:\t a} (q7)
(q7) edge[dashed] node[above,xshift=3] {\t a:\t b, \t b:\t c} node[below] {\t c:\t d, $B^m$:\t a} (q5)
;
\end{tikzpicture}\text{\;.\qedhere}
\end{equation}
\end{primjer}

Zbog $\kr v=y=\kr{\varphi_0(w)}$ i propozicije~\ref{pp:bijkr} zapravo vrijedi $v=\varphi_0(w)$, odnosno (ako uopće dođemo u situaciju da možemo primijeniti lemu~\ref{lm:faza4}) već imamo izlaznu riječ na traci. Još je treba premjestiti na početak i obrisati sve ostalo s trake.

% \subsection{Pospremanje za sobom}

\begin{lema}[{name=[peti fragment transpiliranog stroja]}]\label{lm:faza5}
Postoji fragment Turingovog stroja koji prevodi konfiguraciju\newline $(q_3,\dulj w+1,\t\$\bl^{\dulj w}\t\#uv\t\$\bl\ldots)$, gdje je $u\in(B^m)^*$ te $v,w\in\Sigma_0^*$, u konfiguraciju $(l_{\t\$},\dulj v,v\bl\ldots)$.
\end{lema}
\begin{proof}
Prvi korak je brisanje svih ostataka izračunavanja i separatora, tako da na traci ostanu jedino izlazna riječ $v$ i dva graničnika --- početni i krajnji~\eqref{d:81}. Time dobivamo konfiguraciju $(q_6,l+1,\t\$\bl^l\,v\t\$\bl\ldots)$, gdje je $l:=\dulj w+\dulj u+1$.

Drugi korak je uzimanje jednog po jednog znaka od $v$ i njegov prijenos na početak trake~\eqref{d:83} --- s tim da će prvi znak od $v$ naići na početni graničnik i prebrisati ga~\eqref{d:86}, a svaki sljedeći znak će naići na onaj prije njega koji je već prenesen~\eqref{d:85} i zapisati se desno od njega~\eqref{d:86}. Time dobijemo konfiguraciju $(q_6,l+\dulj v+1,v\bl^{l+1}\t\$\bl\ldots)$ ako je $v\ne\varepsilon$. Za $v=\varepsilon$, ovog koraka nema, odnosno odmah smo u konfiguraciji $(q_6,l+1,\t\$\bl^l\t\$\bl\ldots)$.

Treći korak počinje čitanjem i brisanjem završnog graničnika \t\$ i nastavlja prolaskom kroz praznine ulijevo~\eqref{d:83}, sve dok ne udarimo u zadnji znak od $v$ ako je $v\ne\varepsilon$~\eqref{d:85}, odnosno u početni graničnik (koji tada treba obrisati, ostavivši traku sasvim praznom) u slučaju $v=\varepsilon$~\eqref{d:89}.

    Za nepraznu riječ $v$, završna će pozicija biti neposredno iza zadnjeg znaka od $v$, dakle $\dulj{v}$. Za praznu riječ, prijelaz~\eqref{d:89} će se dogoditi na lijevom rubu trake, pokušati pomak lijevo i ostati na istoj poziciji $0=\dulj{\varepsilon}=\dulj{v}$. U svakom slučaju dobili smo stanje $l_{\t\$}$, traku $v\bl\ldots$ i poziciju $\dulj{v}$.
\end{proof}

\noindent\begin{gather*}
    \SwapAboveDisplaySkip
\label{d:81}\tag{$\delta$5a}\delta(q_3,\t\#):=\delta(q_6,\gamma):=(q_6,\bl,1)\text{, za sve }\gamma\in B^m\\
    \label{d:83}\tag{$\delta$5b}\delta(q_6,\alpha):=\delta(m_\alpha,\bl):=(m_\alpha,\bl,-1)\text{, za sve }\alpha\in\Sigma_0\dcup\{\t\$\}\\
    \label{d:85}\tag{$\delta$5c}\delta(m_\alpha,\beta):=(l_\alpha,\beta,1)\text{, za sve }\alpha\in\Sigma_0\dcup\{\t\$\},\beta\in\Sigma_0\\
\label{d:86}\tag{$\delta$5d}\delta(m_\alpha,\t\$):=\delta(l_\alpha,\bl):=(q_6,\alpha,1)\text{, za sve }\alpha\in\Sigma_0\\
    %\label{d:87}\tag{$\delta$5e}\delta(q_6,\t\$):=\delta(m_{\t\$},\bl):=(m_{\t\$},\bl,-1)\\
    %\label{d:88}\delta(m_{\t\$},\alpha):=(l_{\t\$},\alpha,1)\text{, za sve }\alpha\in\Sigma_0\\
    \label{d:89}\tag{$\delta$5e}\delta(m_{\t\$},\t\$):=(l_{\t\$},\bl,-1)
\end{gather*}

\begin{primjer}[{name=[peti fragment transpiliranog stroja]}]
Za $\Sigma_0:=\{\t a,\t b\}$, peti fragment možemo prikazati dijagramom
\begin{equation}\label{dia:T5}
\begin{tikzpicture}[baseline=(q8)]
%\clip (-1,-2.53) rectangle (8.5,2.16);
\node[state] (q5) {$q_3$};
\node[state,below right of=q5] (q8) {$q_6$};
\node[state,left of=q8] (ma) {$m_{\t a}$};
\node[state,right of=q8] (mb) {$m_{\t b}$};
\node[state,below left of=q8] (la) {$l_{\t a}$};
\node[state,below right of=q8] (lb) {$l_{\t b}$};
    \node[state,above right of=q8] (q9) {$m_{\t\$}$};
    \node[state,accepting,right of=q9] (qz) {$l_{\t\$}$};
\draw
(q5) edge node[above,yshift=5] {\t\#:\bl} (q8)
(q8) edge[loop above] node[above,yshift=-3] {$B^m$:\bl} (q8)
(q8) edge[dashed,bend left] node[above] {\t a:\bl} (ma)
(ma) edge[bend left] node[below,yshift=2] {\t\$:\t a} (q8)
(q8) edge[dashed,bend right] node[above] {\t b:\bl} (mb)
(mb) edge[bend right] node[below,yshift=2] {\t\$:\t b} (q8)
(ma) edge[dashed,loop left] node[left] {\bl} (ma)
(mb) edge[dashed,loop right] node[right] {\bl} (mb)
(ma) edge node[left,xshift=2] {$\Sigma_0$} (la)
(mb) edge node[right] {$\Sigma_0$} (lb)
(la) edge node[below,xshift=3] {\bl:\t a} (q8)
(lb) edge node[below,xshift=-3] {\bl:\t b} (q8)
(q8) edge[dashed] node[right,xshift=1] {\t\$:\bl} (q9)
(q9) edge[dashed,loop left] node[left] {\bl} (q9)
(q9) edge[bend left] node[below] {$\Sigma_0$} (qz)
(q9) edge[bend right,dashed] node[above] {\t\$:\bl} (qz)
;
\end{tikzpicture}\text.
\end{equation}
Za općenitu abecedu $\Sigma_0$, umjesto dva "viseća trokuta" ispod $q_6$, bit će ih $b'$:\newline po jedan $\triangle q_6 m_\alpha l_\alpha$ za svaki znak $\alpha\in\Sigma_0$.
\end{primjer}

%Stigli smo do završnog stanja $l_{\t\$}$. Time smo napokon završili konstrukciju Turingovog stroja $\mathcal T_0$ i sve je spremno za dokaz "obrata" teorema~\ref{tm:tikp}.

\subsection{Objedinjavanje}

\begin{teorem}[{name=[Turing-izračunljivost parcijalno rekurzivnih jezičnih funkcija]}]\label{tm:krit}
Neka je $\Sigma_0$ abeceda, $\N\Sigma_0$ njeno kodiranje te $\varphi_0$ funkcija nad njom.

Ako je $\N\varphi_0\in\mathscr Comp_1$, tada je $\varphi_0$ Turing-izračunljiva.
\end{teorem}
\begin{proof}
Po pretpostavci postoji RAM-program $P_0$ koji računa $\N\varphi_0$.
	
	Označimo $n:=n_{P_0}$ i $m:=m_{P_0^1}$. Tvrdimo da tada funkciju $\varphi_0$ računa Turingov stroj
\begin{align}
    &\qquad\mathcal T_0:=(Q,\Sigma_0,\Gamma,\bl,\delta,n_{\t\$},l_{\t\$})\text,
\shortintertext{s komponentama}
    Q&:=\bdcup_{i\le6}\{q_i\}\dcup\bdcup_{i\le n}\{p_i,s_i,t_i\}\dcup\bdcup_{\mathclap{\alpha\in\Sigma_0\dcup\{\t\$\}}}\,\{n_\alpha,m_\alpha,l_\alpha\}\dcup\{n_{\bl},q_x\}\text,\\
    \Gamma&:=\Sigma_0\dcup\{\t\$,\t\#\}\dcup B^m\text{, gdje je }B^m:=\prod\nolimits_{i<m}\{\t\textopenbullet,\t\textbullet\}\text,\\
    \bl&:=(\t\textopenbullet,\t\textopenbullet,\dotsc,\t\textopenbullet)^\intercal\in B^m\text{ (zapisana kao stupac),}
\end{align}
    i funkcijom prijelaza $\delta$ zadanom jednakostima ($\delta\,\cdots$), %\eqref{d:11}--\eqref{d:12}, \eqref{d:21}--\eqref{d:28}, \eqref{d:31}--\eqref{d:61}, \eqref{d:71}--\eqref{d:70} i \eqref{d:81}--\eqref{d:89},
    tako da se svi parovi $(q,\alpha)\in(Q\setminus\{l_{\t\$}\})\times\Gamma$ nenavedeni u tim jednakostima preslikavaju u $(q_x,\alpha,-1)$.

Doista, neka je $w\in\Sigma_0^*$ proizvoljna riječ i označimo s $k:=\dulj{w}$ njenu duljinu.
	
	Promotrimo prvo slučaj kada je $w\in\dom{\varphi_0}$.

    Tada će početnu konfiguraciju $(n_{\t\$},0,w\bl\ldots)$ prvi fragment od $\mathcal T_0$ (lema~\ref{lm:faza1}) prebaciti u $(q_0,k,\t\$w\t\#\bl\ldots)$, a nju će pak drugi fragment prebaciti u $(p_0,k+2,\t\$\bl^k\,\t\#r_1^{\kr w}\bl\ldots)$, gdje je $r_1=(\t\textopenbullet,\t\textbullet,\t\textopenbullet,\dotsc,\t\textopenbullet)^\intercal$ (lema~\ref{lm:faza2}). Kako je $w\in\dom{\varphi_0}$, vrijedi $x:=\kr w\in\dom{\N\varphi_0}$, pa $P_0$-izračunavanje s $x$ stane, s izlaznim podatkom $y:=\N\varphi_0(x)=\kr{\varphi_0(w)}$. Prema korolaru~\ref{kor:faza3}, $\mathcal T_0$-izračunavanje s $w$ (treća faza) će zato doći u konfiguraciju $(p_n,k+2,\t\$\bl^k\,\t\#t\bl\ldots)$, gdje $t$ u tragu $0$ ima oblik $\t\textbullet^y\t\textopenbullet^{\dulj t-y}$. Tu će pak konfiguraciju četvrti fragment (lema~\ref{lm:faza4}) prebaciti u $(q_3,k+1,\t\$\bl^k\,\t\#uv\t\$\bl\ldots)$, gdje je $v=\varphi_0(w)$ izlazni podatak te $u\in(B^m)^{\dulj t-\dulj{v}}$. Peti će fragment (lema~\ref{lm:faza5}) napokon prebaciti tu konfiguraciju u završnu konfiguraciju $(l_{\t\$},\dulj{v},v\bl\ldots)$, s izlaznim podatkom na traci.

	Ako pak $w\notin\dom{\varphi_0}$, tada će se prve dvije faze izvršiti jednako, ali iz $x\notin\dom{\N\varphi_0}$ po definiciji~\ref{def:compute} slijedi da $P_0$-izračunavanje s $x$ nikad neće stati, pa po korolaru~\ref{kor:faza3} niti $\mathcal T_0$-izračunavanje neće doći do $p_n$. Prema napomeni~\ref{nap:snstane}, to znači da ono nikad neće stati (doći do $l_{\t\$}$), jer jedini put od $n_{\t\$}$ do $l_{\t\$}$ vodi kroz stanje $p_n$.
\end{proof}

Uočimo sličnost upravo provedenog dokaza s dokazom propozicije~\ref{prop:semfmacro} (semantika funkcijskog makroa). Kao što smo tamo omogućili makro-stroju da u izoliranoj okolini izvrši neki konkretni RAM-program, tako smo to ovdje omogućili Turingovom stroju. Kao i u makro-programu iz definicije~\ref{def:funmakro}, postupak se sastoji od pet faza: "otvaranje okvira" pomicanjem udesno, prijenos (kodiranje) argumenata u simulaciju (namještanje početne konfiguracije), sama simulacija (izvršavanje pojedinih instrukcija RAM-programa) te, ako ona stane, prijenos (dekodiranje) njene povratne vrijednosti iz završne konfiguracije te "zatvaranje okvira" (pospremanje nereda koji smo napravili).

Prikažimo postupak Turing-iz\-ra\-ču\-na\-va\-nja kroz faze (sa $z$ je označen najveći broj u ikojem relevantnom registru od $\mathcal S_0$ u završnoj konfiguraciji $P_0$-izračunavanja s $\kr w$):
\begin{equation}
\begin{array}{l|ccl}
                             &\text{stanje}&\text{pozicija}&\text{traka}\\\hline
    \text{početna konfiguracija} & n_{\t\$} & 0         & w\bl\ldots \\
\text{nakon prve faze}       & q_0 & \dulj w   & \t\$w\t\#\bl\ldots \\
\text{nakon druge faze}      & p_0 & \dulj w+2 & \t\$\bl^{\dulj w}\t\#r_1^{\kr w}\bl\ldots \\
\text{nakon treće faze}      & p_n & \dulj w+2 & \renewcommand{\arraystretch}{0.1}
    \t\$\bl^{\dulj w}\t\#\begin{array}{@{}c@{}}\t\textbullet\\?\end{array}^{\kr{v}}?^{z-\kr{v}}\bl\ldots\\
\text{nakon četvrte faze}    & q_3 & \dulj w+1 &
\renewcommand{\arraystretch}{0.1}
\t\$\bl^{\dulj w}\t\#\begin{array}{@{}c@{}}\t\textopenbullet\\?\end{array}^{z-\dulj v}v\t\$\bl\ldots\\
\text{prvi korak pete faze} & q_6 & \dulj w+2+z-\dulj v       & \t\$\bl^{\dulj w+1+z-\dulj v}v\t\$\bl\ldots\\
\text{drugi korak pete faze}& q_6 & \dulj w+2+z & v\bl^{\dulj w+2+z-\dulj v}\t\$\bl\ldots\\
    \text{završna konfiguracija} & l_{\t\$} & \dulj v   & v\bl\ldots
\end{array}
\end{equation}

\begin{primjer}[{name=[transpilirani stroj dodaje znak na kraj riječi]}]\label{pr:+a3}
Primjer~\ref{pr:+a} sad možemo pratiti kroz sve faze:
\begin{align}
	\text{početna konfiguracija}&&&\T{}{n_{\t\$}}{a}{b}\\[-1mm]
	\text{uokvirivanje ulaza}&&{}\leadsto^*{}&\T{\$a}{q_0}{b}{\#}\\[-1mm]
	\text{kodiranje ulaza: }\kr{\t{ab}}=(12)_2=4&&\leadsto^*{}&
\renewcommand{\arraystretch}{0.1}
	\T{\$\bl\bl\#}{\mund{p_0}}{\boi}{\boi\boi\boi}\\[-1mm]
	\text{računanje }2\cdot4+1=9\text{; raspisano u~\eqref{eq:+a2}}&&{}\leadsto^*{}&
\renewcommand{\arraystretch}{0.1}
	\T{\$\bl\bl\#}{\mund{p_4}}{\bio}{\bio\bio\bio\bio\bio\bio\bio\bio}\\[-1mm]
	\text{dekodiranje izlaza: }9=(121)_2=\kr{\t{aba}}&&{}\leadsto^*{}&
\renewcommand{\arraystretch}{0.1}
	\T{\$\bl\bl}{q_3}{\#}{\bl\bl\bl\bl\bl\bl aba\$}\\[-1mm]
%\text{uokvirivanje izlaza}&&{}\leadsto^*{}&
%\renewcommand{\arraystretch}{0.1}
	%\T{\$\bl\bl\bl\bl\bl\bl\bl\bl\bl}{q_6}{a}{ba\$}\\[-1mm]
\text{premještanje izlaza}&&{}\leadsto^*{}&
\renewcommand{\arraystretch}{0.1}
	\T{aba\bl\bl\bl\bl\bl\bl\bl\bl\bl\bl}{q_6}{\$}{}\\[-1mm]
\text{završna konfiguracija}&&{}\leadsto^*{}&
    \T{aba}{l_{\t\$}}{\bl}{}\lcirclearrowleft\text.
\end{align}

Dodati \t a na kraj riječi Turingov stroj očito može daleko jednostavnije --- ali ovaj primjer zapravo pokazuje kako može računati bilo koju funkciju koju RAM-stroj može računati. Pravo računanje odvija se u primjeru~\ref{pr:+a2} --- ovo je samo hrpa "papirologije" prije i poslije, koju treba riješiti da bi taj primjer bio koristan.
\end{primjer}

Ukratko, dokazali smo da je za \emph{jezične} funkcije izračunljivost u jezičnom modelu jednako snažna kao izračunljivost njihovih pratećih funkcija u brojevnom modelu. Time smo napokon opravdali neovisnost o korištenom \emph{encodingu} i riješili "filozofski problem" s početka točke~\ref{sec:kSigma}.

\begin{korolar}[{name=[neovisnost izračunljivosti jezične funkcije o kodiranju abecede]}]\label{kor:ikojiNSigma}
Neka je $\Sigma$ abeceda te $\varphi$ jezična funkcija nad njom.

	Tada parcijalna rekurzivnost funkcije $\N\varphi$ ne ovisi o izboru kodiranja $\N\Sigma$.
\end{korolar}
\begin{proof}
Neka su $\N\Sigma$ i $\N'\Sigma$ dva kodiranja $\Sigma$. Ako je $\N\varphi_0$ parcijalno rekurzivna, tada je po teoremu~\ref{tm:pir} RAM-izračunljiva. Tada je po teoremu~\ref{tm:krit} (primijenjenom na kodiranje $\N\Sigma$) $\varphi$ Turing-izračunljiva, a onda je po teoremu~\ref{tm:tikp} (primijenjenom na kodiranje $\N'\Sigma$) $\N'\varphi$ parcijalno rekurzivna. Zamjenom $\N\Sigma$ i $\N'\Sigma$ se dokaže i drugi smjer, dakle $\N\varphi$ je parcijalno rekurzivna ako i samo ako je $\N'\varphi$ parcijalno rekurzivna.
\end{proof}

%\section{Turing-izračunljivost brojevnih funkcija}
\section{Izračunljivost jezikâ}\label{sec:Todl}

% Ivanov prijedlog: koristiti zareze umjesto / za separator.
% Nisam siguran kako to sklopiti s nulama u nizu ali možda nije strašno.

Sve dosad napravljeno može se iskazati u jednom vrlo općenitom obliku.

\begin{propozicija}[{name=[izomorfizam skupova $\TComp$ i $\mathscr Comp_1$]}]\label{pp:trackbij}
Neka je $\Sigma$ proizvoljna abeceda te $\N\Sigma$ proizvoljno njeno kodiranje. Tada je $\varphi\mapsto\N\varphi$ bijekcija između skupa $\TComp$ svih Turing-izračunljivih jezičnih funkcija nad $\Sigma$, i skupa $\mathscr Comp_1$ svih jednomjesnih RAM-izračunljivih funkcija.
\end{propozicija}

\begin{proof}
Teorem~\ref{tm:tikp} (zajedno s teoremom~\ref{tm:pir}) kaže da za svaku $\varphi\in\TComp$ vrijedi $\N\varphi\in\mathscr Comp_1$. Dakle preslikavanje doista "ide kamo treba".

Za injektivnost, neka su $\varphi_1,\varphi_2\in\TComp$ različite. Ako imaju različite domene, bez smanjenja općenitosti možemo pretpostaviti da postoji $w\in\dom{\varphi_1}\!\!\setminus\dom{\varphi_2}$. Tada je $\kr w\in\dom{\N\varphi_1}$, ali isto tako $\kr w\notin\dom{\N\varphi_2}$ jer je kodiranje riječi injekcija, pa $\kr w$ ne može biti jednak nijednom $\kr{w'}$ za $w'\in\dom{\varphi_2}$. To znači da prateće funkcije $\N\varphi_1$ i $\N\varphi_2$ imaju različite domene, pa su različite.

Ako pak $\varphi_1$ i $\varphi_2$ imaju istu domenu, ali se razlikuju na nekoj riječi $w$ iz te domene, tada (opet po injektivnosti kodiranja riječi) vrijedi
\begin{equation}
    \N\varphi_1(\kr w)=\kr{\varphi_1(w)}\ne\kr{\varphi_2(w)}=\N\varphi_2(\kr w)\text,
\end{equation}
pa su $\N\varphi_1$ i $\N\varphi_2$ različite jer se razlikuju na elementu $\kr w$.

Za surjektivnost, uzmimo proizvoljnu $\f F^1\in\mathscr Comp_1$ i tražimo $\varphi\in\TComp$ takvu da je $\f F=\N\varphi$. Očito, u domeni joj moraju biti upravo sve riječi $w$ za koje je $\kr w\in\dom{\f F}$, a svaku takvu riječ mora preslikavati u (jedinstvenu) riječ $v$ čiji kod je $\f F(\kr w)$. Time je jezična funkcija $\varphi$ potpuno određena, a iz $\N\varphi=\f F\in\mathscr Comp_1$ po teoremu~\ref{tm:krit} slijedi $\varphi\in\TComp$. Dobivenu funkciju $\varphi$ označavamo s $\N^{-1}\f F$.
\end{proof}
Precizno, za $\sigma:=\N\Sigma^*$, imamo vezu
%\begin{equation}
	$\f F=\N\varphi=\sigma\circ\varphi\circ\sigma^{\,-1}\Longleftrightarrow
	\varphi=\N^{-1}\f F=\sigma^{\,-1}\circ\f F\circ\sigma$
%\end{equation}
te vrijedi $\N\,\N^{-1}\f F=\f F$ i $\N^{-1}\N\varphi=\varphi$.

\begin{korolar}[{name=[jednomjesne brojevne funkcije u različitim modelima]}]\label{kor:pent}
Neka je $\Sigma$ abeceda s kodiranjem $\N\Sigma$ te $F^1$ brojevna funkcija.

Tada je $F$ parcijalno rekurzivna ako i samo ako je $\N^{-1}F$ Turing-izračunljiva.
\end{korolar}
\begin{proof}
Ako je $F=\N\,\N^{-1}F$ parcijalno rekurzivna, tada je po teoremu~\ref{tm:pir} RAM-izračunljiva, pa je po teoremu~\ref{tm:krit} $\N^{-1}F$ Turing-izračunljiva. S druge strane, ako je $\N^{-1}F$ Turing-izračunljiva, onda je po teoremu~\ref{tm:tikp} $\N\,\N^{-1}F=F$ parcijalno rekurzivna.
\end{proof}

Vidimo da korolar~\ref{kor:pent} vrijedi bez obzira na abecedu i kodiranje. Važan posebni slučaj dobijemo za jednočlanu abecedu, za koju je kodiranje jedinstveno i podudara se s duljinom (lema~\ref{lm:dulj<=kr}).

\begin{definicija}[{name=[unarna abeceda i unarna reprezentacija brojevne funkcije]}]
\emph{Unarna abeceda} je $\Sigma_{\t\textbullet}:=\{\t\textbullet\}$, s kodiranjem %. Kad nema opasnosti od zabune, umjesto $\Sigma_{\t\textbullet}$ pišemo samo \t\textbullet. Kodiranje unarne abecede je jedino moguće: 
	$\N\Sigma_{\t\textbullet}(\t\textbullet):=1$.

	Za jednomjesnu brojevnu funkciju $F^1$, jezičnu funkciju $\N^{-1}F$ nad $\Sigma_{\t\textbullet}$ zovemo \emph{unarnom reprezentacijom} od $F$ i označavamo je $\t\textbullet F$.
\end{definicija}

Svaka riječ $u$ nad unarnom abecedom ($u\in\Sigma_{\t\textbullet}^*$) je oblika $u=\t\textbullet^n$, gdje je $n=\dulj u=\kr u$. Sada definicija preslikavanja $\N^{-1}$ kaže da je $\t\textbullet F(\t\textbullet^n)=\t\textbullet^{F(n)}$ ako je $n\in\dom F$, a $\t\textbullet^n\notin\dom{\t\textbullet F}$ inače. Po napomeni~\ref{nap:parcdef}, pišemo
$\t\textbullet F(\t\textbullet^n)\simeq\t\textbullet^{F(n)}$.

\begin{primjer}[{name=[unarna reprezentacija]}]
$\t\textbullet\f{factorial}\bigl(
\t\textbullet\f{prime}(
\t\textbullet)
\bigr)=\t\textbullet\f{factorial}(
\t{\textbullet\textbullet\textbullet})
=\t{\textbullet\textbullet\textbullet\textbullet\textbullet\textbullet}$, jer je $p_1!=3\,!=6$.\newline
	Za inicijalne funkcije, $\t\textbullet\f I_1^1(w)=w$, $\t\textbullet\f{Sc}(w)=w\t\textbullet$, a $\t\textbullet\f Z(w)=\varepsilon$ za sve $w\in\Sigma_{\t\textbullet}^*$. %Za parcijalne funkcije, $\varepsilon\notin\dom{\t\textbullet\f{Russell}}$, jer $0\notin K$.
\end{primjer}

\begin{korolar}[{name=[unarno reprezentirane brojevne funkcije u različitim modelima]}]\label{kor:peuf}
Neka je $F^1$ brojevna funkcija.
	
	Tada je $F$ parcijalno rekurzivna ako i samo ako je $\t\textbullet F$ Turing-izračunljiva.
\end{korolar}
\begin{proof}
Već rekosmo, ovo je posebni slučaj korolara~\ref{kor:pent}, za unarnu abecedu.
\end{proof}

%\subsection{Izračunljivost jezika}\label{sec:Todl}

Neka je $\Sigma$ proizvoljna abeceda, s $b'$ znakova, $\N\Sigma$ njeno kodiranje te $L\subseteq\Sigma^*$ jezik nad njom. Što bi značilo da je $L$ izračunljiv?

Izračunljivost $L$ u brojevnom modelu gledamo preko kodova: definiramo jednomjesnu brojevnu relaciju
\begin{equation}\label{eq:kodLdef}
    \kr L:=\{\kr w\mid w\in L\}=\N\Sigma^*[L]\text,
\end{equation}
i prirodno je reći da $L$ ima neko svojstvo izračunljivosti (npr.\ da je primitivno rekurzivan) ako relacija $\kr L$, odnosno njena karakteristična funkcija $\chi_{\kr L}$, ima to svojstvo.

Recimo, prazan jezik $\emptyset$ je primitivno rekurzivan jer je $\chi_{\kr\emptyset}\!=\chi_{\mspace{1mu}\emptyset}\!=\!\f Z$ primitivno rekurzivna (čak inicijalna). Također, univerzalni jezik $\Sigma^*$ je primitivno rekurzivan, jer je $\kr{\Sigma^*}=\N\Sigma^*[\Sigma^*]=\im{\N\Sigma^*}\!=\N$, čija je karakteristična funkcija $\f C_1^1=\f{Sc}\circ\f Z$.

U jezičnom modelu, moramo vidjeti što znači da neki Turingov stroj računa $\chi_L$. To je Turingov stroj nad $\Sigma$, i ulaz $w\in\Sigma^*$ mu se daje na isti način kao i običnom Turingovom stroju: kroz početnu konfiguraciju $(q_0,0,w\bl\ldots)$. No za izlaz ($\mathit{true}$ ili $\mathit{false}$, ovisno o tome je li $w\in L$) postoji samo konačno mnogo mogućnosti, pa ga je prirodnije predstaviti stanjem. Zato takvi Turingovi strojevi imaju \emph{dva} završna stanja, $q_a$ i $q_r$, kojima signaliziraju je li riječ u jeziku ili nije (kažemo da \emph{prihvaćaju} odnosno \emph{odbijaju} riječ).

Na prvi pogled, to je slično stanjima $q_z$ i $q_x$ koje smo imali u našim Turingovim strojevima, samo što konfiguracija sa stanjem $q_r$ (kao i ona s $q_a$) prelazi u samu sebe, a ne po funkciji~$\delta$. Ali postoji jedna bitna razlika: kako su karakteristične funkcije nužno totalne, od takvih strojeva zahtijevamo da \textbf{uvijek stanu} (za svaki ulaz dostignu konfiguraciju s jednim od ta dva završna stanja) i zovemo ih \emph{odlučiteljima} (\emph{deciders}). Svaki odlučitelj $\mathcal T$ dijeli $\Sigma^*$ na dva dijela,
\begin{align}
    \SwapAboveDisplaySkip
    L(\mathcal T):=\{w\in\Sigma^*&\mid\text{$\mathcal T$\!-izračunavanje s $w$ stane u stanju $q_a$}\}
    \text{ i}\\ \bigl(L(\mathcal T)\big)\kompl=\{w\in\Sigma^*&\mid\text{$\mathcal T$\!-izračunavanje s $w$ stane u stanju $q_r$}\}\text,
\end{align} i kažemo da \emph{prepoznaje} $L(\mathcal T)$.

Nije preteško vidjeti da su te dvije karakterizacije, brojevna i jezična, povezane --- pogotovo jer već imamo napravljen najveći dio posla.

\begin{teorem}[{name=[rekurzivnost Turing-odlučivog jezika]}]\label{tm:oikr}
    Neka je $L$ jezik (nad nekom abecedom $\Sigma$).\newline Ako postoji Turingov odlučitelj koji prepoznaje $L$, tada je relacija $\kr L$ rekurzivna.
\end{teorem}
\begin{proof}
Pretpostavimo da je $\mathcal T$ odlučitelj za $L$ i provedimo s tim strojem postupak iz točke~\ref{sec:tikp}. Ovdje navodimo samo detalje koje je potrebno promijeniti.

Prvo, kako imamo dva završna stanja, moramo fiksirati njihove kodove: recimo, $\N Q(q_a):=1$, a $\N Q(q_r):=2$. (Dokažite analogon leme~\ref{lm:bsomp-q0neqz}, da bez smanjenja općenitosti možemo pretpostaviti $q_0\ne q_a$ i $q_0\ne q_r$ --- a po definiciji odlučitelja mora biti $q_a\ne q_r$.)

    Drugo, kako nam $\delta$ sad nije definirana na oba završna stanja, treba promijeniti uvjet u lemi analognoj lemi~\ref{lm:newssdprn}, u $q\notin\{q_a,q_r\}$. Tu se ništa bitno ne mijenja, osim što će $\f{direction}$-tablica (vidjeti primjer~\ref{pr:polatable}) imati dva retka jedinicâ, a ne samo jedan.

I treće, umjesto parcijalno rekurzivne funkcije $\f{stop}$ imat ćemo funkciju zadanu sa
\begin{equation}\label{eq:stop'def}
    \f{stop}'(x):=\mu n\bigl(\f{State}(x,n)\in\{1,2\}\bigr)\text,
\end{equation}
	za koju ćemo moći zaključiti da je rekurzivna. Naime, parcijalno je rekurzivna kao minimizacija rekurzivne (lema~\ref{lm:StateHeadTapeprn} i korolar~\ref{kor:konprn}) relacije. Totalna je (pa smo u~\eqref{eq:stop'def} mogli koristiti \enquote*{:=}) po definiciji odlučitelja: ako je $x=\kr w$, tada mora za neki $n$ biti $\f{State}(x,n)\in\{1,2\}$, jer $\mathcal T$\!-izračunavanje s $w$ mora stati --- i $\f{stop}'(x)=\f{stop}'(\kr w)$ je upravo broj koraka nakon kojeg ono stane.

Tvrdimo da je
\begin{equation}
	x\in\kr L\Longleftrightarrow\f{State}\mspace{2mu}\bigl(x,\f{stop}'(x)\bigr)=1\text,
\end{equation}
iz čega slijedi rekurzivnost od $\kr L$ po lemi~\ref{lm:comprek} i korolaru~\ref{kor:prnrek}.

Doista, ako je $x\in\kr L=\N\Sigma^*[L]$, to znači da postoji $w\in L$ takva da je $x=\kr w$. Tada $\mathcal T$\!-izračunavanje s $w$ mora stati u stanju $q_a$ --- označimo s $n_0$ broj koraka nakon kojeg se to dogodi. Tada je $\f{State}(x,n_0)=1\in\{1,2\}$, dok za sve $n<n_0$ konfiguracija nakon $n$ koraka nije završna (sasvim analogno propoziciji~\ref{prop:ram1zav} --- jer završne konfiguracije prelaze isključivo u same sebe, postoji najviše jedna završna konfiguracija u izračunavanju) pa $\f{State}(x,n)\notin\{1,2\}$. Dakle $n_0=\f{stop}'(x)$, pa je $\f{State}\bigl(x,\f{stop}'(x)\bigr)=\f{State}(x,n_0)=1$.

U drugom smjeru, pretpostavimo $\f{State}\bigl(x,\f{stop}'(x)\bigr)=1$. Po propoziciji~\ref{pp:bijkr}, postoji jedinstvena riječ $w\in\Sigma^*$ takva da je $x=\kr w$. $\mathcal T$ je odlučitelj, pa $\mathcal T$\!-izračunavanje s $w$ mora stati --- označimo s $n_0$ broj koraka nakon kojeg se to dogodi. Tada je (kao i prije) $\f{State}(x,n)\notin\{1,2\}$ za $n<n_0$, a $\f{State}(x,n_0)=1\in\{1,2\}$, pa je $\f{stop}'(x)=n_0$. Sada pretpostavka glasi $\f{State}(x,n_0)=1$, odnosno završno stanje je $q_a$, pa $\mathcal T$ prihvaća $w$. To znači da je $w\in L$, odnosno $x=\kr w\in\kr L$.
\end{proof}
%\newpage

\begin{teorem}[{name=[Turing-odlučivost rekurzivnog jezika]}]\label{tm:krio}
Neka je $L$ jezik (nad nekom abecedom $\Sigma$).\newline Ako je relacija $\kr L$ rekurzivna, tada postoji Turingov odlučitelj za $L$.
\end{teorem}
\begin{proof}
Pretpostavka zapravo znači da je $\chi_{\kr L}$ rekurzivna funkcija. Dakle $\chi_{\kr L}$ je parcijalno rekurzivna, pa po teoremu~\ref{tm:pir} postoji RAM-program $P$ koji je računa; i totalna je, pa $P$-izračunavanje stane sa svakim ulazom. Na $P$ bismo htjeli primijeniti postupak iz točke~\ref{sec:RAM>Turing} --- dakle moramo $\chi_{\kr L}$ prikazati kao $\N\varphi$ za neku jezičnu funkciju $\varphi$ nad $\Sigma$. Možemo li to?

Svakako: dokaz propozicije~\ref{pp:trackbij} kaže da je $\varphi:=\N^{-1}\chi_{\kr L}=\sigma^{\,-\!1}\circ\chi_{\kr L}\circ\sigma$, uz oznaku $\sigma:=\N\Sigma^*$. Funkcija $\varphi$ je totalna kao kompozicija tri totalne, a Turing-izračunljiva je prema korolaru~\ref{kor:pent} (ako uzmemo isto kodiranje pomoću kojeg smo izračunali $\kr L$). Pogledajmo detaljnije kako ona djeluje.

Ako joj damo riječ $w\in\Sigma^*\setminus L$, tada ($\mspace{-1mu}\sigma$ je injekcija) vrijedi $\kr w\notin\kr L$, pa je $\chi_{\kr L}(\kr w)=0$. Tada je
\begin{equation}
    \varphi(w)=\sigma^{\,-\!1}\bigl(\,\chi_{\kr L}\bigl(\sigma(w)\bigr)\bigr)=\sigma^{\,-\!1}\bigl(\,\chi_{\kr L}(\kr w)\bigr)=\sigma^{\,-\!1}(0)=\varepsilon\text.
\end{equation}
Dakle, riječi izvan $L$ ona preslikava u praznu riječ. Riječi \emph{unutar} $L$ onda ne smije preslikavati u $\varepsilon$ zbog injektivnosti $\sigma^{\,-\!1}$, a jer je totalna, mora ih preslikavati nekamo. Dakle, $\varphi(w)$ je neprazna riječ za svaku $w\in L$. Vidimo da, baš kao i u $\N$, imamo prirodnu reprezentaciju $bool$ u $\Sigma^*$: prazna riječ je lažna, neprazne su istinite. Mnogi programski jezici koji definiraju istinitost stringova, definiraju je upravo na taj način.

%Kako izvršavanje RAM-programa o kojem govorimo stane za svaki $\kr w$, po korolaru~\ref{kor:faza3} i napomeni~\ref{nap:snstane} slijedi da će Turingov stroj koji konstruiramo stati za svaki $w$ --- samo trebamo malo promijeniti način na koji stane, da ne komunicira izlaz putem trake nego putem završnog stanja.

    Naš Turingov stroj koji računa $\varphi$ dobiven je transpiliranjem RAM-programa $P$ koji računa $\chi_{\kr L}$. Srećom, u završnoj konfiguraciji pozicija mu je $0$, pa je vrlo lako vidjeti je li izlazna riječ prazna: znak koji čitamo tada je ili praznina ili element ulazne abecede. Proglasimo $l_{\t\$}$ običnim stanjem, i dodajmo prijelaze %U opisu pete (posljednje) faze rada tog Turingovog stroja, razlikovali smo slučajeve kad je izlazna riječ prazna i kad nije. Kao što smo upravo vidjeli, to nam može poslužiti za odabir završnog stanja u kojem ćemo završiti --- samo zamijenimo prijelaze iz $m_{\t\$}$ s
\begin{gather}
    %\SwapAboveDisplaySkip
\label{eq:d88j}\tag{$\delta$5b}
    \delta(l_{\t\$},\alpha):=(q_a,\bl,-1)\text{, za sve }\alpha\in\Sigma\text,\\
\label{eq:d89j}\tag{$\delta$5c}
    \delta(l_{\t\$},\bl):=(q_r,\bl,-1)\text.
\end{gather}
Po teoremu~\ref{tm:krit}, dobiveni stroj će "računati" (pod navodnicima jer odlučitelji zapravo ne služe za računanje funkcija) funkciju $\varphi$, pa će za $w\in L$ rezultat biti neprazan. To znači da će stroj na kraju izračunavanja čitati znak iz $\Sigma$ i ući u stanje $q_a$. Za $w\notin L$ rezultat će biti $\varepsilon$ pa će stroj pročitati prazninu i ući u stanje $q_r$. Budući da za svaku riječ $w\in\Sigma^*$ vrijedi jedna od te dvije mogućnosti, zaključujemo da smo doista konstruirali odlučitelj.
\end{proof}
\vspace{-3mm}

Dijagramatski, na kraj Turingovog stroja dodali smo fragment\; %gornji desni kut~\eqref{dia:T5} zamijenimo s\quad
\begin{tikzpicture}[baseline=(q9.base)]\label{dia:T5j}
    \node[state] (q9) {$l_{\t\$}$};
\node[state,accepting,right of=q9] (qt) {$q_a$};
\node[state,accepting,below right=0.5 and 1.6 of q9] (qf) {$q_r$};
\draw
%(q9) edge[dashed,loop left] node[left] {\t/:\bl} (q9)
(q9) edge[dashed] node[above] {$\Sigma$:\bl} (qt)
(q9) edge[dashed] node[below] {\bl} (qf)
;
\end{tikzpicture}\;.
\vspace{-8mm}

\section{Turing-izračunljivost višemjesnih funkcija}

Dosadašnji rezultati pokazuju da je za jezične funkcije, i za \emph{jednomjesne} brojevne funkcije, svejedno računamo li ih na Turingovom stroju ili na nekom brojevnom modelu izračunavanja (npr.\ RAM-stroju). Da bismo isto dokazali i za \emph{višemjesne} brojevne funkcije, trebamo precizirati reprezentaciju njihovih ulaza.

%Zapravo, sve potrebne ideje smo već vidjeli. Za ulazne podatke jednomjesnih funkcija, kao i za izlazne podatke, koristit ćemo unarni zapis: što je prirodnije nego broj $4$ zapisati na traku Turingovog stroja kao \textbullet\textbullet\textbullet\textbullet, a $0$ kao praznu riječ $\varepsilon$\@? Tako smo već prikazivali (u tragovima) sadržaj registara kod simulacije RAM-izračunavanja. Da smo opterećeni složenošću, morali bismo koristiti binarni ili neki drugi "pravi" pozicijski zapis, ali srećom nismo.

Već smo u uvodu nagovijestili, koristit ćemo kontrakciju --- u abecedu ćemo dodati separator \t/ kojim ćemo razdvojiti više ulaznih podataka: npr.\ $(1,0,5,0)$ prenijet ćemo kao \t{\textbullet//\textbullet\textbullet\textbullet\textbullet\textbullet/}.

\begin{definicija}[{name=[binarna abeceda i binarna reprezentacija]}]\label{def:beta}
\emph{Binarna abeceda} je abeceda $\Sigma_\beta:=\{\t\textbullet,\t/\}$. Za svaki neprazni konačni niz prirodnih brojeva $\vec x=(x_1,x_2,\dotsc,x_k)$, definiramo \emph{binarnu reprezentaciju} kao
\begin{equation}\label{eq:betadef}
    \beta(\vec x):=\t\textbullet^{x_1}\t/\t\textbullet^{x_2}\t/\dotsm\t/\t\textbullet^{x_k}\in\Sigma_\beta^*\text.
\end{equation}
Za svaki $k\in\N_+$, označavamo $\beta^k:=\beta|_{\N^k}$.
\end{definicija}

Ponekad se reprezentacija također zove "kodiranje", ali nama su kodiranja funkcije čije povratne vrijednosti su prirodni brojevi. Funkcija $\beta$ ide u suprotnom smjeru (s dobrim razlogom --- sad trebamo promatrati jezične funkcije kao osnovne, jer za njih imamo teoreme~\ref{tm:tikp} i~\ref{tm:krit}), ali zapravo je možemo promatrati u bilo kojem smjeru.

\begin{propozicija}[{name=[bijektivnost binarne reprezentacije]}]\label{prop:betabij}
Funkcija $\beta$ je bijekcija između $\N^+$ i $\Sigma_\beta^*$.
\end{propozicija}
\begin{proof}
Najlakše je konstruirati inverznu funkciju i pokazati da je to doista inverz. Dakle, za proizvoljnu riječ $u\in\Sigma_\beta^*$, pitamo se kojeg niza je ona kod. Kao i uvijek kod konačnih nizova, trebamo odrediti njegovu duljinu $k$, a zatim pojedine članove $x_1$, $x_2$,~\ldots, $x_k$. Duljina je sljedbenik broja pojavljivanja separatora \t/ u riječi $u$ (jer u~\eqref{eq:betadef} ima $k-1$ separatora) i to je uvijek pozitivan broj, dakle niz je neprazan. Prvi član mu možemo odrediti brojeći kružiće do prvog separatora (ili do kraja riječi ako je $k=1$), drugi brojeći ih između prvog i drugog separatora, i tako dalje. Zadnji član $x_k$ je broj kružića od zadnjeg separatora do kraja riječi $u$.

Tvrdimo da je tako konstruirano preslikavanje desni inverz od $\beta$: odnosno, ako tako dobijemo niz $\vec x$, tada je $\beta(\vec x)=u$. Doista, te dvije riječi imaju isti broj separatora ($k-1$) i isti broj kružića ($\sum\vec x$) te se podudaraju na pozicijama svih separatora (malo pomaknute parcijalne sume od $\vec x$) --- što je dovoljno za zaključak da su jednake.

	To je također i lijevi inverz: odnosno, ako primijenimo taj postupak na riječ $\beta(\vec x)$, dobit ćemo upravo $\vec x$: imat će točnu duljinu $k$, i točan svaki član, jer između $i$-tog i $(i+1\mspace{-1mu})$-vog separatora u~\eqref{eq:betadef} ima točno $x_i$ kružića.
\end{proof}

Za same funkcije, koristit ćemo istu ideju kao u~\eqref{eq:kodfidef}: reprezentaciju ulaza preslikamo u reprezentaciju izlaza (ako je izlaz definiran), dok ne-reprezentacije ne preslikamo nikamo. Ovdje treba biti oprezan zbog propozicije~\ref{prop:betabij}, ali brojevne funkcije imaju fiksnu mjesnost. Zato ćemo ne-reprezentacijama za funkciju $f^k$ proglasiti sve one riječi koje nisu reprezentacije $k$-torki (odnosno one koje su reprezentacije $l$-torki za $l\ne k$). Na isti način, na izlazu ćemo dati reprezentaciju samog broja ($k=1$), dakle riječ bez separatora $\t\textbullet^y=\beta(y)\in\beta[\N]=\im{\beta^{\!1}}=\Sigma_{\t\textbullet}^*$.

\begin{definicija}[{name=[binarna reprezentacija brojevne funkcije]}]
Neka je $k\in\N_+$ te $f^k$ brojevna funkcija. Jezičnu funkciju $\beta f$ nad $\Sigma_\beta$, zadanu s
\begin{equation}\label{eq:betaf}
  \beta f(u):\simeq\beta\bigl(f(\vec x)\bigr)\text{, za } u=\beta(\vec x^k)\text,
\end{equation}
zovemo \emph{binarnom reprezentacijom} funkcije $f$.
\end{definicija}
%\begin{primjer}
%$\beta\f{mul}^2(\t{\textbullet\textbullet/\textbullet\textbullet\textbullet\textbullet\textbullet})=\t{\textbullet\textbullet\textbullet\textbullet\textbullet\textbullet\textbullet\textbullet\textbullet\textbullet}$, jer je $2\cdot5=10$, dok $\t\textbullet$ i $\t{//}$ nisu u $\dom{\beta\f{mul}^2}$, jer $(1)$ i $(0,0,0)$ nisu ulazi ispravne mjesnosti za $\f{mul}^2$.
%\end{primjer}

Vjerojatno ste nekad napisali Turingov stroj za $\beta\f{add}^2$, koristeći $\beta\f{add}^2(u\t/v)=uv$, ako su $u,v\in\Sigma_{\t\textbullet}^*$. Ipak, već tada ste zasigurno vidjeli koliko je teško napisati $\beta\f{mul}^2$, a pogotovo $\beta\f{pow}$ --- ostale divote iz poglavlja~\ref{ch:univ} ($\beta\f{part}$?!) da i ne spominjemo.

Sljedeći veliki cilj je dokazati da Turingovi strojevi doista mogu računati (binarno reprezentirane) \emph{sve} parcijalno rekurzivne funkcije, pa tako i funkciju $\beta\f{univ}$ --- čineći tako jedan od modela univerzalne izračunljivosti. Prvo ćemo dokazati obrat: one brojevne funkcije čije binarne reprezentacije su Turing-izračunljive, parcijalno su rekurzivne. To nije toliko impresivan rezultat, ali lakši je za dokazati, a osnovnu ideju dokaza moći ćemo upotrijebiti i u dokazu obrata.

%\subsection{Prateća funkcija binarne reprezentacije}

Dakle, neka je $k\in\N_+$ te $f^k$ brojevna funkcija takva da je $\beta f$ Turing-izračunljiva. Kodirajmo $\N\Sigma_\beta$: već smo definirali $\kr{\t\textbullet}=1$, dodefinirajmo još $\kr{\t/}:=2$.
Po teoremu~\ref{tm:tikp} $\N\beta f$ je parcijalno rekurzivna --- ali nas zanima $f$. Brojevne funkcije $(\N\beta f)^1$ i $f^k$ su povezane preko jezične funkcije $\beta f$. Možemo li naći brojevnu vezu između njih? Precizno, možemo li naći brojevne funkcije $g^1$ i $h^k$ takve da je $f=g\circ\N\beta f\circ h$? Ako bi $g$ i $h$ bile izračunljive i totalne (recimo primitivno rekurzivne), iz toga bi odmah slijedila parcijalna rekurzivnost od $f$.%, jer je skup parcijalno rekurzivnih funkcija zatvoren na kompoziciju.

U traženju tih funkcija može pomoći dijagram.
\begin{equation}\label{dia:Nbeta}
\begin{tikzcd}
\N^k
\arrow[harpoon]{r}{f}
\arrow{d}{\beta^k}
\arrow[bend right=75,dotted]{dd}{\f{bin}^k}
&\N
\arrow{d}{\beta^1}
\arrow[bend left=75,dotted]{dd}{\f{bin}^1}
\\
\Sigma_\beta^*
\arrow[harpoon]{r}{\beta f}
\arrow{d}{\N\Sigma_\beta^*}
	& \Sigma_{\t\textbullet}^*\!\!\subset\!\Sigma_\beta^*
\arrow{d}{\N\Sigma_\beta^*}
\\
\N
\arrow[harpoon]{r}{\N\beta f}
& \N
\end{tikzcd}
\qquad
\begin{tikzcd}
(3,1,2)
\arrow[mapsto]{r}{\f{mul}^3}
\arrow[mapsto]{d}{\beta^3}
& 6
\arrow[mapsto]{d}{\beta^1}
\\
\t{\textbullet\textbullet\textbullet/\textbullet/\textbullet\textbullet}
\arrow[mapsto]{r}{\beta\f{mul}^3}
\arrow[mapsto]{d}{\N\Sigma_\beta^*}
& \t{\textbullet\textbullet\textbullet\textbullet\textbullet\textbullet}
\arrow[mapsto]{d}{\N\Sigma_\beta^*}
\\
275
\arrow[mapsto]{r}{\N\beta\f{mul}^3}
&
63
\end{tikzcd}
\end{equation}
Lijevo je općenita slika, desno je jedan konkretan primjer (računanje $\f{mul}^3$ na $(3,1,2)$). Brojevi u donjem retku razumljiviji su u pomaknutoj bazi $2$: $275=(11121211)_2$, a $63=(111111)_2$. Gornji pravokutnik predstavlja definiciju~\eqref{eq:betaf}, dok donji predstavlja~\eqref{eq:kodfidef} za posebni slučaj binarne abecede i jezične funkcije $\beta f$ nad njom.

\begin{lema}[{name=[brojevni-jezični-brojevni dijagram komutira]}]\label{lm:pravkomut}
Dijagram prikazan lijevo u~\eqref{dia:Nbeta} (samo pune strelice) komutira.
\end{lema}
Drugim riječima, za svaki od ta dva pravokutnika, pa onda i za veliki pravokutnik sastavljen od ta dva, svejedno je kojim putom dođemo od njegovog gornjeg lijevog do donjeg desnog vrha.
\begin{proof}
Prvo trebamo dokazati (gornji pravokutnik) da je $\beta f\circ\beta^k =\beta^1\circ f$. Domena desne funkcije je $\dom f$ jer je $\beta^1$ totalna, a domena lijeve funkcije je $\{\vec x\in\N^k\mid\beta^k(\vec x)\in\dom{\beta f}\}$, što je opet jednako $\dom f$ po definiciji domene $\dom{\beta f}$. Neka je sad $\vec x\in\dom f$ proizvoljan. Na njemu je vrijednost lijeve funkcije $\beta f\bigl(\beta^k(\vec x)\bigr)$, što je po definiciji~\eqref{eq:betaf} jednako $\beta^1\bigl(f(\vec x)\bigr)=(\beta^1\circ f)(\vec x)$. Dakle, $\beta f\circ\beta^k$ i $\beta^1\circ f$ imaju istu domenu i na njoj se podudaraju, pa su to jednake funkcije.

Donji pravokutnik je jednostavniji jer je $\sigma:=\N\Sigma_\beta^*$ bijekcija: uz oznaku $\varphi:=\beta f$,
\begin{equation}
    \N\beta f\circ\sigma=\N\varphi\circ\sigma=\sigma\circ\varphi\circ\sigma^{\,-\!1}\circ\sigma=\sigma\circ\varphi=\sigma\circ\beta f\text.
\end{equation}

Sada iz te dvije jednakosti slijedi
\begin{equation}\label{eq:pravkomut}
    \sigma\circ\beta^1\circ f=\sigma\circ\beta f\circ\beta^k=\N\beta f\circ\sigma\circ\beta^k\text,
\end{equation}
odnosno čitav "\!vanjski okvir" komutira.
\end{proof}

Upravo dokazana jednakost je korisna, jer pruža način da se u potpunosti izbjegnu jezične funkcije: vrhovi vanjskog pravokutnika su brojevni pa se njegove stranice mogu opisati brojevnim funkcijama. Gornju i donju stranicu već imamo, još je preostalo precizirati lijevu i desnu.

\subsection{Funkcije \texorpdfstring{$\f{bin}^k$}{bin} --- binarno kodiranje}

\begin{definicija}[{name=[binarno kodiranje --- prateća funkcija binarne reprezentacije]}]\label{def:bink}
Za svaki $k\in\N_+$, definiramo $\f{bin}^k:=\N\Sigma_\beta^*\circ\beta^k$.
\end{definicija}

Te brojevne funkcije su na dijagramu prikazane točkanim strelicama: lijeva stranica vanjskog pravokutnika je $\f{bin}^k$, a desna $\f{bin}^1$. Svaka $\f{bin}^k$ je injekcija kao kompozicija dvije injekcije. Jesu li izračunljive bez pozivanja na jezični model? %Štoviše, funkcija $bin^{\cdots}:=\N\Sigma_\beta^*\circ\beta$ (koja nije brojevna funkcija jer nema fiksnu mjesnost) predstavlja mogućnost da u funkciju prenesemo varijabilan broj argumenata: za funkciju $f^{\cdots}$ koja prima \emph{varargs}, $\N\beta f^{\cdots}$ je jednomjesna brojevna funkcija koja dobro modelira sve aspekte izračunljivosti funkcije $f^{\cdots}$, pa bismo mogli reći da je $f^{\cdots}$ izračunljiva (u nekom smislu) ako je $\N\beta f^{\cdots}$ izračunljiva (u tom istom smislu). O tome ćemo reći nešto više malo kasnije --- zasad se usredotočimo na izračunljivost funkcija $\f{bin}^k$.

Kako bismo izračunali $\f{bin}(3,1,2)=(11121211)_2=275$ ili $\f{bin}(6)=(111111)_2=63$ koristeći samo brojevne funkcije? Ovo drugo se čini bitno lakšim.

\begin{lema}[{name=[primitivna rekurzivnost jednomjesnog binarnog kodiranja]}]\label{lm:bin1prn}
Funkcija $\f{bin}^1$\! je primitivno rekurzivna.
\end{lema}
\begin{proof}
Možemo odmah napisati točkovnu definiciju $\f{bin}(x)=\sum_{i<x}2^i$, pa će primitivna rekurzivnost slijediti iz leme~\ref{lm:sumprodrek} --- ali svaki računarac zna da to ima i zatvoreni oblik:
$\f{bin}(x)=2^x-1=\f{pd}\bigl(\f{pow}(2,x)\bigr)$ (jer je uvijek $2^x\ge 1$).
\end{proof}

Kako sada pomoću funkcije $\f{bin}^1$ možemo dobiti $\f{bin}^2$? Riječ $\beta(x,y)=\t\textbullet^x\!\t/\t\textbullet^y$ je sastavljena od tri dijela: $\t\textbullet^x=\beta(x)$, $\t/$ i $\t\textbullet^y=\beta(y)$, čije kodove znamo --- to su redom $\f{bin}(x)$, $2$ i $\f{bin}(y)$. Dakle, da dobijemo njen kod, trebamo ih konkatenirati u pomaknutoj bazi $2$. Operacija je definirana s~\eqref{eq:defconcat}.

\begin{propozicija}[{name=[prateća funkcija konkatenacije nad $\Sigma$]}]\label{pp:krkonkkr}
Neka je $\Sigma$ abeceda, $\N\Sigma$ njeno kodiranje te $b$ broj znakova u njoj.
	
	Tada za sve riječi $u,v\in\Sigma^*$ vrijedi
    $\kr{uv}=\kr u\conc b\kr v$.
\end{propozicija}
\begin{proof}
Uz oznake $u=\alpha_1\alpha_2\dotsm\alpha_{\dulj{u}}$ i $v=\beta_1\beta_2\dotsm\beta_{\dulj{v}}$, vrijedi
\begin{multline*}
\kr{uv}=\kr{\alpha_1\alpha_2\dotsm\alpha_{\dulj u}\beta_1\beta_2\dotsm\beta_{\dulj v}}=\bigl(
\N\Sigma(\alpha_1)
\dotsm
\N\Sigma(\alpha_{\dulj u})
\N\Sigma(\beta_1)
\dotsm
\N\Sigma(\beta_{\dulj v})
\bigr)_b=\\
=\bigl(
\N\Sigma(\alpha_1)
\dotsm
\N\Sigma(\alpha_{\dulj u})
\bigr)_b\conc b\bigl(
\N\Sigma(\beta_1)
\dotsm
\N\Sigma(\beta_{\dulj v})
\bigr)_b=\kr{\alpha_1\dotsm\mspace{1mu}\alpha_{\dulj u}}\conc b\kr{\beta_1\dotsm\mspace{1mu}\beta_{\dulj v}}=\kr u\conc b\kr v\text.
\end{multline*}
Za $u=\varepsilon$ ili $v=\varepsilon$ tvrdnja također vrijedi jer je $\varepsilon$ neutralni element za konkatenaciju, a $\kr\varepsilon=0$ je neutralni element za $\conc b$: doista,
\begin{align}
    0\conc bx&=\f{sconcat}(0,x,b)=0\cdot b^{\f{slh}(x,b)}+x=0+x=x\text,\\
    x\conc b0&=\f{sconcat}(x,0,b)=x\cdot b^{\f{slh}(0,b)}+0=x\cdot b^0=x\text,
\end{align}
pri čemu je $\f{slh}(0,b)=(\mu t\le 0)\bigl(\sum_{i\le t}\!b^i>0\bigr)=0$ jer je već $\sum_{i\le0}b^i=b^0=1>0$.
\end{proof}

\begin{korolar}[{name=[duljina konkatenacije je zbroj duljina]}]\label{kor:lhkonk=lh+lh}
Za sve $x,y\in\N$, za sve $b\in\N_+$, vrijedi $\f{slh}(x\conc b y,b)=\f{slh}(x,b)+\f{slh}(y,b)$.
\end{korolar}
\begin{proof}
Neka su $x,y,b$ proizvoljni ($b$ pozitivan). Uzmimo neku $b$-članu abecedu, recimo $[1\dd b]$.
    Po propoziciji~\ref{pp:bijkr}, postoje $u,v\in[1\dd b]^*$ takve da je $x=\kr u$ i $y=\kr v$. Po definiciji $\f{slh}$, vrijedi $\f{slh}(\kr w,b)=\dulj w$ za sve $w\in[1\dd b]^*$, pa imamo
\begin{multline}
    \f{slh}(x,b)+\f{slh}(y,b)=
    \f{slh}(\kr u,b)+\f{slh}(\kr v,b)=
    \dulj u+\dulj v=
    \dulj{uv}=
    \f{slh}(\kr{uv},b)={}\\
    \text{[propozicija \ref{pp:krkonkkr}]}{}=\f{slh}(\kr u\conc b\kr v,b)=
    \f{slh}(x\conc b y,b)\text{.\;\qedhere}
\end{multline}
\end{proof}

\begin{propozicija}[{name=[primitivna rekurzivnost i asocijativnost konkatenacije]}]\label{pp:concprn}
Za svaki $b\in\N_+$, $\conc b$ je primitivno rekurzivna asocijativna operacija.
\end{propozicija}
\begin{proof}
Primitivna rekurzivnost slijedi iz leme~\ref{lm:ldcpprn}. Za asocijativnost,
\begin{multline}
    \left(x\conc b y\right)\conc b z=\left(x\cdot b^{\f{slh}(y,b)}+y\right)\cdot b^{\f{slh}(z,b)}+z=\\[1mm]
    =\left(x\cdot b^{\f{slh}(y,b)}\cdot b^{\f{slh}(z,b)}+y\cdot b^{\f{slh}(z,b)}\right)+z=
    x\cdot b^{\f{slh}(y,b)\,+\,\f{slh}(z,b)}+\left(y\cdot b^{\f{slh}(z,b)}+z\right)=\\[-1mm]
	\text{[korolar~\ref{kor:lhkonk=lh+lh}]}{}=x\cdot b^{\f{slh}(y\,\conc{{\scriptscriptstyle b}}\,z\,,\,b)}+\left(y\conc b z\right)=x\conc b\left(y\conc b z\right)\text{\qedhere\;.}
\end{multline}
\end{proof}
Zbog asocijativnosti ćemo ubuduće pisati izraze poput $x\conc by\conc bz$ bez zagrada. Uočimo da je za asocijativnost ključno da se radi o \textbf{pomaknutoj} bazi: u običnoj bazi $10$ je $(5\conc{\!1\!0\!}\!'0)\conc{\!1\!0\!}\!'2=50\conc{\!1\!0\!}\!'2=502\ne5\conc{\!1\!0\!}\!'(0\conc{\!1\!0\!}\!'2)=5\conc{\!1\!0\!}\!'2=52$ pa ne vrijedi asocijativnost.

\begin{propozicija}[{name=[primitivna rekurzivnost višemjesnog binarnog kodiranja]}]\label{pp:binkprn}
Za svaki $k\in\N_+$, funkcija $\f{bin}^k$ je primitivno rekurzivna.
\end{propozicija}
\begin{proof}
Matematičkom indukcijom po $k$. Za $k=1$ (baza), to je upravo lema~\ref{lm:bin1prn}. Pretpostavimo da je $\f{bin}^l$ primitivno rekurzivna za neki $l\in\N_+$. Tada je
\begin{multline}
    \f{bin}^{l+1}(\vec x,y)=
    \kr{\beta(\vec x,y)}=
	\kr{\t\textbullet^{x_1}\!\t/\dotsm\t{/\textbullet}^{x_l}\!\t{/\textbullet}^y}=
    \kr{\beta(\vec x)\t/\beta(y)}={}\\
    \text{[propozicija~\ref{pp:krkonkkr}]}{}=\kr{\beta(\vec x)}\conc 2\kr{\t/}\conc 2\kr{\beta(y)}=
    \f{bin}^k(\vec x)\conc22\conc2\f{bin}^1(y)\text,
\end{multline}
što je primitivno rekurzivno po pretpostavci indukcije, propoziciji~\ref{pp:concprn} i lemi~\ref{lm:bin1prn}.
\end{proof}

%\subsection{Turing-izračunljive funkcije su parcijalno rekurzivne}

Promotrimo sada dijagram lijevo u~\eqref{dia:Nbeta}, gledajući i točkane strelice. Definicija~\ref{def:bink} zapravo kaže da "kružni odsječak" sa svake strane tog dijagrama komutira, pa iz toga i leme~\ref{lm:pravkomut} slijedi da čitav "\!vanjski oval" također komutira.

\begin{korolar}[{name=[brojevni-brojevni dijagram komutira]}]\label{kor:binf=Nbetafbin}
Neka je $k\in\N_+$ te $f^k$ funkcija. Tada vrijedi
    $\f{bin}^1\!\circ f=\N\beta f\circ\f{bin}^k$.
\end{korolar}
\begin{proof}
To je jednakost~\eqref{eq:pravkomut} u koju je uvrštena (s obje strane) definicija~\ref{def:bink}.
\end{proof}

%\subsection{Inverz binarnog kodiranja}

Da bismo izrazili $f$ pomoću $\N\beta f$, treba nam (izračunljiv) lijevi inverz za $\f{bin}^1$. Kako možemo iz $63=(111111)_2=\f{bin}(6)$ natrag dobiti $6$? Samo treba prebrojiti znamenke.

\begin{lema}[{name=[primitivna rekurzivnost i specifikacija duljine binarnog zapisa]}]\label{lm:blh}
Definiramo funkciju $\f{blh}$ s $\f{blh}(n):=\f{slh}(n,2)$.

Funkcija $\f{blh}$ je primitivno rekurzivna te vrijedi $\f{blh}\circ\f{bin}^1\!=\f I_1^1$.
\end{lema}
\begin{proof}
Primitivna rekurzivnost slijedi iz simboličke definicije $\f{blh}=\f{slh}\circ(\f I_1^1,\f C_2^1)$, po lemi~\ref{lm:ldcpprn} i propoziciji~\ref{prop:konst}.
Za kompoziciju, uzmimo proizvoljni $n\in\N$ i računamo $\f{blh}\bigl(\f{bin}(n)\bigr)=\f{blh}(2^n-1)$. Relacija koju minimiziramo (po $t\le 2^n-1$) je
\begin{equation}
    \textstyle\sum_{i\le t}2^i>2^n-1
    \Longleftrightarrow
    2^{t+1}-1>2^n-1
    \Longleftrightarrow
    %2^{t+1}>2^n
    %\Longleftrightarrow
    t+1>n
    \Longleftrightarrow
    t\ge n
\end{equation}
pa je najmanji takav $t$ upravo $n=\f I_1^1(n)$.\qedhere
\end{proof}

%Na komutirajućem dijagramu koji cijelo vrijeme promatramo, $\f{blh}$ bismo mogli prikazati kao strelicu prema gore, s desne strane pravokutnika, ispupčenu još više nego strelica $\f{bin}^1$. Upravo dokazana lema pokazuje da tako dobiveni "polumjesec" također komutira, pa i "proširena elipsa" također komutira --- čime smo napokon prikazali $f$ pomoću $\N\beta f$.

\begin{teorem}[{name=[parcijalna rekurzivnost Turing-izračunljivih brojevnih funkcija]}]\label{tm:btip}
Neka je $k\in\N_+$ te $\f f^k$ funkcija takva da je $\beta\f f$ Turing-izračunljiva.\newline Tada je $\f f$ parcijalno rekurzivna.
\end{teorem}
\begin{proof}
    Prema teoremu~\ref{tm:tikp}, primijenjenom na abecedu $\Sigma_\beta$, kodiranje $\{\t\textbullet\mapsto1,\t/\mapsto2\}$ i funkciju $\beta\f f$, $\N\beta\f f$ je parcijalno rekurzivna. Iz leme~\ref{lm:blh} i korolara~\ref{kor:binf=Nbetafbin} sada imamo
\begin{equation}
    \f f=\f I_1^1\circ\f f=\f{blh}\circ\f{bin}^1\!\circ\f f=\f{blh}\circ\N\beta\f f\circ\f{bin}^k\text,
\end{equation}
odnosno $\f f$ je dobivena kompozicijom iz tri funkcije, od kojih je srednja parcijalno rekurzivna, a prva i zadnja su primitivno rekurzivne (lema~\ref{lm:blh} i propozicija~\ref{pp:binkprn}), pa su rekurzivne (korolar~\ref{kor:prnrek}) a time i parcijalno rekurzivne. Kako je skup parcijalno rekurzivnih funkcija zatvoren na kompoziciju, $\f f$ je parcijalno rekurzivna.
\end{proof}

Istim dijagramom možemo dokazati i obrat teorema~\ref{tm:btip}. Jednadžba
\begin{equation}
    \f{bin}^1\!\circ f=\N\beta f\circ\f{bin}^k
\end{equation}
iz korolara~\ref{kor:binf=Nbetafbin} može poslužiti i za dobivanje $\N\beta f$ iz $f$ --- umjesto desne trebamo obrnuti lijevu stranicu pravokutnika, odnosno naći desni inverz za $\f{bin}^k$. Nažalost, to je iz tri razloga zapetljanije nego ovo što smo napravili u prethodnoj točki.

Prvi razlog tiče se napomene~\ref{nap:brip}. Zbog $\dom{\f{bin}^k}=\N^k$, traženi desni inverz ima $k$ izlaznih podataka, pa ga reprezentiramo s $k$ brojevnih funkcija $\f{arg}_1$, $\f{arg}_2$,~\ldots, $\f{arg}_k$ --- tako nazvanih jer ekstrahiraju pojedine argumente funkcije $f$ iz jedinog argumenta $a$ funkcije $\N\beta f$. Te funkcije su neovisne o~$k$: recimo, $\f{arg}_2\bigl(\f{bin}^2(x,y)\bigr)=\f{arg}_2\bigl(\f{bin}^4(x,y,z,t)\bigr)=y$, pa ih ne moramo označavati s dva broja, poput $\f I_n^k$.

Drugo, implementacija je bitno kompliciranija nego u lemi~\ref{lm:blh}. Dok su riječi nad jednočlanom abecedom trivijalno izomorfne s $\N$ ($w\mapsto\dulj w$ je izomorfizam), riječi nad dvočlanom abecedom imaju bogatu strukturu, za čije raščlanjivanje trebamo kompliciranije funkcije, svojevrsnu biblioteku \texttt{<string.h>} za rad s riječima nad binarnom abecedom. U sljedećoj točki napravit ćemo minimum potreban za implementaciju funkcija $\f{arg}_i$.

Treće, funkcija $\f{bin}^k$ nije surjekcija, pa \emph{nema} totalni desni inverz. Precizno, $\N\beta f$ nije \emph{jednaka} $\f{bin}^1\circ f\circ(\f{arg}_1,\dotsc,\f{arg}_k)$, već je restrikcija te funkcije na $\im{\f{bin}^k}$. %ali da bismo dobili jednakost, morat ćemo eksplicitno eliminirati ulaze "krive mjesnosti" --- na kojima ne smijemo pozvati funkcije $\f{arg}_i$. Naime, zbog parcijalne specifikacije one bi mogle dati nešto što jest u domeni od $f$, pa bi desna strana bila definirana, a lijeva bi bila nedefinirana;
Recimo u slučaju kad $a$ ima $7$ znamenaka $2$ u pomaknutoj bazi $2$, izraz $\N\beta\f{mul}^3(a)$ nema smisla, dok je $2^{\f{arg}_1(a)\cdot\f{arg}_2(a)\cdot\f{arg}_3(a)}-1$ sasvim definirano. Srećom, imamo tehniku (restrikcija na rekurzivan skup) koja će uskladiti domene.

\subsection{Standardna biblioteka za \texorpdfstring{$\Sigma_\beta^*$}{binarne stringove}}\label{sec:stdstring}

\begin{propozicija}[{name=["biti prefiks" je parcijalni uređaj]}]\label{pp:prefiksrpu}
Za riječi $v,w\in\Sigma_\beta^*$ kažemo da je $v$ \emph{prefiks} od $w$ ako postoji riječ $u$ takva da je $w=vu$.
Ta relacija je refleksivni parcijalni uređaj na skupu $\Sigma_\beta^*$.
\end{propozicija}
\begin{proof}
Za refleksivnost stavimo $u:=\varepsilon$. Za tranzitivnost, ako je $w=vu$ i $v=v'u'$, tada je $w=(v'u')u=v'(u'u)$ pa je $v'$ prefiks od $w$.

Za antisimetričnost, ako je $w=vu$ i $v=wu'$, tada je kao za tranzitivnost $w=w(uu')$ pa je $\dulj w=\dulj{w(uu')}=\dulj w+\dulj{uu'}$, iz čega $\dulj{uu'}=0$. Kako je $\varepsilon$ jedina riječ duljine $0$, imamo $uu'=\varepsilon$, i analogno $u=u'=\varepsilon$, dakle $w=vu=v\varepsilon=v$.
\end{proof}

\begin{korolar}[{name=[primitivna rekurzivnost relacije "biti prefiks"]}]\label{kor:preceqprnrpu}
Dvomjesna brojevna relacija $\sqsubseteq_\beta$, zadana s
\begin{equation}
    x\sqsubseteq_\beta y:\Longleftrightarrow\text{"$x$ je kod nekog prefiksa riječi čiji kod je $y$"}
\end{equation}
(kodovi su po $\N\Sigma_\beta^*$), primitivno je rekurzivan refleksivni parcijalni uređaj na $\N$.
\end{korolar}
\begin{proof}
    Da se radi o parcijalnom uređaju slijedi iz propozicija~\ref{pp:prefiksrpu} i~\ref{pp:bijkr}, a primitivna rekurzivnost iz lema~\ref{lm:ldcpprn} i~\ref{lm:blh}: $x\sqsubseteq_\beta y\Longleftrightarrow x=\f{sprefix}\bigl(y,\f{blh}(x),2\bigr)$.
\end{proof}

\begin{primjer}[{name=[prateća relacija "prefiks"]}]
Recimo, vrijedi $4=(12)_2=\kr{\t{\textbullet/}}\sqsubseteq_\beta\kr{\t{\textbullet/\textbullet\textbullet\textbullet}}=(12111)_2=39$.
\end{primjer}

Za sljedeću lemu, trebat će nam \emph{uokvireni} brojevi: oni čiji zapisi u pomaknutoj bazi~$2$ počinju i završavaju znamenkom $2$. Želimo "isprogramirati" dokaz propozicije~\ref{prop:betabij}, koji ima nekoliko slučajeva ovisno o tome nalazimo li se prije prvog separatora, nakon zadnjeg, ili između njih. Ako riječ ima separatore na oba kraja, svaki $x_i$ možemo odrediti brojeći jedinice između susjednih separatora.

\begin{lema}[{name=[primitivna rekurzivnost raščlambe binarnih zapisa]}]\label{lm:pos2streak1prn}
Postoje primitivno rekurzivne funkcije $\f{pos2}$ i $\f{streak1}$, takve da za svaki uokvireni broj $x$, za svaki $i\in\N$, vrijedi:
\begin{labeling}{$\f{streak1}(x,i)$}
    \item[$\f{pos2}(x,i)$] je pozicija $i$-te dvojke (ili $\f{blh}(x)$ ako takva ne postoji) te
    \item[$\f{streak1}(x,i)$] je duljina $i$-tog niza uzastopnih jedinica (ili $0$ ako takav ne postoji),
\end{labeling}
počevši od $i=0$ slijeva, u zapisu broja $x$ u pomaknutoj bazi $2$.
\end{lema}
\begin{proof}
Za početak, definirajmo pomoćnu primitivno rekurzivnu (lema~\ref{lm:brojrek}) funkciju koja broji dvojke (separatore) u zapisu broja u pomaknutoj bazi $2$.
\begin{equation}
    \f{count2}(x):=\bigl(\num i<\f{blh}(x)\bigr)\bigl(\f{sdigit}(x,i,2)=2\bigr)
\end{equation}

Tvrdimo da primitivno rekurzivne funkcije zadane točkovno s
\begin{align}
    \f{pos2}(x,i)&:=\f{blh}\bigl((\mu z<x)(z\conc22\,\sqsubseteq_\beta x\,\land\,\f{count2}(z)=i)\bigr),\\
    \f{streak1}(x,i)&:=\f{pos2}\bigl(x,\f{Sc}(i)\bigr)\dotminus\f{Sc}\bigl(\f{pos2}(x,i)\bigr),
\end{align}
zadovoljavaju tražene specifikacije.

Za $i<\f{count2}(x)$, uvjet pod operatorom minimizacije jednoznačno određuje $z$. Doista, kad bi postojala dva broja $z$ i $z'$ s tim svojstvom, morali bi biti kodovi različitih riječi $v$ i $v'$, takvih da su $v\t/$ i $v'\!\t/$ prefiksi od $w$, čiji kod je $x$. Prefiksi iste riječi iste duljine su jednaki ($\f{sprefix}$ je funkcija), pa duljine od $v\t/$ i $v'\!\t/$ moraju biti različite: bez smanjenja općenitosti pretpostavimo $\dulj{v\t/}<\dulj{v'\!\t/}=\dulj{v'}+1$, dakle $v\t/$ je zapravo prefiks od $v'$, pa mora imati manje ili jednako dvojki. Dakle
\begin{multline}
    i=\f{count2}(z')=\f{count2}(\kr{v'})\ge\f{count2}(\kr{v\t/})=\f{count2}(\kr v\conc2\kr{\t/})=\\
    =\f{count2}(\kr v\conc22)=\f{count2}(\kr v)+1=\f{count2}(z)+1=i+1\text,
\end{multline}
    kontradikcija. No kako je $z$ jedinstven, dovoljno je naći \emph{neki} takav i taj će biti minimalan, a $\f{sprefix}$ od $x$ do isključivo pozicije $i$-te dvojke zadovoljava to svojstvo. Njegova duljina je onda upravo ta pozicija (brojeći od $0$), kao što i treba biti.

Za $i\ge\f{count2}(x)$, takav $z$ ne postoji (jer $z\conc22$ ima više dvojki nego $x$, pa ne može biti $z\conc22\sqsubseteq_\beta x$), što znači da minimizacija $(\mu z<x)$ dade $x$, odnosno $\f{pos2}(x,i)=\f{blh}(x)$.

Sada je za $\f{streak1}$ dovoljno oduzeti odgovarajuće pozicije: poziciju prvog znaka nakon $i$-te dvojke, od pozicije sljedeće dvojke. Za $i<\f{count2}(x)-1$, to očito funkcionira: jer su jedine znamenke $1$ i $2$, između susjednih dvojki nalaze se samo jedinice.

	Za $i=\f{count2}(x)-1$, $i$-ta dvojka je upravo zadnja, pa je $\f{pos2}(x,i)=\f{pd}\bigl(\f{blh}(x)\bigr)$. Također po specifikaciji $\f{pos2}$ vrijedi $\f{pos2}(x,i+1)=\f{blh}(x)$, pa je $\f{streak1}(x,i)=\f{blh}(x)\dotminus\f{Sc}\bigl(\f{pd}(\f{blh}(x))\bigr)=0$ (jer je $\f{Sc}\bigl(\f{pd}(t)\bigr)=t\bump\ge t$). A za $i\ge\f{count2}(x)$ imamo $\f{pos2}(x,i+1)=\f{pos2}(x,i)=\f{blh}(x)$, pa je opet $\f{streak1}(x,i)=\f{blh}(x)\dotminus\f{Sc}\bigl(\f{blh}(x)\bigr)=0$.
\end{proof}

%\subsection{Funkcije \texorpdfstring{$\f{arg}_i$}{arg} --- ekstrakcija argumenata}\label{sec:arg}

Za ekstrakciju pojedinog $x_n$ sada trebamo uokviriti $\f{bin}^k(\vec x)$ dvojkama te pozvati "metodu" $\f{streak1}$ s argumentom $n-1$.

\begin{propozicija}[{name=[primitivna rekurzivnost ekstrakcije argumenata]}]\label{pp:argnprn}
    Za svaki $n\in\N_+$ postoji primitivno rekurzivna funkcija $\f{arg}_n$, takva da za sve $\vec x\in\N^+$ vrijedi $\f{arg}_n\bigl(\f{bin}(\vec x)\bigr)=\kr{\vec x}[n-1]$.
\end{propozicija}

\begin{proof}
Za svaki $n\in\N_+$ vrijedi $n-1\in\N$, pa definiramo
\begin{equation}
    \f{arg}_n(x):=\f{streak1}(2\conc2x\conc22,n-1)\text.
\end{equation}
Neka je $\vec x\in\N^k$ proizvoljan ($k\in\N_+$). Tada u $\beta(\vec x)$ ima točno $k-1$ separatora, pa je $\f{count2}\bigl(\f{bin}(\vec x)\bigr)=k-1$, odnosno za $y:=2\conc2\f{bin}(\vec x)\conc22$ vrijedi $\f{count2}(y)=k-1+2=k+1$. Sada za svaki $i\in[1\dd k]$ vrijedi $\f{arg}_i\bigl(\f{bin}(\vec x)\bigr)=\f{streak1}(y,i-1)$, što je upravo $x_i$ (recimo, za $i=2<k$ imat ćemo $\f{streak1}(\kr{\t{/\textbullet}^{x_1}\!\t{/\textbullet}^{x_2}\!\t/\dotsm\t{/\textbullet}^{x_k}\!\t/},1)=\f{pos2}(y,2)-\f{Sc}\bigl(\f{pos2}(y,1)\bigr)=(1+x_1+1+x_2)-\f{Sc}(1+x_1)=x_2$).
Za $i>k$ imat ćemo $i-1\ge k=\f{count2}(y)-1$, pa će po lemi~\ref{lm:pos2streak1prn} biti $\f{streak1}(y,i-1)=0$, kao što i treba.
\end{proof}

%\subsection{Parcijalno rekurzivne funkcije su Turing-izračunljive}

%Napokon je sve spremno za dokaz obrata teorema~\ref{tm:btip}.

\begin{teorem}[{name=[Turing-izračunljivost parcijalno rekurzivnih brojevnih funkcija]}]\label{tm:pibt}
Za svaku parcijalno rekurzivnu funkciju $\f f$, $\beta\f f$ je Turing-izračunljiva.
\end{teorem}
\begin{proof}
Označimo s $k\in\N_+$ mjesnost od $\f f$. %Cilj nam je dokazati da je $\N\beta\f f$ parcijalno rekurzivna.
    Funkciju $\N\beta\f f$ možemo dobiti (iz dijagrama) kao $\f{bin}^1\!\circ\f f\circ(\f{arg}_1,\dotsc,\f{arg}_k)$, restringiranu na $\im{\f{bin}^k}$. Ta slika je primitivno rekurzivna: samo treba provjeriti broj dvojki ($k-1\in\N$ je konstanta).
Dakle,
\begin{equation}
\label{eq:defNbetaf}
    \N\beta\f f(x)\simeq\f{bin}^1\bigl(\f f\bigl(\f{arg}_1(x),\dotsc,\f{arg}_k(x)\bigr)\bigr)\text{, ako je }\f{count2}(x)=k-1
\end{equation}
--- iz čega slijedi, po korolaru~\ref{kor:restrprek}, da je $\N\beta\f f$ parcijalno rekurzivna. Ona je RAM-izračunljiva po teoremu~\ref{tm:pir},  pa je $\beta\f f$ Turing-iz\-rač\-un\-lji\-va po teoremu~\ref{tm:krit}.
\end{proof}

%Kao što smo već rekli, dinamizacijom funkcija $\f{arg}_i$ možemo simulirati i funkcije koje primaju proizvoljan broj argumenata (\emph{varargs}). Definiramo pomoćne primitivno rekurzivne funkcije
%\begin{align}
%\SwapAboveDisplaySkip
%\f{argc}&:=\f{Sc}\circ\f{count2}\text,\\
%\f{arg}(a,i)&:=\f{streak1}(2\conc2a\conc22,i)\text.
%\end{align}
%
%\begin{primjer}[{name=[implementacija \emph{varargs} pomoću binarnog kodiranja]}]
%Recimo, $\f{add}^{\cdots}$, koja zbraja sve svoje argumente (koliko god da ih ima), je primitivno rekurzivna, jer je funkcija zadana s
%\begin{equation}
%\N\beta\f{add}^{\cdots}(a)=\quad\sum_{\mathclap{i\,<\,\f{argc}(a)}}\f{arg}(a,i)
%\end{equation}
%primitivno rekurzivna po lemi~\ref{lm:sumprodrek} i prethodnim rezultatima.
%\end{primjer}
%To nam neće bitno trebati u nastavku, ali zgodno je znati da imamo i tu mogućnost.

