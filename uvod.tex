\chapter{Uvod}

\section{Neka vrst predgovora}

Već nekoliko godina držim, na Matematičkom odsjeku Prirodoslovno-matematičkog fakulteta u Zagrebu, kolegij Izračunljivost. Kolegij je izvorno uveden kao prirodni nastavak kolegija Matematička logika (dugo je vremena gradivo tog kolegija bilo dijelom gradiva kolegija Matematičke logike), s ciljem dokaza Churchovog teorema o neodlučivosti logike prvog reda. Tako je nastala knjiga~\cite{skr:Vuk}, izdvajanjem iz knjige~\cite{skr:VukML}, koja je bila prvenstveno namijenjena studentima teorijske matematike koji žele produbiti svoje znanje o matematičkoj logici.

U međuvremenu, zbog raznih okolnosti, kolegij su počeli upisivati mahom studenti računarstva, kojima pojam algoritma predstavlja puno općenitiji i intuitivno bliži pojam od onog potrebnog da bi se dokazao Churchov teorem. U suvremenom svijetu okruženi smo računalima raznih vrsta, često ih programiramo da bismo ih prilagodili svojim potrebama, i algoritamske sustave više ne doživljavamo kao nešto apstraktno. Pojam izračunljive funkcije (funkcije implementirane u nekom programskom jeziku) počeo je već u umovima studenata računarstva istiskivati skupovnoteorijsku ideju uređene trojke $(\text{domena},\text{kodomena},\text{graf})$ kao asocijaciju na pojam "funkcija". Rekurzija više nije egzotična matematička konstrukcija, već sasvim uobičajen alat u repertoaru gotovo svakog programera. Jezici više nisu reprezentirani črčkarijama na papiru ili otiscima na traci (kao što su bili u Turingovo vrijeme), već tekstnim datotekama, nizovima bajtova u određenom \emph{encodingu}, koji se sasvim prirodno obrađuju programskim alatima. Strukture više nisu dijagrami matematičkih simbola povezanih strelicama, nego memorijski blokovi objekata povezanih pokazivačima ili referencama. \emph{Halting problem} nije više nešto maglovito i daleko od svakodnevnog iskustva: ta svi smo doživjeli da se računalo privremeno smrzne, i bili u nedoumici koliko dugo čekati prije nego što dobijemo nekakav odziv, ili zaključimo da se permanentno smrznulo i ne preostaje nam drugo doli posegnuti za gumbom za ponovo pokretanje. 

U tom svjetlu, počeli su se pokazivati određeni nedostaci knjige~\cite{skr:Vuk}. Zahvaljujući njenom pokušaju da izgradi matematičku intuiciju, potpuno zaboravljajući ili čak namjerno potiskujući intuiciju koju računarci već imaju o tim pojmovima, konačni učinak za većinu studenata bio je vrlo sličan onom koji je primijetio Eric Mazur~\cite{mazur} u svojoj nastavi fizike:
\begin{quotation}
    \emph{Professor Mazur, how should I answer these questions: according to what you taught me, or according to the way I usually think about these things?}
\end{quotation}

Nažalost, znao sam i ja dobivati takva pitanja. Ili sam jednostavno uočio da studenti pri programiranju koriste jedan mentalni model, a pri rješavanju zadataka sasvim drugačiji. I naravno, pritom čine puno više početničkih pogrešaka --- jer taj drugi model izgrađuju tek nekoliko mjeseci, dok prvi izgrađuju desetak godina.

Ova knjiga pokušaj je ispravljanja tog dojma. \textbf{Ne postoje dva svijeta}, svijet modernog računarstva i svijet klasične teorije izračunljivosti. To je jedan te isti svijet, samo što je u matematičkom modelu pojednostavljen (slično kao i model njutnovske mehanike: vakuum, linearno trenje, materijalne točke,~\ldots) --- ali svi bitni pojmovi teorijskog računarstva se u njemu mogu modelirati, svaka intuicija se može validirati, i svaki fenomen se može uočiti. Ako ste "računarac u duši", sve potrebne ideje već imate. I najvažnije, sistematizacija i razumijevanje koje iz toga proizlaze su nezamjenjivi.

Što ako \emph{niste} računarac u duši? Knjiga~\cite{skr:Vuk} je fantastična za izgradnju matematičke intuicije. Praktički jedini njen nedostatak je invalidacija računarske intuicije --- ako tu intuiciju nemate, nedostatka nema. Izuzetno sam se trudio održati \emph{backward compatibility}, tako da čak možete neke pojmove naučiti otamo, a neke odavdje.

\subsection{O knjizi}

Iako je knjigu sasvim moguće isprintati na papir i šarati po njoj, prvenstveno je namijenjena digitalnom čitanju. Zato ima puno referenci (označenih posebnom bojom), na svaku od kojih je moguće kliknuti da bi se vidjelo na što se odnosi. Ako koristite "pravi" PDF čitač (za razliku od ovih što dođu s \emph{browserima}), možete se i vratiti tipkom \keys{\!$\Mapsfrom$}, ili kombinacijom \keys{Alt+\arrowkeyleft}. Pravi čitač omogućuje vam i navigaciju kroz naslove, bolji \emph{rendering}, označavanje omiljenih stranica i još neke stvari zbog kojih se svakako isplati instalirati ga. Moja preporuka je \textsf{SumatraPDF} na Windowsima, \textsf{Okular} na Linuxu, te \textsf{Document Viewer} na Androidu. Naravno, ako imate dovoljno RAM-a, \textsf{Adobe Reader} je također opcija. Ako baš morate čitati u internetskom pregledniku, čujem da \textsf{Firefox} ima relativno dobar \emph{plugin}.

Kao što vjerojatno primjećujete, knjiga je pisana u \textsf{\LaTeX{}}u, klasa \textsf{KOMA-Script book}, koristeći internetsku uslugu \emph{Overleaf} (v2), koja je vjerojatno najbolji besplatni način za produkciju visokokvalitetnih dokumenata "u oblaku". Fontovi su Knuthov Concrete (iz serije \emph{Concrete Mathematics}), a za matematiku Zapfov \AmS{} $Euler$. Ako vas zanima išta detaljnije o produkciji knjige, možete mi pisati na \href{mailto:veky@math.hr}{veky@math.hr} --- ili pogledati u (ne sasvim ažuran) repozitorij na \href{https://github.com/vedgar/izr}{github.com/vedgar/izr}. Naravno, ako uočite bilo kakvu grešku u knjizi, ili smatrate da bi nešto trebalo drugačije prezentirati, pošaljite \emph{email} ili \emph{pull request}.

Duljina knjige prvenstveno je posljedica moje želje da gotovo sve dokaze raspišem do sitnih detalja, kako bih sebi i vama pokazao da ništa nije provučeno "ispod stola", te da biste uočili da osnovnih ideja zapravo nema puno. Ogroman broj dokaza provodi se matematičkom indukcijom, rastavom na slučajeve, ili pak eksplicitnom konstrukcijom (\emph{programiranjem}) objekata koji zadovoljavaju traženu specifikaciju. Ako Vam se bilo koji dokaz učini prelaganim (ili preteškim), ne morate ga čitati --- ali tako se izlažete riziku da propustite uočiti sličnu ideju u nekom idućem dokazu. Drugi uzrok veličine knjige je velik broj primjera. Vjerojatno najčešći prijedlog studenata za poboljšanje nastave na evaluacijskim anketama bio je da stavim više primjera. Čuo sam vas, i nadam se da je ovo dovoljno --- ako nije, recite.

Sastavni dio ove knjige je i Zbirka zadataka, u kojoj se nalaze brojni zadaci, od šablonskih za vježbu, preko svih zadataka sa starih kolokvija i pismenih ispita do kojih sam uspio doći (pri čemu se zahvaljujem bivšim asistentima iz Izračunljivosti, Zvonku Iljazoviću i Marku Doki, na ustupanju zadataka), pa do zadataka koji na određeni način nadopunjuju teoriju izloženu u ovoj knjizi, ali nisu nužni za njeno razumijevanje.

Nadimak ove knjige --- \emph{Computonomicon} --- relativno je doslovni prijevod njene svrhe: prikaz ({\textgreekfont\textepsilon\textiota\textkappa\'o\textnu\textalpha}) zakona (\textnu\'o\textmugreek\textomikron\textvarsigma) računanja (\textsc{compvtvs}). Miješanje grčkih i latinskih korijena je omiljena razonoda računaraca --- promotrite pridjev "heksadecimalni".

Zahvaljujem se kolegama i studentima koji su čitali rane \emph{draftove} ove knjige, i brojnim prijedlozima pridonijeli njenom poboljšanju. Među njima svakako bih istaknuo Marka Horvata, koji još uvijek nije uvjeren da je računarac, ali je pristao igrati tu ulogu za potrebe čitanja ove knjige.

\subsection{Motivacija}

Izračunljivost: matematička obrada pojma algoritma. Čemu to služi? Zar ne znamo pisati algoritme i bez matematičke formalizacije? Koga je zapravo briga za definicije poput "Algoritam je uređena sedmorka, čiji elementi su skupovi~\ldots", i beskorisne propozicije poput "Postoji algoritam za sortiranje liste brojeva"? Dva su moguća odgovora na to pitanje: praktični i teorijski.

Kažemo da znamo pisati algoritme. Ali kako to činimo? Zapravo ih najčešće \emph{implementiramo} u nekom programskom jeziku, prešutno podrazumijevajući da je ono što taj jezik omogućava izraziti, ni više ni manje nego algoritam. S tim shvaćanjem postoje dva problema. Prvo, programski jezici nastaju godišnjim ritmom, a jezik koji postane toliko popularan da se u njemu počnu pisati općeniti algoritmi, nastane možda svakih desetak godina. Zatrpani smo knjigama koje objašnjavaju besmrtne algoritamske koncepte, pokušavajući nam ih "približiti" implementirajući ih u jeziku koji je odavno mrtav. Neke od tih knjiga su toliko popularne da su autori gotovo primorani pisati nova izdanja, u kojima su algoritmi potpuno isti, ali je programski jezik promijenjen. Tu se krije implicitna pretpostavka da, što god jedan programski jezik može izraziti, može i drugi. No kako možemo biti sigurni u to? Na primjer, originalni FORTRAN nije dopuštao rekurziju, dok ALGOL jest~\cite{url:recursionAlgol}. Znamo li da se svaki rekurzivni algoritam može zapisati nerekurzivno? Možemo li to dokazati? Također, ako jest tako, zašto cijelo vrijeme stvaramo nove jezike? Razuman odgovor je da se razlikuju u \emph{nečem drugom}, ne u algoritmima koje prezentiraju. Možemo li "to drugo" eliminirati, svodeći algoritme samo na "čistu esenciju"?

Drugi problem sastoji se u tome da programski jezici nastaju s raznim svrhama, ali izuzetno rijetko s primarnom svrhom modeliranja matematičkih objekata. Još rjeđe takvi jezici postanu planetarno popularni. Popularni jezici su obično opterećeni performansama: optimalnom upotrebom procesora (vremena) i memorije (prostora), i kao posljedica toga njihov dizajn čini razne ustupke hardveru, koji se teško mogu matematički opravdati. Izuzetno je česta pojava, na primjer, da cijele brojeve računala ne reprezentiraju kao elemente skupa $\mathbb Z$, već kao elemente skupa $\mathbb Z\big\slash2^{64}\mathbb Z$. Također, često se instrukcije ne izvršavaju redom kojim se nalaze u izvornom kodu, u svrhu bržeg izvršavanja na višejezgrenim procesorima. Tako nastaje raskorak između algoritma i implementacije, koji ima važne praktične posljedice~\cite{url:wrongBinsearch}.

Teorijski odgovor na motivacijsko pitanje dobijemo kad se zapitamo što, u općenitom smislu, matematičare navede na formalizaciju nekog pojma. Ponekad je to otkriće paradoksa, ali češće se radi o potrebi da se dokaže \emph{nepostojanje} objekta neke klase $K$ s nekim svojstvima. Iako smo se za dokaze postojanja mogli osloniti na intuitivni osjećaj da objekte klase $K$ "prepoznamo kad ih vidimo", to nam očito više nije dovoljno ako slutimo da željeni objekt ne postoji, i želimo to dokazati. A u svakom se području s vremenom pojave problemi koji odolijevaju svim poznatim metodama, i počne se vjerovati da su možda nerješivi.

Dok nitko nije dovodio u pitanje Euklidove konstrukcije, puno je preciznija formulacija geometrijske konstruktibilnosti bila potrebna da se dokaže da je trisekcija kuta ravnalom i šestarom nemoguća. Dok je za pronalazak Cardanove ili Ferrarijeve formule bilo dovoljno znati nekoliko jednostavnih algebarskih manipulacija, tek je Galoisova teorija omogućila dokaz da analogni postupci nisu mogući za algebarske jednadžbe petog i višeg stupnja. Dok je već Galileo vidio da prirodnih brojeva i njihovih kvadrata ima jednako mnogo koristeći intuitivni pojam bijekcije, bitno je stroža formulacija bila potrebna Cantoru za dijagonalni argument kojim je dokazao da bijekcija između $\N$ i $\mathbb R$ ne postoji. Na meta-razini također: Cantor je uspio naći dokaze za mnoge tvrdnje ili njihove negacije u svojoj teoriji skupova, ali tek je formalna aksiomatizacija omogućila da se dokaže da takvi dokazi za neke tvrdnje (kao što je hipoteza kontinuuma), niti za njihove negacije, jednostavno ne postoje.

Od davnina bio je poznat problem rješavanja \emph{diofantskih jednadžbi}, koji se sastoji u traženju prirodnih brojeva koji zajedno s još nekim fiksnim prirodnim brojevima, zbrajanjem i množenjem, čine dva izraza jednakima. Modernim jezikom, zadan je polinom s cjelobrojnim koeficijentima u $k$ varijabli (recimo $x_2^3-x_1^2-1$), i želimo ustanoviti ima li nultočku u $\N^k$ --- ili u $\mathbb Z^k$, što se može svesti na prirodni slučaj. Za mnoge specijalne polinome znali smo odgovor, za mnoge specijalne potklase (recimo kad je broj varijabli $k$, ili stupanj polinoma, jednak $1$) poznavali smo od davnina algoritme za nalaženje nultočaka, ali opći algoritam, koji bi za svaki takav polinom u konačno mnogo koraka odgovarao na pitanje ima li prirodnu nultočku, nismo imali. Na slavnoj Hilbertovoj listi od 23 velika matematička problema, deseti je pronalazak takvog algoritma. Protokom vremena, iskristalizirala se mogućnost da algoritam ne postoji, ali za pravi dokaz toga trebalo je prvo formalizirati pojam algoritma. Nakon što je to učinjeno, relativno brzo (uzevši u obzir da su diofantske jednadžbe bile poznate tisućama godina) je riješen i deseti Hilbertov problem --- naravno, dokazom nepostojanja takvog algoritma.

Nije to bio jedini takav problem: nađeni su brojni drugi problemi za koje se sličnim metodama dokazalo da su algoritamski nerješivi. Danas znamo da je neizračunljivost "posvuda", i nismo njome više toliko fascinirani, ali to je samo znak ogromnog puta koji smo prešli u shvaćanju algoritama tijekom dvadesetog stoljeća. Jedan dio tog puta prikazan je u ovoj knjizi.

\section{Pojam izračunljive funkcije}

Da bismo odgovorili na pitanje što je algoritam, zapitajmo se za početak što algoritam \emph{radi}  --- ili, što mi algoritmom radimo. Očito, algoritam možemo \emph{pokrenuti} na nekim \emph{ulaznim podacima}, izvršavati njegove korake preciznim redom, te u nekom trenutku, kad algoritam to zatraži, zaustaviti postupak, i dobiti \emph{izlazne podatke}. Algoritam, u tom pogledu, obavlja nekakvu \emph{transformaciju} podataka. Štoviše, algoritam bi trebao biti \emph{deterministički}: isti ulazni podaci trebali bi proizvesti iste izlazne podatke. Matematička formalizacija te transformacije je pojam \emph{funkcije}: algoritam \emph{preslikava} ulazne podatke u izlazne. Kažemo da algoritam \emph{računa} funkciju, i takve funkcije (za koje imamo algoritme) zovemo \emph{izračunljivima}. Tako pitanje "Koji su algoritmi mogući?" postaje nešto preciznije pitanje "Koje su funkcije izračunljive?".

% \subsection{Vrste i količina podataka}

No da bismo funkciju mogli matematički zapisati, moramo imati preciziranu domenu i kodomenu. Što su naši podaci?
Na prvi pogled, mogu biti bilo što: imamo algoritme koji rade na cijelim brojevima, realnim brojevima (preciznije, njihovim aproksimacijama --- vidjet ćemo zašto je to bitno), tekstnim podacima (\emph{strings}), datotekama, mrežnim vezama (\emph{sockets}), drugim algoritmima (\emph{higher order programming}), grafovima, objektima, regularnim izrazima, i tko zna čemu. No iskustvo programiranja nas uči da se svi ti raznorazni \emph{tipovi} podataka uvijek mogu --- i moraju, ako želimo nešto raditi s njima --- reprezentirati u memoriji računala kao neki binarni podaci: konačni nizovi nula i jedinica.

Dakle, mogli bismo uzeti $\{0,1\}^*:=\bigcup_{k\in\mathbb N}\,\{0,1\}^k$ kao univerzalni skup naših podataka --- ali pokazuje se da je zgodnije ako umjesto $\{0,1\}$ uzmemo proizvoljni konačni neprazni skup ("abecedu") $\Sigma$. Funkcije iz $\Sigma^*$ u $\Sigma^*$ zovemo \emph{jezične funkcije}, i to je povijesno bio prvi pokušaj formalizacije algoritma: Turingov stroj, kojim ćemo se baviti u poglavlju~\ref{ch:Turing}.

Ipak, skup $\{0,1\}^*$, kao i općeniti $\Sigma^*$, matematički je nespretan; recimo, ako hoćemo nešto o njemu dokazati indukcijom, moramo u koraku posebno razmatrati dodavanje nule, a posebno dodavanje jedinice. Ako hoćemo napraviti neku petlju kroz njega, nije baš lako odrediti sljedbenika zadanog elementa. Nezgoda je i u tome što uobičajenim leksikografskim uređajem nije dobro uređen: na primjer, skup $\{0^n1\mid n\in\N\}$ nema najmanji element.

Za dokazivanje teorema bolje je uzeti jednostavniji skup, te ćemo u najvećem dijelu knjige promatrati \emph{brojevne} funkcije, za koje su ulazni i izlazni podaci \textbf{prirodni brojevi}. U nekom smislu, skup prirodnih brojeva je najjednostavniji mogući skup na kojem se može raditi teorija izračunljivosti --- svakako je najjednostavniji među beskonačnim skupovima, a izračunljivost na konačnim skupovima je trivijalna: svaki algoritam može se napisati jednostavno kao tablica (\emph{lookup table}).

Odabir skupa $\N$ kao osnovnog isplatit će se kroz jednostavnost mnogih dokaza (jer imamo matematičku indukciju, jasan početak i sljedbenika, dobar uređaj,~\ldots), ali s druge strane, zato će biti kompliciranije \emph{kodirati} razne druge matematičke objekte kao prirodne brojeve. Za usporedbu, skup $\Sigma^*$ je zgodniji za kodiranje, jer već imamo intuitivnu predodžbu raznih objekata kao nizova znakova ($\Sigma=$ ASCII): recimo, nitko nam ne mora objasniti kodiranje da bismo znali koji element od $\mathbb Q$ predstavlja \t{'-22/3'}.

Ipak, neintuitivnost kodiranja nadomjestit će lakoća pisanja algoritama: dok je, uz odgovarajuće tehnike (koje ćemo objasniti), lako napisati algoritam za npr.\ zbrajanje racionalnih brojeva kodiranih prirodnim brojevima, odgovarajući algoritam za ASCII-kodirane razlomke gotovo nikada ne stigne dalje od grubog pseudokoda. Treba reći da ćemo ponegdje, gdje su naši objekti već \emph{definirani} kao nizovi znakova (najvažniji primjer su formule logike prvog reda), svakako koristiti njihovu jezičnu reprezentaciju, ali to će biti nakon što objasnimo općenito kodiranje sa $\Sigma^*$ u $\N$ (i obrnuto).

Tehnički detalj oko kojeg se moramo dogovoriti je: smatramo li nulu prirodnim brojem? Treba li brojenje početi od $0$ ili od $1$, dilema je stara koliko i samo računarstvo~\cite{note:EWD831}. Kao i drugdje u matematici, postoje dobri razlozi i za i protiv. Koristit ćemo oba skupa, no kako će nam češće trebati nula među prirodnim brojevima (pogledajmo definiciju od $\{0,1\}^*$, na primjer), skup s nulom imat će kraću oznaku.
\begin{align}
\N&:=\{\,0,1,2,3,\dotsc\}\\
\N_+&:=\{\,1,2,3,4,\dotsc\}
\end{align}

%\subsection{Broj izlaznih i ulaznih podataka}\label{sec:briup}

\begin{napomena}\label{nap:brip}
Pričali smo o izlaznim podacima u množini, no lako je vidjeti da --- s obzirom na to da nas samo zanima postojanje algoritma, ne i njegove performanse --- ništa ne gubimo fiksiranjem broja izlaznih podataka na $1$. Algoritam s $k$ ulaznih i $l$ izlaznih podataka uvijek možemo promatrati kao $l$ algoritama s istih $k$ ulaznih podataka i s po jednim izlaznim podatkom.

Na primjer, u nekim programskim jezicima postoji operacija $\f{divmod}$ iz $\mathbb Z^2$ u $\mathbb Z^2$, koja provodi dijeljenje s ostatkom u $\mathbb Z$, te vraća količnik i ostatak. Nju uvijek možemo, čak i da je nemamo kao osnovnu, emulirati pomoću dvije operacije, $\sslash$ i $\bmod$, koje vraćaju količnik i ostatak istog dijeljenja zasebno. Naravno, razlog zašto neki jezici imaju $\f{divmod}$ kao posebnu funkciju leži u tome da algoritmi za te dvije operacije imaju mnogo zajedničkih koraka, te ako smo odredili npr.\ količnik, obično je lako iz postupka kojim smo to učinili pročitati i ostatak (sjetite se npr.\ algoritma za dijeljenje višeznamenkastih brojeva). Zato bismo ponovnim provođenjem algoritma ispočetka za ostatak nepotrebno duplicirali korake. No ako nas samo zanima koje su funkcije izračunljive, očito postoji algoritam za $\f{divmod}$ ako i samo ako postoje algoritmi za $\sslash$ i za $\bmod$, te nam je dovoljno baviti se pitanjem jesu li \emph{koordinatne funkcije} $\sslash$ i $\bmod$ izračunljive --- u ovom slučaju, dakako, jesu.
\end{napomena}

Kad promatramo broj \emph{ulaznih} podataka (tzv.\ \emph{mjesnost}) algoritma, situacija je bitno drugačija. Jasno je da algoritam za npr.\ potenciranje prirodnih brojeva prima bazu i eksponent kao dva ulazna podatka, i ne može se jednostavno zapisati pomoću algoritama koji primaju po jedan ulazni podatak. %Doduše, korištenjem tehnike \emph{currying}, svaku izračunljivu funkciju možemo računati pomoću algoritama koji primaju po \emph{dva} ulazna podatka, i neki autori doista ograničavaju mjesnost na $2$, ali to stvara dosta tehničkih problema za minimalni dobitak.
Zato ćemo promatrati algoritme proizvoljnih mjesnosti $k\in\N_+$, i smatrati da mjesnost čini važan dio identiteta algoritma. Primjerice, "zbroji 2 broja" i "zbroji 5 brojeva" su različiti algoritmi, štoviše ovaj prvi pojavljuje se kao korak (nekoliko puta) u ovom drugom.

Direktna posljedica toga je da u našem modelu ne postoje algoritmi s "varijabilnim brojem" ulaznih podataka, u računarstvu poznati kao \emph{varargs}. Na nekoliko mjesta gdje nam budu potrebni, modelirat ćemo ih pomoću familije algoritama svih mogućih mjesnosti, recimo zbrajanje kao $\f{add}^k,k\in\N_+$. Mjesnost algoritma ili funkcije ćemo obično pisati u superskriptu ako je želimo naglasiti --- neće dolaziti do zabune s eksponentima jer algoritme niti funkcije nećemo potencirati, niti s oznakom $f^{-1}$ za inverznu funkciju jer mjesnost ne može biti negativna.

Iako mjesnost smatramo neodvojivim dijelom funkcije odnosno algoritma, u slučaju nespecificirane mjesnosti $k$ nespretno je pisati $x_1,x_2,\dotsc,x_k$ svugdje gdje trebamo napisati argumente odnosno ulazne podatke. Zato ćemo često tih $k$ prirodnih brojeva skraćeno pisati $\vec x$, ili $\vec x^{\,k}$ ako želimo naglasiti koliko ih ima --- no najčešće će se to moći zaključiti iz konteksta: recimo, u $f^7(\vec x,y,z)$, očito je duljina od $\vec x$ jednaka $5$.

\begin{napomena}\label{nap:blokovi}
S obzirom na to da promatramo samo determinističke algoritme, naglasimo da nema "implicitnih argumenata": sve vrijednosti o kojima ovisi izlaz funkcije (ako se doista mijenjaju od poziva do poziva) moraju biti prenesene u nju kao argumenti. Često ćemo pisati opće funkcijske pozive kao $f(\vec x,y,z)$ --- gdje su $y$ i $z$ "lokalne varijable" s kojima doista nešto radimo u konkretnom algoritmu, a $\vec x$ predstavlja samo kontekst (\emph{environment}) nekog vanjskog algoritma koji je pozvao $f$ --- koji također moramo prenijeti u $f$ ako želimo da mu ona može pristupiti.
\end{napomena}

Pažljiv čitatelj će primijetiti da zahtijevamo da mjesnost bude pozitivan prirodan broj, odnosno ne promatramo algoritme s $0$ ulaznih podataka. Ovo nije bitna restrikcija (možete se zabaviti pokušavajući otkriti koje sve tehničke detalje u knjizi treba promijeniti da bismo uključili i takve algoritme u razmatranje), ali komplicira izlaganje, a opet, takvi algoritmi nam nisu zanimljivi iz perspektive izračunljivosti: budući da zahtijevamo determinističnost, nul-mjesni algoritmi mogu računati jedino konstante, a one su svakako izračunljive bez obzira na formalizam.

\subsection{Parcijalnost}

Gdje smo u dosadašnjem tekstu pričali o općenitim izračunljivim funkcijama, pazili smo da  upotrijebimo prijedlog "iz": funkcija \emph{iz} $A$ u $B$. Općenito u matematici, takva fraza označava \emph{parcijalne} funkcije, koje ne moraju biti definirane u svim točkama od $A$ (precizno, domena im je podskup od $A$). Recimo, tangens je parcijalna funkcija iz $\mathbb R$ u $\mathbb R$, jer $\frac{\pi}{2}\in\mathbb R\setminus \dom{\text{tg}}$. Takve funkcije označavamo oznakom $f\colon A\rightharpoonup B$, za razliku od \emph{totalnih} funkcija koje označavamo $f\colon A\to B$ i zovemo ih funkcije \emph{sa} $A$ u $B$. 

Dopuštajući algoritmima da računaju parcijalne funkcije, zapravo im omogućavamo da je za neke ulazne podatke njihov rad sasvim dobro definiran (dakle, ovdje ne mislimo na izuzetke, \emph{exceptions}, kao što je dijeljenje nulom), ali da ipak ne postoji završna konfiguracija iz koje bismo mogli pročitati izlazni podatak. Nakon malo razmišljanja dolazimo do zaključka da je to jedino moguće tako da algoritam za neke ulaze beskonačno radi, odnosno nikada ne stane.

Nije li to u kontradikciji s naivnom definicijom algoritma, koja kaže da se radi o \emph{konačnom} postupku? Jest, ali to samo pokazuje da naivne definicije nisu dovoljne, i da nam treba formalizacija. Naime, naivna definicija algoritma, baš kao i naivna definicija skupa, vodi na paradoks sličan Russellovom. \emph{Moramo} u obzir uzeti i parcijalne funkcije, odnosno algoritme koji ne stanu uvijek, ako želimo konzistentnu teoriju. Evo kratke skice argumenta --- precizno ćemo ga provesti kad precizno definiramo pojmove.

Budući da želimo algoritme moći reprezentirati u računalu, moramo ih moći prikazati kao konačne nizove nula i jedinica. Ta reprezentacija mora biti injekcija ako želimo išta raditi s tim algoritmima, a iz teorije skupova znamo da je $\{0,1\}^*$ prebrojiv, dakle \textbf{svih algoritama ima prebrojivo mnogo}. Specijalno, svih jednomjesnih algoritama  ima prebrojivo mnogo (naravno da ih ima beskonačno mnogo). Poredajmo sve jednomjesne algoritme u niz. Pogledajmo sada ovaj jednomjesni algoritam: "Za ulaz $x\in\N$, nađi $x$-ti algoritam u nizu, i primijeni ga na $x$. Izlaz tog algoritma (s ulazom $x$) označi sa $y$. Vrati $y+1$." Jasno je da \emph{taj} algoritam ne može biti na popisu, ako algoritmi računaju totalne funkcije (ako je $r$-ti po redu, tada s ulazom $r$ mora davati i $y$ i $y+1$); ali ako računaju parcijalne funkcije, nema kontradikcije --- jednostavno, $y$ može biti nedefiniran, jer $x$-ti algoritam s ulazom $x$ ne mora stati.

Ipak, bitno je  lakše raditi s algoritmima koji računaju totalne funkcije. Parcijalne funkcije moramo dozvoliti u krajnjoj općenitosti, ali mnoge funkcije koje ćemo koristiti u izgradnji teorije bit će ne samo totalne, nego i \emph{sintaksno} totalne unutar teorije koju gradimo: već iz njihovog oblika bit će jasno da algoritmi koji ih računaju uvijek stanu. Takve funkcije zvat ćemo \emph{primitivno rekurzivnima}.

Kad smo već kod toga, recimo nekoliko riječi i o klasičnim izuzecima poput dijeljenja nulom. Primitivno rekurzivne funkcije po definiciji moraju biti totalne, pa si ne možemo priuštiti jednostavno reći nešto poput "$3\sslash0$ nije definirano" (dokazat ćemo da je $\sslash$ primitivno rekurzivna operacija). U mnogim slučajevima to ćemo rješavati jednostavno tako da kažemo "\emph{postoji} primitivno rekurzivna funkcija $\f{f}$ koja se podudara s traženom funkcijom $g$ na domeni $\dom{g}$", ne govoreći ništa o vrijednostima $\f{f}(\vec x)$ za $\vec x\not\in\dom{g}$. Još općenitije, ponekad ćemo umjesto funkcije $g$ navesti samo neko \emph{svojstvo} koje vrijednosti od $\f{f}$ moraju zadovoljavati na nekom skupu. Kazat ćemo tada da smo \emph{parcijalno specificirali} (totalnu) funkciju $\f{f}$: u smislu, $\f{f}$ je definirana svuda, ali nas zanimaju samo vrijednosti na nekom užem skupu.

Treba napomenuti da je važna odlika ove knjige da će svi algoritmi biti precizno specificirani --- ništa neće ostati na pseudokodu. Dakle, uvijek ćemo moći precizno izračunati vrijednosti primitivno rekurzivne funkcije i izvan skupa na kojem je specificirana --- na primjer, algoritam za $\sslash$ reći će nam da je $3\sslash 0=3$. Ponekad će čak te vrijednosti (\emph{undocumented behavior}) imati neko značenje, u smislu da će takvu funkciju biti lakše uklopiti u kasnije definicije bez puno rastava na slučajeve. No to nećemo često koristiti, i svaki put ćemo naglasiti kad se to dogodi.

\subsection{Relacije}

Iako sva računanja možemo shvatiti kao računanja funkcija, izlaganje je jednostavnije ako uvedemo i \emph{relacije}, koje ćemo računati kao specijalni slučaj računanja funkcija. Iz standardne skupovnoteorijske perspektive to se čini čudnim: nisu li relacije općenit pojam, a funkcije samo specijalni slučaj, relacije s funkcijskim svojstvom?

Iz algoritamske perspektive, nisu. Ako izračunljivu funkciju $\f{f}$ reprezentiramo po\-mo\-ću algoritma koji za dani $\vec x$ računa njenu vrijednost $\f{f}(\vec x)$, izračunljivu relaciju $\f{R}$ prirodno je predstaviti algoritmom koji za dani $\vec x$ računa \emph{istinitosnu} vrijednost ($bool$: $\mathit{true}$ ili $\mathit{false}$), već prema tome je li $\vec x\in\f{R}$ ili nije. Većina programskih jezika uopće nema mogućnost programiranja relacija kao zasebnog tipa algoritma, već ih reprezentiraju funkcijama čiji povratni tip je $bool$.

Na primjer, reći da je dvomjesna relacija uređaja $<$ na prirodnim brojevima iz\-ra\-čun\-lji\-va zapravo znači reći da postoji algoritam koji za sve $(x,y)\in\N^2$ u konačno mnogo koraka vraća $\mathit{true}$ ako je $x<y$, a $\mathit{false}$ inače. Ili, skup prim-brojeva (jednomjesna relacija $\mathbb P$) je izračunljiv jer možemo napisati algoritam $\f{isPrime}\colon\N\to bool$, koji za svaki $x$ u konačno mnogo koraka vraća $\mathit{true}$ ako je $x\in\mathbb P$, a $\mathit{false}$ ako $x\not\in\mathbb P$. Iz navedenih primjera vidimo da je o relacijama prirodno katkad pričati pomoću formula s relacijskim simbolima ($\f{R}(\vec x)$, ili $x\mathrel{\f R}y$ za dvomjesne relacije), a katkad pomoću skupova ($\vec x\in\f{R}$).

U skladu s uobičajenom praksom modernih programskih jezika, prešutno koristimo standardno ulaganje skupa $bool$ u $\N$, tako da preslikamo $\mathit{false}\mapsto 0$, te $\mathit{true}\mapsto 1$. Drugim riječima, na izračunljivost relacije $R$ gledamo kao na izračunljivost njene \emph{karakteristične funkcije} $\chi_R$, koja je iste mjesnosti kao i $R$. U suprotnom smjeru, kad želimo interpretirati proizvoljni prirodni broj kao $bool$, koristimo (opet standardnu) interpretaciju po kojoj se $0$ interpretira kao $\mathit{false}$, a svi ostali prirodni brojevi kao $\mathit{true}$: drugim riječima, podrazumijevamo kompoziciju s karakterističnom funkcijom $\chi_{\N_+}$ (za koju ćemo dokazati da je izračunljiva).

Dakle, relacijama ćemo pripisivati svojstva izračunljivosti koja imaju njihove karakteristične funkcije: na primjer, reći ćemo da je $\f{R}$ primitivno rekurzivna ako je $\chi_{\f{R}}$ primitivno rekurzivna. Primijetimo da kod relacija ne moramo razmišljati o parcijalnosti: karakteristična funkcija je uvijek totalna. Relacije imaju drugi način iskazivanja parcijalnosti, takozvanu \emph{rekurzivnu prebrojivost}, o čemu ćemo više reći u poglavlju~\ref{ch:re}.

\section{Notacija}

Iz prethodne točke zaključujemo da ćemo promatrati (algoritme za) dvije vrste funkcija: jezične i brojevne. Brojevne funkcije su nam važnije i uglavnom ćemo raditi s njima, ali na nekoliko mjesta dobro će nam doći i formalizacija izračunljivosti jezičnih funkcija.

Svaka brojevna funkcija je oblika $f\colon S\to\N$, gdje je $S\subseteq\N^k$ za neki $k\in\N_+$. Skraćeno pišemo $f\colon\N^k\rightharpoonup\N$, ili još kraće $f^k$, ako nam nije bitan skup $S=\dom{f}$. Broj $k$ zovemo \emph{mjesnost} funkcije $f$. Svaka brojevna funkcija ima jedinstvenu mjesnost --- osim prazne funkcije $\varnothing$, čija domena je prazan skup $\emptyset$. Radi jednostavnosti izlaganja, smatrat ćemo da i prazne funkcije imaju fiksnu mjesnost, odnosno umjesto jedne funkcije $\varnothing$ promatrat ćemo familiju $\varnothing^k,k\in\N_+$, i smatrat ćemo da su, na primjer, $\varnothing^3$ i $\varnothing^8$ različite funkcije. Formalno, to možemo napraviti tako da nam "funkcija" znači uređen par, kojem je prva komponenta uobičajena reprezentacija funkcije (skup uređenih parova s nekim svojstvom), a druga komponenta mjesnost --- ali nećemo imati potrebu biti toliko formalni, tim više što prazne funkcije nisu zanimljive iz perspektive izračunljivosti: algoritmi koji ih računaju su jednostavno beskonačne petlje.

(Brojevna) relacija je oblika $R\subseteq\N^k$ za neki $k\in\N_+$. Po analogiji s funkcijama, $k$ zovemo \emph{mjesnost} relacije, i pišemo $R^k$ ako ga želimo naglasiti. Kao i za funkcije, iako su sve prazne relacije skupovno jednake (postoji samo jedan prazan skup), promatrat ćemo familiju $\emptyset^k,k\in\N_+$, i smatrati sve njene elemente različitim relacijama. Na kraju krajeva, njihove karakteristične funkcije \emph{jesu} različite, jer imaju različite domene: recimo, $\dom{\chi_{\emptyset^3}}\!=\N^3$. (Radi se o nulfunkciji $\f C_0^3\colon\N^3\to\N$; potrebno je razlikovati nulfunkciju, koja je totalna, od prazne funkcije koja nije definirana nigdje!)

Jezične funkcije ćemo uvijek definirati nad nekom \emph{abecedom} (konačan neprazan skup) $\Sigma$, i to će nam biti funkcije $\varphi\colon\Sigma^*\rightharpoonup\Sigma^*$. Elementi od $\Sigma^*:=\bigcup_{k\in\N}\Sigma^k$ su konačni nizovi \emph{znakova} iz $\Sigma$, koje zovemo \emph{riječi} i pišemo jednostavno konkatenacijom: recimo, $\t{aab}$ umjesto $(\t{a},\t{a},\t{b})$. \emph{Prazna riječ} je niz duljine $0$: označavamo je s $\varepsilon$. Iz oblika jezičnih funkcija vidimo da su one jednomjesne: u svrhu reprezentacije funkcija veće mjesnosti, obično se abecedi dodaje \emph{separator}, znak koji služi razdvajanju argumenata. Recimo, višemjesne funkcije nad $\{\t{a},\t{b}\}$ možemo reprezentirati kao jednomjesne funkcije nad tročlanom abecedom $\{\t{a},\t{b},\t{,}\}$ --- tako da primjerice $\varphi^4(\t{a},\t{abb},\varepsilon,\t{ba})$ računamo kao $\dot\varphi^1(\t{a{},{}a{}b{}b{},{},{}b{}a})$. Kažemo da je $\dot\varphi$ dobivena \emph{kontrakcijom} iz $\varphi$.
Primijetimo da je ovo generalnije od brojevnih višemjesnih funkcija jer možemo imati \emph{varargs} (mjesnost možemo zaključiti jednostavnim brojenjem separatora u ulaznoj ri\-je\-či), ali i dalje nemamo mogućnost prikazivanja nulmjesnih funkcija: $\dot\varphi(\varepsilon)$ je jednostavno $\varphi^1(\varepsilon)$, ne $\varphi^0()$. Sličan trik možemo primijeniti i kod brojevnih funkcija, nakon što definiramo kodiranje skupa $\N^*$.

Analogon pojmu relacije u jezičnom slučaju, dakle podskup od $\Sigma^*$, zove se jednostavno \emph{jezik}. Iako karakteristična funkcija jezika nije ni brojevna ni jezična funkcija (ide sa $\Sigma^*$ u $bool$), svejedno možemo pomoću kodiranja skupa $\Sigma^*$ reprezentirati i izračunljivost jezikâ. O tome ćemo također više reći kasnije.

Domenu, sliku i graf funkcije $f$ označavamo redom sa $\dom{f}$, $\im{f}$ i $\graf{f}$. Primijetimo da su to sve relacije: za funkciju mjesnosti $k$, domena je mjesnosti $k$, slika je mjesnosti $1$, a graf je mjesnosti $k+1$. Naravno, iste oznake koristimo i za domene, slike i grafove funkcija koje nisu brojevne. Restrikciju funkcije $f$ na skup $S$ (zapravo na $S\cap\dom f$) označavamo s $f|_S$. Sliku te restrikcije za $S\subseteq\dom{f}$ označavamo s $f[S]:=\{f(x)\mid x\in S\}$. Prasliku skupa $T$ označavamo s $f^{-1}[T]:=\{x\in\dom{f}\mid f(x)\in T\}$.

Brojevne izračunljive funkcije i relacije označavamo posebnim fontom: dok nam $g$ označava proizvoljnu funkciju, $\f{g}$ nam označava funkciju za koju imamo neku vrstu algoritma. U tom smislu, $g(x)$ označava uobičajenu funkcijsku vrijednost (drugu komponentu uređenog para u $g$ čija je prva komponenta $x$), dok $\f{g}(x)$ označava izlazni podatak algoritma za $\f{g}$ pokrenutog s ulaznim podatkom $x$.

Za funkcije i relacije koristit ćemo uobičajene matematičke oznake gdje god možemo: pisat ćemo $x+y+z$ za zbroj tri broja, ili $y\mid x$ za djeljivost, ili $p\in\mathbb P$ za prim-brojeve. No treba imati na umu da su to izračunljive funkcije i relacije (što ćemo dokazati), te da u pozadini stoje algoritmi za $\f{add}^3$, $\f{Divides}^2$, odnosno $\f{isPrime}^1$.

\begin{napomena}\label{nap:parcdef}
Algoritamsku jednakost (izračunljivu dvomjesnu brojevnu relaciju, koja se u modernim programskim jezicima često označava `\t{==}') oz\-na\-ča\-va\-mo uobičajenim simbolom `$=$', koji i inače koristimo za jednakost matematičkih objekata (funkcija, relacija, skupova,~\ldots). Kod definicija skupova, i funkcija s prethodno specificiranom domenom (što uključuje totalne funkcije), koristimo simbol `$:=$'. Relacije definiramo formulama koristeći `$:\Longleftrightarrow$'. Često imamo potrebu vrijednosti funkcije specificirati izrazom, uz prešutnu pretpostavku "prirodne domene" (sve ulazne vrijednosti za koje izraz ima smisla). Tada pišemo $f(\vec x):\simeq izraz$. Ovisno o obliku izraza, definirat ćemo precizno značenje fraze "ima smisla".
\end{napomena}

Ponekad ćemo imati potrebu korištenja znaka $\simeq$ između dva izraza, što će značiti da su oni jednaki za one vrijednosti varijabli za koje imaju smisla, te da oba izraza imaju smisla za iste vrijednosti varijabli. Drugačije rečeno, $izraz1\simeq izraz2$ znači da su definicije $f(\vec x):\simeq izraz1$ i $f(\vec x):\simeq izraz2$ ekvivalentne (definiraju istu funkciju), gdje su u $\vec x$ sve varijable koje se pojavljuju bilo u $izraz1$, bilo u $izraz2$. Razlog za izbjegavanje korištenja znaka = i u takvom slučaju leži u tome što relacija $\simeq$, kao i svojstvo "imati smisla", nisu izračunljive. Još jedan razlog za korištenje neuobičajenog znaka je što mnoga "instinktivna pojednostavljenja" više nisu validna: na primjer, ako je $f^3$ totalna a $g^3$ nije, $f(\vec x)+0\cdot g(\vec x)\not\simeq f(\vec x)$ --- jer izraz zdesna ima smisla za sve $\vec x\in\N^3$, dok izraz slijeva ima smisla samo za $\vec x\in\dom{g}$.

Za skupove brojeva koristimo standardne oznake $\mathbb P\subset
\mathbb N\subset
\mathbb Z\subset
\mathbb Q\subset
\mathbb R$. Često koristimo \emph{diskretne intervale}, koje označavamo $[a..b]$ ili $[a..b\rangle$, gdje su $a,b\in\mathbb Z$. Svaki takav interval skup je cijelih brojeva iz odgovarajućeg realnog intervala: na primjer, $[1..5\rangle=[1..4]=\{1,2,3,4\}$.

\section{Opća i univerzalna izračunljivost}

Jedan od velikih ciljeva teorije izračunljivosti je pokazati da pojam izračunljive funkcije zapravo ne ovisi o podlozi na kojoj se njen algoritam izvršava. Iako je lako naći prejednostavne sustave (kao što su na primjer konačni automati, koji ne mogu čak niti uspoređivati proizvoljno velike prirodne brojeve), nakon neke točke dovoljne kompleksnosti svi mehanički sustavi postaju \emph{ekvivalentni} po pitanju toga koje funkcije, uz razumno kodiranje njihovih ulaza i izlaza, računaju. To se vidi iz činjenice da je svaki od njih sposoban \emph{simulirati} sve druge, odnosno reprezentirati njihove konfiguracije (ili bar njihove kodove) unutar svojih, i izvršavati korake njihovog računanja kao (možda komplicirane) procedure na svojim konfiguracijama. Suvremeno računarstvo poznaje taj fenomen pod imenom "virtualizacija". \emph{Church--\!Turingova teza} ide i dalje: kaže da se ne samo svi algoritmi izvršivi na svim formalnim modelima izračunljivosti, već i svi "intuitivno zamislivi" algoritmi, mogu izvršavati na nekom konkretnom modelu izračunljivosti, primjerice na Turingovom stroju. Tu tezu je očito nemoguće formalno dokazati --- bar dok se ne dogovorimo oko općih aksioma izračunljivosti~\cite{dershowitz} --- ali svaki dokaz ekvivalencije raznih sustava izračunljivosti pruža dodatnu empirijsku potvrdu za nju.

Drugim riječima, izračunljivost je \emph{opći} fenomen: u kojem god modelu da je definiramo, ona će obuhvatiti iste funkcije --- ili bar iste s obzirom na prirodno kodiranje ulaza i izlaza. Na primjer, algoritmi za zbrajanje dekadski zapisanih i binarno zapisanih prirodnih brojeva očito su različiti, ali jednako tako je očito da su oba zapravo samo reprezentacije brojevne funkcije $\f{add}^2$, s obzirom na različite zapise (dekadski odnosno binarni) samih prirodnih brojeva.

Također, jedan od tih modela (pa onda i svi ostali, putem simulacije) zapravo posjeduje svojstvo \emph{univerzalnosti}: ne samo da je za svaku izračunljivu funkciju moguće naći algoritam unutar tog modela, već je moguće naći \emph{jedan} algoritam koji, ovisno o ulazima, može računati \emph{sve} izračunljive funkcije, odnosno može simulirati sve ostale algoritme. Štoviše, ta "granica dovoljne kompleksnosti", na kojoj se postiže opća izračunljivost i univerzalnost, je za neke modele začuđujuće nisko. Promotrit ćemo tri vrste takvih sustava: RAM-strojeve, Turingove strojeve, te parcijalno rekurzivne funkcije.