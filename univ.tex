\chapter{Univerzalna izračunljivost}\label{ch:univ}

\section{Kodiranje}

Napokon smo napravili dovoljno alata da možemo i prilično komplicirane funkcije dokazati primitivno rekurzivnima. Sljedeći veliki zalogaj koji ćemo uzeti je kodiranje, specijalno kodiranje konačnih nizova prirodnih brojeva (skraćeno, kodiranje $\N^*$).

Zašto baš to kodiranje? Dva su razloga. Prvo, trebat će nam za opis rada RAM-stroja, tako da možemo raditi s konfiguracijama i izračunavanjima kao s ulaznim podacima. Recimo, imat ćemo funkciju $\f U$ koja će primiti izračunavanje koje je stalo, i vratiti izlazni podatak (rezultat) tog izračunavanja. Ubuduće ćemo često tako neformalno govoriti: "funkcija prima izračunavanje", ili "funkcija vraća RAM-program", misleći  pritom na kanonsko kodiranje izračunavanja odnosno RAM-programa. Znamo da su RAM-programi konačni nizovi instrukcija, a izračunavanja koja stanu se također mogu pamtiti samo do završne konfiguracije kao konačni nizovi konfiguracija. Same pak konfiguracije također se mogu pamtiti kao konačni nizovi --- vrijednost programskog brojača i sadržaj samo relevantnih registara. Vidjet ćemo da se mnogi složeni objekti koje ćemo željeti kodirati mogu prikazati kao konačni nizovi jednostavnijih objekata, te ako već imamo kodiranje tih jednostavnijih objekata, kodiranje $\N^*$ dat će nam odmah mogućnost kodiranja složenih objekata.

Drugi razlog je jednostavnost specifikacije. Kodiranja su zapravo opisana algoritmima, čiji izlazni podaci su prirodni brojevi, a ulazni podaci su \emph{nešto drugo} osim prirodnih brojeva. Općenito može biti komplicirano specificirati takav algoritam, jer on praktički po definiciji ne može biti formalni algoritam (recimo, RAM-algoritam) --- ulazni podaci mu nisu prirodni brojevi, a da bismo ih prikazali kao prirodne brojeve, trebamo upravo kodiranje koje pokušavamo implementirati! Kodiranje $\N^*$ pruža jednostavan izlaz iz tog začaranog kruga, jer imamo formalizaciju algoritama koji primaju konačne nizove prirodnih brojeva: za fiksnu duljinu niza ($k$-torke) to su jednostavno $k$-mjesni algoritmi, a za proizvoljnu duljinu niza bit će to familija algoritama $\f{Code}^k,k\in\N_+$ (i još jedna konstanta kao kod praznog niza, koja igra ulogu $\f{Code}^0$). Ipak, prvo pokušajmo preciznije definirati općenita kodiranja.

\begin{definicija}
Neka je $\mathcal K$ neki skup (koji ne sadrži prirodne brojeve nego neku drugu vrstu objekata). \emph{Kodiranje skupa $\mathcal K$} je izračunljiva (totalna) injekcija $\N\mathcal K\colon\mathcal K\to\N$, kojoj je slika (skup svih kodova) $\im{\N\mathcal K}$ rekurzivan skup, a parcijalni inverz (lijevi inverz s obzirom na kompoziciju) ${\N\mathcal K}^{-1}\colon\N\rightharpoonup\mathcal K$, s domenom $\dom{{\N\mathcal K}^{-1}}=\im{\N\mathcal K}$, je također izračunljiva funkcija.
\end{definicija}

Napominjemo da ${\N\mathcal K}$ i ${\N\mathcal K}^{-1}$ jesu izračunljive, dakle imamo (neformalne) algoritme za njih --- ali to nisu brojevne funkcije, pa te algoritme nemamo u formalnom smislu (recimo, nema smisla reći "$\N\mathcal K\in\mathcal Comp$"). Ipak, možemo biti precizni u pogledu izračunljivosti skupa $\im{\N\mathcal K}$: kako je to obični podskup od $\N$, dakle jednomjesna relacija, zahtijevamo da njena karakteristična funkcija bude rekurzivna.

Pokušajmo još malo preciznije odrediti što mislimo pod izračunljivošću funkcija ${\N\mathcal K}$ i ${\N\mathcal K}^{-1}$. Zamislimo da imamo funkciju $g\colon\mathcal K\rightharpoonup\mathcal K$, za koju želimo utvrditi je li izračunljiva. Tada možemo na $\im{\N\mathcal K}$ definirati tzv.\ \emph{prateću} funkciju (\emph{tracking function}) $\N g:={\N\mathcal K}\circ g\circ{\N\mathcal K}^{-1}$, koja uzme kod $c$, iz njega odredi jedinstveni $\kappa\in\mathcal K$ takav da je ${\N\mathcal K}(\kappa)=c$, primijeni $g$ na $\kappa$, te rezultat (ako je definiran, odnosno ako je $\kappa\in\dom g$) kodira natrag funkcijom ${\N\mathcal K}$. Ključno je primijetiti da je prateća funkcija uvijek brojevna. Ako je ona izračunljiva u nekom smislu (recimo parcijalno rekurzivna), te je kodiranje relativno kanonsko, prirodno je smatrati funkciju $g$ izračunljivom u tom istom smislu. Primijetimo samo da ${\N\mathcal K}\circ g\circ{\N\mathcal K}^{-1}$ ne možemo smatrati simboličkom definicijom prateće funkcije, jer to nije kompozicija brojevnih funkcija. Moramo nekako drugačije, samo koristeći prirodne brojeve, karakterizirati prateću funkciju. Ako to uspijemo za dovoljno intuitivno izračunljivih funkcija $g$, to je argument za tvrdnju da imamo izračunljivo kodiranje.

Slično možemo činiti za funkcije iz $\mathcal K$ u $\N$ (samo ih komponiramo s ${\N\mathcal K}^{-1}$ zdesna), za funkcije iz $\N$ u $\mathcal K$ (samo ih komponiramo s ${\N\mathcal K}$ slijeva --- primijetite da su karakteristične funkcije specijalni slučaj pratećih funkcija, za $\N bool(\mathit{false}):=0,\N bool(\mathit{true}):=1$), te čak za razne "višemjesne" funkcije iz skupova poput $\mathcal K^2\times \mathcal L\times\N$ (gdje je $\mathcal L$ neki drugi skup čije kodiranje $\N\mathcal L$ već imamo): ulaze dekodiramo (ulaze koji već jesu prirodni brojevi ostavimo nepromijenjene), primijenimo funkciju, te kodiramo rezultat ako je definiran. Puna implementacija tog principa odvela bi nas u \emph{objektno programiranje}, gdje je $\mathcal K$ \emph{klasa}, čije razne \emph{metode} kodiramo na opisani način. Često među tim metodama postoji jedna istaknuta surjekcija na $\mathcal K$ (ili više njih čije slike čine particiju od $\mathcal K$) koju onda zovemo \emph{konstruktor}, te koordinatne funkcije njenog inverza (\emph{getters}, ili jednostavno \emph{komponente}) pomoću kojih možemo napraviti sve ostale metode.

\begin{napomena}
U još jednoj stvari budimo precizni: \textbf{neprebrojive skupove ne možemo kodirati}! Doista, ako postoji kodiranje kao injekcija iz $\mathcal K$ u $\N$, tada mora biti $\card\mathcal K\le\card\N=\aleph_0$. Ovo je izuzetno važno, jer pruža opravdanje za intuiciju da samo konačni i prebrojivi skupovi imaju \emph{totalnu reprezentaciju}: sve njihove elemente možemo reprezentirati u (po volji velikom) računalu. 

Neegzaktnosti \texttt{float} tipa, i raznih drugih tipova koji bi trebali reprezentirati realne brojeve, nisu samo tehnički nedostaci pojedinog standarda (kao što je IEEE\;754): one su fundamentalna posljedica činjenice da $\mathbb R$ kao neprebrojiv skup nema totalnu reprezentaciju. Kako god pokušali~\cite{url:calc}, ne možemo u računalu reprezentirati proizvoljan realan broj, i to nema veze s ograničenom veličinom naših 
računala.
\end{napomena}

\subsection{Kodiranje konačnih nizova}
Sada se možemo pozabaviti samom implementacijom kodiranja $\N^*$. Kako bismo najlakše kodirali $\vec x=(x_1,x_2,\dotsc,x_k)$ kao jedan prirodni broj $n$, iz kojeg se kasnije mogu izvući pojedini brojevi $x_i$? U teoriji skupova, za dokaz da je $\N^2$ prebrojiv, promatra se injekcija $p(x,y):=2^x\cdot3^y$, te se iskoristi osnovni teorem aritmetike (rastav na prim-faktore) da bi se iz $p(x,y)$ natrag dobili $x$ i $y$. To se očito lako može proširiti na proizvoljnu mjesnost (jer prim-brojeva ima beskonačno mnogo --- samo ih uzmemo dovoljno redom po veličini, potenciramo odgovarajućim eksponentima, i pomnožimo), ali to nije kodiranje skupa $\N^*$. Naime, takvo preslikavanje nije injekcija, jer se recimo $(1,2,0,0)$ i $(1,2)$ preslikaju u isti broj $2^1\cdot3^2\cdot5^0\cdot7^0=2^1\cdot3^2=18$.

Ipak, mala modifikacija dat će nam kodiranje. Primijetimo da su jedini problem nule u eksponentu --- opisano preslikavanje \emph{jest} kodiranje skupa ${\N_+\!}^*$ konačnih nizova \emph{pozitivnih} prirodnih brojeva. Sada je još samo preostalo komponirati ga s izračunljivom bijekcijom $\f{Sc}$ između $\N$ i $\N_+$, i dobili smo traženo kodiranje.

\begin{definicija}
Za $i\in\N$, sa $p_i$ označimo $i$-ti po redu prim-broj (počevši od $p_0=2$). Definiramo $\langle\rangle:=1$, te za svaki $k\in\N_+$ definiramo funkciju
\begin{equation}\label{eq:Codek}
    \f{Code}^k(x_1,x_2,\dotsc,x_k):=\langle x_1,x_2,\dotsc,x_k\rangle:=2^{x_1+1}\cdot3^{x_2+1}\dotsm p_{k-1}^{x_k+1}\text.
\end{equation}
Time je definirana funkcija $\langle\,\cdots\rangle\colon\N^*\to\N$.
\end{definicija}

To doduše nije brojevna funkcija jer nema fiksnu mjesnost, ali ipak možemo reći da je izračunljiva u istom smislu u kojem su višestruko zbrajanje i množenje izračunljive operacije po lemi~\ref{lm:addmulk}.

\begin{propozicija}\label{prop:Codekprn}
Za svaki $k\in\N_+$, funkcija $\f{Code}^k$ je primitivno rekurzivna.
\end{propozicija}
\begin{proof}
    Jednostavno, svaki faktor u produktu~\eqref{eq:Codek} možemo prikazati kao kompoziciju potenciranja, konstante i koordinatne projekcije. Tada vanjska kompozicija s $\f{mul}^k$ daje% simboličku definiciju
    \begin{equation}\f{Code}^k=\f{mul}^k\circ\bigl(\f{pow}\circ(\f C_2^k,\f I_1^k),\f{pow}\circ(\f C_3^k,\f I_2^k),\dotsc,\f{pow}\circ(\f C_{p_{k-1}}^k,\f I_k^k)\bigr)\text,
    \end{equation}
    iz koje po propozicijama~\ref{prop:konst} i~\ref{prop:symbdef}, primjeru~\ref{pr:addmulpow}, te lemi~\ref{lm:addmulk} slijedi tvrdnja.
\end{proof}

Ovaj put smo napisali simboličku definiciju da bi bilo jasno da nigdje nismo trebali izračunljivost funkcije $n\mapsto p_n$. To svakako vrijedi, i trebat će nam kasnije, ali za pojedinu funkciju $\f{Code}^k$, dovoljno je da znamo da postoji bar $k$ prim-brojeva, te da su konstante s tim vrijednostima izračunljive.

\begin{propozicija}\label{prop:codeinj}
Preslikavanje $\langle\,\cdots\rangle\colon\N^*\to\N$ je injekcija.
\end{propozicija}
\begin{proof}
    Pretpostavimo da su $\vec x^k,\vec y^{\,l}\in\N^*$ takvi da je $\langle\vec x\rangle=\langle\vec y\rangle$. Ako je $k>l$, tada je $k-1\in\N$, te je $\langle\vec x\rangle$ djeljiv s $p_{k-1}$, dok $\langle\vec y\rangle$ to nije (konkretno, $p_{k-1}\not\in\{p_0,p_1,\dotsc,p_{l-1}\}$, pa ne dijeli njihov produkt), kontradikcija. Analogno ne može biti $k<l$. Dakle je $k=l$. Za svaki $i\in[1..k]$, tvrdimo da ne može biti $x_i>y_i$. Kada bi bilo, tada bi kod bio djeljiv s $p_{i-1}^{y_i}$, te bi nakon dijeljenja lijeva strana bila djeljiva s $p_{i-1}$, a desna ne~bi. Analogno se vidi da ne može biti $x_i<y_i$, dakle mora biti $x_i=y_i$. Kako je $i$ bio proizvoljan, zaključujemo $\vec x=\vec y$.
\end{proof}

Sljedeće bismo trebali pokazati da je $\f{Seq}:=\im{\langle\,\cdots\rangle}$ rekurzivan skup, te da na tom skupu imamo izračunljivu funkciju koja nam na neki način daje originalni $\vec x$ iz kojeg je pojedini kod dobiven. Za taj dokaz potrebno nam je ponešto teorije brojeva, no prije toga samo precizirajmo u kakvom obliku tražimo naš inverz.

Recimo odmah da tražimo primitivno rekurzivne, dakle totalne funkcije. Njihovo djelovanje će nas zanimati samo na skupu $\f{Seq}$: bit će parcijalno specificirano, iako naravno, kad damo algoritme, moći će se izračunati u svakom prirodnom broju.

Da bismo odredili konačni niz iz kojeg je dobiven kod $c$, svakako moramo prvo odrediti njegovu duljinu. Dakle, tražimo funkciju $lh$ takvu da za svaki $c\in\f{Seq}$, $lh(c)$ bude duljina konačnog niza čiji je $c$ kod. Takvih funkcija ima neprebrojivo mnogo jer je $\f{Seq}\kompl$ beskonačan (recimo, sadrži sve neparne brojeve osim $1$) a na njemu $lh$ može djelovati proizvoljno, pa sigurno postoje i neizračunljive takve funkcije. Ipak, dokazat ćemo da postoji takva funkcija $\f{lh}$ koja je primitivno rekurzivna.

Kad smo odredili $\f{lh}(c)=:k$, na prvi pogled imamo tipičnu situaciju algoritma s više izlaza: od jednog broja $c$ trebamo dobiti $k$ njih, i u skladu s napomenom~\ref{nap:brip}, to bismo trebali reprezentirati pomoću koordinatnih funkcija $\f{part}_1,\f{part}_2,\dotsc,\f{part}_k$. Ipak, to nema puno smisla jer broj takvih funkcija ovisi o $c$ (po funkciji $\f{lh}$), a osim toga, ponekad će nam trebati i dinamički određeni indeksi. Recimo, kad budemo pisali funkciju $\f U$, koja kod izračunavanja koje stane preslikava u njegov rezultat, bit će potrebno odabrati \emph{zadnju} konfiguraciju u konačnom nizu. A čak i da imamo izračunljive funkcije $\f{part}_i$ za sve $i$, iz toga ne slijedi da je preslikavanje $c\mapsto\f{part}_{\f{lh}(c)}(c)$ izračunljivo (pokušajte napisati simboličku definiciju i vidjet ćete u čemu je problem).

Srećom, ideja dinamizacije i ovdje pomaže: zapravo ćemo imati \emph{dvomjesnu} funkciju $\f{part}^2$, tako da $\f{part}(c,i)$ (skraćena oznaka $c[i]$) bude ono što smo bili nazvali $\f{part}_{i+1}(c)$ --- pomaknuli smo indekse za $1$ jer je ideja shvatiti konačne nizove kao polja (\emph{arrays}), a u većini modernih programskih jezika indeksiranje polja kreće od nule. Bitno nam je da to vrijedi samo za $c\in\f{Seq}$ i $i\in[0..\f{lh}(c)\rangle$ --- za ostale uređene parove mora biti nekako definirana jer je primitivno rekurzivna, ali nije nam bitno kako točno. Zapravo će se ispostaviti da za $c\in\f{Seq}$ i $i\ge\f{lh}(c)$ vrijedi $c[i]=0$, što će biti korisno u jednom trenutku, ali naglasit ćemo to kad nam bude trebalo.

\begin{napomena}
Recimo još nekoliko riječi o dinamizaciji ovdje. Ako za fiksni $k\in\N_+$ promatramo funkciju $\f{Code}^k$ u jednom smjeru i funkcije $\f{part}_1,\dotsc,\f{part}_k$ u drugom, zapravo modeliramo ono što jezik C zove struktura (\t{struct}) s $k$ članova. Razlog zašto C ima i polja i strukture kao zasebne tipove podataka leži u tome da strukture mogu sadržavati podatke različitih tipova. Kako mi sve kodiramo prirodnim brojevima, to nam neće bitno trebati. Ipak, jedna prednost struktura je u tome da umjesto indeksa možemo imati lijepa \emph{imena} za članove, što će nam biti korisno u jednom trenutku kasnije, te ćemo se na ovu ideju vratiti.
\end{napomena}

Kad smo već kod programskog jezika C, vjerojatno znate da se polja u njemu najčešće prenose tako da se u jednom argumentu prenese pokazivač (što u našem slučaju odgovara kodu), a u drugom, zasebnom argumentu njegova duljina. Tada bismo mogli kodirati i bez sljedbenika u eksponentu, jer bi duljina do koje gledamo "memoriju" bila posebno zadana. Ipak, većina modernijih programskih jezika drži duljinu zajedno s pojedinim elementima spremnika (npr.\ Python ima ugrađenu funkciju \t{len}), te je zato i mi kodiramo tako da ju je moguće odrediti iz koda. Nama će ideja kodiranja bez sljedbenika biti bitna u jednom drugačijem slučaju, kada je duljina nespecificirana --- ali o tome kasnije.

Funkcije $\f{Code}^k$ su dovoljno dobre ako imamo fiksnu mjesnost, te možemo argumente zadati posebno. Što dobijemo ako primijenimo ideju dinamizacije na tu familiju funkcija? Očito, argumenti su tada zadani nekom funkcijom $G$, te posebni argument kaže koliko ih ima (usporedite s točkom~\ref{sec:dyn}). Operator koji dinamički kodira zadani broj vrijednosti neke funkcije zovemo operatorom \emph{povijesti}, jer često argument koji kaže koliko vrijednosti treba kodirati zapravo predstavlja neku vrstu vremena, odnosno broji korake u nekom postupku (npr.\ $P$-izračunavanju s $\vec x$).

\begin{definicija}
Neka je $k\in\N_+$, i $G^k$ totalna funkcija. Za funkciju $F^k$ zadanu s
\begin{equation}\label{eq:povijest}
    F(\vec x,y):=\langle G(\vec x,0),G(\vec x,1),\dotsc,G(\vec x,y-1)\rangle
\end{equation}
(početak je $F(\vec x,0):=\langle\rangle=1$) kažemo da je \emph{povijest} funkcije $G$, i pišemo $F:=\overline G$.
\end{definicija}

\begin{primjer}\label{pr:primorijel}
$\overline{\f{Code}}(0,2)=\kr{\kr{0,0},\kr{0,1}}=\kr{6,18}=2^7\cdot3^{19}=148\,769\,467\,776$. Također,
\begin{multline}
    \overline{\f{mul}}(2,1,3)=\langle\f{mul}(2,1,0),\f{mul}(2,1,1),\f{mul}(2,1,2)\rangle=\langle2\cdot1\cdot0,2\cdot1\cdot1,2\cdot1\cdot2\rangle=\\
    =\langle0,2,4\rangle=2^{0+1}\cdot3^{2+1}\cdot5^{4+1}=2\cdot27\cdot3125=168\,750.
\end{multline}
Naravno, $k$ može biti i $1$; jedan važan slučaj je $\f G:=\f Z$. Recimo,
\begin{equation}
    \overline{\f Z}(5)=\langle\f Z(0),\f Z(1),\f Z(2),\f Z(3),\f Z(4)\rangle=\langle0,0,0,0,0\rangle=2\cdot3\cdot5\cdot7\cdot11=2\,310.
\end{equation}
Dakle, funkcija $\overline{\f Z}$ je u uskoj vezi s funkcijom tzv.\ \emph{primorijel}, koja je definirana slično kao faktorijel, osim što množi samo prim-brojeve. 
\end{primjer}

Sada bismo htjeli dokazati rezultat analogan onom u lemi~\ref{lm:sumrek}, samo za povijest umjesto sume. Kao što smo već rekli, to će zahtijevati ponešto teorije brojeva.

\section{Potrebni rezultati iz teorije brojeva}\label{sec:teobroj}

Dokazujemo niz rezultata o primitivnoj rekurzivnosti, navodeći uglavnom samo toč\-kov\-ne definicije (obilno koristeći programiranje s ograničenim petljama: sume, produkte, brojenje i minimizaciju), te obrazlažući ukratko manje poznate rezultate iz teorije brojeva koje ovdje koristimo.

Sjetimo se napomene~\ref{nap:crtica}: često brojevnoteorijske funkcije nisu definirane u nuli, te ćemo je zamijeniti jedinicom, pišući $n\bump$ za $n$ ako je pozitivan, a $1$ ako je $n=0$.

\begin{propozicija}\label{prop:divmodprn}
    Cjelobrojno dijeljenje, zadano s $x\sslash y:=\bigl\lfloor\frac{x}{y\bump\!}\bigr\rfloor$, kao i ostatak cjelobrojnog dijeljenja ($x\bmod y$), primitivno su rekurzivne operacije.
\end{propozicija}
Prije dokaza samo primijetimo da je nula problematična u operaciji $\sslash$, ali nije u operaciji $\bmod$: uobičajeno se u modernoj teoriji brojeva definira $x\bmod 0:=x$.
\begin{proof}
Cjelobrojno dijeljenje zapravo je ponovljeno oduzimanje, onoliko puta koliko se može. Dakle, za zadane $x$ i $y$, tražimo najveći $t$ takav da je $t\cdot y\le x$. Relacija $t\cdot y\le x$ je svakako primitivno rekurzivna, ali nemamo operator maksimizacije. S razlogom ga općenito nemamo: ako malo razmislimo, vidjet ćemo da ne postoji općeniti algoritam kojim bismo odredili najveći broj s nekim svojstvom, čak ni ako znamo da taj broj postoji, jer nikad ne znamo jesmo li tražili dovoljno dugo --- u svakom trenutku iznad najvećeg broja koji smo ispitali postoji još beskonačno mnogo kandidata, svaki od kojih može biti traženi broj.

Ipak, naša relacija je \emph{padajuća} po $t$: jednom kad $t\cdot y$ postane veće od $x$, ostat će veće od $x$ za sve veće $t$. Kod takvih relacija, "prva laž" i "zadnja istina" su susjedne, te jednostavno možemo naći najmanji $t$ za koji vrijedi $t\cdot y>x$, i uzeti njegov prethodnik.

Ipak, to je neograničena minimizacija, a mi bismo htjeli primitivno rekurzivnu funkciju --- možemo li je nekako ograničiti? Svakako: $y\bump\ge 1$ znači $t\cdot y\bump\ge t$ za svaki $t$, te specijalno $(x+1)\cdot y\bump\ge x+1>x$. Dakle, dovoljno je tražiti $t$ do $x+1$ --- odnosno do $x$ uključivo, jer znamo da će ograničena minimizacija tada dati $x+1$ ako ne pronađe nijedan broj s traženim svojstvom u zadanom rasponu. Sve u svemu, tvrdimo da vrijedi
\begin{equation}\label{eq:sslash}
    x\sslash y=\f{pd}\bigl((\mu t\le x)(t\cdot y>x)\bigr)\text.
\end{equation}
Za $y>0$, samo trebamo formalizirati navedeni argument: označimo $z:=x\sslash y$. Tada je $zy\le x$ (i očito $ty\le zy\le x$ za sve $t\le z$) a $(z+1)y>x$, dakle $z+1$ je najmanji $t$ koji pomnožen s $y$ daje broj veći od $x$. S druge strane je $z\le x\sslash1=x$, pa je $z+1\le x+1$. Ako je $z+1\le x$, tada će ga ograničena minimizacija u~\eqref{eq:sslash} naći, a ako je $z+1=x+1$, tada ga neće naći do uključivo $x$ (isključivo $x+1$), pa će po definiciji vratiti upravo $x+1$. U svakom slučaju vrijednost te ograničene minimizacije bit će $z+1$, pa će njen prethodnik biti $z=x\sslash y$, što smo trebali.

Za $y=0$, samo trebamo izračunati lijevu i desnu stranu. Na lijevoj je $x\sslash 0=\bigl\lfloor\frac{x}{1}\bigr\rfloor=x$, a na desnoj je prethodnik minimizacije po $t\le x$, relacije $t\cdot 0=0>x$. Ta relacija je očito uvijek lažna (nema negativnih brojeva), pa minimizacija daje $x+1$, a njen prethodnik onda daje upravo $x$, kao što i treba.

Sada je ostatak jednostavan: tvrdimo da je
\begin{equation}
    x\bmod y=x\ominus x\sslash y\cdot y
\end{equation}
(prioritet operacija je: $\sslash$, pa $\cdot$, pa $\ominus$). Za $y>0$, direktno iz teorema o dijeljenju s ostatkom dobijemo da je ostatak $x-z\cdot y$ (sa $z$ smo označili količnik), a iz provedenog razmatranja se vidi da je $z\cdot y\le x$, pa se oduzimanje može ograničiti nulom. Za $y=0$, na lijevoj strani piše $x$, a na desnoj također $x\ominus x\sslash0\cdot 0=x\ominus x\cdot0=x\ominus0=x$.
\end{proof}

\begin{korolar}\label{kor:midprn}
Djeljivost je primitivno rekurzivna relacija.
\end{korolar}
\begin{proof}
Svatko tko je ikad provjeravao djeljivost u programiranju zna kako se to radi:
\begin{equation}
    y\mid x\Longleftrightarrow x\bmod y=0\text.
\end{equation}
Zaista, za $y>0$, postojanje broja $z$ takvog da je $z\cdot y=x$ (definicija djeljivosti) zapravo znači da je $z=x\sslash y$, pa je $x\bmod y=x\ominus z\cdot y=x\ominus x=0$. U drugom smjeru, ako je $x\ominus x\sslash y\cdot y=0$, iz toga slijedi (kao i prije, vrijedi $x\sslash y\cdot y\le x$) $x=x\sslash y\cdot y$, pa postoji $z:=x\sslash y$ takav da vrijedi $x=z\cdot y$, odnosno $y\mid x$.

Za $y=0$, na desnoj strani stoji $x\bmod 0=x=0$, a to stoji i na lijevoj strani, jer je jedino $0$ djeljiva nulom --- "postoji $z$ takav da je $0\cdot z=x$" zapravo znači da je $x=0$.
\end{proof}

\begin{korolar}
    Skup\, $\mathbb P$ svih prim-brojeva je primitivno rekurzivan.
\end{korolar}
\begin{proof}
Direktno pomoću korolara~\ref{kor:midprn} i ograničenog brojenja (lema~\ref{lm:brojrek}): prim-brojevi imaju točno dva prirodna djelitelja, a za pozitivne $x$, svaki djelitelj od $x$ je manji ili jednak $x$:
\begin{equation}\label{eq:Pprn}
    x\in\mathbb P\Longleftrightarrow(\#d\le x)(d\mid x)=2\text.
\end{equation}
Za nulu će tom metodom ispasti da ima jedan djelitelj (samu sebe), a zapravo ih ima beskonačno mnogo, ali bitno je samo da ih nema točno $2$, odnosno $0\not\in\mathbb P$.
\end{proof}

U "stvarnom životu", provjera je li $x\in\mathbb P$ odvija se drugačije: nulu, jedinicu i dvojku, te sve ostale parne brojeve odvojimo kao specijalne slučajeve, a onda provjeravamo samo neparne kandidate za djelitelje od 3 do uključivo $\lfloor\sqrt{x}\rfloor$. Ipak, to su sve samo praktične optimizacije, koje bitno ubrzavaju algoritam --- ostavljajući ga doduše u istoj klasi složenosti. Zanimljivo je napomenuti da je početkom ovog stoljeća pronađen \emph{polinomni} algoritam za provjeru je li zadani broj prim-broj, ali je dosta kompliciraniji. Kako mi ovdje nismo opterećeni performansama, tražimo samo najelegantniji zapis algoritma, a to je bez sumnje~\eqref{eq:Pprn}.

\begin{propozicija}\label{pp:primeprn}
    Niz $(p_i)_{i\in\N}$ (strogo rastući niz sa slikom $\mathbb P$, tzv.\ \emph{enumeracija} skupa $\mathbb P$) je primitivno rekurzivan.
\end{propozicija}
\begin{proof}
Zapravo trebamo algoritam za funkciju $\f{prime}^1$, koja svakom broju $n$ pridružuje $p_n$, $n$-ti prim-broj po veličini. Očito, taj niz ćemo teško dobiti samo kompozicijom, treba nam primitivna rekurzija. To je degenerirana primitivna rekurzija jer definiramo funkciju mjesnosti $1$, dakle za inicijalizaciju stavimo samo vrijednost $p_0=2$, a u koraku trebamo funkciju $\f{nextprime}^2$ koja prima $n$ i $p_n$, i mora vratiti $p_{n+1}$. Je li ta funkcija izračunljiva?

Kako uopće znamo da je ta funkcija \emph{totalna}, odnosno da možemo pomoću nje graditi primitivnu rekurziju? Prim-brojeva ima beskonačno mnogo: za svaki $p_n$ postoji $q\in\mathbb P$ takav da je $q>p_n$. Štoviše, $p_{n+1}$ je prvi takav $q$, pa ga možemo naći minimizacijom:
\begin{equation}\label{eq:primerek}
    \f{nextprime}(n,p):=\mu q(q\in\mathbb P\land q>p)\text.
\end{equation}
Upravo napisani izraz zapravo kaže da je $\f{nextprime}$ \emph{rekurzivna} funkcija (parcijalno je rekurzivna jer je dobivena minimizacijom konjunkcije dvije rekurzivne relacije, a totalna je zbog beskonačnosti skupa $\mathbb P$), pa je i $\f{prime}$ rekurzivna po korolaru~\ref{kor:F1rek}.

Za primitivnu rekurzivnost, moramo nekako ograničiti minimizaciju, odnosno moramo "isprogramirati" dokaz da je $\mathbb P$ beskonačan. Uzmimo Euklidov dokaz: ako imamo $p_0,p_1,\dotsc,p_n\in\mathbb P$, novi prim-broj možemo dobiti tako da potražimo prim-djelitelj broja $m:=p_0\cdot p_1\dotsm p_n+1>0$. Aha! Znamo da je djelitelje pozitivnog broja dovoljno tražiti do samog tog broja, dakle samo trebamo primitivno rekurzivno izračunati $m$ iz $n$ i $p_n$. Tu bismo mogli upotrijebiti primorijel iz primjera~\ref{pr:primorijel} (konkretno, $m=\f{Sc}\bigl(\overline{\f Z}\bigl(\f{Sc}(n)\bigr)\bigr)$), ali nažalost još ne znamo da je funkcija $\overline{\f Z}$ primitivno rekurzivna, jer nismo dokazali lemu o povijesti. Zapravo, pokazat će se da za tu lemu treba rezultat koji upravo dokazujemo.

Kako se izvući? Spas leži u tome da ne trebamo pomoću $n$ i $p_n$ izračunati baš $m$, nego samo neki broj od kojeg je $m$ manji. Naime, ionako nećemo tražiti njegove prim-djelitelje, nego će nam on samo poslužiti kao gornja granica za traženje sljedećeg prim-broja nakon $p_n$. Sada je lako: umjesto primorijela, možemo upotrijebiti faktorijel, za koji znamo da je primitivno rekurzivan (primjer~\ref{pr:factorialprn}), a očito je $p_n!\ge p_0\cdot p_1\dotsm p_n$ (jer je $p_n!$ umnožak svih brojeva na desnoj strani, i eventualno još nekih, koji su veći ili jednaki $1$).
\begin{equation}
    \f{nextprime}(n,p)=(\mu q\le p!+1)(q\in\mathbb P\land q>p)%\text,
\end{equation}
Dakle $\f{nextprime}$ je dobivena ograničenom minimizacijom primitivno rekurzivne relacije do primitivno rekurzivne (uključive) granice, te je primitivno rekurzivna. Sada je i $\f{prime}=2\pr\f{nextprime}$ primitivno rekurzivna po propoziciji~\ref{prop:F1prn}.
\end{proof}

\subsection{Rastav na prim-faktore}
\begin{lema}\label{lm:exprn}
Za $(n,i)\in\N^2$, označimo s $\f{ex}(n,i)$ eksponent prim-broja $p_i$ u rastavu broja $n\bump$ na prim-faktore. Funkcija $\f{ex}^2$ je primitivno rekurzivna.
\end{lema}
Nula nema rastav na prim-faktore, pa je moramo zamijeniti jedinicom --- koja \emph{ima} jedinstven rastav na prim-faktore: prazan produkt, gdje su svi eksponenti jednaki $0$.
\begin{proof}
Opet, tražimo najveći broj $t$ takav da ${p_i}^t\mid n\bump$. Kao i u dokazu propozicije~\ref{prop:divmodprn}, ta relacija je padajuća po $t$: jednom kad prestane biti istina, ne može ponovo postati istina ni za koji veći $t$. Dakle, isti trik (prethodnik najmanjeg elementa komplementa) prolazi. Također, kako je $\f{prime}$ rastuća funkcija, imamo $n\bump\ge{p_i}^t\ge{p_0}^t=2^t>t$ (Cantorov osnovni teorem za konačne skupove), pa je $t$ dovoljno tražiti do $n\bump$ isključivo, ili do $n$ jer su za $n=0$ ionako svi eksponenti $0$. Sve u svemu, vrijedi
\begin{equation}
    \f{ex}(n,i):=\f{pd}\bigl((\mu t<n)\bigl(\f{pow}(\f{prime}(i),t)\nmid n\bigr)\bigr)\text,
\end{equation}
pa je $\f{ex}$ primitivno rekurzivna.
\end{proof}

Pomoću funkcije $\f{ex}$ napokon možemo dekodirati proizvoljni kod konačnog niza.

\begin{propozicija}\label{prop:lhpartprn}
Postoje primitivno rekurzivne funkcije $\f{lh}^1$ i $\f{part}^2$, takve da za svaki $\vec x=(x_1,x_2,\dotsc,x_k)\in\N^*$, uz oznaku $c:=\langle\vec x\rangle$, te pokratu $c[i]:=\f{part}(c,i)$, vrijedi:
\begin{enumerate}
    \item\label{stav:lh} $\f{lh}(c)=k$;
    \item\label{stav:part<k} za sve $i<k$, $c[i]=x_{i+1}$;
    \item\label{stav:part=0} za sve $i\ge k$, $c[i]=0$.
\end{enumerate}
\end{propozicija}
%Kao što smo već rekli, $\f{part}(c,i)$ ćemo skraćeno pisati $c[i]$, po uzoru na indeksiranje u brojnim programskim jezicima.
\begin{proof}
Tvrdimo da funkcije zadane s
\begin{align}
\f{lh}(c)&:=(\#i<c)\bigl(\f{prime}(i)\mid c\bigr)\text,\\
c[i]:=\f{part}(c,i)&:=\f{pd}\bigl(\f{ex}(c,i)\bigr)\text,
\end{align}
zadovoljavaju sve uvjete. Prije svega, primitivno su rekurzivne: $\f{lh}$ je dobivena og\-ra\-ni\-če\-nim brojenjem primitivno rekurzivne relacije do primitivno rekurzivne granice, a $\f{part}$ je jednostavno kompozicija $\f{pd}\circ\f{ex}$.

Što se tiče tvrdnje~\ref{stav:lh}, prema definiciji $\f{lh}$ broji prim-djelitelje od $c$ (kako je $\f{prime}$ strogo rastuća i $0<p_0$, vrijedi $i<p_i\le c$, pa je dovoljno brojiti do $c$), a kako je $c=\langle\vec x\rangle=2^{x_1+1}\cdot3^{x_2+1}\dotsm p_{k-1}^{x_k+1}$, te su svi napisani eksponenti pozitivni, $c$ ima točno $k$ prim-djelitelja.

Za tvrdnju~\ref{stav:part<k}, neka je $i<k$ proizvoljan. Tada je eksponent od $p_i$ u rastavu $c>0$ na prim-faktore upravo $\f{ex}(c,i)=x_{i+1}+1$, te je $c[i]=\f{pd}(x_{i+1}+1)=x_{i+1}$.

Za tvrdnju~\ref{stav:part=0}, neka je $i\ge k$ proizvoljan. Lako je vidjeti da su jedini prim-djelitelji od $c$ upravo $p_0,p_1,\dotsc,p_{k-1}$, dakle $p_i$ nije među njima. Dakle $(\mu t<c)(p_i^t\nmid c)=1$, pa je $\f{ex}(c,i)=\f{pd}(1)=0$, te je i $c[i]=\f{pd}(0)=0$.
\end{proof}

Primijetimo da je funkcija $\f{part}$ parcijalno specificirana samo po $c$: po $i$ je totalno specificirana, odnosno jednom kad znamo da je $c\in\f{Seq}$, sve vrijednosti $c[i]$ su propisane. Još nam jedino nedostaje dokaz da je $\f{Seq}$ primitivno rekurzivna relacija, no to će slijediti uskoro.

\section{Funkcije definirane pomoću operatora povijesti}

\begin{lema}[Lema o povijesti]\label{lm:povijestrek}
Neka je $k\in\N_+$, i $\f{G}^k$ totalna funkcija.

Tada je ${\f G}$ (primitivno) rekurzivna ako i samo ako je $\overline{\f G}$ (primitivno) rekurzivna.
\end{lema}

Za razliku od, recimo, leme~\ref{lm:sumrek}, ovdje se tvrde dva smjera. (Zapravo, i u lemi~\ref{lm:sumrek} vrijedi obrat, dok recimo u lemi~\ref{lm:prodrek} ne vrijedi. Zgodna je vježba pokušati to dokazati.)
\begin{proof}
Za jedan smjer (zapisati $\overline{\f G}$ primitivno rekurzivno pomoću $\f G$), treba uvrstiti~\eqref{eq:Codek} u~\eqref{eq:povijest}:
\begin{equation}\label{eq:povijestrek}
    \overline{\f G}(\vec x,y)=\prod_{i<y}\f{pow}\bigl(\f{prime}(i),\f{Sc}\bigl(\f{G}(\vec x,i)\bigr)\bigr)\text.
\end{equation}
Ako $i$-ti faktor u~\eqref{eq:povijestrek} označimo s $\f H(\vec x,i)$, tada je $\f{pow}\circ(\f{prime}\circ\f I_{k+1}^{k+1},\f{Sc}\circ\f G)$ sim\-bo\-li\-čka definicija od $\f H^{k+1}$ kompozicijom iz funkcija za koje je već dokazano da su primitivno rekurzivne, te funkcije $\f G$ koja je (primitivno) rekurzivna --- pa je $\f H$ (primitivno) rekurzivna. Sada je $\overline{\f G}$ (primitivno) rekurzivna po lemi~\ref{lm:prodrek}.

Za drugi smjer, zamislimo da nam je zadana $\overline{\f G}$, i želimo iz nje dobiti $\f G$. Pažljivim gledanjem~\eqref{eq:povijest} vidimo da $\overline{\f G}(\vec x,y)$ kodira prvih $y$ vrijednosti oblika $\f G(\vec x,i)$, među kojima \emph{nije} $\f G(\vec x,y)$. (To je čest \emph{bug} u C programima: polje deklarirano s \texttt{int a[7];} \emph{ne} sadrži element \texttt{a[7]}.) Ako želimo dobiti taj broj, moramo zapravo izračunati $\overline{\f G}$ u nekom broju \emph{većem} od $y$ kao zadnjem argumentu. Dovoljno je uzeti $\overline{\f G}(\vec x,y+1)$: tvrdimo da je
\begin{equation}\label{eq:unpovijest}
    \f G(\vec x,y)=\overline{\f G}\bigl(\vec x,\f{Sc}(y)\bigr)[y]\text,
\end{equation}
odakle slijedi (primitivna) rekurzivnost od $\f G$, iz (primitivne) rekurzivnosti od $\overline{\f G}$. Doista, $c:=\overline{\f G}(\vec x,y+1)$ jest kod konačnog niza duljine $y+1$, pa prema propoziciji~\ref{prop:lhpartprn}\eqref{stav:part<k} za svaki $i<y+1$ vrijedi $c[i]=\f G(\vec x,i)$. Specijalno za $i:=y$, imamo jednakost~\eqref{eq:unpovijest}.
\end{proof}

Napokon možemo dokazati primitivnu rekurzivnost slike kodiranja.

\begin{korolar}\label{kor:Seqprn}
    Relacija $\f{Seq}^1:=\im{\langle\,\cdots\rangle}$ je primitivno rekurzivna.
\end{korolar}
\begin{proof}
Formalnije,  \begin{equation}
    \f{Seq}(c)\Longleftrightarrow(\exists\vec x\in\N^*)(c=\langle\vec x\rangle)\text,
\end{equation}
i način za odrediti $\vec x$ je kanonski: pokušamo dekodirati $c$. (To funkcionira i općenito, ali razlog zašto se rekurzivnost slike navodi kao zasebno svojstvo funkcije kodiranja je u tome što ga u slučaju kodiranja nekih drugih objekata možemo provjeriti formalnije nego na ovaj način, koji općenito koristi neformalne algoritme.) Dakle, tvrdimo
\begin{equation}
    \f{Seq}(c)\Longleftrightarrow c=\overline{\f{part}}\bigl(c,\f{lh}(c)\bigr)\text;
\end{equation}
smjer $(\Leftarrow)$ slijedi iz činjenice da je \emph{svaka} vrijednost funkcije oblika $\overline{\f G}$ kod konačnog niza. Za smjer $(\Rightarrow)$, pretpostavimo da je $c=\langle\vec x\rangle$ za neki $\vec x=(x_1,x_2,\dotsc,x_k)\in\N^k\subset\N^*$. Tada je prema propoziciji~\ref{prop:lhpartprn},
\begin{multline}
    \overline{\f{part}}\bigl(c,\f{lh}(c)\bigr)=\langle c[0],c[1],\dotsc,c[\f{lh}(c)-1]\rangle=\\
    =\langle x_{0+1},x_{1+1},\dotsc,x_{\f{lh}(c)-1+1}\rangle=\langle x_1,x_2,\dotsc,x_k\rangle=\langle\vec x\rangle=c\text.
\end{multline}
Za $c=\langle\rangle=1$, također vrijedi $\overline{\f{part}}\bigl(1,\f{lh}(1)\bigr)=\overline{\f{part}}(1,0)=\langle\rangle=1=c$.

Sada primitivna rekurzivnost slijedi na uobičajeni način: prema propoziciji~\ref{prop:lhpartprn} funkcija $\f{part}$ je primitivno rekurzivna, prema lemi~\ref{lm:povijestrek} je tada i $\overline{\f{part}}$ primitivno rekurzivna, a onda je $\chi_{\f{Seq}}=\chi_=\circ\bigl(\f I_1^1,\overline{\f{part}}\circ(\f I_1^1,\f{lh})\bigr)$ simbolička definicija karakteristične funkcije od $\f{Seq}$, iz koje se vidi da je ona primitivno rekurzivna.
\end{proof}

Time smo u potpunosti opisali kodiranje skupa $\N^*$, koje ćemo kasnije koristiti na brojnim mjestima. Što možemo njime? Rekli smo da nam kodiranje pruža mogućnost rada s kodiranim skupom kao \emph{klasom}, gdje konstruktori i komponente pružaju mo\-guć\-nost pisanja svih ostalih metoda. Striktno, ulogu konstruktora za $\N^*$ igra familija funkcija $\f{Code}^k,k\in\N_+$, dinamizirana kroz operator povijesti, a ulogu komponenata funkcije $\f{lh}$ i $\f{part}$. Ideju da sad sve metode klase $\N^*$ možemo napisati pomoću tih funkcija, možemo shvatiti kao da je $\N^*$ \emph{apstraktni tip podataka}, čija konkretna implementacija (umnožak prim-brojeva potenciranih sljedbenicima komponenata) nam nije bitna, dok god koristimo $\langle\,\cdots\rangle$ odnosno $\overline{\cdots}$, te $\f{lh}$ i $\f{part}$ prema njihovoj specifikaciji. Pogledajmo jedan jednostavni primjer.

\begin{primjer}\label{pr:concat}
\emph{Konkatenacija} je preslikavanje  $concat\colon\N^*\times\N^*\to\N^*$, zadano s
\begin{equation}
    concat\bigl((x_1,x_2,\dotsc,x_k),(y_1,y_2,\dotsc,y_l)\bigr):=(x_1,x_2,\dotsc,x_k,y_1,y_2,\dotsc,y_l)\text.
\end{equation}
Pomalo neprecizno, ali sasvim jasno, pišemo $concat(\vec x^k,\vec y^{\,l}):=(\vec x,\vec y)$. I ubuduće ćemo smatrati da su takvi konstrukti "spljošteni" na jednu razinu --- duljina od $(\vec x,\vec y)$ nije $2$, već $k+l$. To smo zapravo već koristili svaki put kad smo napisali npr.\ $\f{f}(\vec x,y)$.
\end{primjer}

Brojevna funkcija $\N concat$ definirana je samo na $\f{Seq}\times\f{Seq}$, ali u svrhu primitivne rekurzivnosti, to se shvaća kao parcijalna specifikacija. Također, kodirana konkatenacija često se piše infiksno kao operacija.

\begin{lema}\label{lm:starprn}
Postoji primitivno rekurzivna operacija $*$ takva da za sve $\vec x,\vec y\in\N^*$ vrijedi \begin{equation}
    \langle\vec x\rangle*\langle\vec y\rangle=\langle\vec x,\vec y\rangle\text.
\end{equation}
\end{lema}
\begin{proof}
Zapravo, operator povijesti nam daje mogućnost pisanja "točkovne definicije" željenog konačnog niza: samo kažemo koju duljinu želimo, i zadamo funkciju koja propisuje elemente. Konkretno, ovdje želimo duljinu koja je zbroj duljina od $\vec x$ i od $\vec y$, te (ako sa $x$ i $y$ označimo operande od $*$) prvih $k:=\f{lh}(x)$ elemenata trebaju biti dobiveni pomoću $\f{part}$ iz $x$, a preostali elementi iz $y$ (s indeksima pomaknutima za $k$).

Preciznije, definirajmo funkciju $\f G^3$ točkovno po slučajevima:
\begin{align}
    \label{eq:starG}\f G(x,y,i)&:=\begin{cases}
    x[i],&i<\f{lh}(x)\\
    y[i\ominus\f{lh}(x)],&\text{inače}
    \end{cases}\text,
\shortintertext{i pomoću nje}
\label{eq:star}
x*y&:=\overline{\f G}\bigl(x,y,\f{lh}(x)+\f{lh}(y)\bigr)\text.
\end{align}
Sada sigurno vrijedi da je $x*y\in\f{Seq}$ (jer je vrijednost funkcije dobivene poviješću), vrijedi $t:=\f{lh}(x*y)=\f{lh}(x)+\f{lh}(y)$, te za svaki $i<t$ vrijedi $(x*y)[i]=\f G(x,y,i)$. Kako kod konačnog niza $concat(\vec x,\vec y)$ ima sva ta svojstva, kodiranje je injekcija, a konačan niz je jednoznačno zadan svojom duljinom i elementima, zaključujemo da je $x*y$ upravo $\langle concat(\vec x,\vec y)\rangle=\langle\vec x,\vec y\rangle$.

Jednadžbe~\eqref{eq:starG} i~\eqref{eq:star} mogu poslužiti kao točkovna definicija od $\f G$ i $*$ za sve $x$ i $y$, ne samo za one iz $\f{Seq}$ --- i iz toga slijedi primitivna rekurzivnost funkcije $\f G$ po teoremu~\ref{tm:grek} (primitivno rekurzivna verzija), a onda i funkcije $\overline{\f G}$ po lemi~\ref{lm:povijestrek}, pa tako i operacije $*$ dobivene iz nje kompozicijom.
\end{proof}

Zanimljivo je da smo koristili samo javno sučelje kodiranja konačnih nizova --- isti dokaz funkcionira za liste u Pythonu, koje sigurno imaju drugačiju implementaciju:
\begin{verbatim}
   >>> x, y = [2, 5, 8], [0, 3]
   >>> [x[i] if i<len(x) else y[i-len(x)] for i in range(len(x)+len(y))]
   [2, 5, 8, 0, 3]
\end{verbatim}

\subsection{Primitivna rekurzija kroz prostor i vrijeme}

Primitivna rekurzija pruža nam jednostavan način pisanja ograničenih petlji, koje prate jedan podatak. Što ako ih želimo pratiti više, recimo $l$ njih? Rekli smo da ćemo više izlaznih podataka reprezentirati kroz više nezavisnih algoritama, i ponekad se to doista može rastaviti: recimo ako tražimo kumulativni zbroj i umnožak nekog niza, možemo prvo naći zbroj pa onda umnožak, zasebnim primitivnim rekurzijama. No kod kompliciranijih rekurzija, algoritmi više nisu lako odvojivi, jer je zamislivo da sljedeće vrijednosti svih $l$ podataka ovise o prethodnim vrijednostima svih njih.

Važno mjesto gdje se to pojavljuje je simulacija kompliciranih (npr.\ univerzalnih) modela izračunljivosti, primitivnom rekurzijom kroz vrijeme. Recimo, u RAM-stroju moramo pratiti registre i programski brojač. Stanje registara u koraku $n+1$ ovisi o stanju registara u koraku $n$, ali ovisi i o tome koja se instrukcija izvršava, što (za fiksni program) ovisi o vrijednosti programskog brojača. Vrijednost pak programskog brojača u koraku $n+1$ definirana je po slučajevima, i najčešće je jednostavno sljedbenik vrijednosti programskog brojača u koraku $n$, ali kod izvršavanja instrukcije tipa $\dec$ ovisi i o stanju registra na koji ta instrukcija djeluje, u koraku $n$.

Takva ovisnost zove se \emph{simultana} primitivna rekurzija, jer moramo simultano ra\-ču\-na\-ti svih $l$ vrijednosti --- ali kodiranje $\N^*$ (zapravo samo $\N^l$) omogućuje nam da to implementiramo pomoću obične primitivne rekurzije. Ideja je jednostavna: umjesto $l$ vrijednosti $a_1$, $a_2$,~\ldots, $a_l$, pratimo jednu vrijednost $\langle\vec a\rangle$. Slikovito, pratimo jedan \t{struct} s $l$ članova.

\begin{propozicija}\label{prop:simultrek}
Neka su $k,l\in\N_+$, te neka su $\f G_1^k$, $\f G_2^k$,~\ldots, $\f G_l^k$, i još $\f H_1^{k+l+1}$, $\f H_2^{k+l+1}$,~\ldots, $\f H_l^{k+l+1}$ (primitivno) rekurzivne funkcije.

Tada su i funkcije $\f F_1^k$, $\f F_2^k$,~\ldots, $\f F_l^k$, zadane s
\begin{align}
    \label{eq:simultG}\f F_i(\vec x,0)&:=\f G_i(\vec x)\text,\\
    \label{eq:simultH}
    \f F_i(\vec x,y+1)&:=\f H_i\bigl(\vec x,y,\f F_1(\vec x,y),\f F_2(\vec x,y),\dotsc,\f F_l(\vec x,y)\bigr)\text,
\end{align}
za sve $i\in[1..l]$, (primitivno) rekurzivne.
\end{propozicija}
\begin{proof}
Kao što smo već rekli, cilj nam je primitivnom rekurzijom prvo dobiti funkciju $\f F:=\f{Code}^l\circ(\f F_1,\f F_2,\dotsc,\f F_l)$, iz koje ćemo onda lako dobiti svaki $\f F_i$ kompozicijom s funkcijom $\f{part}$. Inicijalizacija je jednostavna: iz~\eqref{eq:simultG} imamo
\begin{equation}\label{eq:simultGd}
    \f F(\vec x,0)=\langle \f F_1(\vec x,0),\dotsc,\f F_l(\vec x,0)\rangle=\langle \f G_1(\vec x),\dotsc,\f G_l(\vec x)\rangle=:\f G(\vec x)\text,
\end{equation}
te je $\f G=\f{Code}^l\circ(\f G_1,\dotsc,\f G_l)$ (primitivno) rekurzivna kao kompozicija takvih. Za korak petlje je definicija dulja, ali zapravo sasvim jednostavna: iz~\eqref{eq:simultH} je
\begin{multline}\label{eq:simultHd}
\f F(\vec x,y+1)=\langle\f F_1(\vec x,y+1),\dotsc,\f F_l(\vec x,y+1)\rangle=\\
=\bigl\langle\f H_1\bigl(\vec x,y,\f F_1(\vec x,y),\dotsc,\f F_l(\vec x,y)\bigr),\dotsc,\f H_l\bigl(\vec x,y,\f F_1(\vec x,y),\dotsc,\f F_l(\vec x,y)\bigr)\bigr\rangle=\\
=\langle
\f H_1(\vec x,y,
\f F(\vec x,y)[0],
\dotsc,
\f F(\vec x,y)[l-1]),
\dotsc,
\f H_l(\vec x,y,
\f F(\vec x,y)[0],
\dotsc,
\f F(\vec x,y)[l-1])\rangle\\
=:\f H\bigl(\vec x,y,\f F(\vec x,y)\bigr)\text,
\end{multline}
te je $\f H$ (primitivno) rekurzivna kao kompozicija funkcija $\f{Code}^l$, $\f H_i$, $\f{part}$, konstanti $\f C_0^{k+1}$ do $\f C_{l-1}^{k+1}$, i koordinatnih projekcija.

Jednakosti~\eqref{eq:simultGd} i~\eqref{eq:simultHd} kažu nam da je $\f F=\f G\pr\f H$, dakle funkcija $\f F$ je dobivena primitivnom rekurzijom iz (primitivno) rekurzivnih funkcija, pa je i sama (primitivno) rekurzivna. 

A onda je za svaki $i\in[1..l]$, funkcija $\f F_i$ zadana s $\f F_i(\vec x, y)=\f F(\vec x,y)[i-1]$, simbolički $\f F_i=\f{part}\circ(\f F,\f C_{i-1}^{k+1})$. To znači da su sve $\f F_i$ dobivene kompozicijom iz (primitivno) rekurzivnih funkcija $\f F$, $\f{part}$ i konstanti, te su i one sve (primitivno) rekurzivne.
\end{proof}

\begin{napomena}
Gdje se u točkovnoj definiciji pojavljuje sintaksno isti izraz više puta, uvodit ćemo pokrate koje će nam omogućiti da kompliciranije izraze zapišemo lakše i preglednije. Recimo,~\eqref{eq:simultHd} bismo mogli zapisati kao
\begin{align}
    \f H(\vec x,y,z)&:=\langle \f H_1(\vec a),\f H_2(\vec a),\dotsc,\f H_l(\vec a)\rangle\text,\\
    \text{uz pokratu } \vec a&:=\bigl(\vec x,y,z[0],z[1],\dotsc,z[l-1]\bigr)\text.
\end{align}
Treba napomenuti da je to samo kraći \emph{zapis} za~\eqref{eq:simultHd}, ne uvođenje pomoćnih funkcija --- jer tada bi $\vec a$ kao funkcija trebala imati više izlaznih podataka, te primati $\vec x$, $y$ i $z$ kao argumente, što bi uništilo dobar dio kratkoće zapisa.

Analogija u programskom jeziku C je korištenje preprocesora (\verb+#define+). Na neki način, uvodimo "makroe" u funkcijski jezik, ali ih nećemo formalizirati jer nam neće biti potrebni tako često, nećemo uopće koristiti makroe s parametrima (koje C preprocesor podržava), a "grafičko" uvrštavanje izraza na određena mjesta u većem izrazu nije pretjerano zahtjevna operacija --- samo smanjuje preglednost, koja je zapravo jedina motivacija za uvođenje pokrata.
\end{napomena}

Dokazali smo da je moguće u primitivnoj rekurziji simultano graditi $l$ funkcija, tako da svaka sljedeća vrijednost ovisi "prostorno" o prethodnim vrijednostima različitih funkcija. Što dobijemo ako pokušamo dinamizirati taj $l$? Dobit ćemo funkciju koja ovisi o \emph{povijesti} neke druge funkcije, no svakako je najzanimljiviji slučaj kad ovisi "vremenski" o povijesti same sebe --- kad je definirana \emph{rekurzijom s poviješću}, koja može koristiti ne samo neposredno prethodnu vrijednost, nego sve ranije.

Matematički, ako obična primitivna rekurzija odgovara običnom principu matematičke indukcije --- gdje u dokazu $\wp(n+1)$ smijemo koristiti $\wp(n)$, ali još moramo zasebno dokazati bazu $\wp(0)$ --- tada rekurzija s poviješću odgovara principu \emph{jake} indukcije --- gdje u dokazu $\wp(n)$ smijemo koristiti $\wp(m)$ za sve $m<n$, te ne trebamo odvajati bazu kao zasebni slučaj: za $n=0$ ionako nema pretpostavki $\wp(m)$ koje bismo mogli koristiti.

\begin{propozicija}\label{prop:rekpov}
Neka je $k\in\N_+$, te $\f G^k$ (primitivno) rekurzivna funkcija. Tada je i funkcija $\f F^k$\!, zadana s
\begin{equation}\label{eq:rekpov}
    \f F(\vec x,y):=\f G\bigl(\vec x,\overline{\f F}(\vec x,y)\bigr)\text,
\end{equation}
također (primitivno) rekurzivna.
\end{propozicija}
Primijetimo da po Dedekindovom teoremu rekurzije  (pogledajte~\cite[str.\ 60]{skr:VukTS} za detalje: $\overline{\f F}(\vec x,y)$ ovdje kodira $\varphi|_y$) za svaku totalnu funkciju $G^k$ postoji jedinstvena (totalna) funkcija $F^k$ koja zadovoljava jednadžbu~\eqref{eq:rekpov}. Zato u njoj možemo pisati simbol $:=$, odnosno reći da je $\f F$ \emph{definirana} rekurzijom s poviješću.
\begin{proof}
Ideja je slična kao u dokazu propozicije~\ref{prop:simultrek}, samo umjesto kodiranja fiksne mjesnosti $\f{Code}^l$ imamo dinamički operator povijesti. Dakle, trebamo dobiti $\overline{\f F}$ primitivnom rekurzijom (degeneriranom u slučaju $k=1$). Inicijalizacija: svaka povijest počinje kodom praznog niza,
\begin{equation}
    \overline{\f F}(\vec x,0)=\langle\rangle=1\text.
\end{equation}
Za korak, moramo izraziti $\overline{\f F}(\vec x,y+1)$ pomoću $\vec x$, $y$ i $z:=\overline{\f F}(\vec x,y)$ --- primijetimo da nam ovdje "kontrolna varijabla" zapravo i ne treba, jer $y$ uvijek možemo dobiti kao $\f{lh}(z)$. Kao što smo, primjerice, operator $\sum$ mogli dobiti iteriranjem operacije $+$~\eqref{eq:sumH} na početnoj vrijednosti $0$~\eqref{eq:sumG}, tako operator $\overline{\cdots}$ možemo dobiti iteriranjem operacije $*$ na početnoj vrijednosti $\langle\rangle$. Dakle, vrijedi
\begin{align}
    \overline{\f F}(\vec x,y+1)&=\overline{\f F}(\vec x,y)*\langle\f F(\vec x,y)\rangle=\overline{\f F}(\vec x,y)*\bigl\langle\f G\bigl(\vec x,\overline{\f F}(\vec x,y)\bigr)\bigr\rangle\text,
\shortintertext{odnosno}
    \f H(\vec x,y,z)&:=z*\f{Code}^1\bigl(\f G(\vec x,z)\bigr)\text.
\end{align}
Sada je funkcija $\f H$ (primitivno) rekurzivna prema lemi~\ref{lm:starprn} i propoziciji~\ref{prop:Codekprn}, pa je i $\overline{\f F}=\f C_1^{k-1}\pr\f H$ (primitivno) rekurzivna jer je dobivena primitivnom rekurzijom iz (primitivno) rekurzivnih funkcija (za $k=1$ to je degenerirana primitivna rekurzija $\overline{\f F}=1\pr\f H$, pa je $\overline{\f F}$ (primitivno) rekurzivna po propoziciji~\ref{prop:F1prn} odnosno po korolaru~\ref{kor:F1rek}). Prema lemi~\ref{lm:povijestrek}, tada je i $\f F$ (primitivno) rekurzivna.
\end{proof}

\subsection{Primjeri korištenja rekurzije s poviješću}

\begin{primjer}
Fibonaccijev niz je vjerojatno najpoznatija funkcija definirana rekurzijom s poviješću:
\begin{align}
\SwapAboveDisplaySkip
    \f{Fib}(n)&:=n\text{, za  $n<2$;}\\
    \f{Fib}(n)&:=\f{Fib}(n-1)+\f{Fib}(n-2)\text{, inače.}
\end{align}
Dokažimo da je $\f{Fib}^1$ primitivno rekurzivna. Po propoziciji~\ref{prop:rekpov}, dovoljno je naći primitivno rekurzivnu funkciju $\f G$ koja prima povijest $p:=\overline{\f{Fib}}(n)$ (kod prvih $n$ vrijednosti Fibonaccijevog niza), te vraća sljedeću vrijednost $\f{Fib}(n)$. Kao što smo rekli, $n$ uvijek možemo dobiti kao $\f{lh}(p)$. Pomoću njega možemo i napisati pomoćnu funkciju za indeksiranje "s kraja" (koja je i inače prilično korisna: moderni programski jezici često dozvoljavaju indeksiranje s kraja pomoću negativnih indeksa, ali mi nemamo negativne brojeve pa ćemo upotrijebiti drugu funkciju):
\begin{align}\label{eq:rpartdef}
    \f{rpart}(c,i)&:=\begin{cases}
    c[\f{lh}(c)\ominus\f{Sc}(i)],&i<\f{lh}(c)\\
    0,&\text{inače}
    \end{cases}\text.
\intertext{Sada nije teško napisati točkovnu definiciju funkcije $\f G$:}
    \f G(p)&:=\begin{cases}
    \f{lh}(p),&\f{lh}(p)<2\\
    \f{rpart}(p,0)+\f{rpart}(p,1),&\text{inače}
    \end{cases}\text.
\end{align}
Prema teoremu o grananju (primitivno rekurzivna verzija), funkcija $\f{rpart}$, pa onda i funkcija $\f G$, je primitivno rekurzivna, a tada je i $\f{Fib}$ primitivno rekurzivna jer je dobivena rekurzijom s poviješću iz $\f G$.
\end{primjer}

Iako smo rekurziju s poviješću uveli koristeći funkcije, prelaskom na karakteristične funkcije možemo analogni rezultat dobiti za relacije. Ugrubo, ako pri odlučivanju vrijedi li $\f R(\vec x,n)$ koristimo samo istinitosti $\f R(\vec x,m)$ za $m<n$, i to na neki način koji čuva (primitivnu) rekurzivnost, tada je i $\f R$ (primitivno) rekurzivna. Evo jednog važnog primjera, koji će također poslužiti kao uvod u sljedeću točku, pokazivanjem da se mogu kodirati razni objekti, ne samo konačni nizovi prirodnih brojeva.

\begin{primjer}\label{pr:lskod}
Kodiramo formule logike sudova, tako da propozicijsku varijablu $P_i$ kodiramo kao $\langle0,i\rangle$, negaciju $\lnot\varphi$ kao $\langle1,u\rangle$ gdje je $u$ kod od $\varphi$, te $(\varphi\to\psi)$ kao $\langle2,u,v\rangle$ gdje je $u$ kod od $\varphi$, a $v$ kod od $\psi$. Ostali veznici se mogu dobiti pomoću negacije i kondicionala na dobro poznat način --- vidjeti~\cite{skr:VukML}. Recimo, kod varijable $P_0$ je $\langle0,0\rangle=2^1\cdot3^1=6$, a kod formule $(P_0\to P_0)$ je \begin{equation}
    \langle2,6,6\rangle=2^3\cdot3^7\cdot5^7=1\,366\,875\,000\text.
\end{equation}

Ovakva vrsta kodiranja, gdje se tip zapisuje na početku kao element nekog početnog komada od $\N$ (tzv.\ \texttt{enum}), a nakon njega ostali podaci ili kodovi (koji odgovaraju \emph{pokazivačima} na podatke kod rekurzivno definiranih struktura), česta je u računarstvu. U imperativnim jezicima (Pascal, Ada,~\ldots) obično se koristi pojam \emph{variant record}, a u funkcijskima (Haskell, Scala,~\ldots) pojam \emph{algebraic data type}. Recimo, u Haskellu bi deklaracija tog tipa izgledala ovako:
\begin{equation}
\texttt{data PF = PropVar Integer | Not PF | Implies PF PF}
\end{equation}
i reprezentacija elementa takvog tipa u računalnoj memoriji bila bi vrlo slična kodiranju koje smo mi napravili. Još jedan primjer, jednostavniji jer nije rekurzivan pa ne treba "pokazivače", vidjet ćemo na početku sljedeće točke.

Može se vidjeti da je to kodiranje injekcija, jer je kompozicija dvije injekcije: prva se dobije tako da "zamijenimo šiljate zagrade oblima", pa dobijemo elemente od $\N^*$, a druga je kodiranje $\N^*$. Ova druga je injekcija prema propoziciji~\ref{prop:codeinj}, ali zašto je prva injekcija? Ako su dvije formule različitih tipova (recimo, jedna je propozicijska varijabla, a druga negacija), preslikavaju se u konačne nizove s različitim prvim elementom. No ako su istog tipa, zapravo trebamo provesti neku indukciju po složenosti formule da bismo dokazali injektivnost. (Pokušajte --- to je dobra vježba.) Uostalom, i sama definicija je rekurzivna po izgradnji formula: recimo, u kodiranju $\lnot\varphi$ pretpostavljamo da već imamo kod od $\varphi$.

Pokušajmo sada dokazati da je slika tog kodiranja (nazovimo je $\f{PF}$) primitivno rekurzivna. Karakterističnu funkciju te slike, $\chi_{\f{PF}}$, definirat ćemo rekurzijom s poviješću. Kao i prije, trebamo naći primitivno rekurzivnu funkciju $\f G$ koja prima povijest $\overline{\chi_{\f{PF}}}(n)$, i vraća je li $n$ kod formule logike sudova. Kako vraća $0$ ili $1$ ($bool$), funkciju $\f G$ možemo dobiti kao karakterističnu funkciju neke relacije. Precizno, $\f G=\chi_{\f R}$, gdje $\f R$ relacija čija parcijalna specifikacija glasi: za $p:=\overline{\chi_{\f{PF}}}(n)$,
\begin{equation}
    \f R(p)\Longleftrightarrow\text{"$n=\f{lh}(p)$ je kod neke formule logike sudova".}
\end{equation}
Dakle, pretpostavimo da imamo $n$, i razmislimo kako bismo odlučili je li kod neke formule. Očito, to će vrijediti ako i samo ako je ili kod propozicijske varijable, ili kod negacije, ili kod kondicionala. Prvi disjunkt možemo napisati kao $\exists i(n=\langle0,i\rangle)$, ali to nije dovoljno dobro jer je kvantifikacija neograničena. Možemo li je ograničiti? Svakako. Lako je vidjeti da je svaki element konačnog niza manji od koda tog niza:
\begin{equation}\label{eq:elem<kod}
    x_i<x_i+1<2^{x_i+1}=p_0^{x_i+1}\le p_i^{x_i+1}\le(\,\dotsm)\cdot p_i^{x_i+1}\cdot(\,\dotsm)=\langle\dotsc,x_i,\dotsc\rangle\text.
\end{equation}
Dakle, ako postoji takav $i$, on je sigurno manji od $n$, te prvi disjunkt možemo napisati u obliku ograničene kvantifikacije $(\exists i<n)(n=\langle0,i\rangle)$, što je primitivno rekurzivno.

Za drugi disjunkt, opet možemo napisati $\exists u\bigl(n=\langle1,u\rangle\,\land\,\f{PF}(u)\bigr)$, i kao i prije možemo ograničiti kvantifikaciju na $(\exists u<n)$, ali što ćemo s rekurzivnim $\f{PF}(u)$? Izvući ćemo ga iz povijesti $p$, u kojoj su zapisane sve vrijednosti karakteristične funkcije od $\f{PF}$ na brojevima manjim od $n$. Dakle, drugi disjunkt je
\begin{equation}
    (\exists u<n)\bigl(n=\langle1,u\rangle\land p[u]=1\bigr)\text,
\end{equation}
što je primitivno rekurzivno. Analogno bismo dobili i treći disjunkt (s dvije ograničene kvantifikacije),  primitivno rekurzivan dvostrukom primjenom propozicije~\ref{prop:okvantrek} (i brojnih drugih rezultata koji pokazuju da je kvantificirana relacija primitivno rekurzivna).

Tada je $\f R$ primitivno rekurzivna kao disjunkcija tri primitivno rekurzivne relacije (propozicija~\ref{prop:skupl}), što znači da je njena karakteristična funkcija $\f G=\chi_{\f R}$ primitivno rekurzivna. I za kraj, prema propoziciji~\ref{prop:rekpov}, $\chi_{\f{PF}}$ je tada primitivno rekurzivna, dakle $\f{PF}^1$ je primitivno rekurzivna relacija.
\end{primjer}

\section{Kodiranje RAM-modela izračunljivosti}

Napokon imamo dovoljno alata da možemo proći kroz poglavlje~\ref{sec:RAMizr}, i sve bitno u njemu kodirati prirodnim brojevima. Tako ćemo dobiti mogućnost simulacije rada RAM-stroja primitivno rekurzivnim funkcijama, a time i parcijalnu rekurzivnost RAM-izračunljivih funkcija. Krenimo redom: prvo su na redu RAM-instrukcije.

%\subsection{Kodiranje instrukcija i programa}

Dok još nije dio RAM-programa, instrukcija ima samo tip (\inc, \dec\ ili \goto), te ovisno o tipu, adresu registra na koji djeluje, i\slash ili odredište (na koje zasad nema nikakvih uvjeta). Posljednje dvoje već jesu prirodni brojevi, a tip instrukcije možemo kodirati na način koji je standardan za konačne skupove: fiksiramo neki poredak. Konkretno, uzet ćemo kodiranje koje preslikava $\inc\mapsto0$, $\dec\mapsto1$, te $\goto\mapsto2$, točno onako kao što bi bio efekt, u programskom jeziku C, naredbe
\begin{equation}
\texttt{enum ins\_type \string{ INC, DEC, GOTO \string};}
\end{equation}
te smo time dobili injekciju s $\mathcal Ins$ u $\N^*$, čije kodiranje iskoristimo. Primijetimo da smo sličnu stvar već napravili manje formalno u dokazu leme~\ref{lema:alef0ins} --- disjunktifikacija unije skupova $A$ i $B$ u obliku $\{0\}\times A\cup\{1\}\times B$, koju poznajemo iz teorije skupova, upravo odgovara ovakvom kodiranju.

\begin{definicija}\label{def:kodIns}
Za proizvoljnu RAM-instrukciju $I$, definiramo \emph{kod instrukcije} $\kins{I}$, jednadžbama:
\begin{alignat}{3}
\SwapAboveDisplaySkip
    \kins{\incr j}&:=\langle0,j\rangle&\;=\;&&\f{codeINC}(j)&=6\cdot3^j\text,\\
    \kins{\decr jl}&:=\langle1,j,l\rangle&\;=\;&&\f{codeDEC}(j,l)&=60\cdot3^j\cdot5^l\text,\\
    \kins{\goto\;l}&:=\langle2,l\rangle&\;=\;&&\f{codeGOTO}(l)&=24\cdot3^l\text,
\end{alignat}
za sve $j,l\in\N$.
\end{definicija}
U desnom stupcu napisani su konstruktori kao aritmetički izrazi, iz kojih se vidi kako ih možemo primitivno rekurzivno izračunati iz $j$ i\slash ili $l$. Da je $\kins{\cdots}\colon\mathcal Ins\to\N$ injekcija, već smo objasnili. Da mu je slika $\f{Ins}:=\im{\,\kins{\cdots}}$ primitivno rekurzivna, možemo vidjeti na sličan način kao u primjeru~\ref{pr:lskod} (s istom argumentacijom zašto to funkcionira):
\begin{multline}\label{eq:Insprn}
    \f{Ins}(i)\Longleftrightarrow(\exists j<i)\bigl(i=\f{codeINC}(j)\bigr)\lor(\exists j<i)(\exists l<i)\bigl(i=\f{codeDEC}(j,l)\bigr)\lor{}\\
    {}\lor(\exists l<i)\bigl(i=\f{codeGOTO}(l)\bigr)\text.
\end{multline}

Što se inverza tiče, definiramo komponente $j$ odnosno $l$ kao dvije funkcije, te tip instrukcije kroz tri relacije.

\begin{lema}\label{lm:InsTYPEprn}
Skupovi $\f{InsINC}$, $\f{InsDEC}$ i $\f{InsGOTO}$, kodova instrukcija pojedinog tipa, primitivno su rekurzivni.
\end{lema}
\begin{proof}
Lako vidimo
\begin{align}
\SwapAboveDisplaySkip
    \label{eq:InsINC}\f{InsINC}(i)&\Longleftrightarrow\f{Ins}(i)\land i[0]=0\text,\\
    \label{eq:InsDEC}\f{InsDEC}(i)&\Longleftrightarrow\f{Ins}(i)\land i[0]=1\text,\\
    \label{eq:InsGOTO}\f{InsGOTO}(i)&\Longleftrightarrow\f{Ins}(i)\land i[0]=2\text.
\end{align}
Dokažimo samo ekvivalenciju~\eqref{eq:InsINC}, ostale su sasvim analogne. Za smjer $(\Rightarrow)$, ako je $i$ kod instrukcije tipa $\inc$, recimo $i=\kins{\incr j}$, tada po definiciji~\ref{def:kodIns} vrijedi $i=\langle0,j\rangle$, pa je $i[0]=0$ po propoziciji~\ref{prop:lhpartprn}. Štoviše, svaki element niza je manji od koda tog niza~\eqref{eq:elem<kod}, pa je $j<i$, odnosno vrijedi prvi disjunkt u~\eqref{eq:Insprn}, pa vrijedi i $\f{Ins}(i)$.

Za smjer $(\Leftarrow)$, pretpostavimo da vrijedi $\f{Ins}(i)$ te $i[0]=0$. Tada u disjunkciji~\eqref{eq:Insprn} mora vrijediti prvi disjunkt, pa postoji $j$ takav da je $i=\f{codeINC}(j)=\kins{\incr j}$, k\^od instrukcije tipa $\inc$.
\end{proof}

\begin{lema}\label{lm:regndestprn}
    Definiramo $\f{regn}(t)$ kao adresu registra na koji djeluje instrukcija koda $t$, ako takva postoji i djeluje na neki registar, a $0$ inače. Analogno definiramo $\f{dest}(t)$ za odredište. Funkcije $\f{regn}^1$ i\, $\f{dest}^1$ su primitivno rekurzivne.
\end{lema}
\begin{proof}
Koristeći prethodnu lemu i teorem o grananju (primitivno rekurzivnu verziju), odmah pišemo točkovne definicije:
\begin{align}
    \f{regn}(t)&=\begin{cases}
    t[1],&\f{InsINC}(t)\lor\f{InsDEC}(t)\\
    0,&\text{inače}
    \end{cases}\text,\\
    \f{dest}(t)&=\begin{cases}
    t[2],&\f{InsDEC}(t)\\
    t[1],&\f{InsGOTO}(t)\\
    0,&\text{inače}
    \end{cases}\text.
\end{align}
Naravno, napisane jednakosti su samo "obrnuto" napisana definicija~\ref{def:kodIns}.
\end{proof}

% \subsection{Kodiranje RAM-programa}

Sad kada imamo kodiranje instrukcija, lako je dobiti i kodiranje programa. Naime, programi su konačni nizovi instrukcija, te ih tako možemo i kodirati.

\begin{definicija}
Za proizvoljni RAM-program
$P:=\begin{prog}
t.&I_t
\end{prog}_{t<n}$
definiramo \emph{kod programa} $\kprog P$ kao
$\kprog P:=\langle\kins{I_0},\kins{I_1},\dotsc,\kins{I_{n-1}}\rangle$.
\end{definicija}

To je također kodiranje, iako bismo za konstruktore trebali napraviti dinamičke \emph{generatore koda} (jer većina programskih konstrukcija koje smo napravili nemaju fiksnu, statičku duljinu), koji primaju izračunljivu funkciju koja za svaki $i<n$ daje kod instrukcije s rednim brojem $i$. To se može napraviti sasvim općenito, i vidjeti da su sve konstrukcije programa koje smo dosad sreli (i koje ćemo još sresti u nastavku) primitivno rekurzivne, ali dva su razloga zašto to nećemo raditi.

Prvo, makroi kompliciraju stvar: većinu zanimljivih programa (recimo, one za računanje kompozicije, primitivne rekurzije i minimizacije) nismo napisali kao RAM-programe, nego kao makro-programe. Mogli bismo kodirati makroe kao zaseban tip instrukcija (prirodno se nameće $\kins{P^*}:=\langle3,\kprog{P}\rangle$ kao logičan nastavak definicije~\ref{def:kodIns}) i onda dobiti spljoštenje kao primitivno rekurzivnu funkciju $\f{flat}^1$ na kodovima, ali~\ldots\ postoji i drugi razlog, a taj je da je takvo razmišljanje u potpunoj općenitosti nepotrebno. Kad dokažemo univerzalnost našeg modela, postat će jasno da možemo jednu malu i jednostavnu transformaciju programa --- parametrizaciju --- napraviti kao primitivno rekurzivnu funkciju, a sve ostale transformacije napraviti pomoću nje.

Ipak, možemo vidjeti jedan jednostavni primjer.

\begin{primjer}\label{pr:kodkonst}
Pri samom početku, u dokazu teorema~\ref{tm:alef0izr}, napisali smo RAM-pro\-gra\-me $P_n$~\eqref{eq:konstRAM} koji računaju konstantne funkcije. Za svaki $n$ tako možemo dobiti kod tog programa, i to preslikavanje $n\mapsto\kprog{P_n}$ je primitivno rekurzivno. Doista, svaka instrukcija u tim programima je $\incr0$, s kodom $\f{codeINC}(0)=6$, pa je
\begin{equation}
    \kprog{P_n}=\langle6,6,\dotsc,6\rangle\text{ ($n$ šestica)}=\langle \f C_6(0),\f C_6(1),\dotsc,\f C_6(n-1)\rangle=\overline{\f C_6}(n)\text.
\end{equation}
$\overline{\f C_6^1}$ je primitivno rekurzivna po propozicijama~\ref{prop:konst} i~\ref{lm:povijestrek}.
(Kuriozitet: ista se stvar može dobiti i potenciranjem primorijela, $\kprog{P_n}=\bigl(\overline{\f Z}(n)\bigr)^7$\!; vidite li zašto?)
\end{primjer}

\begin{primjer}\label{pr:Qflatkod}
Evo i jednog statičkog primjera: u primjeru~\ref{pr:flat} naveden je RAM-program $Q^\flat$~\eqref{eq:Qflat}. Njegov kod je
\begin{multline}
    \kprog{Q^\flat}=\langle\kins{\decr12},\!\kins{\goto\;0},\!\kins{\decr22},\!\kins{\decr16},\!\kins{\incr0},\dotsc,\!\kins{\goto\;9}\rangle\\
    =\langle\langle1,\!1,\!2\rangle,\langle2,\!0\rangle,\langle1,\!2,\!2\rangle,\langle1,\!1,\!6\rangle,\langle0,\!0\rangle,\langle2,\!3\rangle,\langle1,\!2,\!9\rangle,\langle0,\!0\rangle,\langle2,\!6\rangle,\langle1,\!3,\!12\rangle,\langle0,\!0\rangle,\langle2,\!9\rangle\rangle=\\
    =\langle4\,500,24,13\,500,2\,812\,500,6,648,1\,054\,687\,500,6,\dotsc,6,472\,392\rangle=\\
    =2^{4501}\cdot3^{25}\cdot5^{13501}\dotsm31^7\cdot37^{472393}=:e_0\text.
\end{multline}
Taj kod ćemo koristiti u kasnijim primjerima.
\end{primjer}

Injektivnost preslikavanja $\kprog{\cdots}$ slijedi direktno iz injektivnosti od $\kins{\cdots}$ i $\langle\,\cdots\rangle$: dva različita programa se razlikuju ili po duljini (pa njihovi kodovi imaju različit $\f{lh}$), ili u nekoj instrukciji, recimo onoj na rednom broju $i$ (pa njihovi kodovi imaju različit $\f{part}$ na mjestu $i$, jer je kodiranje instrukcija injektivno).

\begin{lema}
    Slika kodiranja RAM-programa, $\f{Prog}^1:=\im{\,\kprog{\cdots}}=\{\kprog P\mid P\in\mathcal Prog\}$, primitivno je rekurzivna.
\end{lema}
\begin{proof}
Samo treba reći da se radi o konačnom nizu instrukcija, čija odredišta (ako postoje) su manja ili jednaka duljini programa:
\begin{equation}\label{eq:Progget}
    \f{Prog}(e)\Longleftrightarrow\f{Seq}(e)\wedge\bigl(\forall i<\f{lh}(e)\bigr)\bigl(\f{Ins}(e[i])\land\f{dest}(e[i])\le \f{lh}(e)\bigr)\text.\\
\end{equation}
Primijetimo samo da koristimo činjenicu da je $\f{dest}(t)=0$ ako instrukcija koda $t$ nema od\-re\-di\-šte, te je $0\le\f{lh}(e)$ uvijek. Marljiva evaluacija od $\land$ (odnosno $\f{mul}^2$) znači da će se $\f{lh}(e)$ uvijek izračunati, čak i kad $e$ nije kod konačnog niza, ali rezultat će sigurno biti $\mathit{false}$ (odnosno $0$), a nećemo zapeti u beskonačnoj petlji jer je $\f{lh}$ totalna. Izračunavanje će biti dulje jer nemamo \emph{shortcircuiting}, ali izračunljivost se neće promijeniti.
\end{proof}

Za kodiranje programa ne koristimo nikakve posebne komponente osim onih koje već imamo za konačne nizove: $\f{lh}(e)$ za duljinu programa, te $e[i]$ za dohvaćanje instrukcije rednog broja $i$. Uostalom, već smo ih koristili u~\eqref{eq:Progget}.

Samo još napomenimo da kodove RAM-programa obično označavamo slovom $e$. To je neka blaga vrsta "mađarske notacije", ponekad korištene u programiranju, u kojoj iz oblika imena zaključujemo tip vrijednosti koju tim imenom reprezentiramo.

I to je sve što se programa tiče. Spremni smo za idući korak.

\subsection{Kodiranje stanja registara i RAM-konfiguracija}

Konfiguraciju RAM-stroja smo definirali kao jednu funkciju, ali zapravo je jasno da trebamo kodirati dvije odvojene stvari: stanje registara, i vrijednost programskog brojača. Vrijednost programskog brojača već jest prirodan broj, pa ga ne kodiramo (odnosno kodiramo ga identitetom). Sa stanjem registara imamo više posla.

Na prvi pogled, skup svih mogućih stanja registara ne možemo uopće kodirati jer je neprebrojiv: imamo prebrojivo mnogo registara, svaki može sadržavati jednu od prebrojivo mnogo vrijednosti, a $\aleph_0^{\aleph_0}=\mathfrak c>\aleph_0$. Ipak, definicija~\ref{def:RAMconf} kaže da promatramo samo one konfiguracije koje su s konačnim nosačem --- i to je doista dovoljno. Naime, početna konfiguracija (bilo kojeg RAM-stroja s bilo kojim ulazom) je $0$ na svim registrima osim najviše $k$ ulaznih, te se u svakom koraku izračunavanja broj elemenata nosača može povećati najviše za $1$ --- izvršavanjem instrukcije $\incr j$ na nekom registru čija je vrijednost prije bila $r_j=0$.

Iz teorije skupova znamo da nizova prirodnih brojeva s konačnim nosačem ima prebrojivo mnogo, no kako ih kodirati? Jedna elegantna metoda koristi istu ideju kao za konačne nizove, samo bez sljedbenika eksponenata. Dakle, za proizvoljan niz s konačnim nosačem $(r_0,r_1,r_2,\dotsc,0,0,0,\dotsc)$ definiramo kod s
\begin{equation}\label{eq:defkreg}
    \knk{r_0,r_1,r_2,\dotsc,0,0,0,\dotsc}:=\prod_{i\in\N}p_i^{r_i}\,\text.
\end{equation}
Primijetimo da je, zbog konačnosti nosača, samo konačno mnogo eksponenata pozitivno (svi ostali su $0$), pa samo konačno mnogo prim-brojeva sudjeluje u produktu.

Važan razlog zašto je to idealno kodiranje stanja registara je to što 
ne ovisi o konkretnom RAM-stroju odnosno programu. Mogli bismo npr.\ proći kroz program i naći mu širinu $m_P$, pa kodirati stanje kao konačan niz registara do $\reg{m_P}$, ali to bi ovisilo o programu, i dodavanje potpuno irelevantnih instrukcija koje se možda uopće ne mogu izvršiti (\emph{unreachable code}) bi promijenilo kod stanja registara za isto izračunavanje (isti niz konfiguracija), te kodiranje ne bi bilo funkcija.

Treba nam jedan bitan konstruktor, a to je %\emph{proširenje nulom} zadanog konačnog niza. Kako već imamo kodiranje konačnih nizova, možemo ga definirati na kodovima.
početna konfiguracija RAM-stroja s ulazom $\vec x$ (prenesenim preko koda, jer želimo imati jednu funkciju za sve mjesnosti).

\begin{lema}\label{lm:startprn}
    Postoji primitivno rekurzivna funkcija $\f{start}^1$\!, s parcijalnom specifikacijom: za svaki $\vec x\in\N^+$\! (neprazni konačni niz), $\f{start}(\langle\vec x\rangle)=\knk{0,\vec x,0,0,0,\dotsc}$.
\end{lema}
\begin{proof}
Samo treba napisati definiciju:
\begin{equation}
    \f{start}(x):=\;\prod_{\mathclap{i<\f{lh}(x)}}p_{i+1}^{x[i]}=\;\prod_{\mathclap{i<\f{lh}(x)}}\f{pow}\bigl(\f{prime}\bigl(\f{Sc}(i)\bigr),\f{part}(x,i)\bigr)\text.
\end{equation}
Primitivna rekurzivnost slijedi iz propozicija~\ref{pp:primeprn} i~\ref{prop:lhpartprn}, napomene~\ref{nap:sumprodH} i primjera~\ref{pr:addmulpow}. Prim-brojevi su pomaknuti jer adrese ulaznih registara počinju od $1$.
\end{proof}

Napomenimo da funkcija $\f{start}$ nije injekcija: recimo, $\f{start}(\langle2,1\rangle)=\f{start}(\langle2,1,0,0\rangle)=\knk{0,2,1,0,0,\dotsc}$, odnosno $\f{start}(72)=\f{start}(2520)=45$. Ali preslikavanje $\knk{\cdots}$ jest injekcija (kao što kodiranje i treba biti), po osnovnom teoremu aritmetike. Također po osnovnom teoremu aritmetike, slika mu je $\im{\knk{\cdots}}=\N_+$, što smo dokazali primitivno rekurzivnim još odavno (primjer~\ref{pr:N+prn}).

Što se komponenata tiče, $\f{lh}$ nema smisla, a za indeksiranje služi funkcija $\f{ex}$ iz leme~\ref{lm:exprn}. Doista, $\f{ex}(\knk{r_0,r_1,\dotsc,0,0,\dotsc},i)=r_i$, jer je to upravo eksponent od $p_i$ u tom kodu. Koristit ćemo je u očitavanju rezultata (sadržaj registra $\reg0$ u završnoj konfiguraciji).
\begin{equation}\label{eq:resultdef}
    \f{result}(c):=\f{ex}(c,0)
\end{equation}

Za samo izvršavanje instrukcija na registrima, koristit ćemo množenje odnosno dijeljenje s $p_j$ za inkrement odnosno dekrement registra $\reg j$. Jasno je da time povećavamo odnosno smanjujemo eksponent odgovarajućeg prim-broja za $1$. Konkretno,
\begin{align}
\label{eq:knkppj}\knk{r_0,r_1,\dotsc,r_{j-1},r_j+1,r_{j+1},r_{j+2},\dotsc,0,0,\dotsc}
&=
\knk{r_0,r_1,\dotsc,0,0,\dotsc}\cdot p_j\text,
\\%\intertext{te ako je $r_i>0$,}
\label{eq:knkkpj}\knk{r_0,r_1,\dotsc,r_{j-1},r_j-1,r_{j+1},r_{j+2},\dotsc,0,0,\dotsc}
&=
\knk{r_0,r_1,\dotsc,0,0,\dotsc}\sslash p_j
\end{align}
(jasno, \eqref{eq:knkkpj} vrijedi samo ako je $r_j>0$).

Sada možemo kodirati dokaz leme~\ref{lema:ramdet}. Drugim riječima, za preslikavanje $nextconf$ koje svaku konfiguraciju RAM-stroja preslikava u "sljedeću" konfiguraciju (u koju ova prelazi), konstruirat ćemo $\N nextconf$ kao izračunljivu funkciju. Ona ima dva izlaza, pa ćemo je reprezentirati kroz dvije primitivno rekurzivne funkcije.

Jedan tehnički detalj: funkcijska relacija $\leadsto$ ovisi o programu $P$, no ako zanemarimo završne konfiguracije (i pravilo da one prelaze u same sebe), zapravo ovisi samo o trenutnoj instrukciji $I_{c(\textsc{pc})}$ tog programa. Zato će naše primitivno rekurzivne funkcije primati tu instrukciju, odnosno njen kod, i u parcijalnoj specifikaciji nećemo precizirati što rade ako je ulazna konfiguracija završna --- time ćemo se baviti kasnije.

\begin{lema}\label{lm:NextRegCountprn}
Za proizvoljnu RAM-konfiguraciju $c$, označimo kod stanja njenih registara s $c(\reg{*}):=\knk{c(\reg0),c(\reg1),\dotsc}$.

Postoje primitivno rekurzivne funkcije $\f{NextReg}^2$ i $\f{NextCount}^3$, koje zadovoljavaju sljedeću parcijalnu specifikaciju: za RAM-instrukciju $I$, i za RAM-kon\-fi\-gu\-ra\-ci\-je $c$ i $d$ takve da $c$ nije završna, te $c\leadsto d$ po instrukciji $I$, vrijedi
\begin{align}
    \f{NextReg}\bigl(\kins{I},c(\reg{*})\bigr)&=d(\reg{*})\text{, i}\\
    \f{NextCount}\bigl(\kins{I},c(\reg{*}),c(\textsc{pc})\bigr)&=d(\textsc{pc})\text.
\end{align}
\end{lema}
\begin{proof}
Označimo argumente tih funkcija redom s $i:=\kins{I}\in\f{Ins}$, $r:=c(\reg{*})\in\N_+$, te $p:=c(\textsc{pc})\in\N$ (naravno, ovaj treći samo za $\f{NextCount}$).

$\f{NextReg}$ treba pomnožiti ili podijeliti (ako je djeljiv) $r$ s $p_j$, ako $I$ djeluje na $\reg j$ --- ili ga ostaviti na miru, u suprotnom. $\f{NextCount}$ treba inkrementirati $p$ ako je $I$ promijenila neki registar, ili ga postaviti na njeno odredište ako nije.
\begin{align}
%\SwapAboveDisplaySkip
    \label{eq:NextRegdef}\f{NextReg}(i,r)&:=\begin{cases}
    r\cdot pj,&Up\\
    r\sslash pj,&Down\\
    r,&\text{inače}
    \end{cases}%\text,
\\
    \f{NextCount}(i,r,p)&:=\begin{cases}
    \f{Sc}(p),&Up\lor Down\\
    \f{dest}(i),&\text{inače}
    \end{cases}%\text,
\\
\text{uz pokrate }
pj&:=\f{prime}\bigl(\f{regn}(i)\bigr)%\text,
\\
Up&:\Longleftrightarrow\f{InsINC}(i)%\text,
\\
Down&:\Longleftrightarrow\f{InsDEC}(i)\land pj\mid r%\text.
\end{align}
$Up$ i $Down$ su disjunktni jer su već $\f{InsINC}$ i $\f{InsDEC}$ disjunktni ($i[0]$ ne može istovremeno biti $0$ i $1$). Također, uvjeti su primitivno rekurzivni, kao i pojedine grane od $\f{NextReg}$ i $\f{NextCount}$, pa su one primitivno rekurzivne po teoremu~\ref{tm:grek}.

Dokažimo da $\f{NextReg}$ i $\f{NextCount}$ doista zadovoljavaju željenu parcijalnu specifikaciju. U tu svrhu, neka su $I$ i $c$, odnosno $i$, $r$ i $p$ kao na početku dokaza. Tvrdimo da $\f{NextReg}(i,r)$ i $\f{NextCount}(i,r,p)$ upravo kodiraju konfiguraciju u koju $c$ prelazi po $I$.

Ako je $I$ tipa $\inc$, recimo $I=(\incr j)$, tada vrijedi $Up$ i $pj=\f{prime}(j)=p_j$, pa je $r'=r\cdot p_j$, što prema~\eqref{eq:knkppj} kodira upravo stanje registara nakon izvršavanja $I$.

Ako je $I$ tipa $\goto$, recimo $I=(\goto\;l)$, opet ne vrijedi ni $Up$ ni $Down$, te je $r'=r$, a $p'=\f{dest}(i)=\f{dest}(\kins{\goto\;l})=l$, kao što i treba biti.

Ako je pak $I$ tipa $\dec$, recimo $I=(\decr jl)$, tada je opet $pj=p_j$, te $Up$ ne vrijedi, a $Down\Leftrightarrow p_j\mid r=c(\reg{*})\Leftrightarrow c(\reg j)>0$. Ako je to istina, $p'=p+1$ i $r'=r\sslash p_j$, a ako nije, $p'=\f{dest}(\kins{\decr jl})=l$ i $r'=r$, kao što i treba biti po semantici instrukcije tipa $\dec$.
\end{proof}

\subsection{Kodiranje RAM-izračunavanja}

Primijetimo da sada raspolažemo svime potrebnim da bismo kodirali postupak iz\-ra\-ču\-na\-va\-nja kao niz konfiguracija. Konkretno, neka je $P^k$ RAM-algoritam, te $\vec x\in\N^k$ ulaz za njega. Htjeli bismo konstruirati izračunljivu funkciju koja prima $k\in\N_+$, $\vec x\in\N^k$ i $P\in\mathcal Prog$, te vraća niz $(c_n)_{n\in\N}$ koji predstavlja $P$-izračunavanje s $\vec x$. 

Izrazimo taj zadatak preko brojevnih funkcija. Umjesto $\vec x$ i $P$ očito možemo prenijeti njihove kodove $x:=\langle\vec x\rangle\in\f{Seq}':=\f{Seq}\setminus\{\langle\rangle\}$ i $e:=\kprog{P}\in\f{Prog}$. Tada ne treba prenositi $k$ jer ga uvijek možemo odrediti kao $\f{lh}(x)$. No kako vratiti niz? Skup kodova stanja registara je $\im{\knk{\cdots}}=\N_+$, a skup "kodova" vrijednosti programskog brojača je jednostavno $\N$ (zapravo $[0..\f{lh}(e)]$, ali hoćemo imati jednu kodomenu za sve $e\in\f{Prog}$), dakle skup kodova konfiguracija je $\N_+\!\times\N$, a skup kojem pripadaju izračunavanja je onda skup nizova $(\N_+\!\times\N)^\N$. Preslikavanje koje konstruiramo tada je element skupa
\begin{equation}
    \bigl((\N_+\!\times\N)^\N\bigr)^{\f{Seq}'\times\f{Prog}}\cong
    (\N_+\!\times\N)^{\f D}\cong\N_+^{\f D}\times\N^{\f D}\text,
\end{equation}
gdje smo označili $\f D:=\f{Seq}'\times\f{Prog}\times\N$ --- dakle zapravo želimo dvije tromjesne funkcije s ulaznim podacima $(x,e,n)\in\f D$, gdje je $n$ broj koraka izračunavanja koje smo napravili. Ukratko, nizove kodiramo "točkovno" tako da zapravo kodiramo njihove članove, dodavši indeks u nizu kao još jedan ulazni podatak, a algoritam s dva izlazna podatka (kod stanja registara i vrijednost programskog brojača) već standardno shvaćamo kao dva algoritma s istim ulaznim podacima. Sada je još samo potrebno proširiti te funkcije do totalnih tako da budu primitivno rekurzivne.

\begin{lema}\label{lm:RegCountprn}
Postoje primitivno rekurzivne funkcije $\f{Reg}^3$ i\, $\f{Count}^3$, čija parcijalna specifikacija glasi:
za svaki RAM-program $P$, za svaki neprazni konačni niz $\vec x$, za svaki prirodni broj $n$, $\f{Reg}(\langle \vec x\rangle,\kprog{P},n)$ je kod stanja registara, a $\f{Count}(\langle \vec x\rangle,\kprog{P},n)$ je vrijednost programskog brojača, nakon $n$ koraka $P$-izračunavanja s $\vec x$.
\end{lema}
Preciznije, ako je $P$-izračunavanje s $\vec x$ niz $(c_n)_{n\in\N}$, tada je $\f{Reg}(\langle\vec x\rangle, \kprog{P},n)=c_n(\reg{*})$, a $\f{Count}(\langle\vec x\rangle,\kprog{P},n)=c_n(\textsc{pc})$.
\begin{proof}
To što smo razdvojili reprezentaciju izračunavanja na dvije funkcije ne znači da ih možemo računati odvojenim algoritmima. Lako je vidjeti da, prema lemi~\ref{lema:ramdet}, sljedeća konfiguracija ovisi samo o neposredno prethodnoj (kao Markovljev lanac, samo deterministički), ali o oba njena dijela --- i o registrima i o brojaču. To znači da je prirodni način definiranja tih funkcija \emph{simultana rekurzija}.

Za nju nam prvo trebaju inicijalizacije, odnosno vrijednosti funkcija u $n=0$. To je lako: nakon $0$ koraka stanje registara je početna konfiguracija s ulazom $\vec x$, čiji se kod može dobiti iz koda $x:=\langle\vec x\rangle$ funkcijom $\f{start}$ iz leme~\ref{lm:startprn}:
\begin{equation}
    \f{Reg}(x,e,0)=\knk{0,x_1,x_2,\dotsc,x_k,0,0,\dotsc}=\f{start}(x)\text,
\end{equation}
dakle za općenitu točkovnu definiciju inicijalizacijske funkcije za $\f{Reg}$ (gdje je $x\in\N$ proizvoljan) možemo uzeti
\begin{equation}
    \f{Reg}(x,e,0):=\f G_1(x,e):=\f{start}(x)\text.
\end{equation}
Inicijalizacija za $\f{Count}$ je još jednostavnija: $\f G_2:=\f C_0^2$, jer je početna vrijednost programskog brojača uvijek $0$.

Prelazimo na korak računanja. Za njega nam trebaju funkcije $\f H_1^5$ i $\f H_2^5$, koje primaju $x$ i $e$, broj već napravljenih koraka $n$, te stare vrijednosti koda stanja registara $r$ i programskog brojača $p$ --- a vraćaju nove vrijednosti $r'$ i $p'$ redom. Njih dobijemo pomoću $\f{NextReg}$ i $\f{NextCount}$, s tim da sad moramo voditi računa i o završnim konfiguracijama, i preslikati ih u same sebe. $\f{NextReg}$ to već čini, jer ako $t$ nije kod instrukcije,~\eqref{eq:NextRegdef} kaže da je $\f{NextReg}(t,r)=r$ --- no za $\f H_2$ ćemo morati granati.
\begin{align}
%\SwapAboveDisplaySkip
    \f H_1(x,e,n,r,p)&:=\f{NextReg}(e[p],r)\text,\\
    \f H_2(x,e,n,r,p)&:=\begin{cases}
        \f{NextCount}(e[p],r,p),&
        p<\f{lh}(e)\\
        p,&\text{inače}
    \end{cases}\text.
\end{align}

Funkcije $\f H_1$ i $\f H_2$ su primitivno rekurzivne po teoremu o grananju~\ref{tm:grek}, propoziciji~\ref{prop:lhpartprn}, lemi~\ref{lm:NextRegCountprn} i primjeru~\ref{pr:m-v}. Sada je jasno da su $\f{Reg}$ i $\f{Count}$, definirane s
\begin{align}
%\SwapAboveDisplaySkip
    \f{Reg}(x,e,0)&:=\f G_1(x,e)=\f{start}(x)\text,\\
    \f{Count}(x,e,0)&:=\f G_2(x,e)=0\text,\\
    \label{eq:stepReg}\f{Reg}(x,e,n+1)&:=\f H_1\bigl(x,e,n,\f{Reg}(x,e,n),\f{Count}(x,e,n)\bigr)\text,\\
    \label{eq:stepCount}\f{Count}(x,e,n+1)&:=\f H_2\bigl(x,e,n,\f{Reg}(x,e,n),\f{Count}(x,e,n)\bigr)\text,
\end{align}
dobivene simultanom primitivnom rekurzijom iz primitivno rekuzivnih funkcija $\f G_1$, $\f G_2$, $\f H_1$ i $\f H_2$, te su primitivno rekurzivne po propoziciji~\ref{prop:simultrek}.

Dokažimo da doista ispunjavaju navedenu parcijalnu specifikaciju. U tu svrhu, neka je $\vec x$ neprazni konačni niz, $P$ RAM-program, te $x$ i $e$ njihovi kodovi redom. Tvrdnju dokazujemo indukcijom po $n$, broju koraka. Za $n=0$, $\f{Count}(x,e,n)=0$, što je po definiciji vrijednost programskog brojača nakon $0$ koraka izračunavanja. Također, iz definicije funkcije $\f{start}$ slijedi da je $\f{Reg}(x,e,0)$ kod stanja registara na početku izračunavanja.

Pretpostavimo da je parcijalna specifikacija zadovoljena nakon $n$ koraka, i pogledajmo što se događa u ($n+1$).\ koraku. Ako je konfiguracija $c_n$ (nakon $n$ koraka) završna, tada je po pretpostavci indukcije $\f{Count}(x,e,n)=c_n(\textsc{pc})=\f{lh}(e)$, te ne vrijedi uvjet grananja u definiciji $\f H_2$, pozvanoj  iz~\eqref{eq:stepCount}. Također je prema propoziciji~\ref{prop:lhpartprn} $e[\f{lh}(e)]=0\not\in\f{Ins}$, što znači da funkcija $\f H_1$ vraća nepromijenjen kod stanja registara (ne vrijedi ni $Up$ ni $Down$ u~\eqref{eq:NextRegdef}). Iz toga je $\f{Reg}(x,e,n+1)=\f{Reg}(x,e,n)$ i $\f{Count}(x,e,n+1)=\f{Count}(x,e,n)$, što i treba biti jer je $c_{n+1}=c_n$.

Ako pak $c_n$ nije završna konfiguracija, tada je $p:=c_n(\textsc{pc})<\f{lh}(e)$, te je $i:=e[p]$ upravo kod instrukcije koja se izvršava u ($n+1$).\ koraku. Tada je prema pretpostavci indukcije i lemi~\ref{lm:NextRegCountprn},
\begin{multline}
    \f{Reg}(x,e,n+1)=\f H_1\bigl(x,e,n,\f{Reg}(x,e,n),p\bigr)=\\
    =\f{NextReg}\bigl(e[p],\f{Reg}(x,e,n)\bigr)=\f{NextReg}\bigl(i,c_n(\reg{*})\bigr)=c_{n+1}(\reg{*})\text,
\end{multline}
i analogno $\f{Count}(x,e,n+1)=c_{n+1}(\textsc{pc})$, čime je proveden korak indukcije.
\end{proof}

\subsection{Prepoznavanje završne konfiguracije i čitanje rezultata}

Jednom kad imamo funkciju $\f{Count}$ koja daje vrijednost programskog brojača nakon zadanog broja koraka izračunavanja, lako je detektirati kad smo došli do završne konfiguracije.
\begin{equation}
    \f{Final}'(x,e,n):\Longleftrightarrow\f{Count}(x,e,n)=\f{lh}(e)
\end{equation}
%(Razlog zašto nam definicije imaju crtice je u tome da ovaj pristup, iako intuitivan, ima nekoliko mana i to zapravo neće biti "prave" definicije. Njih ćemo napisati kasnije.)

Zbog $\chi_{\f{Final}'}^3=\chi_=\circ(\f{Count},\f{lh}\circ\f I_2^3)$, to je primitivno rekurzivna relacija, te za sve $P\in\mathcal Prog$, $\vec x\in\N^+$ i $n\in\N$, $\f{Final}'(\langle\vec x\rangle,\kprog P,n)$ upravo znači da je konfiguracija nakon $n$ koraka $P$-izračunavanja s $\vec x$, završna.

Ipak, za proizvoljne $(x,e,n)\in\N^3$, može se dogoditi da vrijedi $\f{Final}'(x,e,n)$, iako $e$ uopće nije kod RAM-programa, ili $x$ nije kod nepraznog niza.

\begin{primjer}\label{pr:Final'}
Recimo, za $x=100$ i $e=10^{217}$, prvo bismo odredili
$\f{start}(100)$. Broj $100$ ima dva prim-djelitelja ($2$ i $5$), pa je $\f{lh}(100)=2$. Iz toga $\f{start}(100)=3^{100[0]}\cdot5^{100[1]}=\knk{0,1,0,0,\dotsc}=3=\f{Reg}(100,10^{217},0)$. Dakle, $\reg1$ kreće od $1$, a svi ostali registri od nule.  $\f{Count}(100,10^{217},0)=0$ jer \textsc{pc} uvijek kreće od nule.

Sada odredimo "trenutnu instrukciju": $e[0]=216=2^3\cdot3^3=\langle2,2\rangle=\kins{\goto\;2}\in\f{InsGOTO}$, te je $\f{Count}(100,10^{217},1)=\f{dest}(216)=2=\f{lh}(e)$. Dakle, $\f{Final}'(100,10^{217},1)$ vrijedi --- no nema smisla reći da to reprezentira zaustavljanje nekog iz\-ra\-ču\-na\-va\-nja, jer ne postoje $P\in\mathcal Prog$ i $\vec x\in\N^+$ takvi da bi to bilo $P$-izračunavanje s $\vec x$.
\end{primjer}

Primjer~\ref{pr:Final'} pokazuje da trebamo biti oprezni s parcijalnim specifikacijama. Dok su za primitivno rekurzivne funkcije one ponekad "nužno zlo" ili bar "manje zlo", za primitivno rekurzivne \emph{relacije} je mnogo prirodnije isključiti sve nespecificirane točke (efektivno, promatrati presjek s parcijalnom specifikacijom). To ćemo učiniti ovdje.
\begin{equation}
\f{Final}(x,e,n):\Longleftrightarrow\f{Seq'}(x)\land\f{Prog}(e)\land\f{Final}'(x,e,n)
\end{equation}

\begin{lema}\label{lm:Finalspec}
Relacija $\f{Final}^3$ je primitivno rekurzivna, te
za proizvoljne $(x,e,n)\in\N^3$, vrijedi $\f{Final}(x,e,n)$ ako i samo ako je $x$ kod nekog nepraznog konačnog niza $\vec x$ prirodnih brojeva, $e$ je kod nekog RAM-programa $P$, te $P$-izračunavanje s $\vec x$ stane nakon najviše $n$ koraka.
\end{lema}
\begin{proof}
Za primitivnu rekurzivnost: jednočlana relacija $\{\langle\rangle\}=\{1\}$ je primitivno rekurzivna po lemi~\ref{lm:r1prn}. Tada su po propoziciji~\ref{prop:vezn}, $\f{Seq'}:=\f{Seq}\setminus\{1\}$, a onda i $\f{Final}$, primitivno rekurzivne.

Za specifikaciju, smjer slijeva nadesno, raspetljavanjem definicije od $\f{Final}(x,e,n)$ vidimo da mora vrijediti:
\begin{itemize}
    \item $\f{Seq}(x)$, dakle $x=\langle\vec x\rangle$ za neki $\vec x\in\N^*$, jer je $\f{Seq}=\im{\langle\,\cdots\rangle}$;
    \item $x\not=1=\langle\rangle$, dakle $\vec x$ je neprazan, jer je $\langle\,\cdots\rangle$ injekcija;
    \item $\f{Prog}(e)$, dakle $e=\kprog{P}$ za neki RAM-program $P$, jer je $\f{Prog}=\im{\kprog{\cdots}}$;
    \item $\f{Count}(x,e,n)=\f{lh}(e)$, što prema lemi~\ref{lm:RegCountprn} (koju možemo primijeniti zbog prethodne tri stavke) znači da je $n$-ta konfiguracija u $P$-izračunavanju s $\vec x$ završna --- iz čega slijedi da ono stane, te je prvi indeks završne konfiguracije u izračunavanju manji ili jednak $n$.
\end{itemize}
Analogno, zaključivanjem u suprotnom smjeru, dobijemo i drugi smjer tvrdnje.
\end{proof}

To je u redu ako već imamo (kandidat za) $n$ --- ali možemo li \emph{izračunati} $n$? Naravno, to je upravo minimizacija:  $\f{step}:=\mu\,\f{Final}$ je parcijalno rekurzivna, te je $\f{step}(\langle\vec x\rangle,\kprog P)=\mu n\,\f{Final}(\langle\vec x\rangle,\kprog{P},n)$ upravo broj koraka $P$-izračunavanja s $\vec x$, do dolaska u završnu konfiguraciju. To je definirano samo za izračunavanja koja stanu: $HALT:=\dom{\f{step}}=\exists_*\f{Final}$ je dvomjesna relacija takva da $HALT(\langle \vec x\rangle,\kprog P)$ vrijedi ako i samo ako $P$-izračunavanje s $\vec x$ stane.

\begin{napomena}\label{nap:HaltProg}
Štoviše, zahvaljujući oprezu sa specifikacijom prije, sad možemo biti precizniji: kad god $e$ nije kod nekog programa, ili $x$ nije kod nepraznog konačnog niza, ne vrijedi  $\f{Final}(x,e,n)$, ni za koji $n$. Dakle, tada $(x,e)\not\in HALT$, te izraz $\f{step}(x,e)$ nema smisla.
\end{napomena}

Primijetimo da $HALT$ nismo napisali u "izračunljivom fontu", jer nemamo algoritam za računanje $\chi_{HALT}$. To što je $HALT$ domena parcijalno rekurzivne funkcije, odnosno projekcija primitivno rekurzivne relacije, nije dovoljno: ako je $(x,e)\in HALT$, to ćemo doznati čim nađemo broj koraka $n$ --- ali ako $(x,e)\not\in HALT$, to nećemo nikada doznati tim pristupom. Pokazat ćemo, štoviše, da \textbf{nijedan pristup ne radi za \emph{sve} parove} $(x,e)$, iako u nekim slučajevima (poput $e\not\in\f{Prog}$) možemo utvrditi $\lnot HALT(x,e)$. Precizno, pokazat ćemo da skup $HALT$ nije rekurzivan --- što će biti jedan od prvih primjera nepostojanja algoritma za neki problem. No to će morati još malo pričekati.

Pozabavimo se sada čitanjem rezultata izračunavanja (izlaznog podatka algoritma, odnosno vrijednosti funkcije koja se računa na ulaznim podacima). Za $HALT(x,e)$, izraz $\f{Reg}\bigl(x,e,\f{step}(x,e)\bigr)$ ima smisla, i tada predstavlja stanje registara u završnoj konfiguraciji izračunavanja. Za $\lnot HALT(x,e)$, što uključuje i $\lnot\f{Prog}(e)$ i $\lnot\f{Seq}'(x)$, taj izraz je nedefiniran. Još treba dokomponirati primitivno rekurzivnu funkciju $\f{result}$ iz~\eqref{eq:resultdef}, i dobiti
\begin{equation}
\label{eq:univdef}\f{univ}(x,e):\simeq\f{result}\bigl(\f{Reg}\bigl(x,e,\f{step}(x,e)\bigr)\bigr)\text,
\end{equation}
\emph{univerzalnu} funkciju koja preslikava RAM-program i ulaz za njega u rezultat iz\-ra\-ču\-na\-va\-nja, ako i samo ako je taj rezultat definiran.

\begin{lema}\label{lm:univspec}
Funkcija $\f{univ}^2$ je parcijalno rekurzivna, i za sve $x,e\in\N$ vrijedi:
\begin{enumerate}
    \item\label{tv:univ} Ako je $x$ kod nekog nepraznog konačnog niza $\vec x$, ako je $e$ kod nekog RAM-programa $P$, te ako $P$-izračunavanje s $\vec x$ stane, tada je $\f{univ}(x,e)$ rezultat tog izračunavanja.
    \item\label{tv:nuniv} U svim ostalim slučajevima, $(x,e)\not\in\dom{\f{univ}}$.
\end{enumerate}
\end{lema}
\begin{proof}
Parcijalna rekurzivnost slijedi direktno iz~\eqref{eq:univdef}, iz primitivne rekurzivnosti funkcije $\f{result}=\f{ex}\circ(\f I_1^1,\f Z)$ i $\f{Reg}$, te parcijalne rekurzivnosti funkcije $\f{step}$.

Za tvrdnju~\ref{tv:univ}, pretpostavke znače da postoji konfiguracija u $P$-izračunavanju s $\vec x$ koja je završna. Prema lemi~\ref{lm:Finalspec}, to znači $\exists n\,\f{Final}(x,e,n)$, odnosno $(x,e)\in\exists_*\f{Final}=\dom{\f{step}}$, pa postoji $n_0:=\f{step}(x,e)$, i $\f{Reg}(x,e,n_0)$ je kod stanja registara u završnoj konfiguraciji $c_{n_0}$ tog izračunavanja.
\begin{equation}
    z:=\f{Reg}(x,e,n_0)=c_{n_0}(\reg{*})=\knk{c_{n_0}(\reg0),c_{n_0}(\reg1),\dotsc}
\end{equation}
Sada je $\f{univ}(x,e)=\f{result}(z)=\f{ex}(\knk{c_{n_0}(\reg0),\dotsc},0)=c_{n_0}(\reg0)$, sadržaj registra $\reg0$ u završnoj konfiguraciji, što smo i trebali.

Za tvrdnju~\ref{tv:nuniv}, ako neki od  uvjeta nije zadovoljen, prema lemi~\ref{lm:Finalspec} (drugi smjer) ni za koji $n$ ne vrijedi $\f{Final}(x,e,n)$, te zato $(x,e)\not\in\dom{\f{step}}$. Po definiciji domene kompozicije (marljiva evaluacija), $(x,e)$ ne može biti niti u domeni od $\f{univ}$.
\end{proof}

\section{Kleenejev teorem o normalnoj formi}

Funkcija $\f{univ}$ je univerzalna, jer može simulirati bilo koji RAM-stroj, ondnosno ra\-ču\-na\-ti bilo koju RAM-izračunljivu funkciju, pa time (po teoremu~\ref{tm:pir}) i svaku parcijalno rekurzivnu funkciju. Na neki način, $\f{univ}$ predstavlja funkcijski \emph{interpreter} za RAM-stroj, nasuprot imperativnom \emph{kompajleru} za simboličke definicije izgrađenom u točki~\ref{sec:pir}. No iz raznih razloga, uglavnom tehničkih i povijesnih, zapravo se univerzalna funkcija uvodi na malo drugačiji način.

Htjeli bismo napisati što jednostavniju "simboličku definiciju" univerzalne funkcije. Definicija~\eqref{eq:univdef} ima nezgodno svojstvo da dvaput koristi $x$ i $e$, odnosno napisana je kao kompozicija dvije funkcije, svaka od kojih ovisi o ulaznim podacima. To je očito prekomplicirano: vanjska funkcija ne mora primati $x$ i $e$, ako joj unutarnja funkcija jednostavno pošalje završnu konfiguraciju, ili nešto iz čega se završna konfiguracija može odrediti. Razlog zašto ih trenutno prima je što joj unutarnja funkcija pošalje samo broj koraka, koji je očito sam po sebi nedovoljan za određivanje završne konfiguracije. No koristeći kodiranje $\N^*$, poslana vrijednost može biti proizvoljno komplicirana.

Drugo, očito od osnovnih operatora ($\circ$, $\pr$ i $\mu$) moramo koristiti minimizaciju jer $\f{univ}$ nije totalna --- ali htjeli bismo je koristiti što "kasnije", tako da što veći dio stabla koje predstavlja njenu simboličku definiciju bude primitivno rekurzivan. Napomenimo samo, iako to nećemo dokazati jer je jako teško, da se ne može postići da minimizacija bude u korijenu (drugim riječima, ne postoji primitivno rekurzivna relacija $\f{R}^3$ takva da je $\f{univ}=\mu\,\f{R}$) --- dakle moramo još nakon minimizacije primijeniti još neki operator. To ne može biti $\pr$ jer bismo time opet dobili totalnu funkciju, dakle mora biti kompozicija. Najjednostavnija "normalna forma" koja zadovoljava te uvjete je $\f{univ}^2=\f U^1\circ\mu\f{\dot T}^3$ za primitivno rekurzivne $\f U$ i $\f{\dot T}$ ($\f U$ je funkcija, $\f{\dot T}$ je relacija).

%Još jedan tehnički detalj je da želimo primiti argumente zasebno, a ne kao kod konačnog niza. To znači da ćemo umjesto $x$ imati $\vec x$, duljine $k$, te ćemo umjesto jedne relacije $\f{\dot T}^3$ imati po jednu relaciju $\f T_k^{k+2}$ za svaki $k\in\N_+$, a umjesto $\f{univ}^2$ imat ćemo familiju parcijalnih funkcija $\f{comp}_k^{k+1}$, $k\in\N_+$, koje će zadovoljavati
%\begin{equation}\label{eq:comp=univ}
    %\f{comp}_k(\vec x^k,e)\simeq\f{univ}(\langle\vec x\rangle,e)\text.
%\end{equation}
%Drugim riječima, za svaki $k\in\N_+$, $\f{comp}_k$ je dobivena kompozicijom iz parcijalno rekurzivnih funkcija $\f{univ}$ i $\f{Code}^k$, pa je parcijalno rekurzivna. Još je preostalo naći $\f U$ i $\f T_k$ takve da je $\f{comp}_k=\f U\circ\mu\,\f T_k$.

Najjednostavniji način da funkciji $\f U$ pošaljemo dovoljno podataka o izračunavanju je da joj pošaljemo \emph{čitavo} izračunavanje. Možemo li ga kodirati prirodnim brojem? Ako ne stane, teško: može se ponavljati ciklički, ali se može i ponašati vrlo komplicirano. Ali izračunavanja koja ne stanu ionako nas ne zanimaju, jer ne želimo da $\f U\circ\mu\f{\dot T}$ bude definirano u tom slučaju.

Dakle, promotrimo izračunavanje koje stane. Vidjeli smo da ono mora biti oblika $(c_0,c_1,\dotsc,c_{n_0},c_{n_0},c_{n_0},\dotsc)$, gdje je $c_{n_0}$ završna, a nijedna $c_i$ za $i<n_0$ nije završna. Za potrebe kodiranja, dovoljno je gledati samo konačan niz $(c_0,c_1,\dotsc,c_{n_0})$ duljine $n_0+1$, jer se iz njega beskonačnim ponavljanjem zadnjeg elementa može dobiti i čitavo izračunavanje. Za kodirati pojedinu $c_i$, trebali bismo poslati i $\f{Reg}(x,e,i)$ i $\f{Count}(x,e,i)$, no s obzirom na to da nam je u završnoj konfiguraciji po definiciji poznata ova druga vrijednost ($c_{n_0}(\textsc{pc})=\f{lh}(e)$) i zapravo nam treba sadržaj registra $\reg0$, slat ćemo samo vrijednosti funkcije $\f{Reg}$. Drugim riječima, treba nam povijest $\overline{\f{Reg}}(x,e,n_0+1)$, što je izračunljivo jednom kad imamo $n_0:=\f{step}(x,e)$.

\begin{definicija}\label{def:kodizr}
Neka je $P^k$ RAM-algoritam, te $\vec x\in\N^k$ takav da $P$-izračunavanje s $\vec x$ stane. \emph{Kod} tog izračunavanja definiramo kao povijest kodova stanja registara, do uključivo prvog indeksa završne konfiguracije u tom izračunavanju. Za izračunavanja koja ne stanu kod nije definiran.
\end{definicija}

Propozicija~\ref{prop:ramdet}, restringirana samo na izračunavanja koja stanu (jer jedino takva znamo kodirati), može se iskazati kao: tromjesna relacija \begin{equation}\label{eq:Trace}
    Trace\bigl(\vec x,P,(c_n)_n\bigr):\Longleftrightarrow\text{"$(c_n)_n$ je $P$-izračunavanje s $\vec x$, koje stane",}
\end{equation}
ima funkcijsko svojstvo. Tada će i $\f{\dot T}^3:=\N Trace$ imati funkcijsko svojstvo. Cilj nam je dokazati da je $Trace$ izračunljiva, odnosno da je $\f{\dot T}$ primitivno rekurzivna.

U skladu s napomenom u primjeru~\ref{pr:concat}, $Trace$ ćemo shvatiti kao ($k+2$)-mjesnu relaciju, tako da svaki $x_i$ shvatimo kao zasebni argument. To vodi na promatranje familije brojevnih relacija $\f T_k^{k+2},k\in\N_+$. A jednom kad dobijemo primitivnu rekurzivnost od $\f{\dot T}$, dobit ćemo i primitivnu rekurzivnost svih $\f T_k$, jer vrijedi
\begin{equation}\label{eq:TkviadotT}
\f T_k(\vec x,e,y)\Longleftrightarrow\f{\dot T}(\langle\vec x\rangle,e,y)\text,
\end{equation}
dakle $\f T_k$ je dobivena iz $\f{\dot T}$ kompozicijom s $\f{Code}^k$ i koordinatnim projekcijama.

\begin{primjer}
U primjeru~\ref{pr:makro} je naveden primjer makro-programa $Q$ i $Q$-izračunavanja s $(2,4)$, koje stane. Kako $Q^\flat$-izračunavanje s $(2,4)$ prolazi kroz ista stanja registara, možemo iz~\eqref{ml:Qstane} izračunati kod tog izračunavanja (iz definicije~\ref{def:makrolead} trebamo zanemariti prijelaze tipa~\ref{stav:carry}, jer oni ne odgovaraju nikakvim prijelazima RAM-stroja, kao i one tipa~\ref{stav:zav}, jer smo već stigli do završne konfiguracije):
\begin{multline}
    c_0:=\langle5625,1875,1875,625,625,625,125,125,25,50,50,10,20,20,4,8,8,8,8\rangle=\\
    =2^{5626}\cdot3^{1876}\cdot5^{1876}\cdot7^{626}\cdot11^{626}\cdot13^{626}\cdot17^{126}\dotsm61^9\cdot67^9\text.
\end{multline}
Recimo, $c_0[6]=\f{part}(c_0,6)=\f{pd}\bigl(\f{ex}(c_0,6)\bigr)=\f{pd}(126)=125=5^3=\knk{0,0,3,0,\dotsc}$, jer nakon $6$ koraka $Q^\flat$-izračunavanja s $(2,4)$ u $\reg2$ bude broj $3$, a u svim ostalim registrima broj $0$.

Drugi način za iskazati to isto je $\f{Reg}(\langle2,4\rangle,\kprog{Q^\flat},6)=\knk{0,0,3,0,\dotsc}$. Ako iz\-ra\-ču\-na\-mo $\langle2,4\rangle=2^3\cdot3^5=1944$ i upotrijebimo $e_0:=\kprog{Q^\flat}$ iz primjera~\ref{pr:Qflatkod}, možemo napisati i $\f{Reg}(1944,e_0,6)=5^3$. Vidimo da je $c_0$ povijest funkcije $\f{Reg}$, konkretno $\f{\dot T}(1944,e_0,c_0)$ znači $c_0=\overline{\f{Reg}}(1944,e_0,19)$, gdje je $\f{pd}(19)=18=\f{step}(1944,e_0)$.
\end{primjer}

%\subsection{Grafovi totalnih funkcija}

Funkcija $\f{step}$ jest izračunljiva, ali budući da nam treba \emph{relacija} za minimizaciju, koristit ćemo njen graf $\f{Step}^3:=\graf{\f{step}}$ kao relaciju iz koje možemo dobiti njene vrijednosti. 

\begin{napomena}\label{nap:parc=}
Primijetimo da \textbf{ne možemo} napisati $n=\f{step}(x,e)$ kao točkovnu definiciju relacije $\f{Step}$ --- to bi simbolički glasilo $\chi_{\f{Step}}=\chi_=\circ\bigl(\f I_3^3,\f{step}\circ(\f I_1^3,\f I_2^3)\bigr)$, što jednostavno nije istina jer te dvije funkcije imaju različite domene: lijeva je totalna, a desna je definirana samo na $HALT\times\N$. To je samo jedan od problema koje imamo s parcijalnim funkcijama, koji su osnovni razlog zašto se držimo primitivno rekurzivnih funkcija dok god možemo: vidjet ćemo kasnije da općenito graf (baš kao ni domena) izračunljive funkcije ne mora biti izračunljiv.
\end{napomena}

Ipak, za totalne funkcije takav rezultat vrijedi, što možemo već sada dokazati. Prvo strogo definirajmo graf, i dokažimo jednu tehničku lemu.

\begin{definicija}
Neka je $k\in\N_+$ i $f^k$ funkcija. \emph{Graf} funkcije $f$ je relacija $\graf f^{k+1}$ zadana s
\begin{equation}
    \label{eq:defgraf}\graf f(\vec x,y):\Longleftrightarrow\vec x\in\dom f\land y=f(\vec x)\text.
\end{equation}
Zapravo, uvjet $\vec x\in\dom f$ ne treba pisati jer slijedi iz $y=f(\vec x)$, ali je naveden zbog jasnoće (pogledajte problem koji smo maloprije imali sa $\graf{\f{step}}$).
\end{definicija}

\begin{napomena}\label{nap:graf=funsv}
Iz elementarne matematike znamo da je relacija $R$ graf neke funkcije ako i samo ako ima \emph{funkcijsko svojstvo}: $x\mathrel R y_1\land x\mathrel R y_2\Rightarrow y_1=y_2$. Riječima, "svaka vertikala siječe graf u najviše jednoj točki".
\end{napomena}

\begin{lema}\label{lm:projmugraf}
Za svaki $k\in\N_+$, za svaku funkciju $f^k$, vrijede sljedeće jednakosti:
\begin{align}
    \label{eq:projgraf}\exists_*\graf f&=\dom f\text,\\
    \label{eq:mugraf}\mu\,\graf f&=f\text.
\end{align}
\end{lema}
\begin{proof}
Za~\eqref{eq:projgraf}, iz $\vec x\in\exists_*\graf f$ slijedi da postoji $y\in\N$ takav da je $\vec x\in\dom f$ i $y=f(\vec x)$. Specijalno to znači $\vec x\in\dom f$. U drugom smjeru, $\vec x\in\dom f$ znači da postoji $f(\vec x)\in\N$, a onda vrijedi i $\graf f\bigl(\vec x,f(\vec x)\bigr)$, pa je $\vec x\in\exists_*\graf f$.

Dokažimo sada~\eqref{eq:mugraf}. Iz~\eqref{eq:projgraf} slijedi da te dvije funkcije imaju istu domenu~\eqref{eq:dommu}, trebamo samo vidjeti da se podudaraju u svim točkama te domene. U tu svrhu, neka je $\vec x\in\dom f$. Tada postoji $f(\vec x)=:y_0\in\N$, i vrijedi $\graf f(\vec x,y_0)$. Štoviše, zbog funkcijskog svojstva, ni za koji drugi $y\not=y_0$ ne vrijedi $\graf f(\vec x,y)$, dakle skup $\{y\in\N\mid\graf f(\vec x,y)\}$ je jednočlan skup $\{y_0\}$, pa mu je najmanji element $\mu y\,\graf f(\vec x,y)=y_0=f(\vec x)$.
\end{proof}

\begin{teorem}[Teorem o grafu za totalne funkcije]\label{tm:graftot}
Neka je $k\in\N_+$, te\, $\f{f}^k$ totalna funkcija. Tada je\, $\graf{\f f}$ rekurzivan ako i samo ako je\, $\f f$ rekurzivna.
\end{teorem}
\begin{proof}
Za smjer ($\Rightarrow$): po pretpostavci, $\chi_{\graf{\f f}}$ je rekurzivna, dakle parcijalno rekurzivna. Skup parcijalno rekurzivnih funkcija je zatvoren na minimizaciju, pa je $\mu\,\graf f$ također parcijalno rekurzivna --- no ta funkcija je jednaka $\f f$ po~\eqref{eq:mugraf}. Dakle, $\f f$ je parcijalno rekurzivna, a po pretpostavci teorema je totalna, pa je rekurzivna.

Za smjer ($\Leftarrow$): kako je $\f f$ totalna, $\vec x^k\in\dom f=\N^k$ uvijek vrijedi, pa~\eqref{eq:defgraf} postaje $\graf{\f f}(\vec x,y)\Longleftrightarrow y=\f f(\vec x)$, odnosno $\chi_{\graf{\f f}}$ je dobivena kompozicijom iz $\chi_=$, $\f f$ i koordinatnih projekcija. $\chi_=$ i koordinatne projekcije su rekurzivne po korolarima~\ref{kor:jednakost} i~\ref{kor:prnrek}, a $\f f$ je rekurzivna po pretpostavci, pa je $\chi_{\graf{\f f}}$ rekurzivna po lemi~\ref{lm:comprek}.
\end{proof}

Napomenimo samo da u teoremu~\ref{tm:graftot} ne možemo staviti riječ "primitivno" u zagrade, kao što smo činili u mnogim rezultatima do sada: postoje rekurzivne funkcije čiji grafovi su primitivno rekurzivni, ali one same nisu primitivno rekurzivne. \emph{Ackermannova} funkcija, uvedena u~\cite[dodatak]{skr:Vuk}, primjer je takve funkcije.

\subsection{Relacije \texorpdfstring{$\f{Step}$ i $\f T_k$}{Step i Tk}}

Kako $\f{step}$ nije totalna, ne možemo direktno primijeniti teorem~\ref{tm:graftot}, ali možemo neke druge rezultate. Konkretno, prema~\eqref{eq:mugraf}, imamo $\f{step}=\mu\,\f{Step}$, no $\f{step}$ je već definirana minimizacijom primitivno rekurzivne relacije $\f{Final}$. Relacije $\f{Step}$ i $\f{Final}$ nisu jednake jer jedna ima funkcijsko svojstvo a druga nema (čim vrijedi $\f{Final}(x,e,n_0)$, vrijedi i $\f{Final}(x,e,n)$ za sve $n>n_0$), ali možemo li dobiti jednu pomoću druge? Svakako, $\f{Final}(x,e,n)\Leftrightarrow(\exists m\le n)\f{Step}(x,e,m)$ znači da je $\f{Final}$ dobivena ograničenom egzistencijalnom kvantifikacijom iz $\f{Step}$, ali nama treba drugi smjer.

Relacija $\f{Step}(x,e,m)\Leftrightarrow m= \mu n\,\f{Final}(x,e,n)$, koju dobijemo direktnim čitanjem definicije $\f{Step}=\graf{\f{step}}$, čini se kao dobar početak: jedino što joj nedostaje je totalnost ove minimizacije na desnoj strani. No taj problem smo već imali nekoliko puta u točki~\ref{sec:teobroj}, i uvijek smo ga uspješno rješavali ograničavanjem minimizacije. Postoji li gornja granica za $n$ do koje je dovoljno provjeravati vrijedi li $\f{Final}(x,e,n)$, da bismo znali je li $m$ najmanji takav $n$? Naravno --- to je upravo $m+1$! Odnosno, dovoljno je provjeravati do uključivo $m$.

\begin{lema}
Relacija $\f{Step}^3:=\graf{\f{step}^2}$ je primitivno rekurzivna.
\end{lema}

\begin{proof}
Kao što smo upravo rekli, cilj nam je dokazati
\begin{equation}
    \f{Step}(x,e,m)\Longleftrightarrow m=(\mu n\le m)\f{Final}(x,e,n)
\end{equation}
--- iz toga će onda slijediti primitivna rekurzivnost prema (redom) lemi~\ref{lm:Finalspec}, propoziciji~\ref{prop:muHprn}, napomeni~\ref{nap:muSc} i korolaru~\ref{kor:jednakost}.

Pretpostavimo da vrijedi $\f{Step}(x,e,m)$. Tada vrijedi $(x,e)\in\dom{\f{step}}=HALT$, i $m=\f{step}(x,e)$. Dakle vrijedi $\f{Final}(x,e,m)$, i ni za koji $n<m$ ne vrijedi $\f{Final}(x,e,n)$, te je $\mu n\bigl(n\le m\to\f{Final}(x,e,n)\bigr)=\mu n\,\f{Final}(x,e,n)=\f{step}(x,e)=m$.

Ako pak ne vrijedi $\f{Step}(x,e,m)$, tada negiranjem~\eqref{eq:defgraf} vidimo da ili ne vrijedi $HALT(x,e)$, ili pak postoji $s:=\f{step}(x,e)$, ali je različit (veći ili manji) od $m$. Tvrdimo da ni u kojem od tih slučajeva broj $t:=(\mu n\le m)\f{Final}(x,e,n)$ nije jednak $m$.

Ako $\lnot HALT(x,e)$, tada ne postoji $n$ takav da vrijedi $\f{Final}(x,e,n)$, pa je
\begin{equation}
    t = \mu n(n\le m\to\bot)=\mu n\lnot(n\le m)=\mu n(n>m)= m+1\not=m\text.
\end{equation}
Ako je $s<m$, tada je $t=s$, pa je opet $t\not=m$. Ako je $s>m$, tada (po definiciji funkcije $\f{step})$ za svaki $n<s$ --- pa specijalno za svaki $n\le m$ --- vrijedi $\lnot\f{Final}(x,e,n)$, te je opet $t=m+1\not=m$.
\end{proof}

Sada se napokon možemo pozabaviti relacijama $\f{\dot T}$ i $\f T_k,k\in\N_+$. Prisjetimo se, one su dobivene kodiranjem argumenata relacije $Trace$, prva kodirajući ulazne podatke $\vec x$ kao element od $\N^*$, a druga gledajući ih zasebno. 

\begin{propozicija}\label{prop:Tkprn}
Za svaki $k\in\N_+$, relacija $\f T_k$, zadana s
\begin{equation}\label{eq:Tkdef}
    \f T_k(\vec x^k,e,y):\Longleftrightarrow(\exists P\in\mathcal Prog)(e=\kprog{P}\land \text{"$y$ je kod $P$-izračunavanja s $\vec x$"})\text,
\end{equation}
primitivno je rekurzivna.
\end{propozicija}
\begin{proof}
Kao što smo već rekli, prvo ćemo dokazati primitivnu rekurzivnost relacije $\f{\dot T}$, zadane s
\begin{equation}\label{eq:dotTdef}
    \f{\dot T}(x,e,y):\Longleftrightarrow(\exists\vec x\in\N^+)\bigl(x=\langle\vec x\rangle\land\f T_{\f{lh}(x)}(\vec x,e,y)\bigr)\text.
\end{equation}

Tvrdimo da je njena točkovna definicija
\begin{align}
    \dot{\f T}(x,e,y)\Longleftrightarrow\f{Step}&(x,e,n)\land y=\overline{\f{Reg}}\bigl(x,e,\f{Sc}(n)\bigr)\text,\\
    &\text{uz pokratu }n:=\f{pd}\big(\f{lh}(y)\bigr)\text.
    %\f T_k(\vec x^k,e,y)&:\Longleftrightarrow\dot{\f T}(\langle\vec x\rangle,e,y)\text{, za svaki $k\in\N_+$.}
\end{align}
U jednom smjeru, pretpostavimo da vrijedi $\f{\dot T}(x,e,y)$. Tada prema~\eqref{eq:dotTdef} i~\eqref{eq:Tkdef} postoje $\vec x\in\N^+$ (njegovu duljinu označimo s $k$) i $P\in\mathcal Prog$ takvi da je $x$ kod od $\vec x$, $e$ je kod od $P$, a $y$ je kod $P$-izračunavanja s $\vec x$. Po definiciji~\ref{def:kodizr}, to znači da $P$-izračunavanje s $\vec x$ stane (inače kod ne bi bio definiran) --- odnosno vrijedi $HALT(x,e)$, pa postoji $n_0:=\f{step}(x,e)$ --- i $y$ je upravo povijest stanja registara prvih $n_0+1$ konfiguracija u tom izračunavanju, od $c_0$ do uključivo $c_{n_0}$.

Iz $n_0=\f{step}(x,e)$ slijedi $\f{Step}(x,e,n_0)$, povijest stanja registara opisana je funkcijom $\overline{\f Reg}$, a upravo smo vidjeli da je $\f{lh}(y)=\f{Sc}(n_0)$. Dakle, $n_0=\f{pd}\bigl(\f{lh}(y)\bigr)$, što se upravo tvrdi u točkovnoj definiciji.

U drugom smjeru, pretpostavimo da vrijedi točkovna definicija. Tada iz $(x,e,n)\in\f{Step}=\graf{\f{step}}$ slijedi $(x,e)\in\dom{\f{step}}=HALT$ i $n:=\f{pd}\bigl(\f{lh}(y)\bigr)=\f{step}(x,e)=(\mu\,\f{Final})(x,e)$. Specijalno vrijedi $\f{Final}(x,e,n)$, pa je (po lemi~\ref{lm:Finalspec}) $x$ kod nekog nepraznog konačnog niza $\vec x$, $e$ je kod nekog RAM-programa $P$, te $P$-izračunavanje s $\vec x$ stane nakon najviše $n$ koraka --- zapravo u ovom slučaju nakon točno $n$ koraka, jer je $n=\f{step}(x,e)$.

Također vrijedi $y=\overline{\f{Reg}}\bigl(x,e,\f{Sc}(n)\bigr)$, dakle $\f{lh}(y)=\f{Sc}(n)>0$. To znači da je $y$ povijest stanja registara prvih $n+1$ konfiguracija, što je upravo kod tog izračunavanja.

Primijetimo još samo da nismo mogli zapisati jednostavnije $y=\f{Reg}\bigl(x,e,\f{lh}(y)\bigr)$, jer bi to zadovoljavao i kod praznog niza: $1=\overline G\bigl(\vec x,\f{lh}(1)\bigr)$ bez obzira na specifikaciju funkcije $G$ i vrijednosti argumenata $\vec x$. Na neki način, $0$ je problematična kao $m=\f{lh}(y)$, pa smo umjesto $m$ napisali $\f{Sc}\bigl(\f{pd}(m)\bigr)=m\bump$ u duhu napomene~\ref{nap:crtica}.

Sada za proizvoljni $k$, primitivna rekurzivnost $\f T_k$ slijedi iz~\eqref{eq:TkviadotT}.
\end{proof}

\begin{korolar}\label{kor:funHaltTk}
    Za svaki $k\in\N_+$, relacija $\f T_k$ ima funkcijsko svojstvo, te je njena projekcija $\exists_*\f T_k=\{(\vec x,e)\in\N^{k+1}\mid HALT(\langle\vec x\rangle,e)\}=:Halt_k$.
\end{korolar}
\begin{proof}
Direktno iz definicije: za proizvoljne $\vec x$ i $e$, postoji najviše jedan RAM-program $P$ s kodom $e$ ($\kprog{\cdots}$ je injekcija), pa onda postoji jedinstveno $P$-izračunavanje s $\vec x$ (propozicija~\ref{prop:ramdet}), pa ako ono stane, postoji jedinstven njegov kod. Ako $P$ ne postoji, ili ako izračunavanje ne stane, ne postoji nijedan $y$ takav da vrijedi $\f T_k(\vec x,e,y)$. Dakle, uvijek postoji \emph{najviše} jedan takav $y$, te postoji \emph{točno} jedan takav $y$ ako i samo ako je $(\vec x,e)\in Halt_k$ --- a to je upravo tvrdnja koju smo željeli dokazati.
\end{proof}

\subsection{Funkcije \texorpdfstring{$\f U$ i $\f{comp}_k$}{U i comp}}

Korolar~\ref{kor:funHaltTk} prema napomeni~\ref{nap:graf=funsv} kaže da je za svaki pozitivni $k$, relacija $\f T_k$ graf neke funkcije. Štoviše, po lemi~\ref{lm:projmugraf}, ta funkcija je upravo $\mu\f T_k$, njena domena je $Halt_k$, i ona preslikava svaki $(\vec x^k,\kprog{P})\in Halt_k$ (svaki element od $Halt_k$ mora biti tog oblika, po napomeni~\ref{nap:HaltProg}) u kod $P$-izračunavanja s $\vec x$. Sada, da bismo dobili \emph{rezultat} tog izračunavanja, samo treba dokomponirati slijeva funkciju zadanu s
\begin{equation}\label{eq:Udef}
    \f U(y):=\f{result}\bigl(\f{rpart}(y,0)\bigr)=\f{ex}(y[\f{pd}(\f{lh}(y))],0)\text,
\end{equation}
doslovno, "sadržaj izlaznog registra zadnje konfiguracije kodirane s $y$", te ćemo dobiti
\begin{equation}
    \f{comp}_k(\vec x,y):\simeq \f U\bigl(\mu y\,\f T_k(\vec x,e,y)\bigr)\text.
\end{equation}

\begin{propozicija}\label{prop:compspec}
Funkcija $\f U$ je primitivno rekurzivna. Za svaki $k\in\N_+$, funkcija $\f{comp}_k$ je parcijalno rekurzivna, s domenom $\dom{\f{comp}_k}=Halt_k$, te vrijedi \begin{equation}\label{eq:comp=univ}
    \f{univ}(\langle\vec x\rangle,e)\simeq\f{comp}_k(\vec x,e)\text.
\end{equation}
\end{propozicija}
\begin{proof}
Prvo, $\f U=\f{result}\circ\f{rpart}\circ(\f I_1^1,\f Z)$ je simbolička definicija od $\f U$: $\f{result}$ je točkovno definirana u~\eqref{eq:resultdef}, a $\f{rpart}$ u~\eqref{eq:rpartdef}, pomoću primitivno rekurzivnih funkcija ($\f{ex}$, $\f{part}$, $\f{pd}$ i $\f{lh}$), pa su $\f{result}$ i $\f{rpart}$ --- a onda i $\f U$ --- primitivno rekurzivne.

Sada je $\f{comp}_k=\f U\circ\mu\f T_k$ parcijalno rekurzivna jer je dobivena kompozicijom i minimizacijom iz primitivno rekurzivnih $\f U$ i $\f T_k$. Prema korolaru~\ref{kor:funHaltTk} domena joj je $Halt_k$, baš kao i domena funkcije $(\vec x,e)\mapsto\f{univ}(\langle\vec x\rangle,e)$ (po definiciji domene kompozicije, uzevši u obzir da je $\f{Code}^k$ totalna). Još treba samo vidjeti da se te dvije funkcije podudaraju na toj domeni. Za svaki $(\vec x,e)\in Halt_k$ vrijedi (uz oznake $x:=\langle\vec x\rangle$ i $n:=\f{step}(x,e)$, dakle vrijedi $\f{Step}(x,e,n)$):
\begin{multline}
    \f{univ}(x,e)=
    \f{result}\bigl(\f{Reg}(x,e,n)\bigr)=
    \f{result}\bigl(\overline{\f{Reg}}(x,e,\f{Sc}(n))[n]\bigr)=\\
    =
    \f{result}\bigl(\f{rpart}\bigl(\overline{\f{Reg}}(x,e,\f{Sc}(n)),0\bigr)\bigr)=
    \f U\bigl(\overline{\f{Reg}}(x,e,\f{Sc}(n))\bigr)
    =\\
    =\f U\bigl(\mu y\bigl(y=\overline{\f{Reg}}(x,e,\f{Sc}(n))\bigr)\bigr)=
    \f U\bigl(\mu y\bigl(y=\overline{\f{Reg}}(x,e,\f{Sc}(n))\land\f{Step}(x,e,n)\bigr)\bigr)=\\
    =\f U\bigl(\mu y\f{\dot T}(x,e,y)\bigr)=
    \f U\bigl(\mu y\f T_k(\vec x,e,y)\bigr)=
    \f{comp}_k(\vec x,e)\text,
\end{multline}
iz čega slijedi tražena parcijalna jednakost.
\end{proof}

\begin{korolar}\label{kor:compspec}
Za svaki $k\in\N_+$, za sve $(\vec x,e)\in\N^{k+1}$ vrijedi:
\begin{enumerate}
    \item\label{tv:comp} Ako je $e$ kod nekog RAM-programa $P$, te ako $P$-izračunavanje s $\vec x$ stane, tada je $\f{comp}_k(\vec x,e)$ rezultat tog izračunavanja.
    \item\label{tv:ncomp} U ostalim slučajevima ($e\not\in\f{Prog}$, ili $e=\kprog{P}$ ali $P$-izračunavanje s $\vec x$ ne stane), izraz $\f{comp}_k(\vec x,e)$ nema smisla.
\end{enumerate}
\end{korolar}
\begin{proof}
Ovo je zapravo lema~\ref{lm:univspec}, iskazana na malo drugačiji način koristeći upravo dokazanu parcijalnu jednakost~\eqref{eq:comp=univ} --- i pojednostavljena jer znamo da vrijedi $\langle\vec x^k\rangle\in\f{Seq}'$ za $k\in\N_+$, pa to ne treba pisati u uvjete.
\end{proof}

Sve bitno što smo dosad napravili u ovoj točki može se iskazati u jednom teoremu.

\begin{teorem}[Kleenejev teorem o normalnoj formi]\label{tm:Kleene}
Postoji primitivno rekurzivna funkcija $\f U$, takva da za svaki $k\in\N_+$ postoji primitivno rekurzivna relacija $\f T_k$, takva da za svaku parcijalno rekurzivnu funkciju $\f f$ postoji broj $e$, takav da za svaki $\vec x\in\N^k$ vrijede sljedeće dvije tvrdnje:
\begin{gather}
    \label{eq:kleenedom}\vec x\in\dom{\f f}\Longleftrightarrow\exists y\,\f T_k(\vec x,e,y)\text,\\
    \label{eq:kleeneval}\f f(\vec x)\simeq\f U\bigl(\mu y\,\f T_k(\vec x,e,y)\bigr)\text.
\end{gather}
\end{teorem}
\begin{proof}
Funkcija $\f U$ je definirana s~\eqref{eq:Udef}, i dokazano je da je primitivno rekurzivna u propoziciji~\ref{prop:compspec}. Za svaki $k\in\N_+$, relacija $\f T_k$ je definirana, i dokazano je da je primitivno rekurzivna, u propoziciji~\ref{prop:Tkprn}. Neka je sada $\f f$ proizvoljna parcijalno rekurzivna funkcija. Prema teoremu~\ref{tm:pir}, postoji RAM-program koji računa $\f f$. Uzmimo $e:=\kprog P$, i neka je $\vec x\in\N^k$ proizvoljan.

Prema definiciji~\ref{def:compute}, ako je $\vec x\in\dom{\f f}$, tada $P$-izračunavanje s $\vec x$ stane, pa postoji njegov kod $y_0$. Tada po definiciji~\eqref{eq:Tkdef} vrijedi $\f T_k(\vec x,e,y_0)$, pa postoji $y$ (konkretno, $y_0$) takav da vrijedi $\f T_k(\vec x,e,y)$. Ako pak $\vec x\not\in\dom{\f f}$, tada opet po definiciji~\ref{def:compute} $P$-izračunavanje ne stane, te ne postoji njegov kod, odnosno ne postoji $y$ takav da vrijedi $\f T_k(\vec x,e,y)$. Time je dokazano~\eqref{eq:kleenedom}, odnosno dokazano je da u~\eqref{eq:kleeneval} lijeva i desna strana imaju smisla za iste $\vec x$ (jer lijeva ima smisla za $\vec x\in\dom{\f f}$, a desna za $(\vec x,e)\in\exists_*\f T_k$).

Za takve $\vec x$, kao što smo vidjeli, postoji kod $P$-izračunavanja s $\vec x$, koji smo označili s $y_0$ i vidjeli da vrijedi $\f T_k(\vec x,e,y_0)$. Ali $\f T_k$ ima funkcijsko svojstvo, dakle $y_0$ je \emph{jedini}, pa onda i najmanji, $y$ takav da vrijedi $\f T_k(\vec x,e,y)$. To znači da na desnoj strani zapravo piše $\f U(y_0)$, odnosno stanje registra $\reg0$ u zadnjoj konfiguraciji kodiranoj s $y_0$ --- što je opet po definiciji~\ref{def:compute} jednako $\f f(\vec x)$, jer ta konfiguracija mora biti završna.
\end{proof}

\begin{korolar}
Za svaku parcijalno rekurzivnu funkciju postoji simbolička definicija u kojoj se operator $\mu$ pojavljuje točno jednom.
\end{korolar}
\begin{proof}
Samo treba~\eqref{eq:kleeneval} zapisati u simboličkom obliku:
\begin{equation}
    \f f=\f U\circ\mu\f T_k\circ(\f I_1^k,\f I_2^k,\dotsc,\f I_k^k,\f C_e^k)\text.
\end{equation}
Vidimo da su $\f U$, $\f T_k$, $\f I_n^k$ i $\f C_e^k$ primitivno rekurzivne, dakle u njihovim simboličkim definicijama se ne pojavljuje minimizacija. Iz toga slijedi da se pojavljuje isključivo u dobivanju funkcije $\mu\f T_k$.
\end{proof}

%Štoviše, za fiksni $k$, postoje brojevi $z_k$, $r_k$ i $i_{nk}$, $n\in[1..k]$, takvi da svaka $k$-mjesna parcijalno rekurzivna funkcija ima simboličku definiciju u kojoj se $\f Z$ pojavljuje točno $z_k$ puta, $\f I_n^k$ točno $i_{nk}$ puta, te primitivna rekurzija točno $r_k$ puta. Jedino što se mijenja je broj pojava funkcije $\f{Sc}$ i operatora kompozicije, zbog $\f C_e^k=\f{Sc}\circ\f{Sc}\circ\dotsb\circ\f{Sc}\circ\f Z\circ\f I_1^k$.

\subsection{Indeksi izračunljivih funkcija}

Kad fiksiramo $\f U$ i sve $\f T_k$, te pomoću njih definiramo funkcije $\f{comp}_k$ (kao što smo i učinili), Kleenejev teorem o normalnoj formi može se izreći jednostavnije.

\begin{korolar}\label{kor:pimi}
Za svaku $\f f\in\mathcal Comp$ postoje $k\in\N_+$ i $e\in\N$ takvi da je funkcija $\vec x\mapsto\f{comp}_k(\vec x,e)$ jednaka $\f f$.
\end{korolar}

\begin{proof}
Naravno, za $k$ stavimo mjesnost funkcije $\f f$, a za $e$ kod RAM-programa koji računa $\f f$ (koji postoji jer je $\f f$ RAM-izračunljiva). Tada po Kleenejevom teoremu o normalnoj formi, funkcije o kojima tvrdnja govori imaju istu domenu i podudaraju se na toj domeni, dakle to su jednake funkcije.
\end{proof}

\begin{definicija}\label{def:indeks}
Za fiksne $k\in\N_+$ i $e\in\N$, $k$-mjesnu brojevnu funkciju $\vec x\mapsto\f{comp}_k(\vec x,e)$ označavamo s $\kf e^k$, ili jednostavno s $\kf e$ ako joj ne trebamo istaknuti mjesnost.

Broj $e$ zovemo \emph{indeksom} funkcije $\kf e^k$ (broj $k$ zovemo \emph{mjesnošću} funkcije $\kf e^k$). 

Za funkciju $f^k$ kažemo da \emph{ima indeks} ako postoji $e\in\N$ takav da je $\kf e^k=f^k$.
\end{definicija}

\begin{korolar}\label{kor:iip}
Za sve $k\in\N_+$, za sve $e\in\N$, funkcija $\kf e^k$ je parcijalno rekurzivna.
\end{korolar}
\begin{proof}
Samo treba zapisati upravo navedenu definiciju simbolički pomoću kompozicije: $\kf e^k:=\f{comp}_k\circ(\f I_1^k,\f I_2^k,\dotsc\f I_k^k,\f C_e^k)$. Sada tvrdnja slijedi iz propozicije~\ref{prop:compspec}.
\end{proof}

\begin{korolar}\label{kor:pii}
Svaka parcijalno rekurzivna funkcija ima indeks.
\end{korolar}
\begin{proof}
Neka je $k\in\N_+$, i $\f f^k$ parcijalno rekurzivna funkcija. Prema teoremu~\ref{tm:pir}, $\f f\in\mathcal Comp$. Sada prema korolaru~\ref{kor:pimi} postoje $k'$ i $e$ takvi da je $\f f^k=\kf e^{k'}$. Jednake funkcije moraju imati iste mjesnosti, pa je zapravo $k'=k$, odnosno $\f f^k=\kf e^k$.
\end{proof}

\begin{napomena}\label{nap:>1ind}
Često se govori: "\ldots,~dakle funkcija $f$ ima indeks, označimo ga s $e_0$". Strogo govoreći, to je pogrešno, jer $f$, ako već ima indeks, sigurno \textbf{nema jedinstveni indeks}: programerska intuicija nam kaže da za svaki RAM-program postoji beskonačno mnogo RAM-programa koji su mu ekvivalentni (možemo dodavati irelevantne ili nedostupne instrukcije). Ipak, kako kasnije najčešće ne koristimo nikakva svojstva tog broja $e_0$ osim da je indeks od $f$, zapravo taj izričaj možemo shvatiti kao pokratu za "\ldots,~dakle funkcija $f$ ima indeks; odaberimo jedan njen indeks, fiksirajmo ga i nazovimo ga $e_0$".

Alternativno, možemo smatrati da smo uzeli \emph{najmanji} indeks za $f$, koji sigurno postoji ako $f$ ima indeks, i jednoznačno je određen (skup prirodnih brojeva je dobro uređen), ali to ima jedan bitan nedostatak: često iz specifikacije neke funkcije $f$ možemo \emph{izračunati} neki indeks za $f$, ali ne možemo izračunati najmanji indeks za nju. Ugrubo, možemo provjeravati brojeve $e$ redom, ali uvjet zaustavljanja $\kf e=f$ je jednakost funkcija, koja nije izračunljiva. Čak i da su funkcije totalne (što ne moraju biti), morali bismo provjeriti sve $\vec x$ iz $\N^k$, a ima ih beskonačno mnogo.
\end{napomena}

Drugim, riječima, tromjesna relacija $index$ (zadana s $\kf e^k=\f f$) između $\N$, $\N_+$ i $\mathcal Comp$ nema funkcijsko svojstvo po prvoj varijabli --- ali ima po trećoj. %Za svaki broj $e$ možemo karakterizirati funkciju $\kf e^k$.

\begin{propozicija}\label{prop:computeind}
Za svaki $k\in\N_+$, za svaki $e\in\N$, vrijedi:
\begin{enumerate}
    \item\label{it:progind} Ako je $e\in\f{Prog}$, tada je $\kf e^k$ jedinstvena funkcija koju računa RAM-algoritam $P^k$, gdje je $P$ jedinstveni RAM-program takav da je $\kprog P=e$.
    \item\label{it:nprogind} Ako $e\not\in\f{Prog}$, tada je $\kf e^k=\varnothing^k$.
\end{enumerate}
\end{propozicija}
\begin{proof}
Ako je $e\in\f{Prog}$, tada postoji $P\in\mathcal Prog$ takav da je $\kprog P=e$, i jedinstven je jer je $\kprog{\cdots}$ injekcija. Po korolaru~\ref{kor:ram1fun}, postoji jedinstvena funkcija $\f f^k$ koju $P^k$ računa. Po definiciji~\ref{def:compute}, za tu funkciju vrijedi: za sve $\vec x\in\dom{\f f}$, $P$-izračunavanje s $\vec x$ stane, pa je po korolaru~\ref{kor:compspec}\eqref{tv:comp}, $\f f(\vec x)=\f{comp}_k(\vec x,e)=\kf e^k(\vec x)$. Za sve pak $\vec x\not\in\dom{\f f}$, $P$-izračunavanje s $\vec x$ ne stane, pa po korolaru~\ref{kor:compspec}\eqref{tv:ncomp} izraz $\f{comp}_k(\vec x,e)\simeq\kf e^k(\vec x)$ nema smisla, baš kao ni $\f f(\vec x)$. Dakle uvijek je $\f f(\vec x)\simeq\kf e(\vec x)$, odnosno $\f f=\kf e$.

S druge strane, ako $e\not\in\f{Prog}$, tada opet po korolaru~\ref{kor:compspec}\eqref{tv:ncomp} izraz $\f{comp}(\vec x,e)\simeq\kf e(\vec x)$ nema smisla, ali ovaj put neovisno o $\vec x\in\N^k$. Drugim riječima $\dom{\kf e^k}=\emptyset^k$, odnosno jedino je moguće $\kf e^k=\varnothing^k$. 
\end{proof}

\begin{korolar}
Svaki RAM-algoritam $P^k$ računa funkciju $\kf{\kprog P}^k$.
\end{korolar}
\begin{proof}
Ovo je samo jezgrovit zapis propozicije~\ref{prop:computeind}\eqref{it:progind}.
\end{proof}

Relacija $index$ na taj način može poslužiti kao neka vrsta kodiranja: kad želimo dati algoritmu izračunljivu funkciju kao ulazni podatak, ili vratiti iz algoritma izračunljivu funkciju kao izlazni podatak, možemo prenijeti njen indeks $e$ ($k$ se obično vidi iz konteksta). To nije pravo kodiranje, jer nije jednoznačna funkcija ako kažemo "bilo koji indeks", a nije izračunljiva funkcija ako kažemo "najmanji indeks" --- pogledajte napomenu~\ref{nap:>1ind}. Ipak, ako to slanje indeksa u algoritam i vraćanje indeksa iz algoritama shvatimo kao parcijalnu specifikaciju, to može funkcionirati --- pogledajmo primjer.

\begin{primjer}\label{pr:parcspecind}
Recimo da imamo funkciju $F\colon\mathcal Comp_2\rightharpoonup\mathcal Comp_3$. Drugim ri\-je\-či\-ma, $F$ preslikava dvomjesne RAM-izračunljive funkcije u tromjesne RAM-izračunljive funkcije. Kao i prije, htjeli bismo izračunljivost funkcije $F$ opisati pomoću izračunljivosti prateće funkcije $\N F$, koja prima indeks funkcije $\f f$, i vraća indeks funkcije $F(\f f)$. Zbog napomene~\ref{nap:>1ind} funkciju $\N F$ ne možemo time potpuno opisati, ali možemo reći što od nje tražimo: to je da \emph{za svaki} $e\in\N$ bude
\begin{equation}\label{eq:parcspecind}
    \kf{\N F(e)}^3=F(\kf e^2)\text.
\end{equation}
Preciznije, ako $F$ preslikava $\f f^2$ u $\f g^3$, tada $\N F$ mora preslikavati \emph{svaki} indeks za $\f f^2$ u \emph{neki} (ne nužno isti) indeks za $\f g^3$. A ako $\f f\not\in\dom F$, tada \emph{nijedan} indeks za $\f f$ ne smije biti u $\dom{\N F}$. Takvih funkcija općenito ima neprebrojivo mnogo --- jer indeksa za svaku $\f g\in\im F$ ima beskonačno mnogo --- pa ih ima i neizračunljivih. Ali ako \emph{postoji} bar jedna takva funkcija koja je izračunljiva u nekom smislu i zadovoljava~\eqref{eq:parcspecind}, kažemo da je $F$ izračunljiva u istom tom smislu.

Takve funkcije možemo i komponirati: ako imamo $G\colon\mathcal Comp_3\rightharpoonup\mathcal Comp_1$ sa sličnom specifikacijom, koja preslikava $\f g^3$ u $\f h^1$, tada iako ne znamo koji će nam indeks od $\f g$ funkcija $\N F$ dati, funkcija $\N G$ mora ispravno raditi za \emph{sve} indekse od $\f g$, pa će na kraju $\N G\circ\N F$ dati neki indeks za $\f h$, ako joj damo (bilo koji) indeks za $\f f$.
\end{primjer}

Naravno, to možemo kombinirati s dosad napravljenim kodiranjima: recimo, funkcija $apply_k\colon\N^k\times\mathcal Comp_k\rightharpoonup\N$, koja prima $\vec x^k$ i $\f f^k$, i vraća $\f f(\vec x)$ ako je $\vec x\in\dom{\f f}$, ima svoju prateću funkciju koja prima $\langle\vec x\rangle$ i bilo koji indeks za $\f f$, i vraća $\f f(\vec x)$ ako postoji.
\begin{equation}
    \N apply_k(x,e):\simeq\f{univ}(x,e)\text{, za }\f{lh}(x)=k
\end{equation}

\begin{napomena}\label{nap:restrprek}
Nažalost, iz toga još uvijek ne slijedi da je takva funkcija izračunljiva, iako su $\f{univ}$, $\f{lh}$ i $=$ izračunljive --- jer trebamo neki rezultat koji kaže da je restrikcija parcijalno rekurzivne funkcije na (primitivno) rekurzivan skup ponovo parcijalno rekurzivna. To sigurno možemo "na prste" --- recimo, probajte pokazati da je 
\begin{equation}
    \f f|_{{\f S}^k}(\vec x^k)\simeq\f f(\vec x)+\mu y\bigl(\f{Sc}(y)=\chi_{\f S}(\vec x)\bigr)
\end{equation}
--- ali zapravo će to trivijalno slijediti iz rezultata u idućoj točki. Ipak, ideja prikaza restrikcije kao zbroja s funkcijom koja je na nekom lijepom skupu nula, a izvan njega nedefinirana --- bit će nam od koristi kasnije, u poglavlju~\ref{ch:re}.
\end{napomena}

\subsection{Parcijalno rekurzivna verzija teorema o grananju}

Na početku točke~\ref{sec:grtot} nešto smo rekli o teškoćama koje marljiva evaluacija nosi ako želimo u našem funkcijskom jeziku implementirati grananje s funkcijama koje nisu nužno totalne. Zapravo, jedino što možemo koristiti je kompozicija, jer primitivna rekurzija je definirana samo na totalnim funkcijama, a minimizacija na karakterističnim funkcijama koje su također totalne --- ali marljiva evaluacija znači da se u kompoziciji sve "unutarnje" funkcije evaluiraju uvijek, i parcijalnost bilo koje od njih znači parcijalnost čitave kompozicije.

Dakle, način da se izvučemo je da našoj funkciji $if$ ne damo već izračunate \emph{vrijednosti} $\f g(\vec x)$ i $\f h(\vec x)$, već neevaluirane \emph{funkcije} $\f g$ i $\f h$. Tada s obzirom na uvjet odaberemo jednu od njih, i onda je tek izračunamo na $\vec x$: umjesto $\f f(\vec x,y)\simeq if\bigl(y,\f g(\vec x),\f h(\vec x)\bigr)$ imamo $\f f(\vec x,y)\simeq if(y,\f g,\f h)(\vec x)$. Koristeći indekse, to možemo i doslovno napraviti.

\begin{teorem}[Teorem o grananju, parcijalno rekurzivna verzija]\label{tm:gprek}
Neka su $k,l\in\N_+$, neka su $\f G_0^k$, $\f G_1^k$, $\f G_2^k$,~\ldots, $\f G_l^k$ parcijalno rekurzivne funkcije, te $\f R_1^k$, $\f R_2^k$,~\ldots, $\f R_l^k$ u parovima disjunktne rekurzivne relacije, sve iste mjesnosti. Tada je i funkcija $\f F:=\{\f R_1\colon\f G_1,\f R_2\colon\f G_2,\dotsc,\f R_l\colon\f G_l,\f G_0\}$ također parcijalno rekurzivna.
\end{teorem}
\begin{proof}
Za svaki $i\in[0..l]$, funkcija $\f G_i^k$ je parcijalno rekurzivna, pa prema korolaru~\ref{kor:pii} ima indeks, označimo ga s $e_i$ (pogledajte napomenu~\ref{nap:>1ind} za značenje ove fraze). Prema teoremu~\ref{tm:grek} (rekurzivna verzija), funkcija
\begin{equation}
    \f H^k:=\{\f R_1\colon\f C_{e_1}^k,\f R_2\colon\f C_{e_2}^k,\dotsc,\f R_l\colon\f C_{e_l}^k,\f C_{e_0}^k\}
\end{equation}
je rekurzivna (konstante su primitivno rekurzivne pa su rekurzivne, a uvjeti su u parovima disjunktni i rekurzivni po pretpostavci). Tvrdimo da je
\begin{equation}\label{eq:gprekdef}
    \f F(\vec x)\simeq\f{comp}_k\bigl(\vec x,\f H(\vec x)\bigr).
\end{equation}
Doista, neka je $\vec x\in\N^k$ proizvoljan. Ako vrijedi $\f R_i(\vec x)$ za neki (jedinstveni, zbog disjunktnosti) $i\in[1..l]$, tada je prema teoremu~\ref{tm:grek}, $\f H(\vec x)=\f C_{e_i}(\vec x)=e_i$, pa je
\begin{equation}
    \f{comp}_k\bigl(\vec x,\f H(\vec x)\bigr)\simeq\f{comp}_k(\vec x,e_i)\simeq\kf{e_i}(\vec x)\simeq\f G_i(\vec x)\simeq\f F(\vec x)\text.
\end{equation}
Ako pak ne vrijedi $\f R_i(\vec x)$ ni za koji $i$, tada je opet prema teoremu~\ref{tm:grek}, $\f H(\vec x)=C_{e_0}(\vec x)=e_0$, pa je kao i prije $\f{comp}_k\bigl(\vec x,\f H(\vec x)\bigr)\simeq\f G_0(\vec x)\simeq\f F(\vec x)$.

Sada parcijalna rekurzivnost slijedi iz~\eqref{eq:gprekdef}, jer je $\f F$ dobivena kompozicijom iz parcijalno rekurzivnih funkcija $\f{comp}$ i $\f H$.
\end{proof}

U teoremu~\ref{tm:grek} nismo mogli ispustiti $\f G_0$, jer njena podrazumijevana vrijednost $\varnothing^k$ nije rekurzivna (nije uopće totalna). Ali $\varnothing^k$ jest parcijalno rekurzivna, tako da je ovdje možemo ispustiti --- samo nam treba neki indeks za nju, da bi $\f H$ bila totalna funkcija.

Za to možemo iskoristiti propoziciju~\ref{prop:computeind}\eqref{it:nprogind} --- svi brojevi iz $\f{Prog}\kompl$ indeksi su prazne funkcije. Posebno lijep takav broj je $0$, koji nije u $\f{Prog}$ jer uopće nije u $\f{Seq}$: naime, $\overline{\f{part}}\bigl(0,\f{lh}(0)\bigr)=\overline{\f{part}}(0,0)=\langle\rangle=1\not=0$.

\begin{korolar}\label{kor:gprek}
Neka su $k,l\in\N_+$, neka su $\f G_1^k$, $\f G_2^k$,~\ldots, $\f G_l^k$ parcijalno rekurzivne funkcije, te $\f R_1^k$, $\f R_2^k$,~\ldots, $\f R_l^k$ u parovima disjunktne rekurzivne relacije, sve iste mjesnosti. Tada je i funkcija $\f F:=\{\f R_1\colon\f G_1,\f R_2\colon\f G_2,\dotsc,\f R_l\colon\f G_l\}$ također parcijalno rekurzivna.
\end{korolar}
\begin{proof}
Direktno iz teorema~\ref{tm:gprek}, uvrštavajući za $\f G_0$ parcijalno rekurzivnu funkciju $\varnothing^k$, odnosno njen indeks $e_0=0$ u dokaz.
\end{proof}

\begin{korolar}\label{kor:restrprek}
Neka je $k\in\N_+$, te neka je $\f G^k$ parcijalno rekurzivna funkcija, i $\f R^k$ rekurzivna relacija iste mjesnosti. Tada je restrikcija $\f G|_{\f R}$ također parcijalno rekurzivna.
\end{korolar}
\begin{proof}
Direktno iz korolara~\ref{kor:gprek}, uvrštavajući $l:=1$, $\f G_1:=\f G$, te $\f R_1:=R$. Lako se vidi da je $\{\f R\colon\f G\}$ upravo jednako $\f G|_{\f R}$: domena joj je $\dom{\f G}\cap\f R$, a vrijednosti su joj upravo jednake vrijednostima funkcije $\f G$ na tom skupu.
\end{proof}

%\begin{napomena}\label{nap:restrprek}
%Jedan duhovit dokaz korolara~\ref{kor:restrprek} koristi sljedeću jednakost:
%\begin{equation}
    %(\f G|_{\f R})(\vec x)\simeq\f G(\vec x)+\f{comp}\bigl(\vec x,\chi_{\f R}(\vec x)\bigr)\text,
%\end{equation}
%koja proizlazi iz $\chi_{\f R}(\vec x)\in\{0,1\}$, s tim da je $0$ indeks prazne funkcije, a $1=\kprog{[\,]}$ indeks nulfunkcije, koja zbrojena s $\f G$ opet daje $\f G$. Ta ideja --- 
%\end{napomena}