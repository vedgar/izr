\section{Izračunljivost jezika}\label{sec:Todl}

Neka je $\Sigma$ proizvoljna abeceda, $b'$ njena baza, $\N\Sigma$ njeno kodiranje, te $L\subseteq\Sigma^*$ jezik nad njom. Što bi značilo da je $L$ izračunljiv?

Izračunljivost od $L$ u brojevnom modelu gledali bismo preko kodova: lako je definirati jednomjesnu brojevnu relaciju
\begin{equation}\label{eq:kodLdef}
    \kr L:=\{\kr w\mid w\in L\}=\N\Sigma^*[L]\text,
\end{equation}
i prirodno je reći da $L$ ima neko svojstvo izračunljivosti (npr.\ da je primitivno rekurzivan) ako relacija $\kr L$ (dakle njena karakteristična funkcija $\chi_{\kr L}$) ima to svojstvo.

Recimo, prazan jezik $\emptyset$ je primitivno rekurzivan jer je $\chi_{\kr\emptyset}=\chi_\emptyset=\f Z$ primitivno rekurzivna (štoviše, inicijalna). Također, univerzalni jezik $\Sigma^*$ je primitivno rekurzivan, jer je $\kr{\Sigma^*}=\N\Sigma^*[\Sigma^*]=\im{\N\Sigma^*}$, što je primitivno rekurzivno po propoziciji~\ref{pp:injkr}.

U jezičnom modelu, moramo vidjeti što znači da neki Turingov stroj računa $\chi_L$. Očito, to treba biti Turingov stroj nad $\Sigma$, i ulaz $w\in\Sigma^*$ mu se daje na isti način kao i običnom Turingovom stroju: kroz početnu konfiguraciju $(q_0,0,w\bl\ldots)$. No izlaz ($\mathit{true}$ ili $\mathit{false}$, ovisno o tome je li $w\in L$) je jedan od konačno mnogo njih, pa ga je prirodnije predstaviti stanjem. Zato takvi Turingovi strojevi imaju \emph{dva} završna stanja, $q_{\mathit{true}}$ i $q_{\mathit{false}}$, te njima signaliziraju je li riječ u jeziku ili nije (kažemo da \emph{prihvaćaju} odnosno \emph{odbijaju} riječ).

Na prvi pogled, to je slično stanjima $q_z$ i $q_x$ koje smo imali u našim Turingovim strojevima, samo što je propisano da konfiguracija sa stanjem $q_{\mathit{false}}$ (kao i ona s $q_{\mathit{true}}$) prelazi u samu sebe, a ne po funkciji $\delta$. Ali postoji jedna bitna razlika: kako su karakteristične funkcije nužno totalne, od takvih strojeva zahtijevamo da \textbf{uvijek stanu} (za svaki ulaz dostignu konfiguraciju s jednim od ta dva završna stanja). Za takve strojeve kažemo da su \emph{odlučivači}. Svaki odlučivač $\mathcal T$ time dijeli $\Sigma^*$ na dva dijela,
\begin{align}
    L(\mathcal T):=\{w\in\Sigma^*&\mid\text{$\mathcal T$-izračunavanje s $w$ stane u stanju $q_{\mathit{true}}$}\}
    \text{ i}\\ \bigl(L(\mathcal T)\big)\kompl=\{w\in\Sigma^*&\mid\text{$\mathcal T$-izračunavanje s $w$ stane u stanju $q_{\mathit{false}}$}\}\text,
\end{align} i kažemo da \emph{prepoznaje} $L(\mathcal T)$.

Nije preteško vidjeti da su te dvije karakterizacije povezane --- pogotovo jer već imamo napravljen najveći dio posla.

\begin{teorem}\label{tm:oikr}
Neka je $L$ jezik. Ako postoji Turingov odlučivač koji prepoznaje $L$, tada je jednomjesna relacija $\kr L$ rekurzivna.
\end{teorem}
\begin{proof}
Pretpostavimo da je $\mathcal T$ odlučivač za $L$, i provedimo s tim strojem postupak iz točke~\ref{sec:tikp}. Ovdje navodimo samo detalje koje je potrebno promijeniti.

Prvo, kako imamo dva završna stanja, moramo fiksirati njihove kodove: recimo, $\N Q(q_\mathit{true}):=1$, a $\N Q(q_\mathit{false}):=2$. (Možemo dokazati analogon leme~\ref{lm:bsomp-q0neqz}, da bez smanjenja općenitosti možemo pretpostaviti $q_0\not=q_\mathit{true}$ i $q_0\not=q_\mathit{false}$ --- a po definiciji odlučivača mora biti $q_\mathit{true}\not=q_\mathit{false}$.)

Drugo, kako nam $\delta$ sad nije definirana na oba završna stanja, treba promijeniti uvjet u lemi analognoj lemi~\ref{lm:newssdprn}, u $q\not\in\{q_\mathit{true},q_\mathit{false}\}$. Tu se ništa bitno ne mijenja, osim što će $\f{direction}$-tablica~\eqref{eq:directiontable} imati dva retka jedinica, a ne samo jedan.

I treće, umjesto parcijalno rekurzivne funkcije $\f{stop}$, imat ćemo funkciju zadanu sa
\begin{equation}
    \f{stop}'(x):=\mu n\bigl(x\in\im{\N\Sigma^*}\to\f{State}(x,n)\in\{1,2\}\bigr)\text,
\end{equation}
za koju lako vidimo da je rekurzivna. Naime, parcijalno je rekurzivna po propoziciji~\ref{prop:vezn}, jer su skupovi $\im{\N\Sigma^*}$ i $\{1,2\}$ primitivno rekurzivni (prvi po propoziciji~\ref{pp:injkr}, a drugi po korolaru~\ref{kor:konprn}), pa je dobivena minimizacijom primitivno rekurzivne relacije. Totalna je (pa smo mogli koristiti znak `:=' u definiciji) po definiciji odlučivača, jer ako je $x=\kr w$, tada mora postojati $n$ takav da je $\f{State}(x,n)\in\{1,2\}$, jer $\mathcal T$-izračunavanje s $w$ mora stati.

Naravno, tada je $\f{stop}'(x)=\f{stop}'(\kr w)$ upravo broj koraka nakon kojeg to iz\-ra\-ču\-na\-va\-nje stane.
Ako pak $x\not\in\im{\N\Sigma^*}$, tada je uvjet pod minimizacijom ispunjen za svaki $n$, pa je $\f{stop}'(x)=\mu\N=0$.

Tvrdimo da je
\begin{equation}
    x\in\kr L\Longleftrightarrow\f{State}\bigl(x,\f{stop}'(x)\bigr)=1\text,
\end{equation}
iz čega odmah slijedi rekurzivnost od $\kr L$ po lemi~\ref{lm:comprek} i korolaru~\ref{kor:prnrek}. 

Doista, ako je $x\in\kr L=\N\Sigma^*[L]$, to znači da postoji $w\in L$ takva da je $x=\kr w$. Tada $\mathcal T$-izračunavanje s $w$ mora stati u stanju $q_\mathit{true}$ --- označimo s $n_0$ broj koraka nakon kojeg se to dogodi. Tada je $\f{State}(x,n_0)=1\in\{1,2\}$, dok za sve $n<n_0$ konfiguracija nakon $n$ koraka nije završna (sasvim analogno propoziciji~\ref{prop:ram1zav} --- postoji najviše jedna završna konfiguracija u izračunavanju), pa vrijedi $\f{State}(x,n)\not\in\{1,2\}$. Dakle $n_0=\f{stop}'(x)$, pa je $\f{State}\bigl(x,\f{stop}'(x)\bigr)=\f{State}(x,n_0)=1$.

U drugom smjeru, ako je $\f{State}\bigl(x,\f{stop}'(x)\bigr)=1$, mora biti $x\in\im{\N\Sigma^*}$ --- inače bi, kao što smo obrazložili, $\f{stop}'(x)$ bilo $0$, ali $\f{State}(x,0)=0\not=1$. Dakle postoji $w\in\Sigma^*$ takva da je $x=\kr w$. Kao i u prošlom odlomku, označimo s $n_0$ broj koraka nakon kojeg $\mathcal T$-izračunavanje s $w$ stane --- taj broj postoji jer je $\mathcal T$ odlučivač, i štoviše $n_0=\f{stop}'(x)$. Dakle pretpostavka glasi $\f{State}(x,n_0)=1$, odnosno stanje konfiguracije od $\mathcal T$ nakon $n_0$ koraka računanja s $w$ je $\N Q^{-1}(1)=q_\mathit{true}$. No to znači da $\mathcal T$ prihvaća $w$, odnosno $w\in L(\mathcal T)=L$, pa je $x=\kr w\in\kr L$.
\end{proof}

\begin{teorem}\label{tm:krio}
Neka je $L$ jezik (nad nekom abecedom $\Sigma$). Ako je relacija $\kr L$ rekurzivna, tada postoji Turingov odlučivač za $L$.
\end{teorem}
\begin{proof}
Pretpostavka zapravo znači da je $\chi_{\kr L}$ rekurzivna funkcija. Dakle $\chi_{\kr L}$ je parcijalno rekurzivna, pa po teoremu~\ref{tm:pir} postoji RAM-program koji je računa; i totalna je, pa taj RAM-program stane za svaki ulaz. Na taj RAM-program htjeli bismo primijeniti postupak iz točke~\ref{sec:RAM>Turing}. Da bismo to mogli, moramo $\chi_{\kr L}$ prikazati kao $\N\varphi$ za neku jezičnu funkciju $\varphi$. Možemo li to?

Očito ne, jer na brojevima koji nisu kodovi riječi, $\N\varphi$ mora poprimiti vrijednost $b'\ge2$, a $\chi_{\kr L}$, kao i svaka karakteristična funkcija, može poprimati samo vrijednosti $0$ i $1$. Ovdje se možemo izvući primjenjujući napomenu~\ref{nap:approxNfi}, da je dovoljno tražiti $\chi_{\kr L}\approx\N\varphi$ --- ili umjesto na $\chi_{\kr L}$ možemo upravo napisani odlomak primijeniti na funkciju $\chi_{\kr L}':=\{\im{\N\Sigma^*}\colon\chi_{\kr L},C_{b'}^1\}$, koja jest rekurzivna po teoremu~\ref{tm:grek}.

U svakom slučaju, tražimo jezičnu funkciju $\varphi$ takvu da $\N\varphi(\kr w)$ bude $1$ ako je $w\in L$, a $0$ inače. Čitajući definiciju~\ref{def:kodfi}, vidimo da $1$ i $0$ moramo prikazati kao kodove dviju riječi, $v_\mathit{true}$ i $v_\mathit{false}$, nad $\Sigma$, jednu od kojih će $\varphi$ vratiti ovisno o tome je li njen ulaz iz $L$ ili nije. Za to možemo primijeniti postupak na kraju dokaza propozicije~\ref{pp:injkr}: sve što nam treba je $\f{cd}(0,0,b')=\f{cd}(1,0,b')=0$ (pa $0$ i $1$ jesu u slici kodiranja), te je uvijek $0=\kr\varepsilon$, a zbog injektivnosti kodiranja je $1$ kod neke riječi različite od $\varepsilon$ (konkretno, to je riječ koja se sastoji od jednog znaka $\alpha_1\in\Sigma$).

Kako izvršavanje RAM-programa o kojem govorimo stane za svaki $\kr w$, po korolaru~\ref{kor:faza3} i napomeni~\ref{nap:snstane} slijedi da će Turingov stroj koji konstruiramo stati za svaki $w$ --- samo trebamo malo promijeniti način na koji stane, da ne komunicira izlaz putem trake nego putem završnog stanja.

U opisu pete (zadnje) faze rada tog Turingovog stroja, razlikovali smo slučajeve kad je povratna riječ prazna i kad nije. Kao što smo upravo vidjeli, s obzirom na to da je $v_\mathit{false}=\varepsilon\not=v_\mathit{true}$, to nam može poslužiti za odabir završnog stanja u kojem ćemo završiti. Samo trebamo zamijeniti prijelaze~\eqref{d:88} i~\eqref{d:89} sa
\begin{gather}
\label{eq:d88j}
    \delta(q_9,\alpha_1):=(q_\mathit{true},\alpha_1,1)\text,\\
\label{eq:d89j}
    \delta(q_9,\t\$):=(q_\mathit{false},\bl,-1)\text.
\end{gather}
Po teoremu~\ref{tm:krit}, $\mathcal T$ će "računati" (pod navodnicima jer odlučivači zapravo ne služe za računanje funkcija) funkciju $\varphi$, pa će za $w\in L$ rezultat biti $v_\mathit{true}=\alpha_1$, te će stroj u trećem koraku pete faze, krećući se lijevo, doći do znaka $\alpha_1$ i ući u stanje $q_\mathit{true}$. Za $w\not\in L$, rezultat će biti $v_\mathit{false}=\varepsilon$, te će stroj doći do graničnika, obrisati ga i ući u stanje $q_\mathit{false}$. Budući da za svaku riječ $w\in\Sigma^*$ vrijedi $w\in L$ ili $w\not\in L$, zaključujemo da smo doista konstruirali odlučivač.
\end{proof}

Dijagramatski, gornji desni kut~\eqref{dia:T5} zamijenimo s\quad \begin{tikzpicture}[baseline=(q9.base)]\label{dia:T5j}
\node[state,above right of=q8] (q9) {$q_9$};
\node[state,accepting,above right=0.1 and 2 of q9] (qt) {$q_\mathit{true}$};
\node[state,accepting,below right=0.1 and 2 of q9] (qf) {$q_\mathit{false}$};
\draw
%(q9) edge[dashed,loop left] node[left] {\t/:\bl} (q9)
(q9) edge node[above] {$\alpha_1$} (qt)
(q9) edge[dashed] node[below] {\t\$:\bl} (qf)
;
\end{tikzpicture}.

\section{Turing-izračunljivost brojevnih funkcija}

Dosadašnji rezultati pokazuju da je za jezične funkcije svejedno računamo li ih na Turingovom stroju ili na nekom brojevnom modelu izračunavanja (npr.\ RAM-stroju). Možemo li to isto dokazati i za \emph{brojevne} funkcije? Svakako, samo prvo trebamo definirati što uopće znači računati brojevnu funkciju na Turingovom stroju. Drugim riječima, moramo predstaviti (reprezentirati) brojevne funkcije u jezičnom modelu.

Zapravo, sve potrebne ideje smo već vidjeli. Za ulazne podatke jednomjesnih funkcija, kao i za izlazne podatke, koristit ćemo unarni zapis: što je prirodnije nego broj $4$ zapisati na traku Turingovog stroja kao \textbullet\textbullet\textbullet\textbullet, a $0$ kao praznu riječ $\varepsilon$\@? Tako smo već prikazivali (u tragovima) sadržaj registara kod simulacije RAM-izračunavanja. Da smo opterećeni složenošću, morali bismo koristiti binarni ili neki drugi "pravi" pozicijski zapis, ali srećom nismo.

Kod ulaza višemjesnih funkcija, koristit ćemo kontrakciju: u abecedu ćemo dodati separator \t/, te ćemo više ulaznih podataka razdvojiti separatorima: npr.\ $(1,0,5,0)$ prenijet ćemo kao \t{\textbullet//\textbullet\textbullet\textbullet\textbullet\textbullet/}.

\begin{definicija}\label{def:upsilon}
\emph{Unarna abeceda} je abeceda $\Sigma_\upsilon:=\{\text\textbullet,\t/\}$. Za svaki neprazni niz prirodnih brojeva $\vec x=(x_1,x_2,\dotsc,x_k)$, definiramo \emph{unarnu reprezentaciju} kao
\begin{equation}\label{eq:upsilondef}
    \upsilon(\vec x):=\text\textbullet^{x_1}\t/\text\textbullet^{x_2}\t/\dotsm\t/\text\textbullet^{x_k}\in\Sigma_\upsilon^*
\end{equation}
($\upsilon$ je grčko slovo "upsilon"). Za svaki $k\in\N_+$, označavamo $\upsilon^k:=\upsilon|_{\N^k}$.
\end{definicija}

Ponekad se reprezentacija također zove "kodiranje", ali nama su kodiranja funkcije čije povratne vrijednosti su prirodni brojevi. Funkcija $\upsilon$ ide u suprotnom smjeru (s dobrim razlogom --- sad trebamo promatrati jezične funkcije kao osnovne, jer za njih imamo teoreme~\ref{tm:tikp} i~\ref{tm:krit}), ali zapravo je možemo promatrati u bilo kojem smjeru.

\begin{propozicija}\label{prop:upsilonbij}
Funkcija $\upsilon$ je bijekcija između $\N^+$ i $\Sigma_\upsilon^*$.
\end{propozicija}
\begin{proof}
Najlakše je konstruirati inverznu funkciju, i pokazati da je to doista inverz. Dakle, uzmimo proizvoljnu riječ $u\in\Sigma_\upsilon^*$, i pitamo se kojeg niza je ona kod. Kao i uvijek kod konačnih nizova, trebamo odrediti njegovu duljinu $k$, i zatim pojedine članove $x_1$, $x_2$,~\ldots, $x_k$. Duljina je očito sljedbenik broja pojavljivanja separatora \t/ u riječi $u$ (jer u~\eqref{eq:upsilondef} ima $k-1$ separatora), i to je doista pozitivan broj, dakle niz je neprazan. Prvi član mu možemo odrediti brojeći kružiće do prvog separatora (ili do kraja riječi ako je $k=1$), drugi brojeći ih između prvog i drugog separatora, itd. Posljednji član $x_k$ je broj kružića od zadnjeg separatora do kraja riječi $u$.

Tvrdimo da je tako konstruirano preslikavanje desni inverz od $\upsilon$: odnosno, ako tako dobijemo niz $\vec x$, tada je $\upsilon(\vec x)=u$. Doista, te dvije riječi imaju isti broj separatora ($k-1$) i isti broj kružića ($\sum\vec x$), te se podudaraju na pozicijama svih separatora (pomaknute parcijalne sume od $\vec x$) --- što je dovoljno da zaključimo da su jednake.

To je također i lijevi inverz: odnosno, ako primijenimo taj postupak na riječ $\upsilon(\vec x)$, dobit ćemo upravo $\vec x$: imat će točnu duljinu $k$, i točan svaki član, jer između $i$-tog i $(i+1)$-vog separatora u~\eqref{eq:upsilondef} ima točno $x_i$ kružića.
\end{proof}

Za same funkcije, koristit ćemo istu ideju kao u~\eqref{eq:kodfidef}: reprezentaciju ulaza preslikamo u reprezentaciju izlaza (ako je izlaz definiran), dok ne-reprezentacije preslikamo u "najmanju" ne-reprezentaciju ("najmanju" je pod navodnicima jer su reprezentacije riječi, ali vidjet ćemo da svejedno postoji prirodni izbor). Ovdje treba biti oprezan zbog propozicije~\ref{prop:upsilonbij}, ali brojevne funkcije imaju fiksnu mjesnost. Zato ćemo ne-reprezentacijama za funkciju $f^k$ proglasiti sve one riječi koje nisu reprezentacije $k$-torki (odnosno one koje jesu reprezentacije $l$-torki za $l\not=k$). Na isti način, na izlazu ćemo dati reprezentaciju samog broja ($k=1$), dakle riječ bez separatora $\upsilon(y)\in\upsilon[\N]=\im{\upsilon^1}=\{\text\textbullet\}^*$, ili ako ulaz nije bila reprezentacija odgovarajuće mjesnosti, "najmanju" (zapravo najkraću) riječ iz $(\{\text\textbullet\}^*)\kompl$, što je očito \t/.

\begin{definicija}
Neka je $k\in\N_+$, i $f^k$ brojevna funkcija. Za jezičnu funkciju $\upsilon f$ zadanu s \begin{equation}\label{eq:upsilonf}
    \upsilon f(u):\simeq\begin{cases}
    \upsilon\bigl(f(\vec x)\bigr),&u=\upsilon(\vec x^k)\\
    \t/,&\text{inače}
    \end{cases}\text,
\end{equation}
kažemo da je \emph{unarna reprezentacija} funkcije $f$.
\end{definicija}
\begin{primjer}
$\upsilon\f{mul}^2(\t{\textbullet\textbullet/\textbullet\textbullet\textbullet\textbullet\textbullet})=\t{\textbullet\textbullet\textbullet\textbullet\textbullet\textbullet\textbullet\textbullet\textbullet\textbullet}$, jer je $2\cdot5=10$, dok je na primjer $\upsilon\f{mul}^2(\t{\textbullet})=\upsilon\f{mul}^2(\t{//})=\t/$, jer $(1)$ i $(0,0,0)$ nisu ulazi ispravne mjesnosti za $\f{mul}^2$.
\end{primjer}

Vjerojatno ste nekad u dosadašnjem obrazovanju napisali Turingov stroj koji računa funkciju $\upsilon\f{add}^2$ (možda samo na ispravnim reprezentacijama, odnosno na $\upsilon[\N^2]$), koristeći svojstvo $\upsilon\f{add}^2(u\t/v)=uv$, ako su $u,v\in\{\text\textbullet\}^*$. Ipak, već tada ste zasigurno vidjeli koliko je teško napisati $\upsilon\f{mul}^2$, a pogotovo $\upsilon\f{pow}$ --- ostale divote iz poglavlja~\ref{ch:univ} ($\upsilon\f{prime}$?!) da i ne spominjemo.

Sljedeći veliki cilj je dokazati da Turingovi strojevi doista mogu računati (unarno reprezentirane) \emph{sve} parcijalno rekurzivne funkcije, pa tako i funkciju $\f{univ}$ --- čineći tako jedan od modela univerzalne izračunljivosti. Ipak, do tada ima još nekoliko tehničkih poteškoća, pa ćemo prvo dokazati obrat: brojevne funkcije, čije su unarne reprezentacije Turing-izračunljive, su parcijalno rekurzivne. To nije toliko impresivan rezultat, ali lakši je za dokazati, a mnoge dijelove dokaza moći ćemo upotrijebiti i u dokazu obrata.

\subsection{Prateća funkcija unarne reprezentacije}

Dakle, neka je $k\in\N_+$, i $f^k$ brojevna funkcija takva da je $\upsilon f$ Turing-izračunljiva. Ono što sada sigurno možemo zaključiti je da ako fiksiramo neko kodiranje od $\Sigma_\upsilon$ (čija je baza $b'=3$), funkcija $\N\upsilon f=\N(\upsilon f)$ jest parcijalno rekurzivna. Štoviše, korolar~\ref{kor:ikojiNSigma} kaže da je svejedno koje kodiranje uzmemo, pa određenosti radi, stavimo
\begin{equation}\label{eq:kodSigmaups}
    \N\Sigma_\upsilon(\text\textbullet):=1\text,\qquad
    \N\Sigma_\upsilon(\t/):=2\,\text.
\end{equation}
Sada znamo, po teoremu~\ref{tm:tikp}, da je $\N\upsilon f$ parcijalno rekurzivna. Možemo li nekako iz toga dobiti da je $f$ parcijalno rekurzivna? Drugim riječima, $f^k$ i $(\N\upsilon f)^1$ su obje brojevne funkcije, ali su povezane preko jezične funkcije $\upsilon f$. Možemo li nekako naći brojevnu vezu između njih? Precizno, možemo li naći brojevne funkcije $g^1$ i $h^k$ takve da je $f=g\circ\N\upsilon f\circ h$? Ako bi $g$ i $h$ bile izračunljive i totalne (recimo primitivno rekurzivne), iz toga bi odmah slijedila parcijalna rekurzivnost od $f$, jer je skup parcijalno rekurzivnih funkcija zatvoren na kompoziciju.

U traženju tih funkcija može pomoći dijagram.
\begin{equation}\label{dia:Nupsilon}
\begin{tikzcd}
\N^k
\arrow[harpoon]{r}{f}
\arrow{d}{\upsilon^k}
\arrow[bend right=70,dotted]{dd}{\f u^k}
&\N
\arrow{d}{\upsilon^1}
\arrow[bend left=70,dotted]{dd}{\f u^1}
\\
\Sigma_\upsilon^* 
\arrow[harpoon]{r}{\upsilon f}
\arrow{d}{\N\Sigma_\upsilon^*}
& \Sigma_\upsilon^* 
\arrow{d}{\N\Sigma_\upsilon^*}
\\
\N
\arrow[harpoon]{r}{\N\upsilon f}
& \N
\end{tikzcd}
\qquad
\begin{tikzcd}
(3,1,2)
\arrow[mapsto]{r}{\f{mul}^3}
\arrow[mapsto]{d}{\upsilon^3}
& 6
\arrow[mapsto]{d}{\upsilon^1}
\\
\t{\textbullet\textbullet\textbullet/\textbullet/\textbullet\textbullet}
\arrow[mapsto]{r}{\upsilon\f{mul}^3}
\arrow[mapsto]{d}{\N\Sigma_\upsilon^*}
& \t{\textbullet\textbullet\textbullet\textbullet\textbullet\textbullet} 
\arrow[mapsto]{d}{\N\Sigma_\upsilon^*}
\\
%\begin{array}{c}
%(11121211)_3\\
3370
%\end{array}
\arrow[mapsto]{r}{\N\upsilon\f{mul}^3}
&
%\begin{array}{c}
%(111111)_3\\
364
%\end{array}
\end{tikzcd}
\end{equation}
Lijevo je općenita slika, desno je jedan konkretan primjer (računanje $\f{mul}^3$ na $(3,1,2)$). Brojevi na dnu primjera su razumljiviji zapisani u bazi $3$: $3370=(11121211)_3$, $364=(111111)_3$. Gornji pravokutnik predstavlja definiciju~\eqref{eq:upsilonf}, dok donji predstavlja~\eqref{eq:kodfidef} za specijalni slučaj unarne abecede, i jezične funkcije $\upsilon f$ nad njom.

\begin{lema}\label{lm:pravkomut}
Dijagram prikazan lijevo u~\eqref{dia:Nupsilon} (bez točkanih strelica) komutira.
\end{lema}
\begin{proof}
Drugim riječima, za svaki od ta dva pravokutnika, pa onda i za veliki pravokutnik sastavljen od ta dva, je svejedno kojim putem dođemo od njegovog gornjeg lijevog do donjeg desnog vrha.

Prvo, dakle, trebamo dokazati (gornji pravokutnik) da je $\upsilon f\circ\upsilon^k =\upsilon^1\circ f$. Domena desne funkcije je $\dom f$ jer je $\upsilon^1$ totalna, a domena lijeve funkcije je $\{\vec x\in\N^k\mid\upsilon^k(\vec x)\in\dom{\upsilon f}\}$, što je opet jednako $\dom f$ po definiciji domene $\dom{\upsilon f}$. Neka je sad $\vec x\in\dom f$ proizvoljan. Na njemu je lijeva funkcija $\upsilon f\bigl(\upsilon^k(\vec x)\bigr)$, što je po definiciji~\eqref{eq:upsilonf} jednako $\upsilon^1\bigl(f(\vec x)\bigr)$, što je upravo jednako $(\upsilon^1\circ f)(\vec x)$. Dakle, $\upsilon f\circ\upsilon^k$ i $\upsilon^1\circ f$ imaju istu domenu i na njoj se podudaraju, pa su to jednake funkcije.

Prilično analogno, koristeći~\eqref{eq:kodfidef}, bi se dokazalo $\N\upsilon f\circ\N\Sigma_\upsilon^*=\N\Sigma_\upsilon^*\circ\upsilon f$, odnosno donji pravokutnik također komutira. Sada iz toga slijedi
\begin{equation}\label{eq:pravkomut}
    \N\Sigma_\upsilon^*\circ\upsilon^1\circ f=\N\Sigma_\upsilon^*\circ\upsilon f\circ\upsilon^k=\N\upsilon f\circ\N\Sigma_\upsilon^*\circ\upsilon^k\text,
\end{equation}
odnosno čitav "\!vanjski okvir" komutira.
\end{proof}

Upravo dokazana jednakost je korisna, jer pruža način da se u potpunosti izbjegnu jezične funkcije. Naime, vrhovi vanjskog pravokutnika su sasvim brojevni, i njegove stranice se mogu opisati kao brojevne funkcije. Gornju i donju stranicu već imamo, još je preostalo precizirati lijevu i desnu.

\subsection{Funkcije \texorpdfstring{$\f u^k$}{u}}

\begin{definicija}\label{def:uk}
Za svaki $k\in\N_+$, definiramo $\f u^k:=\N\Sigma_\upsilon^*\circ\upsilon^k$.
\end{definicija}

Te funkcije su brojevne, i na dijagramu su prikazane točkanim strelicama: lijeva stranica vanjskog pravokutnika je $\f u^k$, a desna $\f u^1$. Svaka $\f u^k$ je očito injekcija kao kompozicija dvije injekcije. Štoviše, funkcija $u^{\cdots}:=\N\Sigma_\upsilon^*\circ\upsilon$ (koja nije brojevna funkcija jer nema fiksnu mjesnost) predstavlja jednostavan način da u funkciju prenesemo varijabilan broj argumenata: za funkciju $f^{\cdots}$ koja prima \emph{varargs}, $\N\upsilon f^{\cdots}$ je jednomjesna brojevna funkcija koja dobro modelira sve aspekte izračunljivosti funkcije $f^{\cdots}$, pa bismo mogli reći da je $f^{\cdots}$ izračunljiva (u nekom smislu) ako je $\N\upsilon f^{\cdots}$ izračunljiva (u tom istom smislu). O tome ćemo reći nešto više malo kasnije --- zasad se usredotočimo na izračunljivost funkcija $\f u^k$.

Kako bismo mogli iz $(3,1,2)$ izračunati $(11121211)_3=3370$, ili iz $6$ izračunati $(111111)_3=364$, koristeći samo brojevne funkcije? Ovo drugo se svakako čini lakšim.

\begin{lema}\label{lm:u1prn}
Funkcija $\f u^1$ je primitivno rekurzivna.
\end{lema}
\begin{proof}
Možemo direktno napisati točkovnu definiciju $\f u(x)=\sum_{i<x}3^i$, pa će primitivna rekurzivnost slijediti iz propozicije~\ref{lm:sumrek} --- ali možemo to čak i izračunati po formuli za sumu geometrijskog niza:
\begin{equation}\label{eq:u1prn}
    \f u(x)=\sum_{i=0}^{x-1}3^i=\frac{3^{x-1+1}-1}{3-1}=\frac{3^x-\,3^x\bmod 2}{2}=3^x\!\sslash2
\end{equation}
($3^x$ je uvijek neparan, pa razlomak upravo predstavlja cjelobrojno dijeljenje).
\end{proof}

Kako sada pomoću funkcije $\f u^1$ možemo dobiti $\f u^2$? Naravno, $\upsilon(x,y)=\text\textbullet^x\!\t/\text\textbullet^y$ je sastavljena od tri dijela: $\text\textbullet^x=\upsilon(x)$, $\t/$ i $\text\textbullet^y=\upsilon(y)$, čije kodove znamo --- to su $\f u(x)$, $2$ i $\f u(y)$. Dakle, da dobijemo njen kod, trebamo još konkatenaciju prikazati kao brojevnu funkciju na kodovima. Ovo je slično primjeru~\ref{pr:concat}, samo što umjesto konačnih nizova bilo kakvih brojeva ovdje imamo konačne nizove znamenaka u bazi $b$ (ovdje konkretno u bazi $3$).

\begin{definicija}
Za $x,y,b\in\N$ takve da je $b\ge 2$, s $x\conc by$ označimo broj čiji je zapis u bazi $b$ dobiven konkatenacijom takvih zapisa od $x$ i od $y$ --- s tim da (uobičajeno) ne promatramo vodeće nule, tako da je broj $0$ neutralni element operacije $\conc b$.
\end{definicija}

\begin{primjer}
$23\conc{10}456=23\,456$, $3\conc25=(11\,101)_2=29$, $0\conc732=32$.
\end{primjer}

\begin{lema}\label{lm:Concatprn}
Postoji primitivno rekurzivna funkcija $\f{Concat}^3$, takva da za sve brojeve $x,y,b\in\N$ takve da je $b\ge2$, vrijedi $x\conc by=\f{Concat}(x,y,b)$.
\end{lema}

\begin{proof}
Možemo kombinirati tehnike iz lema~\ref{lm:recodeprn} i~\ref{lm:starprn}, i direktno pomoću $\f{Digit}$ ekstrahirati znamenke iz brojeva $x$ i $y$, i slagati ih u novi broj (pokušajte!), ali pokazuje se da postoji jednostavniji način: pomaknemo $x$ za onoliko mjesta udesno koliko $y$ ima znamenaka u bazi $b$, i dodamo $y$. Dakle, tvrdimo da
\begin{equation}
    \f{Concat}(x,y,b):=x\cdot b^{\f{Lh}(y,b)}+y
\end{equation}
zadovoljava specifikaciju (primitivno je rekurzivna po primjeru~\ref{pr:addmulpow} i lemi~\ref{lm:LhDigitcdprn}). Doista, ako napišemo $x=(x_{k-1}\dotsm x_1x_0)_b$ i $y=(y_{l-1}\dotsm y_1y_0)_b$ tako da ni $x_{k-1}$ ni $y_{l-1}$ nisu $0$, tada je (po lemi~\ref{lm:LhDigitcdprn}) $\f{Lh}(y,b)=l$, pa je
\begin{multline}
    \f{Concat}(x,y,b)=x\cdot b^l+y=\Bigl(\sum_{i<k}x_i\cdot b^i\Bigr)\cdot b^l+\sum_{i<l}y_i\cdot b^i
    =\sum_{i<l}y_ib^i+\sum_{i<k}x_ib^{i+l}=
    %=\sum_{i=0}^{l-1}y_ib^i+\sum_{i=l}^{\mathclap{k+l-1}}x_{i-l}b^l=
    \\
    =y_0b^0+y_1b^1+\dotsb+y_{l-1}b^{l-1}+x_0b^l+x_1b^{l+1}+\dotsb+x_{k-1}b^{k+l-1}=\\
    =(x_{k-1}\dotsm x_1x_0y_{l-1}\dotsm y_1y_0)_b=(x_{k-1}\dotsm x_0)_b\conc b(y_{l-1}\dotsm y_0)_b=x\conc b y\text.
\end{multline}
Naravno, taj izvod nema smisla za $x=0$ ili $y=0$, ali lako se vidi da je $\f{Lh}(0,b)=0$, pa je
$\f{Concat}(x,0,b)=x\cdot b^0+0=x$, i $\f{Concat}(0,y,b)=0\cdot b^{\f{Lh}(y,b)}+y=y$.
\end{proof}

\begin{propozicija}\label{pp:krkonkkr}
Neka je $\Sigma$ abeceda, $\N\Sigma$ njeno kodiranje, te $b$ njena baza. Tada za sve $u,v\in\Sigma^*$ vrijedi
    $\kr{uv}=\kr u\conc b\kr v$.
\end{propozicija}
\begin{proof}
Uz oznake $u=\alpha_1\alpha_2\dotsm\alpha_{\left|u\right|}$ i $v=\beta_1\beta_2\dotsm\beta_{\left|v\right|}$, vrijedi
\begin{multline}
\kr{uv}=\kr{\alpha_1\alpha_2\dotsm\alpha_{\left|u\right|}\beta_1\beta_2\dotsm\beta_{\left|v\right|}}=\bigl(
\N\Sigma(\alpha_{\left|u\right|})
\dotsm
\N\Sigma(\alpha_1)
\N\Sigma(\beta_{\left|v\right|})
\dotsm
\N\Sigma(\beta_1)
\bigr)_b=\\
=\bigl(
\N\Sigma(\alpha_{\left|u\right|})
\dotsm
\N\Sigma(\alpha_1)
\bigr)_b\conc b\bigl(
\N\Sigma(\beta_{\left|v\right|})
\dotsm
\N\Sigma(\beta_1)
\bigr)_b=\\
=\kr{\alpha_1\dotsm\alpha_{\left|u\right|}}\conc b\kr{\beta_1\dotsm\beta_{\left|v\right|}}=\kr u\conc b\kr v\text.
\end{multline}
Za $u=\varepsilon$ ili $v=\varepsilon$ tvrdnja također vrijedi jer je $\varepsilon$ neutralni element za konkatenaciju, a $\kr\varepsilon=0$ neutralni element za $\conc b$.
\end{proof}

\begin{propozicija}\label{pp:concprn}
Za svaki $b\ge 2$, $\conc b$ je primitivno rekurzivna asocijativna operacija.
\end{propozicija}
\begin{proof}
Primitivna rekurzivnost slijedi direktno iz leme~\ref{lm:Concatprn}. Također iz njenog dokaza slijedi $\f{Lh}(x\conc b y,b)=\f{Lh}(x,b)+\f{Lh}(y,b)$, pa vrijedi
\begin{multline}
    \left(x\conc b y\right)\conc b z=\left(x\cdot b^{\f{Lh}(y,b)}+y\right)\cdot b^{\f{Lh}(z,b)}+z=\\[1mm]
    =\left(x\cdot b^{\f{Lh}(y,b)}\cdot b^{\f{Lh}(z,b)}+y\cdot b^{\f{Lh}(z,b)}\right)+z=
    x\cdot b^{\f{Lh}(y,b)+\f{Lh}(z,b)}+\left(y\cdot b^{\f{Lh}(z,b)}+z\right)=\\[-1mm]
    =x\cdot b^{\f{Lh}\bigl(y\conc b z,b\bigr)}+\left(y\conc b z\right)=x\conc b\left(y\conc b z\right)\text,
\end{multline}
odnosno asocijativnost.
\end{proof}
Zbog asocijativnosti ćemo ubuduće pisati izraze poput $x\conc by\conc bz$ bez zagrada. Primijetimo samo da je zapis nule praznom rječju umjesto znamenkom $0$ bitan i ovdje: kad bismo nulu pisali "uobičajeno", imali bismo $(1\conc{10}'0)\conc{10}'2=10\conc{10}'2=102\not=1\conc{10}'(0\conc{10}'2)=1\conc{10}'2=12$, pa asocijativnost ne bi vrijedila.

Pomoću konkatenacije sada nije teško izračunati i ostale funkcije $\f u^k$.

\begin{propozicija}\label{pp:ukprn}
Za svaki $k\in\N_+$, funkcija $\f u^k$ je primitivno rekurzivna.
\end{propozicija}
\begin{proof}
Matematičkom indukcijom po $k$. Za $k=1$ (baza), to je upravo lema~\ref{lm:u1prn}. Pretpostavimo da je $\f u^l$ primitivno rekurzivna za neki $l\in\N_+$. Tada je
\begin{equation}
    \f u^{l+1}(\vec x,y)=
    \kr{\upsilon(\vec x,y)}=
    \kr{\text\textbullet^{x_1}\!\t/\dotsm\t/\text\textbullet^{x_l}\!\t/\text\textbullet^y}=
    \kr{\upsilon(\vec x)\t/\upsilon(y)}\text,
\end{equation}
što je po propoziciji~\ref{pp:krkonkkr} jednako
\begin{equation}
    \kr{\upsilon(\vec x)}\conc 3\kr{\t/}\conc 3\kr{\upsilon(y)}=
    \f u^k(\vec x)\conc32\conc3\f u^1(y)\text,
\end{equation}
te primitivna rekurzivnost od $\f u^{l+1}$ slijedi iz pretpostavke indukcije, propozicije~\ref{pp:concprn} i leme~\ref{lm:u1prn}.
\end{proof}

\subsection{Turing-izračunljive funkcije su parcijalno rekurzivne}

Promotrimo sada dijagram lijevo u~\eqref{dia:Nupsilon}, u svjetlu definicije~\ref{def:uk}, odnosno gledajući i točkane strelice. Definicija~\ref{def:uk} zapravo kaže da "kružni isječak" sa svake strane tog dijagrama komutira, pa iz toga i leme~\ref{lm:pravkomut} slijedi da čitava "\!vanjska elipsa" također komutira.

\begin{korolar}\label{kor:uf=Nupsfu}
Neka je $k\in\N_+$, te $f^k$ funkcija. Tada vrijedi
    $\f u^1\circ f=\N\upsilon f\circ\f u^k$.
\end{korolar}
\begin{proof}
To je jednakost~\eqref{eq:pravkomut} u koju je uvrštena (s obje strane) definicija~\ref{def:uk}.
\end{proof}

Sjetimo se, naš cilj je izraziti $f$ pomoću $\N\upsilon f$. S korolarom~\ref{kor:uf=Nupsfu} smo već jako blizu, treba nam samo još (izračunljiv) lijevi inverz za $\f u^1$. Kako možemo iz $364=(111111)_3=\f u(6)$ natrag dobiti $6$? Samo treba prebrojiti jedinice --- srećom, imamo pravu funkciju baš za to. Mogli bismo to zpisati kao $n=\f{cd}\bigl(\f u(n),1,3\bigr)$, ali zapravo možemo još lakše: budući da su sve znamenke jedinice, prebrojiti jedinice isto je kao i prebrojiti sve znamenke.

\begin{lema}\label{lm:Lh3}
Definiramo funkciju $\f{Lh3}$ s $\f{Lh3}(n):=\f{Lh}(n,3)$. 

Funkcija $\f{Lh3}$ je primitivno rekurzivna, i vrijedi $\f{Lh3}\circ\f u^1=\f I_1^1$.
\end{lema}
\begin{proof}
Primitivna rekurzivnost slijedi iz simboličke definicije $\f{Lh3}=\f{Lh}\circ(\f I_1^1,C_3^1)$, leme~\ref{lm:LhDigitcdprn} i propozicije~\ref{prop:konst}.

Uzmimo sada $n\in\N$, i označimo $y:=\f u^1(n)=3^n\!\sslash2$. Tada je $\f{Lh3}(y)=(\mu t<y)(3^t>y)$, a očito vrijedi $1<2<3$ i $3^n>0$, iz čega $3^n=\frac{3^n}{1}>\frac{3^n}{2}>\frac{3^n}{3}=3^{n-1}$, odnosno $\frac{3^n}{2}\in\langle3^{n-1}..\,3^n\rangle$. Za $n>0$, krajevi toga intervala su cjelobrojni, pa je $y=3^n\!\sslash2=\left\lfloor\frac{3^n}{2}\right\rfloor\in[3^{n-1}..\,3^n\rangle$, odnosno $3^{n-1}\le y<3^n$: $n-1$ još nije takav $t$ da je $3^t>y$, a $n$ jest takav. Dakle, $n$ je upravo najmanji takav $t$, odnosno $n=\f{Lh3}(y)$. Za $n=0$ je $y=3^0\!\sslash2=1\sslash2=0$, pa je također $\f{Lh3}(y)=\f{Lh3}(0)=0=n$.
\end{proof}

Na komutirajućem dijagramu koji cijelo vrijeme promatramo, $\f{Lh3}$ bismo mogli prikazati kao strelicu prema gore, s desne strane pravokutnika, ispupčenu još više nego strelica $\f u^1$. Upravo dokazana lema pokazuje da tako dobiveni "polumjesec" također komutira, pa i "proširena elipsa" također komutira --- čime smo napokon prikazali $f$ pomoću $\N\upsilon f$.

\begin{teorem}\label{tm:utip}
Neka je $k\in\N_+$, i\, $\f f^k$ funkcija takva da je $\upsilon\f f$ Turing-izračunljiva. Tada je $\f f$ parcijalno rekurzivna.
\end{teorem}
\begin{proof}
Prema teoremu~\ref{tm:tikp}, primijenjenom na abecedu $\Sigma_\upsilon$, kodiranje~\eqref{eq:kodSigmaups} i funkciju $\upsilon\f f$, funkcija $\N\upsilon\f f$ je parcijalno rekurzivna. Iz leme~\ref{lm:Lh3} i korolara~\ref{kor:uf=Nupsfu} sada imamo
\begin{equation}
    \f f=\f I_1^1\circ\f f=\f{Lh3}\circ\f u^1\!\circ \f f=\f{Lh3}\circ\N\upsilon\f f\circ\f u^k\text,
\end{equation}
odnosno $f$ je dobivena kompozicijom iz tri funkcije, od kojih je srednja parcijalno rekurzivna, a prva i zadnja su primitivno rekurzivne (lema~\ref{lm:Lh3} i propozicija~\ref{pp:ukprn}), pa su rekurzivne (korolar~\ref{kor:prnrek}) a time i parcijalno rekurzivne. Kako je skup parcijalno rekurzivnih funkcija zatvoren na kompoziciju, $\f f$ je parcijalno rekurzivna.
\end{proof}

\subsection{Funkcije \texorpdfstring{$\f{arg}_i$}{arg} --- ekstrakcija argumenata}\label{sec:arg}

Sad je jasno da isti komutirajući dijagram može poslužiti i za dokazivanje obrata teorema~\ref{tm:utip}. Jednadžba
\begin{equation}
    \f u^1\!\circ f=\N\upsilon f\circ\f u^k
\end{equation}
iz korolara~\ref{kor:uf=Nupsfu} može poslužiti i za dobivanje $\N\upsilon f$ iz $f$ --- samo trebamo umjesto desne obrnuti lijevu stranicu pravokutnika, odnosno naći desni inverz za $\f u^k$. Nažalost, to je iz tri razloga zapetljanije nego ovo što smo napravili u prethodnoj točki.

Prvi razlog je sasvim birokratski, i tiče se naše odluke sa samog početka, da ćemo promatrati brojevne funkcije s jednim izlaznim podatkom. Kako je $\dom{\f u^k}=\N^k$, traženi desni inverz bi morao imati $k$ izlaznih podataka, te ćemo ga reprezentirati kroz $k$ brojevnih funkcija $\f{arg}_1$, $\f{arg}_2$,~\ldots, $\f{arg}_k$ --- tako nazvanih jer ekstrahiraju pojedine argumente funkcije $f$ iz jedinog argumenta $a$ funkcije $\N\upsilon f$. (Te funkcije bit će neovisne o $k$: recimo, $\f{arg}_2\bigl(\f u^2(x,y)\bigr)=\f{arg}_2\bigl(\f u^4(x,y,z,t)\bigr)=y$, pa ih nećemo morati označavati s dva broja, kao što smo recimo morali označavati koordinatne projekcije $\f I_n^k$.)

Drugi razlog je tehnički: koristit ćemo funkcije $\f{Lh3}$ i $\f{cd}$ u bazi $3$, ali svejedno će implementacija biti kompliciranija nego u lemi~\ref{lm:Lh3}. Na neki način, dok su riječi nad jednočlanom abecedom trivijalno izomorfne s $\N$, riječi nad dvočlanom abecedom imaju bogatiju strukturu, i za njeno raščlanjivanje treba nam neka vrst "standardne biblioteke" za rad sa stringovima. Ovdje ćemo napraviti samo minimum potreban za primitivnu rekurzivnost funkcija $\f{arg}_i$, ali lako je vidjeti kako tu ima puno mogućnosti za implementaciju raznih algoritama.

Treći razlog, napokon, sasvim je stvaran i nije tako lako rješiv programskim tehnikama: funkcija $\f u^k$ nije surjekcija, i kao takva \emph{nema} totalni desni inverz. Ugrubo, imamo $\N\upsilon f\approx\f u^1\circ f\circ(\f{arg}_1,\dotsc,\f{arg}_k)$, ali da bismo dobili jednakost, morat ćemo razmišljati o specijalnim slučajevima --- riječ \t/ za unarnu reprezentaciju, broj $b'$ za prateću funkciju --- u kojima ne smijemo pozvati funkcije $\f{arg}_i$. Naime, zbog parcijalne specifikacije one bi mogle dati nešto što nije u domeni od $f$, pa bi desna strana bila nedefinirana zbog marljive evaluacije, a lijeva bi bila definirana; recimo u slučaju kad $a$ ima znamenku $0$ u bazi $3$, vrijednost $\N\upsilon f(a)$ je definirana kao $3$. Srećom, imamo tehniku (teorem o grananju, parcijalno rekurzivna verzija) za zaobilaženje marljive evaluacije u tom slučaju.

Krenimo s razvojem "minimalne standardne biblioteke" za rad sa stringovima (nad $\Sigma_\upsilon$). Za početak, za riječi $v,w\in\Sigma_\upsilon^*$ kažemo da je $v$ \emph{prefiks} od $w$ ako postoji riječ $u$ takva da je $w=vu$.

\begin{propozicija}
Relacija "prefiks" je refleksivni parcijalni uređaj na skupu $\Sigma_\upsilon^*$.
\end{propozicija}
\begin{proof}
Za refleksivnost, dovoljno je staviti $u:=\varepsilon$. Za tranzitivnost, samo treba iskoristiti asocijativnost kompozicije: ako je $w=vu$ i $v=v'u'$, tada je $w=(v'u')u=v'(u'u)$, pa je $v'$ prefiks od $w$.

Za antisimetričnost, lako se vidi da je duljina konkatenacije zbroj duljina, pa ako je $w=vu$ i $v=wu'$, tada je kao za tranzitivnost $w=w(uu')$, te vrijedi $\dulj w=\dulj{w(uu')}=\dulj w+\dulj{uu'}$, iz čega $\dulj{uu'}=0$. Kako je $\varepsilon$ jedina riječ duljine $0$, imamo $uu'=\varepsilon$, i analogno $u=u'=\varepsilon$, dakle $w=vu=v\varepsilon=v$.
\end{proof}

\begin{lema}
Dvomjesna brojevna relacija $\preceq_\upsilon$, zadana s
\begin{equation}
    x\preceq_\upsilon y:\Longleftrightarrow\text{"$x$ je kod prefiksa riječi čiji kod je $y$"}
\end{equation}
(kodovi su s obzirom na $\N\Sigma_\upsilon^*$), primitivno je rekurzivna.
\end{lema}
Recimo, vrijedi $5=(12)_3=\kr{\text\textbullet\t/}\preceq_\upsilon\kr{\t{\textbullet/\textbullet\textbullet\textbullet}}=(12111)_3=148$.

\begin{proof}
Mogli bismo se snalaziti s cjelobrojnim dijeljenjem ($y\sslash3^{\f{Lh3(y)\ominus Lh3(x)}}=x$, otprilike), ali možemo i samo pročitati definiciju. $\exists z\,(y=x\conc3z)$ je sasvim u redu (po propoziciji~\ref{pp:krkonkkr}), treba samo ograničiti ovaj $z$. Iz $y=x\cdot3^{\f{Lh3(z)}}+z\ge z$ slijedi $z\le y$, pa imamo
\begin{equation}
    x\preceq_\upsilon y\Longleftrightarrow\f{cd}(y,0,3)=0\land(\exists z\le y)(y=x\conc3z)\text,
\end{equation}
iz čega slijedi primitivna rekurzivnost. Prvi konjunkt je potreban da osiguramo da $y$ doista jest kod neke riječi iz $\Sigma_\upsilon^*$ --- ne trebamo pisati isti uvjet za $x$ jer po definiciji "prefiks" od $y$ ne može imati neku znamenku koju $y$ nema.
\end{proof}

Za sljedeću lemu, trebat će nam \emph{uokvireni} brojevi: oni čiji zapisi u bazi $3$ ne sadrže nulu, imaju bar dvije znamenke, te počinju i završavaju znamenkom $2$. Naime, ono što zapravo želimo napraviti je "isprogramirati" dokaz propozicije~\ref{prop:upsilonbij}, no taj dokaz ima nekoliko slučajeva ovisno o tome nalazimo li se prije prvog separatora, nakon zadnjeg, ili između njih. Jednostavnije je kad riječ ima separatore na lijevom i na desnom kraju: tada svaki $x_i$ možemo odrediti brojeći jedinice između susjednih separatora.

\begin{lema}\label{lm:pos2streak1prn}
Postoje primitivno rekurzivne funkcije $\f{pos2}$ i $\f{streak1}$, takve da za svaki uokvireni broj $x$, za svaki $i\in\N$, vrijedi:
\begin{labeling}{$\f{streak1}(x,i)$}
    \item[$\f{pos2}(x,i)$] je pozicija $i$-te znamenke $2$ (ili\, $\f{Lh3}(x)$ ako takva ne postoji), te
    \item[$\f{streak1}(x,i)$] je duljina $i$-tog niza uzastopnih jedinica (ili\, $0$ ako takav ne postoji),
\end{labeling}
počevši od $i=0$ slijeva, u zapisu broja $x$ u bazi $3$.
\end{lema}
\begin{proof}
Tvrdimo da funkcije zadane točkovno s
\begin{align}
    \f{pos2}(x,i)&:=\f{Lh3}\bigl((\mu z<x)(z\conc32\,\preceq_\upsilon x\,\land\,\f{cd}(z,2,3)=i)\bigr),\\
    \f{streak1}(x,i)&:=\f{pos2}\bigl(x,\f{Sc}(i)\bigr)\ominus\f{Sc}\bigl(\f{pos2}(x,i)\bigr),
\end{align}
zadovoljavaju tražene specifikacije (očito su primitivno rekurzivne).

Za $i<\f{cd}(x,2,3)$, uvjet pod operatorom minimizacije jednoznačno određuje $z$. Doista, kad bi postojala dva broja $z$ i $z'$ s tim svojstvom, morali bi biti kodovi različitih riječi $v$ i $v'$, takvih da su $v\t/$ i $v'\!\t/$ prefiksi od $w$, čiji kod je $x$. Prefiksi iste riječi iste duljine su jednaki, pa duljine od $v\t/$ i $v'\!\t/$ moraju biti različite: bez smanjenja općenitosti pretpostavimo $\dulj{v\t/}<\dulj{v'\!\t/}=\dulj{v'}+1$, dakle $v\t/$ je zapravo prefiks od $v'$, pa mora imati manje ili jednako znamenaka $2$ u bazi $3$. Dakle
\begin{multline}
    i=\f{cd}(z',2,3)=\f{cd}(\kr{v'},2,3)\ge\f{cd}(\kr{v\t/},2,3)=\f{cd}(\kr v\conc3\kr{\t/},2,3)=\\
    =\f{cd}(\kr v\conc32,2,3)=\f{cd}(\kr v,2,3)+1=\f{cd}(z,2,3)+1=i+1\text,
\end{multline}
kontradikcija. No ako je $z$ jedinstven, to znači da je dovoljno naći \emph{neki} takav i taj će biti minimalan, a kod "prefiksa" od $x$ do isključivo $i$-te dvojke zadovoljava to svojstvo. Njegova duljina je onda upravo ta pozicija (brojeći od $0$), kao što i treba biti.

Za $i\ge\f{cd}(x,2,3)$, takav $z$ ne postoji (jer $z\conc32$ ima više dvojki nego $x$, pa ne može biti $z\preceq_\upsilon x$), te minimizacija $(\mu z<x)$ dade $x$, odnosno vrijedi $\f{pos2}(x,i)=\f{Lh3}(x)$, opet, kao što i treba biti po specifikaciji.

Sada je za funkciju $\f{streak1}$ dovoljno oduzeti odgovarajuće pozicije: poziciju prvog znaka nakon $i$-te dvojke, od pozicije sljedeće dvojke. Za $i<\f{cd}(x,2,3)-1$, to funkcionira točno kako treba: kako uokvireni brojevi ne sadrže znamenku $0$, između susjednih dvojki nalaze se samo jedinice.

Za $i=\f{cd}(x,2,3)-1$, $i$-ta dvojka je upravo zadnja, pa je $\f{pos2}(x,i)=\f{Lh3}(x)-1$ (možemo pisati obične minuse jer uokvireni brojevi imaju bar dvije dvojke --- na početku i na kraju). Također po specifikaciji $\f{pos2}$ vrijedi $\f{pos2}(x,i+1)=\f{Lh3}(x)$, pa je $\f{streak1}(x,i)=\f{Lh3}(x)\ominus\f{Sc}\bigl(\f{Lh3}(x)-1\bigr)=0$.

Za $i\ge\f{cd}(x,2,3)$ imamo $\f{pos2}(x,i+1)=\f{pos2}(x,i)=\f{Lh3}(x)$, pa je opet $\f{streak1}(x,i)=\f{Lh3}(x)\ominus\f{Sc}\bigl(\f{Lh3}(x)\bigr)=0$.
\end{proof}

Sada je za ekstrakciju pojedinog $x_i$ samo potrebno uokviriti $\f u^k(\vec x)$ dvojkama, i pozvati funkciju $\f{streak1}$ s odgovarajućim argumentom.

\begin{propozicija}\label{pp:argnprn}
Za svaki $n\in\N_+$ postoji primitivno rekurzivna funkcija $\f{arg}_n$, takva da za sve $\vec x=(x_1,x_2,\dotsc,x_k)\in\N^*$, za sve $i\in\N_+$ vrijedi $\f{arg}_i\bigl(\f u(\vec x)\bigr)=\begin{cases}
x_i,&i\le k\\
0,&\text{inače}
\end{cases}$.
\end{propozicija}

\begin{proof}
Za svaki $n\in\N_+$ vrijedi $n-1\in\N$, pa definiramo
\begin{equation}
    \f{arg}_n(x):=\f{streak1}(2\conc3x\conc32,n-1)\text.
\end{equation}
Neka je $\vec x\in\N^k$ proizvoljan ($k\in\N_+$). Tada u $\upsilon(\vec x)$ ima točno $k-1$ separatora, pa je $\f{cd}\bigl(\f u(\vec x),2,3\bigr)=k-1$, odnosno za $y:=2\conc3\f u(\vec x)\conc32$ vrijedi $\f{cd}(y,2,3)=k-1+2=k+1$. Sada za svaki $i\in[1..k]$ vrijedi $\f{arg}_i\bigl(\f u(\vec x)\bigr)=\f{streak1}(y,i-1)$, što je upravo $x_i$ (recimo, za $i=2<k$ imat ćemo $\f{streak}(\kr{\t{/\textbullet}^{x_1}\!\t{/\textbullet}^{x_2}\!\t/\dotsm\t{/\textbullet}^{x_k}\!\t/},0)=\f{pos2}(y,2)-\f{Sc}\bigl(\f{pos2}(y,1)\bigr)=(1+x_1+1+x_2)-\f{Sc}(1+x_1)=x_2$).

Za $i>k$ imat ćemo $i-1\ge k=\f{cd}(y,2,3)-1$, pa će po lemi~\ref{lm:pos2streak1prn} $\f{streak1}(y,i-1)$ biti $0$, kao što i treba.
\end{proof}

\subsection{Parcijalno rekurzivne funkcije su Turing-izračunljive}

Napokon je sve spremno za dokaz obrata teorema~\ref{tm:utip}.

\begin{teorem}\label{tm:piut}
Za svaku parcijalno rekurzivnu funkciju $\f f$, $\upsilon\f f$ je Turing-izračunljiva.
\end{teorem}
\begin{proof}
Označimo s $k\in\N_+$ mjesnost od $\f f$. Cilj nam je dokazati da je $\N\upsilon\f f$ parcijalno rekurzivna. Tu funkciju možemo dobiti (iz dijagrama) kao $\f u^1\!\circ\f f\circ(\f{arg}_1,\dotsc,\f{arg}_k)$, na slici od $\f u^k$. Ta slika je primitivno rekurzivan skup, jer samo treba provjeriti da njegov element nema nijedne nule, i da ima točno $k-1$ dvojku.
\begin{equation}\label{eq:imuk}
    x\in\im{\f u^k}\Longleftrightarrow\f{cd}(x,0,3)=0\land\f{cd}(x,2,3)=k-1
\end{equation}

Što ako taj uvjet nije ispunjen? Može ne biti ispunjen na dva načina. Prvo, ako je $\f{cd}(x,0,3)>0$, to zapravo znači da $x$ nije u slici od $\f u^k=\N\Sigma_\upsilon^*\circ\upsilon^k$ jer uopće nije u slici od $\N\Sigma_\upsilon^*$. Tada $\N\upsilon\f f(x)$, kao i svako kodiranje jezične funkcije, po~\eqref{eq:kodfidef} poprima u $x$ vrijednost baze ulazne abecede, koja u ovom slučaju iznosi $3$.

Ako je $\f{cd}(x,0,3)=0$, to zapravo znači da $x$ jest kod neke riječi $w\in\Sigma_\upsilon^*$, i kako je $\upsilon$ bijekcija (propozicija \ref{prop:upsilonbij}), sigurno postoji neki $\vec y\in\N^+$ takav da je $w=\upsilon(\vec y)$. Ako u tom slučaju uvjet~\eqref{eq:imuk} ipak nije ispunjen, to zapravo znači da duljina od $\vec y$ nije $k$. Tada $\upsilon \f f(x)$ mora vratiti $\t/$ po~\eqref{eq:upsilonf}, pa $\N\upsilon\f f$ daje $\kr{\t/}=2$.

Dakle, vrijedi
\begin{align}
\SwapAboveDisplaySkip
\label{eq:defNupsf}
    \N\upsilon\f f(x)&\simeq\begin{cases}
    \f u^1\bigl(f(\vec a)\bigr),&x\in\im{\f u^k}\\
    3,&\f{cd}(x,0,3)>0\\
    2,&\text{inače}
    \end{cases}\text,\\
    \text{uz pokratu }\vec a&:=\bigl(\f{arg}_1(x),\f{arg}_2(x),\dotsc,\f{arg}_k(x)\bigr)\text,
\end{align}
iz čega slijedi, po teoremu~\ref{tm:gprek}, da je $\N\upsilon\f f$ parcijalno rekurzivna. Sada je po teoremu~\ref{tm:pir}, $\N\upsilon\f f$ RAM-izračunljiva, te je po korolaru~\ref{kor:krit} funkcija $\upsilon\f f$ Turing-iz\-rač\-un\-lji\-va, što smo i trebali.
\end{proof}

Primijetimo da srednji uvjet u~\eqref{eq:defNupsf} zapravo nije bio nužan (mogli smo i takve $x$ poslati u $2$, na primjer), jer za primjenu teorema~\ref{tm:krit} dovoljno da imamo RAM-program koji računa $\N\upsilon\f f$ na kodovima riječi. Ali ovako je jednakost jasnija i točnija, a budući da ionako moramo imati grananje, nije bitno kompliciranija.

Za kraj pogledajmo nekoliko vrijednosti koje $\N\upsilon\f f(x)$ može poprimiti, i što one znače.
\begin{itemize}
    \item[$0$] znači da je sve u redu, i $\f f(\vec a)=0$ ($\mathit{false}$). Tada je $\N\upsilon\f f(x)=\kr{\upsilon(0)}=\kr{\varepsilon}=0$.
    \item[$1$] znači da je sve u redu, i $\f f(\vec a)=1$ ($\mathit{true}$). Tada je $\N\upsilon\f f(x)=\kr{\upsilon(1)}=\kr{\text\textbullet}=1$.
    \item[$2$] znači da je $\vec a$ krive duljine, pa $\f f^k(\vec a)$ nema smisla (možemo $2$ shvatiti kao \emph{error code} za grešku "\emph{wrong argument count}")
    \item[$3$] znači da $\vec a$ ne postoji, jer se znamenka $0$ nalazi u nečem što bi trebao biti kod riječi (greška "\emph{zero byte in source}")
    \item[$4$] znači da je sve u redu, i $\f f(\vec a)=2$. Tada je $\N\upsilon\f f(x)=\kr{\upsilon(2)}=\kr{\t{\textbullet\textbullet}}=(11)_3=4$.
\end{itemize}
Sljedeća legalna vrijednost za $\N\upsilon\f f(x)$ je tek $(111)_3=13$, i ona odgovara $\f f(\vec a)=3$.

Sada je jasno da dinamizacijom funkcija $\f{arg}_i$ možemo simulirati i funkcije koje primaju proizvoljan broj parametara: recimo, $\f{add}^{\cdots}$, koja zbraja sve svoje argumente (koliko god da ih ima), je primitivno rekurzivna, jer je funkcija zadana s
\begin{equation}
    \N\upsilon\f{add}^{\cdots}(x)=\begin{cases}\quad\,\displaystyle\sum_{\mathclap{i\le\f{cd}(x,2,3)}}\f{streak1}(2\conc3x\conc32,i),&\f{cd}(x,0,3)=0\\
    \quad\,3,&\text{inače}
    \end{cases}
\end{equation}
primitivno rekurzivna po  teoremu~\ref{tm:grek}, lemama~\ref{lm:sumrek},~\ref{lm:LhDigitcdprn} i~\ref{lm:pos2streak1prn}, te propoziciji~\ref{pp:concprn}.
To nam neće bitno trebati u nastavku, ali zgodno je znati da imamo i tu mogućnost.